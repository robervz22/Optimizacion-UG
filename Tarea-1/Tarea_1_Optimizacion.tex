\documentclass[11pt]{article}

    \usepackage[breakable]{tcolorbox}
    \usepackage{parskip} % Stop auto-indenting (to mimic markdown behaviour)
    
    \usepackage{iftex}
    \ifPDFTeX
    	\usepackage[T1]{fontenc}
    	\usepackage{mathpazo}
    \else
    	\usepackage{fontspec}
    \fi

    % Basic figure setup, for now with no caption control since it's done
    % automatically by Pandoc (which extracts ![](path) syntax from Markdown).
    \usepackage{graphicx}
    % Maintain compatibility with old templates. Remove in nbconvert 6.0
    \let\Oldincludegraphics\includegraphics
    % Ensure that by default, figures have no caption (until we provide a
    % proper Figure object with a Caption API and a way to capture that
    % in the conversion process - todo).
    \usepackage{caption}
    \DeclareCaptionFormat{nocaption}{}
    \captionsetup{format=nocaption,aboveskip=0pt,belowskip=0pt}

    \usepackage[Export]{adjustbox} % Used to constrain images to a maximum size
    \adjustboxset{max size={0.9\linewidth}{0.9\paperheight}}
    \usepackage{float}
    \floatplacement{figure}{H} % forces figures to be placed at the correct location
    \usepackage{xcolor} % Allow colors to be defined
    \usepackage{enumerate} % Needed for markdown enumerations to work
    \usepackage{geometry} % Used to adjust the document margins
    \usepackage{amsmath} % Equations
    \usepackage{amssymb} % Equations
    \usepackage{textcomp} % defines textquotesingle
    % Hack from http://tex.stackexchange.com/a/47451/13684:
    \AtBeginDocument{%
        \def\PYZsq{\textquotesingle}% Upright quotes in Pygmentized code
    }
    \usepackage{upquote} % Upright quotes for verbatim code
    \usepackage{eurosym} % defines \euro
    \usepackage[mathletters]{ucs} % Extended unicode (utf-8) support
    \usepackage{fancyvrb} % verbatim replacement that allows latex
    \usepackage{grffile} % extends the file name processing of package graphics 
                         % to support a larger range
    \makeatletter % fix for grffile with XeLaTeX
    \def\Gread@@xetex#1{%
      \IfFileExists{"\Gin@base".bb}%
      {\Gread@eps{\Gin@base.bb}}%
      {\Gread@@xetex@aux#1}%
    }
    \makeatother

    % The hyperref package gives us a pdf with properly built
    % internal navigation ('pdf bookmarks' for the table of contents,
    % internal cross-reference links, web links for URLs, etc.)
    \usepackage{hyperref}
    % The default LaTeX title has an obnoxious amount of whitespace. By default,
    % titling removes some of it. It also provides customization options.
    \usepackage{titling}
    \usepackage{longtable} % longtable support required by pandoc >1.10
    \usepackage{booktabs}  % table support for pandoc > 1.12.2
    \usepackage[inline]{enumitem} % IRkernel/repr support (it uses the enumerate* environment)
    \usepackage[normalem]{ulem} % ulem is needed to support strikethroughs (\sout)
                                % normalem makes italics be italics, not underlines
    \usepackage{mathrsfs}
    

    
    % Colors for the hyperref package
    \definecolor{urlcolor}{rgb}{0,.145,.698}
    \definecolor{linkcolor}{rgb}{.71,0.21,0.01}
    \definecolor{citecolor}{rgb}{.12,.54,.11}

    % ANSI colors
    \definecolor{ansi-black}{HTML}{3E424D}
    \definecolor{ansi-black-intense}{HTML}{282C36}
    \definecolor{ansi-red}{HTML}{E75C58}
    \definecolor{ansi-red-intense}{HTML}{B22B31}
    \definecolor{ansi-green}{HTML}{00A250}
    \definecolor{ansi-green-intense}{HTML}{007427}
    \definecolor{ansi-yellow}{HTML}{DDB62B}
    \definecolor{ansi-yellow-intense}{HTML}{B27D12}
    \definecolor{ansi-blue}{HTML}{208FFB}
    \definecolor{ansi-blue-intense}{HTML}{0065CA}
    \definecolor{ansi-magenta}{HTML}{D160C4}
    \definecolor{ansi-magenta-intense}{HTML}{A03196}
    \definecolor{ansi-cyan}{HTML}{60C6C8}
    \definecolor{ansi-cyan-intense}{HTML}{258F8F}
    \definecolor{ansi-white}{HTML}{C5C1B4}
    \definecolor{ansi-white-intense}{HTML}{A1A6B2}
    \definecolor{ansi-default-inverse-fg}{HTML}{FFFFFF}
    \definecolor{ansi-default-inverse-bg}{HTML}{000000}

    % commands and environments needed by pandoc snippets
    % extracted from the output of `pandoc -s`
    \providecommand{\tightlist}{%
      \setlength{\itemsep}{0pt}\setlength{\parskip}{0pt}}
    \DefineVerbatimEnvironment{Highlighting}{Verbatim}{commandchars=\\\{\}}
    % Add ',fontsize=\small' for more characters per line
    \newenvironment{Shaded}{}{}
    \newcommand{\KeywordTok}[1]{\textcolor[rgb]{0.00,0.44,0.13}{\textbf{{#1}}}}
    \newcommand{\DataTypeTok}[1]{\textcolor[rgb]{0.56,0.13,0.00}{{#1}}}
    \newcommand{\DecValTok}[1]{\textcolor[rgb]{0.25,0.63,0.44}{{#1}}}
    \newcommand{\BaseNTok}[1]{\textcolor[rgb]{0.25,0.63,0.44}{{#1}}}
    \newcommand{\FloatTok}[1]{\textcolor[rgb]{0.25,0.63,0.44}{{#1}}}
    \newcommand{\CharTok}[1]{\textcolor[rgb]{0.25,0.44,0.63}{{#1}}}
    \newcommand{\StringTok}[1]{\textcolor[rgb]{0.25,0.44,0.63}{{#1}}}
    \newcommand{\CommentTok}[1]{\textcolor[rgb]{0.38,0.63,0.69}{\textit{{#1}}}}
    \newcommand{\OtherTok}[1]{\textcolor[rgb]{0.00,0.44,0.13}{{#1}}}
    \newcommand{\AlertTok}[1]{\textcolor[rgb]{1.00,0.00,0.00}{\textbf{{#1}}}}
    \newcommand{\FunctionTok}[1]{\textcolor[rgb]{0.02,0.16,0.49}{{#1}}}
    \newcommand{\RegionMarkerTok}[1]{{#1}}
    \newcommand{\ErrorTok}[1]{\textcolor[rgb]{1.00,0.00,0.00}{\textbf{{#1}}}}
    \newcommand{\NormalTok}[1]{{#1}}
    
    % Additional commands for more recent versions of Pandoc
    \newcommand{\ConstantTok}[1]{\textcolor[rgb]{0.53,0.00,0.00}{{#1}}}
    \newcommand{\SpecialCharTok}[1]{\textcolor[rgb]{0.25,0.44,0.63}{{#1}}}
    \newcommand{\VerbatimStringTok}[1]{\textcolor[rgb]{0.25,0.44,0.63}{{#1}}}
    \newcommand{\SpecialStringTok}[1]{\textcolor[rgb]{0.73,0.40,0.53}{{#1}}}
    \newcommand{\ImportTok}[1]{{#1}}
    \newcommand{\DocumentationTok}[1]{\textcolor[rgb]{0.73,0.13,0.13}{\textit{{#1}}}}
    \newcommand{\AnnotationTok}[1]{\textcolor[rgb]{0.38,0.63,0.69}{\textbf{\textit{{#1}}}}}
    \newcommand{\CommentVarTok}[1]{\textcolor[rgb]{0.38,0.63,0.69}{\textbf{\textit{{#1}}}}}
    \newcommand{\VariableTok}[1]{\textcolor[rgb]{0.10,0.09,0.49}{{#1}}}
    \newcommand{\ControlFlowTok}[1]{\textcolor[rgb]{0.00,0.44,0.13}{\textbf{{#1}}}}
    \newcommand{\OperatorTok}[1]{\textcolor[rgb]{0.40,0.40,0.40}{{#1}}}
    \newcommand{\BuiltInTok}[1]{{#1}}
    \newcommand{\ExtensionTok}[1]{{#1}}
    \newcommand{\PreprocessorTok}[1]{\textcolor[rgb]{0.74,0.48,0.00}{{#1}}}
    \newcommand{\AttributeTok}[1]{\textcolor[rgb]{0.49,0.56,0.16}{{#1}}}
    \newcommand{\InformationTok}[1]{\textcolor[rgb]{0.38,0.63,0.69}{\textbf{\textit{{#1}}}}}
    \newcommand{\WarningTok}[1]{\textcolor[rgb]{0.38,0.63,0.69}{\textbf{\textit{{#1}}}}}
    
    
    % Define a nice break command that doesn't care if a line doesn't already
    % exist.
    \def\br{\hspace*{\fill} \\* }
    % Math Jax compatibility definitions
    \def\gt{>}
    \def\lt{<}
    \let\Oldtex\TeX
    \let\Oldlatex\LaTeX
    \renewcommand{\TeX}{\textrm{\Oldtex}}
    \renewcommand{\LaTeX}{\textrm{\Oldlatex}}
    % Document parameters
    % Document title
    \title{Tarea 1}
    
    
    
    
    
% Pygments definitions
\makeatletter
\def\PY@reset{\let\PY@it=\relax \let\PY@bf=\relax%
    \let\PY@ul=\relax \let\PY@tc=\relax%
    \let\PY@bc=\relax \let\PY@ff=\relax}
\def\PY@tok#1{\csname PY@tok@#1\endcsname}
\def\PY@toks#1+{\ifx\relax#1\empty\else%
    \PY@tok{#1}\expandafter\PY@toks\fi}
\def\PY@do#1{\PY@bc{\PY@tc{\PY@ul{%
    \PY@it{\PY@bf{\PY@ff{#1}}}}}}}
\def\PY#1#2{\PY@reset\PY@toks#1+\relax+\PY@do{#2}}

\expandafter\def\csname PY@tok@w\endcsname{\def\PY@tc##1{\textcolor[rgb]{0.73,0.73,0.73}{##1}}}
\expandafter\def\csname PY@tok@c\endcsname{\let\PY@it=\textit\def\PY@tc##1{\textcolor[rgb]{0.25,0.50,0.50}{##1}}}
\expandafter\def\csname PY@tok@cp\endcsname{\def\PY@tc##1{\textcolor[rgb]{0.74,0.48,0.00}{##1}}}
\expandafter\def\csname PY@tok@k\endcsname{\let\PY@bf=\textbf\def\PY@tc##1{\textcolor[rgb]{0.00,0.50,0.00}{##1}}}
\expandafter\def\csname PY@tok@kp\endcsname{\def\PY@tc##1{\textcolor[rgb]{0.00,0.50,0.00}{##1}}}
\expandafter\def\csname PY@tok@kt\endcsname{\def\PY@tc##1{\textcolor[rgb]{0.69,0.00,0.25}{##1}}}
\expandafter\def\csname PY@tok@o\endcsname{\def\PY@tc##1{\textcolor[rgb]{0.40,0.40,0.40}{##1}}}
\expandafter\def\csname PY@tok@ow\endcsname{\let\PY@bf=\textbf\def\PY@tc##1{\textcolor[rgb]{0.67,0.13,1.00}{##1}}}
\expandafter\def\csname PY@tok@nb\endcsname{\def\PY@tc##1{\textcolor[rgb]{0.00,0.50,0.00}{##1}}}
\expandafter\def\csname PY@tok@nf\endcsname{\def\PY@tc##1{\textcolor[rgb]{0.00,0.00,1.00}{##1}}}
\expandafter\def\csname PY@tok@nc\endcsname{\let\PY@bf=\textbf\def\PY@tc##1{\textcolor[rgb]{0.00,0.00,1.00}{##1}}}
\expandafter\def\csname PY@tok@nn\endcsname{\let\PY@bf=\textbf\def\PY@tc##1{\textcolor[rgb]{0.00,0.00,1.00}{##1}}}
\expandafter\def\csname PY@tok@ne\endcsname{\let\PY@bf=\textbf\def\PY@tc##1{\textcolor[rgb]{0.82,0.25,0.23}{##1}}}
\expandafter\def\csname PY@tok@nv\endcsname{\def\PY@tc##1{\textcolor[rgb]{0.10,0.09,0.49}{##1}}}
\expandafter\def\csname PY@tok@no\endcsname{\def\PY@tc##1{\textcolor[rgb]{0.53,0.00,0.00}{##1}}}
\expandafter\def\csname PY@tok@nl\endcsname{\def\PY@tc##1{\textcolor[rgb]{0.63,0.63,0.00}{##1}}}
\expandafter\def\csname PY@tok@ni\endcsname{\let\PY@bf=\textbf\def\PY@tc##1{\textcolor[rgb]{0.60,0.60,0.60}{##1}}}
\expandafter\def\csname PY@tok@na\endcsname{\def\PY@tc##1{\textcolor[rgb]{0.49,0.56,0.16}{##1}}}
\expandafter\def\csname PY@tok@nt\endcsname{\let\PY@bf=\textbf\def\PY@tc##1{\textcolor[rgb]{0.00,0.50,0.00}{##1}}}
\expandafter\def\csname PY@tok@nd\endcsname{\def\PY@tc##1{\textcolor[rgb]{0.67,0.13,1.00}{##1}}}
\expandafter\def\csname PY@tok@s\endcsname{\def\PY@tc##1{\textcolor[rgb]{0.73,0.13,0.13}{##1}}}
\expandafter\def\csname PY@tok@sd\endcsname{\let\PY@it=\textit\def\PY@tc##1{\textcolor[rgb]{0.73,0.13,0.13}{##1}}}
\expandafter\def\csname PY@tok@si\endcsname{\let\PY@bf=\textbf\def\PY@tc##1{\textcolor[rgb]{0.73,0.40,0.53}{##1}}}
\expandafter\def\csname PY@tok@se\endcsname{\let\PY@bf=\textbf\def\PY@tc##1{\textcolor[rgb]{0.73,0.40,0.13}{##1}}}
\expandafter\def\csname PY@tok@sr\endcsname{\def\PY@tc##1{\textcolor[rgb]{0.73,0.40,0.53}{##1}}}
\expandafter\def\csname PY@tok@ss\endcsname{\def\PY@tc##1{\textcolor[rgb]{0.10,0.09,0.49}{##1}}}
\expandafter\def\csname PY@tok@sx\endcsname{\def\PY@tc##1{\textcolor[rgb]{0.00,0.50,0.00}{##1}}}
\expandafter\def\csname PY@tok@m\endcsname{\def\PY@tc##1{\textcolor[rgb]{0.40,0.40,0.40}{##1}}}
\expandafter\def\csname PY@tok@gh\endcsname{\let\PY@bf=\textbf\def\PY@tc##1{\textcolor[rgb]{0.00,0.00,0.50}{##1}}}
\expandafter\def\csname PY@tok@gu\endcsname{\let\PY@bf=\textbf\def\PY@tc##1{\textcolor[rgb]{0.50,0.00,0.50}{##1}}}
\expandafter\def\csname PY@tok@gd\endcsname{\def\PY@tc##1{\textcolor[rgb]{0.63,0.00,0.00}{##1}}}
\expandafter\def\csname PY@tok@gi\endcsname{\def\PY@tc##1{\textcolor[rgb]{0.00,0.63,0.00}{##1}}}
\expandafter\def\csname PY@tok@gr\endcsname{\def\PY@tc##1{\textcolor[rgb]{1.00,0.00,0.00}{##1}}}
\expandafter\def\csname PY@tok@ge\endcsname{\let\PY@it=\textit}
\expandafter\def\csname PY@tok@gs\endcsname{\let\PY@bf=\textbf}
\expandafter\def\csname PY@tok@gp\endcsname{\let\PY@bf=\textbf\def\PY@tc##1{\textcolor[rgb]{0.00,0.00,0.50}{##1}}}
\expandafter\def\csname PY@tok@go\endcsname{\def\PY@tc##1{\textcolor[rgb]{0.53,0.53,0.53}{##1}}}
\expandafter\def\csname PY@tok@gt\endcsname{\def\PY@tc##1{\textcolor[rgb]{0.00,0.27,0.87}{##1}}}
\expandafter\def\csname PY@tok@err\endcsname{\def\PY@bc##1{\setlength{\fboxsep}{0pt}\fcolorbox[rgb]{1.00,0.00,0.00}{1,1,1}{\strut ##1}}}
\expandafter\def\csname PY@tok@kc\endcsname{\let\PY@bf=\textbf\def\PY@tc##1{\textcolor[rgb]{0.00,0.50,0.00}{##1}}}
\expandafter\def\csname PY@tok@kd\endcsname{\let\PY@bf=\textbf\def\PY@tc##1{\textcolor[rgb]{0.00,0.50,0.00}{##1}}}
\expandafter\def\csname PY@tok@kn\endcsname{\let\PY@bf=\textbf\def\PY@tc##1{\textcolor[rgb]{0.00,0.50,0.00}{##1}}}
\expandafter\def\csname PY@tok@kr\endcsname{\let\PY@bf=\textbf\def\PY@tc##1{\textcolor[rgb]{0.00,0.50,0.00}{##1}}}
\expandafter\def\csname PY@tok@bp\endcsname{\def\PY@tc##1{\textcolor[rgb]{0.00,0.50,0.00}{##1}}}
\expandafter\def\csname PY@tok@fm\endcsname{\def\PY@tc##1{\textcolor[rgb]{0.00,0.00,1.00}{##1}}}
\expandafter\def\csname PY@tok@vc\endcsname{\def\PY@tc##1{\textcolor[rgb]{0.10,0.09,0.49}{##1}}}
\expandafter\def\csname PY@tok@vg\endcsname{\def\PY@tc##1{\textcolor[rgb]{0.10,0.09,0.49}{##1}}}
\expandafter\def\csname PY@tok@vi\endcsname{\def\PY@tc##1{\textcolor[rgb]{0.10,0.09,0.49}{##1}}}
\expandafter\def\csname PY@tok@vm\endcsname{\def\PY@tc##1{\textcolor[rgb]{0.10,0.09,0.49}{##1}}}
\expandafter\def\csname PY@tok@sa\endcsname{\def\PY@tc##1{\textcolor[rgb]{0.73,0.13,0.13}{##1}}}
\expandafter\def\csname PY@tok@sb\endcsname{\def\PY@tc##1{\textcolor[rgb]{0.73,0.13,0.13}{##1}}}
\expandafter\def\csname PY@tok@sc\endcsname{\def\PY@tc##1{\textcolor[rgb]{0.73,0.13,0.13}{##1}}}
\expandafter\def\csname PY@tok@dl\endcsname{\def\PY@tc##1{\textcolor[rgb]{0.73,0.13,0.13}{##1}}}
\expandafter\def\csname PY@tok@s2\endcsname{\def\PY@tc##1{\textcolor[rgb]{0.73,0.13,0.13}{##1}}}
\expandafter\def\csname PY@tok@sh\endcsname{\def\PY@tc##1{\textcolor[rgb]{0.73,0.13,0.13}{##1}}}
\expandafter\def\csname PY@tok@s1\endcsname{\def\PY@tc##1{\textcolor[rgb]{0.73,0.13,0.13}{##1}}}
\expandafter\def\csname PY@tok@mb\endcsname{\def\PY@tc##1{\textcolor[rgb]{0.40,0.40,0.40}{##1}}}
\expandafter\def\csname PY@tok@mf\endcsname{\def\PY@tc##1{\textcolor[rgb]{0.40,0.40,0.40}{##1}}}
\expandafter\def\csname PY@tok@mh\endcsname{\def\PY@tc##1{\textcolor[rgb]{0.40,0.40,0.40}{##1}}}
\expandafter\def\csname PY@tok@mi\endcsname{\def\PY@tc##1{\textcolor[rgb]{0.40,0.40,0.40}{##1}}}
\expandafter\def\csname PY@tok@il\endcsname{\def\PY@tc##1{\textcolor[rgb]{0.40,0.40,0.40}{##1}}}
\expandafter\def\csname PY@tok@mo\endcsname{\def\PY@tc##1{\textcolor[rgb]{0.40,0.40,0.40}{##1}}}
\expandafter\def\csname PY@tok@ch\endcsname{\let\PY@it=\textit\def\PY@tc##1{\textcolor[rgb]{0.25,0.50,0.50}{##1}}}
\expandafter\def\csname PY@tok@cm\endcsname{\let\PY@it=\textit\def\PY@tc##1{\textcolor[rgb]{0.25,0.50,0.50}{##1}}}
\expandafter\def\csname PY@tok@cpf\endcsname{\let\PY@it=\textit\def\PY@tc##1{\textcolor[rgb]{0.25,0.50,0.50}{##1}}}
\expandafter\def\csname PY@tok@c1\endcsname{\let\PY@it=\textit\def\PY@tc##1{\textcolor[rgb]{0.25,0.50,0.50}{##1}}}
\expandafter\def\csname PY@tok@cs\endcsname{\let\PY@it=\textit\def\PY@tc##1{\textcolor[rgb]{0.25,0.50,0.50}{##1}}}

\def\PYZbs{\char`\\}
\def\PYZus{\char`\_}
\def\PYZob{\char`\{}
\def\PYZcb{\char`\}}
\def\PYZca{\char`\^}
\def\PYZam{\char`\&}
\def\PYZlt{\char`\<}
\def\PYZgt{\char`\>}
\def\PYZsh{\char`\#}
\def\PYZpc{\char`\%}
\def\PYZdl{\char`\$}
\def\PYZhy{\char`\-}
\def\PYZsq{\char`\'}
\def\PYZdq{\char`\"}
\def\PYZti{\char`\~}
% for compatibility with earlier versions
\def\PYZat{@}
\def\PYZlb{[}
\def\PYZrb{]}
\makeatother


    % For linebreaks inside Verbatim environment from package fancyvrb. 
    \makeatletter
        \newbox\Wrappedcontinuationbox 
        \newbox\Wrappedvisiblespacebox 
        \newcommand*\Wrappedvisiblespace {\textcolor{red}{\textvisiblespace}} 
        \newcommand*\Wrappedcontinuationsymbol {\textcolor{red}{\llap{\tiny$\m@th\hookrightarrow$}}} 
        \newcommand*\Wrappedcontinuationindent {3ex } 
        \newcommand*\Wrappedafterbreak {\kern\Wrappedcontinuationindent\copy\Wrappedcontinuationbox} 
        % Take advantage of the already applied Pygments mark-up to insert 
        % potential linebreaks for TeX processing. 
        %        {, <, #, %, $, ' and ": go to next line. 
        %        _, }, ^, &, >, - and ~: stay at end of broken line. 
        % Use of \textquotesingle for straight quote. 
        \newcommand*\Wrappedbreaksatspecials {% 
            \def\PYGZus{\discretionary{\char`\_}{\Wrappedafterbreak}{\char`\_}}% 
            \def\PYGZob{\discretionary{}{\Wrappedafterbreak\char`\{}{\char`\{}}% 
            \def\PYGZcb{\discretionary{\char`\}}{\Wrappedafterbreak}{\char`\}}}% 
            \def\PYGZca{\discretionary{\char`\^}{\Wrappedafterbreak}{\char`\^}}% 
            \def\PYGZam{\discretionary{\char`\&}{\Wrappedafterbreak}{\char`\&}}% 
            \def\PYGZlt{\discretionary{}{\Wrappedafterbreak\char`\<}{\char`\<}}% 
            \def\PYGZgt{\discretionary{\char`\>}{\Wrappedafterbreak}{\char`\>}}% 
            \def\PYGZsh{\discretionary{}{\Wrappedafterbreak\char`\#}{\char`\#}}% 
            \def\PYGZpc{\discretionary{}{\Wrappedafterbreak\char`\%}{\char`\%}}% 
            \def\PYGZdl{\discretionary{}{\Wrappedafterbreak\char`\$}{\char`\$}}% 
            \def\PYGZhy{\discretionary{\char`\-}{\Wrappedafterbreak}{\char`\-}}% 
            \def\PYGZsq{\discretionary{}{\Wrappedafterbreak\textquotesingle}{\textquotesingle}}% 
            \def\PYGZdq{\discretionary{}{\Wrappedafterbreak\char`\"}{\char`\"}}% 
            \def\PYGZti{\discretionary{\char`\~}{\Wrappedafterbreak}{\char`\~}}% 
        } 
        % Some characters . , ; ? ! / are not pygmentized. 
        % This macro makes them "active" and they will insert potential linebreaks 
        \newcommand*\Wrappedbreaksatpunct {% 
            \lccode`\~`\.\lowercase{\def~}{\discretionary{\hbox{\char`\.}}{\Wrappedafterbreak}{\hbox{\char`\.}}}% 
            \lccode`\~`\,\lowercase{\def~}{\discretionary{\hbox{\char`\,}}{\Wrappedafterbreak}{\hbox{\char`\,}}}% 
            \lccode`\~`\;\lowercase{\def~}{\discretionary{\hbox{\char`\;}}{\Wrappedafterbreak}{\hbox{\char`\;}}}% 
            \lccode`\~`\:\lowercase{\def~}{\discretionary{\hbox{\char`\:}}{\Wrappedafterbreak}{\hbox{\char`\:}}}% 
            \lccode`\~`\?\lowercase{\def~}{\discretionary{\hbox{\char`\?}}{\Wrappedafterbreak}{\hbox{\char`\?}}}% 
            \lccode`\~`\!\lowercase{\def~}{\discretionary{\hbox{\char`\!}}{\Wrappedafterbreak}{\hbox{\char`\!}}}% 
            \lccode`\~`\/\lowercase{\def~}{\discretionary{\hbox{\char`\/}}{\Wrappedafterbreak}{\hbox{\char`\/}}}% 
            \catcode`\.\active
            \catcode`\,\active 
            \catcode`\;\active
            \catcode`\:\active
            \catcode`\?\active
            \catcode`\!\active
            \catcode`\/\active 
            \lccode`\~`\~ 	
        }
    \makeatother

    \let\OriginalVerbatim=\Verbatim
    \makeatletter
    \renewcommand{\Verbatim}[1][1]{%
        %\parskip\z@skip
        \sbox\Wrappedcontinuationbox {\Wrappedcontinuationsymbol}%
        \sbox\Wrappedvisiblespacebox {\FV@SetupFont\Wrappedvisiblespace}%
        \def\FancyVerbFormatLine ##1{\hsize\linewidth
            \vtop{\raggedright\hyphenpenalty\z@\exhyphenpenalty\z@
                \doublehyphendemerits\z@\finalhyphendemerits\z@
                \strut ##1\strut}%
        }%
        % If the linebreak is at a space, the latter will be displayed as visible
        % space at end of first line, and a continuation symbol starts next line.
        % Stretch/shrink are however usually zero for typewriter font.
        \def\FV@Space {%
            \nobreak\hskip\z@ plus\fontdimen3\font minus\fontdimen4\font
            \discretionary{\copy\Wrappedvisiblespacebox}{\Wrappedafterbreak}
            {\kern\fontdimen2\font}%
        }%
        
        % Allow breaks at special characters using \PYG... macros.
        \Wrappedbreaksatspecials
        % Breaks at punctuation characters . , ; ? ! and / need catcode=\active 	
        \OriginalVerbatim[#1,codes*=\Wrappedbreaksatpunct]%
    }
    \makeatother

    % Exact colors from NB
    \definecolor{incolor}{HTML}{303F9F}
    \definecolor{outcolor}{HTML}{D84315}
    \definecolor{cellborder}{HTML}{CFCFCF}
    \definecolor{cellbackground}{HTML}{F7F7F7}
    
    % prompt
    \makeatletter
    \newcommand{\boxspacing}{\kern\kvtcb@left@rule\kern\kvtcb@boxsep}
    \makeatother
    \newcommand{\prompt}[4]{
        \ttfamily\llap{{\color{#2}[#3]:\hspace{3pt}#4}}\vspace{-\baselineskip}
    }
    

    
    % Prevent overflowing lines due to hard-to-break entities
    \sloppy 
    % Setup hyperref package
    \hypersetup{
      breaklinks=true,  % so long urls are correctly broken across lines
      colorlinks=true,
      urlcolor=urlcolor,
      linkcolor=linkcolor,
      citecolor=citecolor,
      }
    % Slightly bigger margins than the latex defaults
    
    \geometry{verbose,tmargin=1in,bmargin=1in,lmargin=1in,rmargin=1in}
    
    

\begin{document}
    \title{Tarea 1 Optimización}
    \author{Roberto Vásquez Martínez \\ Profesor: Joaquín Peña Acevedo}
    \date{13/Febrero/2022}
    \maketitle
    
    

    
    \hypertarget{ejercicio-1-6-puntos}{%
\section{Ejercicio 1 (6 puntos)}\label{ejercicio-1-6-puntos}}

Programar y probar el método de la iteración de Halley para el cálculo
de raíces de una función de una variable.

\hypertarget{descripciuxf3n-del-muxe9todo}{%
\subsection{Descripción del
método}\label{descripciuxf3n-del-muxe9todo}}

El método de Halley usa una aproximación de la función \(f(x)\) de
segundo orden del desarrollo de Taylor de \(f(x)\).

\[ f(x_{k+1}) \approx f(x_k) +  f'(x_k) \Delta x + \frac{1}{2} f''(x_k) (\Delta x)^2 \]

Si igualamos a cero la aproximación tenemos que

\[
\Delta x = - \frac{f(x_k)}{  f'(x_k) + \frac{1}{2} f''(x_k) \Delta x }
\]

El valor \(\Delta x\) en el lado izquierdo de la igualdad corresponde a
\(\Delta x = x_{k+1} - x_{k}\), mientras que el que está en el
denominador se aproxima por el paso de Newton-Raphson:

\[ \Delta x = -\frac{f(x_k)}{f'(x_k)}, \]

de modo que

\[
x_{k+1} - x_{k} = 
- \frac{f(x_k)}{  f'(x_k) - \frac{1}{2} f''(x_k)f(x_k)/f'(x_k)  },
\]

es decir, el método de Halley propone generar la secuencia de puntos
mediante la siguiente regla:

\[
x_{k+1} = x_{k} 
- \frac{f(x_k)}{  f'(x_k) - \frac{f''(x_k)f(x_k)}{2f'(x_k)}  }.
\]

\begin{enumerate}
\def\labelenumi{\arabic{enumi}.}
\item
  Escriba la función que aplique el método de Halley. Debe recibir como
  argumentos un punto inicial \(x_0\), las función \(f(x)\), sus
  derivadas \(f'(x)\) y \(f''(x)\), el número máximo de iteraciones y un
  tolerancia \(\tau>0\), similar a la función \texttt{NewtonRaphson()}
  vista en el ejemplo de la clase, de modo que se detenga cuando se
  cumpla que \(|f(x_k)|< \tau\). Defina la variable \texttt{res} que
  indique el resultado obtenido (\texttt{res=0} se acabaron las
  iteraciones y no se encontró un punto que satisfaga el criterio de
  convergencia, \texttt{res=1} el algoritmo converge, \texttt{res=-1}
  hay un problema al evaluar la expresión. La función debe devolver el
  último punto \(x_k\), \(f(x_k)\), el número de iteraciones realizadas
  y la variable \texttt{res}.
\item
  Pruebe el algoritmo de Halley con las siguientes funciones y puntos
  inciales:
\end{enumerate}

\[ f_1(x) =  x^3 - 2x + 1, x_0=-1000,1000. \]

\[ f_2(x) =  1 + x - \frac{3}{2}x^2 + \frac{1}{6}x^3 + \frac{1}{4}x^4, x_0=-1000,1000. \]

En cada caso imprima \(x_0\), \(f(x_0)\), \(x_k\), \(f(x_k)\), el número
de iteraciones \(k\) realizadas y el valor de la variable \(res\).

\begin{enumerate}
\def\labelenumi{\arabic{enumi}.}
\setcounter{enumi}{2}
\tightlist
\item
  Repita las pruebas anteriores con el método de Newton-Raphson y
  escriba un comentario sobre los resultados.
\end{enumerate}

\hypertarget{soluciuxf3n}{%
\subsection{Solución:}\label{soluciuxf3n}}

    La funciones que implementan tanto el método de Halley como el algoritmo
de Newton-Raphson se encuentran el el módulo \texttt{lib\_t1.py} con los
nombres \texttt{Halley()} y \texttt{NewtonRaphson()}. Al igual que en la
clase, considerarmos a lo más \(100\) iteraciones, es decir,
\(iterMax=100\) y una tolerancia de \(\tau=10^{-8}\) que en el código
definimos como \texttt{tol=1e-8}.

    En las siguientes celdas de código probamos el algorimo de Halley con la
función \(f_1\).

Para el punto inicial \(x_0=-1000\) tenemos el siguiente resultado.

    \begin{tcolorbox}[breakable, size=fbox, boxrule=1pt, pad at break*=1mm,colback=cellbackground, colframe=cellborder]
\prompt{In}{incolor}{1}{\boxspacing}
\begin{Verbatim}[commandchars=\\\{\}]
\PY{k+kn}{from} \PY{n+nn}{lib\PYZus{}t1} \PY{k+kn}{import} \PY{o}{*}
\PY{n}{iterMax}\PY{o}{=}\PY{l+m+mi}{100}
\PY{n}{tol}\PY{o}{=}\PY{l+m+mf}{1e\PYZhy{}8}
\PY{n}{x0}\PY{o}{=}\PY{p}{[}\PY{o}{\PYZhy{}}\PY{l+m+mf}{1000.0}\PY{p}{,}\PY{l+m+mf}{1000.0}\PY{p}{]}
\PY{n}{dic\PYZus{}f1\PYZus{}Halley}\PY{o}{=}\PY{n}{Halley}\PY{p}{(}\PY{n}{x0}\PY{p}{[}\PY{l+m+mi}{0}\PY{p}{]}\PY{p}{,}\PY{n}{f1}\PY{p}{,}\PY{n}{der1\PYZus{}f1}\PY{p}{,}\PY{n}{der2\PYZus{}f1}\PY{p}{,}\PY{n}{iterMax}\PY{p}{,}\PY{n}{tol}\PY{p}{)}
\PY{n}{print\PYZus{}results}\PY{p}{(}\PY{n}{dic\PYZus{}f1\PYZus{}Halley}\PY{p}{)}
\end{Verbatim}
\end{tcolorbox}

    \begin{Verbatim}[commandchars=\\\{\}]
El algoritmo converge y los resultados son:
x0  :  -1000.0
f(x0)  :  -999997999.0
Raiz  :  -1.6180339887504553
f(Raiz)  :  -3.2809310823722626e-12
Iteraciones  :  12
Estado  :  1
    \end{Verbatim}

    Por otro lado, con el punto inicial \(x_0=1000\) obtenemos el siguiente
resultado

    \begin{tcolorbox}[breakable, size=fbox, boxrule=1pt, pad at break*=1mm,colback=cellbackground, colframe=cellborder]
\prompt{In}{incolor}{2}{\boxspacing}
\begin{Verbatim}[commandchars=\\\{\}]
\PY{n}{dic\PYZus{}f1\PYZus{}Halley}\PY{o}{=}\PY{n}{Halley}\PY{p}{(}\PY{n}{x0}\PY{p}{[}\PY{l+m+mi}{1}\PY{p}{]}\PY{p}{,}\PY{n}{f1}\PY{p}{,}\PY{n}{der1\PYZus{}f1}\PY{p}{,}\PY{n}{der2\PYZus{}f1}\PY{p}{,}\PY{n}{iterMax}\PY{p}{,}\PY{n}{tol}\PY{p}{)}
\PY{n}{print\PYZus{}results}\PY{p}{(}\PY{n}{dic\PYZus{}f1\PYZus{}Halley}\PY{p}{)}
\end{Verbatim}
\end{tcolorbox}

    \begin{Verbatim}[commandchars=\\\{\}]
El algoritmo converge y los resultados son:
x0  :  1000.0
f(x0)  :  999998001.0
Raiz  :  1.000000008768147
f(Raiz)  :  8.768147319315744e-09
Iteraciones  :  13
Estado  :  1
    \end{Verbatim}

    Notamos que en ambos casos obtenemos una raíz aunque estas raíces son
distintas, una positiva y una negativa. Podemos ver que las raíces de
\(f_1\) son
\[ x_1=1,\ x_2=\frac{-1-\sqrt{5}}{2}\approx -1.61\text{ y }x_3=\frac{-1+\sqrt{5}}{2}\approx 0.6180\]
Con las diferences condiciones iniciales hemos encontrado \(x_1\) y
\(x_2\), para hallar \(x_3\) quízas funcione inicializar el algoritmo de
Halley con un valor menor a \(x_1\), por ejemplo \(x_0=0.5\), eso lo
probamos en la siguiente celda.

    \begin{tcolorbox}[breakable, size=fbox, boxrule=1pt, pad at break*=1mm,colback=cellbackground, colframe=cellborder]
\prompt{In}{incolor}{3}{\boxspacing}
\begin{Verbatim}[commandchars=\\\{\}]
\PY{n}{dic\PYZus{}f1\PYZus{}Halley}\PY{o}{=}\PY{n}{Halley}\PY{p}{(}\PY{l+m+mf}{0.5}\PY{p}{,}\PY{n}{f1}\PY{p}{,}\PY{n}{der1\PYZus{}f1}\PY{p}{,}\PY{n}{der2\PYZus{}f1}\PY{p}{,}\PY{n}{iterMax}\PY{p}{,}\PY{n}{tol}\PY{p}{)}
\PY{n}{print\PYZus{}results}\PY{p}{(}\PY{n}{dic\PYZus{}f1\PYZus{}Halley}\PY{p}{)}
\end{Verbatim}
\end{tcolorbox}

    \begin{Verbatim}[commandchars=\\\{\}]
El algoritmo converge y los resultados son:
x0  :  0.5
f(x0)  :  0.125
Raiz  :  0.6180339887498948
f(Raiz)  :  0.0
Iteraciones  :  3
Estado  :  1
    \end{Verbatim}

    De esta forma hemos encontrado \(x_3\). Ahora procedemos a hallar raíces
de \(f_2\) a través del algoritmo de Halley con las condiciones
iniciales \(x_0=-1000,1000\). Para \(x_0=-1000\) tenemos el siguiente
resultado.

    \begin{tcolorbox}[breakable, size=fbox, boxrule=1pt, pad at break*=1mm,colback=cellbackground, colframe=cellborder]
\prompt{In}{incolor}{4}{\boxspacing}
\begin{Verbatim}[commandchars=\\\{\}]
\PY{n}{dic\PYZus{}f2\PYZus{}Halley}\PY{o}{=}\PY{n}{Halley}\PY{p}{(}\PY{n}{x0}\PY{p}{[}\PY{l+m+mi}{0}\PY{p}{]}\PY{p}{,}\PY{n}{f2}\PY{p}{,}\PY{n}{der1\PYZus{}f2}\PY{p}{,}\PY{n}{der2\PYZus{}f2}\PY{p}{,}\PY{n}{iterMax}\PY{p}{,}\PY{n}{tol}\PY{p}{)}
\PY{n}{print\PYZus{}results}\PY{p}{(}\PY{n}{dic\PYZus{}f2\PYZus{}Halley}\PY{p}{)}
\end{Verbatim}
\end{tcolorbox}

    \begin{Verbatim}[commandchars=\\\{\}]
El algoritmo converge y los resultados son:
x0  :  -1000.0
f(x0)  :  249831832334.33334
Raiz  :  -2.979654185792593
f(Raiz)  :  0.0
Iteraciones  :  15
Estado  :  1
    \end{Verbatim}

    Y para \(x_0=1000\) tenemos lo siguiente

    \begin{tcolorbox}[breakable, size=fbox, boxrule=1pt, pad at break*=1mm,colback=cellbackground, colframe=cellborder]
\prompt{In}{incolor}{5}{\boxspacing}
\begin{Verbatim}[commandchars=\\\{\}]
\PY{n}{dic\PYZus{}f2\PYZus{}Halley}\PY{o}{=}\PY{n}{Halley}\PY{p}{(}\PY{n}{x0}\PY{p}{[}\PY{l+m+mi}{1}\PY{p}{]}\PY{p}{,}\PY{n}{f2}\PY{p}{,}\PY{n}{der1\PYZus{}f2}\PY{p}{,}\PY{n}{der2\PYZus{}f2}\PY{p}{,}\PY{n}{iterMax}\PY{p}{,}\PY{n}{tol}\PY{p}{)}
\PY{n}{print\PYZus{}results}\PY{p}{(}\PY{n}{dic\PYZus{}f2\PYZus{}Halley}\PY{p}{)}
\end{Verbatim}
\end{tcolorbox}

    \begin{Verbatim}[commandchars=\\\{\}]
El algoritmo converge y los resultados son:
x0  :  1000.0
f(x0)  :  250165167667.66666
Raiz  :  -0.546729731825445
f(Raiz)  :  1.734723475976807e-17
Iteraciones  :  19
Estado  :  1
    \end{Verbatim}

    Se puede ver que ambas son raíces y además son las únicas dos raíces
reales de la función \(f_2\).

    Repetiremos lo anterior con el método de Newton-Raphson. Para \(f_1\)
con condición inicial \(x_0=-1000\) tenemos lo siguiente.

    \begin{tcolorbox}[breakable, size=fbox, boxrule=1pt, pad at break*=1mm,colback=cellbackground, colframe=cellborder]
\prompt{In}{incolor}{6}{\boxspacing}
\begin{Verbatim}[commandchars=\\\{\}]
\PY{n}{dic\PYZus{}f1\PYZus{}NR}\PY{o}{=}\PY{n}{NewtonRaphson}\PY{p}{(}\PY{n}{x0}\PY{p}{[}\PY{l+m+mi}{0}\PY{p}{]}\PY{p}{,}\PY{n}{f1}\PY{p}{,}\PY{n}{der1\PYZus{}f1}\PY{p}{,}\PY{n}{iterMax}\PY{p}{,}\PY{n}{tol}\PY{p}{)}
\PY{n}{print\PYZus{}results}\PY{p}{(}\PY{n}{dic\PYZus{}f1\PYZus{}NR}\PY{p}{)}
\end{Verbatim}
\end{tcolorbox}

    \begin{Verbatim}[commandchars=\\\{\}]
El algoritmo converge y los resultados son:
x0  :  -1000.0
f(x0)  :  -999997999.0
Raiz  :  -1.6180339888222295
f(Raiz)  :  -4.234541606251696e-10
Iteraciones  :  20
Estado  :  1
    \end{Verbatim}

    Y para \(x_0=1000\) obtenemos lo siguiente

    \begin{tcolorbox}[breakable, size=fbox, boxrule=1pt, pad at break*=1mm,colback=cellbackground, colframe=cellborder]
\prompt{In}{incolor}{7}{\boxspacing}
\begin{Verbatim}[commandchars=\\\{\}]
\PY{n}{dic\PYZus{}f1\PYZus{}NR}\PY{o}{=}\PY{n}{NewtonRaphson}\PY{p}{(}\PY{n}{x0}\PY{p}{[}\PY{l+m+mi}{1}\PY{p}{]}\PY{p}{,}\PY{n}{f1}\PY{p}{,}\PY{n}{der1\PYZus{}f1}\PY{p}{,}\PY{n}{iterMax}\PY{p}{,}\PY{n}{tol}\PY{p}{)}
\PY{n}{print\PYZus{}results}\PY{p}{(}\PY{n}{dic\PYZus{}f1\PYZus{}NR}\PY{p}{)}
\end{Verbatim}
\end{tcolorbox}

    \begin{Verbatim}[commandchars=\\\{\}]
El algoritmo converge y los resultados son:
x0  :  1000.0
f(x0)  :  999998001.0
Raiz  :  1.0000000028195468
f(Raiz)  :  2.8195468182445893e-09
Iteraciones  :  22
Estado  :  1
    \end{Verbatim}

    El algoritmo de Newton-Raphson encontró las mismas raíces que el
algoritmo de Halley respecto a las condiciones iniciales, sin embargo,
lo hizo en más iteraciones. Ahora probamos el método de Newton-Raphson
con \(f_2\) y la condición inicial \(x_0=-1000\).

    \begin{tcolorbox}[breakable, size=fbox, boxrule=1pt, pad at break*=1mm,colback=cellbackground, colframe=cellborder]
\prompt{In}{incolor}{8}{\boxspacing}
\begin{Verbatim}[commandchars=\\\{\}]
\PY{n}{dic\PYZus{}f2\PYZus{}NR}\PY{o}{=}\PY{n}{NewtonRaphson}\PY{p}{(}\PY{n}{x0}\PY{p}{[}\PY{l+m+mi}{0}\PY{p}{]}\PY{p}{,}\PY{n}{f2}\PY{p}{,}\PY{n}{der1\PYZus{}f2}\PY{p}{,}\PY{n}{iterMax}\PY{p}{,}\PY{n}{tol}\PY{p}{)}
\PY{n}{print\PYZus{}results}\PY{p}{(}\PY{n}{dic\PYZus{}f2\PYZus{}NR}\PY{p}{)}
\end{Verbatim}
\end{tcolorbox}

    \begin{Verbatim}[commandchars=\\\{\}]
El algoritmo converge y los resultados son:
x0  :  -1000.0
f(x0)  :  249831832334.33334
Raiz  :  -2.9796541859258454
f(Raiz)  :  1.6091874499579717e-09
Iteraciones  :  25
Estado  :  1
    \end{Verbatim}

    E inicializando con \(x_0=1000\) tenemos

    \begin{tcolorbox}[breakable, size=fbox, boxrule=1pt, pad at break*=1mm,colback=cellbackground, colframe=cellborder]
\prompt{In}{incolor}{9}{\boxspacing}
\begin{Verbatim}[commandchars=\\\{\}]
\PY{n}{dic\PYZus{}f2\PYZus{}NR}\PY{o}{=}\PY{n}{NewtonRaphson}\PY{p}{(}\PY{n}{x0}\PY{p}{[}\PY{l+m+mi}{1}\PY{p}{]}\PY{p}{,}\PY{n}{f2}\PY{p}{,}\PY{n}{der1\PYZus{}f2}\PY{p}{,}\PY{n}{iterMax}\PY{p}{,}\PY{n}{tol}\PY{p}{)}
\PY{n}{print\PYZus{}results}\PY{p}{(}\PY{n}{dic\PYZus{}f2\PYZus{}NR}\PY{p}{)}
\end{Verbatim}
\end{tcolorbox}

    \begin{Verbatim}[commandchars=\\\{\}]
El algortimo no converge
    \end{Verbatim}

    Con esta función con la condición inicial \(x_0=-1000\) encontramos la
mismas raíz que con el algoritmo de Halley, aunque otra vez con más
iteraciones. Por otro lado, inicializando con \(x_0=1000\) el algoritmo
no converge, es decir, \texttt{res=0}, se alcanza el número máximo de
iteraciones sin que se cumpla la condición de paro, por lo que se puede
inferir que para hallar la segunda raíz de \(f_2\) podría converge con
un número mucho mayor de iteraciones que con las que lo hizo el
algoritmo de Halley, y esto se debe a que la aproximación que usa el
algoritmo de Halley es de orden mayor que la usado en Newton Raphson, y
por lo tanto hay menos propagación de error en el algoritmo de Halley.
Otra posibilidad es que se requiere una mejor elección de la condición
inicial para el algoritmo de Newton-Raphson, si consideramos inicializar
con un número mucho más cercano a la segunda raíz por la derecha que
\(1000\), por ejemplo \(x_0=1\) como condición inicial obtenemos lo
siguiente.

    \begin{tcolorbox}[breakable, size=fbox, boxrule=1pt, pad at break*=1mm,colback=cellbackground, colframe=cellborder]
\prompt{In}{incolor}{10}{\boxspacing}
\begin{Verbatim}[commandchars=\\\{\}]
\PY{n}{dic\PYZus{}f2\PYZus{}NR}\PY{o}{=}\PY{n}{NewtonRaphson}\PY{p}{(}\PY{l+m+mf}{1.0}\PY{p}{,}\PY{n}{f2}\PY{p}{,}\PY{n}{der1\PYZus{}f2}\PY{p}{,}\PY{n}{iterMax}\PY{p}{,}\PY{n}{tol}\PY{p}{)}
\PY{n}{print\PYZus{}results}\PY{p}{(}\PY{n}{dic\PYZus{}f2\PYZus{}NR}\PY{p}{)}
\end{Verbatim}
\end{tcolorbox}

    \begin{Verbatim}[commandchars=\\\{\}]
El algoritmo converge y los resultados son:
x0  :  1.0
f(x0)  :  0.9166666666666666
Raiz  :  -0.5467297318263349
f(Raiz)  :  -2.336894566745684e-12
Iteraciones  :  51
Estado  :  1
    \end{Verbatim}

    Por lo que también podemos inferir que el algoritmo de Newton-Raphson es
mucho mas susceptible a la elección de condiciones iniciales. Y a pesar
de la condición inicial seleccionada, el algoritmo de Halley es mucho
mas eficiente en el número de iteraciones lo que muestra el poder de los
métodos basados en una aproximación de orden mayor, con el costo de
tener que obtener la segunda derivada.

    \hypertarget{ejercicio-2-4-puntos}{%
\section{Ejercicio 2 (4 puntos)}\label{ejercicio-2-4-puntos}}

Una manera de aproximar la función \(\cos(x)\) es mediante la función

\[ C(x; n) =  \sum_{i=0}^n c_i \]

donde \(n\) es un parámetro que indica la cantidad de términos en la
suma y

\[ c_i = -c_{i-1} \frac{x^2}{2i(2i-1)} \quad \text{y} \quad c_0 = 1.\]

\begin{enumerate}
\def\labelenumi{\arabic{enumi}.}
\tightlist
\item
  Programe la función \(C(x;n)\).
\item
  Imprima el valor del error \(C(x;n)-1\) para
  \(x \in \{2\pi, 8\pi, 12\pi \}\) y \(n = 10, 50, 100, 200\).
\item
  Imprima el valor del error \(C(x;n)+1\) para
  \(x \in \{\pi, 9\pi, 13\pi \}\) y \(n = 10, 50, 100, 200\).
\item
  Comente sobre el comportamiento de los errores obtenidos y cuál sería
  una manera apropiada de usar esta función.
\end{enumerate}

\hypertarget{soluciuxf3n}{%
\subsection{Solución:}\label{soluciuxf3n}}

    En primer lugar, programamos esta función cuyo código fuente se
encuentra en el módulo \texttt{lib\_t1.py}.

Para el numeral 2, imprimimos el vector \(|C(x;n)-1|\) para
\(n\in\{10,50,100,200\}\) y \(x\in\{2\pi,8\pi,12\pi\}\), por propiedades
básicas de la función coseno en cada uno de los valores \(x\) se tiene
\(\cos(x)=1\), por lo que \(|C(x;n)-1|\) es el vector de errores que se
comete para cada \(x\) dado \(n\). Imprimimos la evaluación de la
función coseno en el vector de valores \([2\pi,8\pi,12\pi]\) iterando
sobre cada valor de \(n\).

    \begin{tcolorbox}[breakable, size=fbox, boxrule=1pt, pad at break*=1mm,colback=cellbackground, colframe=cellborder]
\prompt{In}{incolor}{12}{\boxspacing}
\begin{Verbatim}[commandchars=\\\{\}]
\PY{k+kn}{import} \PY{n+nn}{math} \PY{k}{as} \PY{n+nn}{mt}
\PY{k+kn}{import} \PY{n+nn}{numpy} \PY{k}{as} \PY{n+nn}{np}
\PY{n}{list\PYZus{}n}\PY{o}{=}\PY{p}{[}\PY{l+m+mi}{10}\PY{p}{,}\PY{l+m+mi}{50}\PY{p}{,}\PY{l+m+mi}{100}\PY{p}{,}\PY{l+m+mi}{200}\PY{p}{]}
\PY{n}{values\PYZus{}even}\PY{o}{=}\PY{n}{np}\PY{o}{.}\PY{n}{array}\PY{p}{(}\PY{p}{[}\PY{l+m+mf}{2.0}\PY{o}{*}\PY{n}{mt}\PY{o}{.}\PY{n}{pi}\PY{p}{,}\PY{l+m+mf}{8.0}\PY{o}{*}\PY{n}{mt}\PY{o}{.}\PY{n}{pi}\PY{p}{,}\PY{l+m+mf}{12.0}\PY{o}{*}\PY{n}{mt}\PY{o}{.}\PY{n}{pi}\PY{p}{]}\PY{p}{)}
\PY{k}{for} \PY{n}{n} \PY{o+ow}{in} \PY{n}{list\PYZus{}n}\PY{p}{:}
    \PY{n+nb}{print}\PY{p}{(}\PY{l+s+sa}{f}\PY{l+s+s1}{\PYZsq{}}\PY{l+s+s1}{Para n=}\PY{l+s+si}{\PYZob{}}\PY{n}{n}\PY{l+s+si}{\PYZcb{}}\PY{l+s+s1}{ |C(x;n)\PYZhy{}1| es igual a: }\PY{l+s+s1}{\PYZsq{}}\PY{p}{,}\PY{n+nb}{abs}\PY{p}{(}\PY{n}{pol\PYZus{}taylor\PYZus{}cos}\PY{p}{(}\PY{n}{values\PYZus{}even}\PY{p}{,}\PY{n}{n}\PY{p}{)}\PY{o}{\PYZhy{}}\PY{l+m+mf}{1.0}\PY{p}{)}\PY{p}{)}
\end{Verbatim}
\end{tcolorbox}

    \begin{Verbatim}[commandchars=\\\{\}]
Para n=10 |C(x;n)-1| es igual a:  [3.01224042e-04 2.51526888e+09 1.08015596e+13]
Para n=50 |C(x;n)-1| es igual a:  [4.66293670e-15 4.86928924e-07 1.91766915e-01]
Para n=100 |C(x;n)-1| es igual a:  [4.66293670e-15 4.86928924e-07
1.35724564e-01]
Para n=200 |C(x;n)-1| es igual a:  [4.66293670e-15 4.86928924e-07
1.35724564e-01]
    \end{Verbatim}

    Haciendo el mismo ejercicio, imprimimos el vector de errores
\(|C(x;n)+1|\) para \(n\in\{10,50,100,200\}\) y
\(x\in\{3\pi,9\pi,13\pi\}\)

    \begin{tcolorbox}[breakable, size=fbox, boxrule=1pt, pad at break*=1mm,colback=cellbackground, colframe=cellborder]
\prompt{In}{incolor}{13}{\boxspacing}
\begin{Verbatim}[commandchars=\\\{\}]
\PY{n}{values\PYZus{}odd}\PY{o}{=}\PY{n}{np}\PY{o}{.}\PY{n}{array}\PY{p}{(}\PY{p}{[}\PY{l+m+mf}{3.0}\PY{o}{*}\PY{n}{mt}\PY{o}{.}\PY{n}{pi}\PY{p}{,}\PY{l+m+mf}{9.0}\PY{o}{*}\PY{n}{mt}\PY{o}{.}\PY{n}{pi}\PY{p}{,}\PY{l+m+mf}{13.0}\PY{o}{*}\PY{n}{mt}\PY{o}{.}\PY{n}{pi}\PY{p}{]}\PY{p}{)}
\PY{k}{for} \PY{n}{n} \PY{o+ow}{in} \PY{n}{list\PYZus{}n}\PY{p}{:}
    \PY{n+nb}{print}\PY{p}{(}\PY{l+s+sa}{f}\PY{l+s+s1}{\PYZsq{}}\PY{l+s+s1}{Para n=}\PY{l+s+si}{\PYZob{}}\PY{n}{n}\PY{l+s+si}{\PYZcb{}}\PY{l+s+s1}{ |C(x;n)\PYZhy{}1| es igual a: }\PY{l+s+s1}{\PYZsq{}}\PY{p}{,}\PY{n+nb}{abs}\PY{p}{(}\PY{n}{pol\PYZus{}taylor\PYZus{}cos}\PY{p}{(}\PY{n}{values\PYZus{}odd}\PY{p}{,}\PY{n}{n}\PY{p}{)}\PY{o}{+}\PY{l+m+mf}{1.0}\PY{p}{)}\PY{p}{)}
\end{Verbatim}
\end{tcolorbox}

    \begin{Verbatim}[commandchars=\\\{\}]
Para n=10 |C(x;n)-1| es igual a:  [2.07517810e+00 2.90378712e+10 5.53930634e+13]
Para n=50 |C(x;n)-1| es igual a:  [2.14495088e-13 1.09779085e-05 1.94408879e+02]
Para n=100 |C(x;n)-1| es igual a:  [2.14495088e-13 1.09779085e-05
1.46360832e+00]
Para n=200 |C(x;n)-1| es igual a:  [2.14495088e-13 1.09779085e-05
1.46360832e+00]
    \end{Verbatim}

    Sabemos que teóricamente el valor de \(cos(x)\) es \(1\) para todo
\(x\in \{2\pi,8\pi,12\pi\}\) y \(-1\) para todo \(\{3\pi,9\pi,12\pi\}\),
sin embargo, en base a los resultados anteriores la aproximación
\(C(x;n)\) para \(n\) fijo comete mayor error cuanto más crece el valor
de \(x\).

Para explicar esto recurrimos el \textbf{Teorema de Taylor}. Es claro
que \(C(x;n)\) es el polinomio de Taylor de grado \(2n\) de la función
coseno, luego por el Teorema de Taylor se tiene que
\[\cos x=C(x;n)+R_{2n}(x),\] donde
\[ R_{2n}(x)=\cos^{(2n+1)}(t)\frac{x^{2n+1}}{(2n+1)!} \text{ para algún }t\in [0,x],\]
y \(\cos^{(2n+1)}\) representa la derivada de orden \(2n+1\) de la
función coseno.

Por lo tanto, tenemos que el error en la aproximación tiene la forma
\[ |\cos x- C(x;n)|=|\cos^{(2n+1)}(t)|\cdot |\frac{x^{2n+1}}{(2n+1)!}|,\]
luego cuando \(x\) se hace grande el factor \(\frac{x^{2n+1}}{(2n+1)!}\)
crece exponencialmente con \(n\) fijo, por tanto el error en la
aproximación crece exponencialmente a medida que \(x\) crece.

Sin embargo, podemo reducir el factor de crecimiento del error
utilizando que la función coseno es periódica. Para \(x\geq 0\)
arbitrario sea
\[ n_x=\max\{n\in\mathbb{N}\cup\{0\} |\ x-2\pi\cdot n>0\}.\]

Por lo tanto, si \(t_x=x-2\pi\cdot n_x\) entonces por la definición de
\(n_x\) se tiene que \(t_x\in [0,2\pi)\) y \(t_x\) es el menor real
positivo tal que \[ \cos(x)=\cos(t_x).\]

Para \(x\leq 0\) utilizamos la paridad de la función coseno y
consideramos lo anterior con \(x:=-x\).

Por lo que aproximamos \(\cos x\) a través de \(C(t_x;n)\), y en este
caso podemos acotar el error de la siguiente forma
\[ |\cos(x)-C(t_x;n)|\leq |\frac{t_x^{2n+1}}{(2n+1)!}|\leq \frac{(2\pi)^{2n+1}}{(2n+1)!},\]
y observamos que el error se va \(0\) más rápido cuando \(n\to \infty\),
pues hemos acotado por un valor que ya no depende de \(x\).

Concluimos así que para utilizar la aproximación a \(\cos(x)\) es más
conveniente usar \(C(t_x;n)\) que \(C(x;n)\), especialmente para \(n\)
pequeños pues es el caso en el que el denominador \((2n+1)!\) no tiene
tanto impacto; además al utilizar \(t_x\) el error se irá mucho más
rápido a \(0\) a medida que \(n\to\infty\).


    % Add a bibliography block to the postdoc
    
    
    
\end{document}
