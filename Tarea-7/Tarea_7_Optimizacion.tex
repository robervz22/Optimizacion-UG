\documentclass[11pt]{article}

    \usepackage[breakable]{tcolorbox}
    \usepackage{parskip} % Stop auto-indenting (to mimic markdown behaviour)
    
    \usepackage{iftex}
    \ifPDFTeX
    	\usepackage[T1]{fontenc}
    	\usepackage{mathpazo}
    \else
    	\usepackage{fontspec}
    \fi

    % Basic figure setup, for now with no caption control since it's done
    % automatically by Pandoc (which extracts ![](path) syntax from Markdown).
    \usepackage{graphicx}
    % Maintain compatibility with old templates. Remove in nbconvert 6.0
    \let\Oldincludegraphics\includegraphics
    % Ensure that by default, figures have no caption (until we provide a
    % proper Figure object with a Caption API and a way to capture that
    % in the conversion process - todo).
    \usepackage{caption}
    \DeclareCaptionFormat{nocaption}{}
    \captionsetup{format=nocaption,aboveskip=0pt,belowskip=0pt}

    \usepackage[Export]{adjustbox} % Used to constrain images to a maximum size
    \adjustboxset{max size={0.9\linewidth}{0.9\paperheight}}
    \usepackage{float}
    \floatplacement{figure}{H} % forces figures to be placed at the correct location
    \usepackage{xcolor} % Allow colors to be defined
    \usepackage{enumerate} % Needed for markdown enumerations to work
    \usepackage{geometry} % Used to adjust the document margins
    \usepackage{amsmath} % Equations
    \usepackage{amssymb} % Equations
    \usepackage{textcomp} % defines textquotesingle
    % Hack from http://tex.stackexchange.com/a/47451/13684:
    \AtBeginDocument{%
        \def\PYZsq{\textquotesingle}% Upright quotes in Pygmentized code
    }
    \usepackage{upquote} % Upright quotes for verbatim code
    \usepackage{eurosym} % defines \euro
    \usepackage[mathletters]{ucs} % Extended unicode (utf-8) support
    \usepackage{fancyvrb} % verbatim replacement that allows latex
    \usepackage{grffile} % extends the file name processing of package graphics 
                         % to support a larger range
    \makeatletter % fix for grffile with XeLaTeX
    \def\Gread@@xetex#1{%
      \IfFileExists{"\Gin@base".bb}%
      {\Gread@eps{\Gin@base.bb}}%
      {\Gread@@xetex@aux#1}%
    }
    \makeatother

    % The hyperref package gives us a pdf with properly built
    % internal navigation ('pdf bookmarks' for the table of contents,
    % internal cross-reference links, web links for URLs, etc.)
    \usepackage{hyperref}
    % The default LaTeX title has an obnoxious amount of whitespace. By default,
    % titling removes some of it. It also provides customization options.
    \usepackage{titling}
    \usepackage{longtable} % longtable support required by pandoc >1.10
    \usepackage{booktabs}  % table support for pandoc > 1.12.2
    \usepackage[inline]{enumitem} % IRkernel/repr support (it uses the enumerate* environment)
    \usepackage[normalem]{ulem} % ulem is needed to support strikethroughs (\sout)
                                % normalem makes italics be italics, not underlines
    \usepackage{mathrsfs}
    

    
    % Colors for the hyperref package
    \definecolor{urlcolor}{rgb}{0,.145,.698}
    \definecolor{linkcolor}{rgb}{.71,0.21,0.01}
    \definecolor{citecolor}{rgb}{.12,.54,.11}

    % ANSI colors
    \definecolor{ansi-black}{HTML}{3E424D}
    \definecolor{ansi-black-intense}{HTML}{282C36}
    \definecolor{ansi-red}{HTML}{E75C58}
    \definecolor{ansi-red-intense}{HTML}{B22B31}
    \definecolor{ansi-green}{HTML}{00A250}
    \definecolor{ansi-green-intense}{HTML}{007427}
    \definecolor{ansi-yellow}{HTML}{DDB62B}
    \definecolor{ansi-yellow-intense}{HTML}{B27D12}
    \definecolor{ansi-blue}{HTML}{208FFB}
    \definecolor{ansi-blue-intense}{HTML}{0065CA}
    \definecolor{ansi-magenta}{HTML}{D160C4}
    \definecolor{ansi-magenta-intense}{HTML}{A03196}
    \definecolor{ansi-cyan}{HTML}{60C6C8}
    \definecolor{ansi-cyan-intense}{HTML}{258F8F}
    \definecolor{ansi-white}{HTML}{C5C1B4}
    \definecolor{ansi-white-intense}{HTML}{A1A6B2}
    \definecolor{ansi-default-inverse-fg}{HTML}{FFFFFF}
    \definecolor{ansi-default-inverse-bg}{HTML}{000000}

    % commands and environments needed by pandoc snippets
    % extracted from the output of `pandoc -s`
    \providecommand{\tightlist}{%
      \setlength{\itemsep}{0pt}\setlength{\parskip}{0pt}}
    \DefineVerbatimEnvironment{Highlighting}{Verbatim}{commandchars=\\\{\}}
    % Add ',fontsize=\small' for more characters per line
    \newenvironment{Shaded}{}{}
    \newcommand{\KeywordTok}[1]{\textcolor[rgb]{0.00,0.44,0.13}{\textbf{{#1}}}}
    \newcommand{\DataTypeTok}[1]{\textcolor[rgb]{0.56,0.13,0.00}{{#1}}}
    \newcommand{\DecValTok}[1]{\textcolor[rgb]{0.25,0.63,0.44}{{#1}}}
    \newcommand{\BaseNTok}[1]{\textcolor[rgb]{0.25,0.63,0.44}{{#1}}}
    \newcommand{\FloatTok}[1]{\textcolor[rgb]{0.25,0.63,0.44}{{#1}}}
    \newcommand{\CharTok}[1]{\textcolor[rgb]{0.25,0.44,0.63}{{#1}}}
    \newcommand{\StringTok}[1]{\textcolor[rgb]{0.25,0.44,0.63}{{#1}}}
    \newcommand{\CommentTok}[1]{\textcolor[rgb]{0.38,0.63,0.69}{\textit{{#1}}}}
    \newcommand{\OtherTok}[1]{\textcolor[rgb]{0.00,0.44,0.13}{{#1}}}
    \newcommand{\AlertTok}[1]{\textcolor[rgb]{1.00,0.00,0.00}{\textbf{{#1}}}}
    \newcommand{\FunctionTok}[1]{\textcolor[rgb]{0.02,0.16,0.49}{{#1}}}
    \newcommand{\RegionMarkerTok}[1]{{#1}}
    \newcommand{\ErrorTok}[1]{\textcolor[rgb]{1.00,0.00,0.00}{\textbf{{#1}}}}
    \newcommand{\NormalTok}[1]{{#1}}
    
    % Additional commands for more recent versions of Pandoc
    \newcommand{\ConstantTok}[1]{\textcolor[rgb]{0.53,0.00,0.00}{{#1}}}
    \newcommand{\SpecialCharTok}[1]{\textcolor[rgb]{0.25,0.44,0.63}{{#1}}}
    \newcommand{\VerbatimStringTok}[1]{\textcolor[rgb]{0.25,0.44,0.63}{{#1}}}
    \newcommand{\SpecialStringTok}[1]{\textcolor[rgb]{0.73,0.40,0.53}{{#1}}}
    \newcommand{\ImportTok}[1]{{#1}}
    \newcommand{\DocumentationTok}[1]{\textcolor[rgb]{0.73,0.13,0.13}{\textit{{#1}}}}
    \newcommand{\AnnotationTok}[1]{\textcolor[rgb]{0.38,0.63,0.69}{\textbf{\textit{{#1}}}}}
    \newcommand{\CommentVarTok}[1]{\textcolor[rgb]{0.38,0.63,0.69}{\textbf{\textit{{#1}}}}}
    \newcommand{\VariableTok}[1]{\textcolor[rgb]{0.10,0.09,0.49}{{#1}}}
    \newcommand{\ControlFlowTok}[1]{\textcolor[rgb]{0.00,0.44,0.13}{\textbf{{#1}}}}
    \newcommand{\OperatorTok}[1]{\textcolor[rgb]{0.40,0.40,0.40}{{#1}}}
    \newcommand{\BuiltInTok}[1]{{#1}}
    \newcommand{\ExtensionTok}[1]{{#1}}
    \newcommand{\PreprocessorTok}[1]{\textcolor[rgb]{0.74,0.48,0.00}{{#1}}}
    \newcommand{\AttributeTok}[1]{\textcolor[rgb]{0.49,0.56,0.16}{{#1}}}
    \newcommand{\InformationTok}[1]{\textcolor[rgb]{0.38,0.63,0.69}{\textbf{\textit{{#1}}}}}
    \newcommand{\WarningTok}[1]{\textcolor[rgb]{0.38,0.63,0.69}{\textbf{\textit{{#1}}}}}
    
    
    % Define a nice break command that doesn't care if a line doesn't already
    % exist.
    \def\br{\hspace*{\fill} \\* }
    % Math Jax compatibility definitions
    \def\gt{>}
    \def\lt{<}
    \let\Oldtex\TeX
    \let\Oldlatex\LaTeX
    \renewcommand{\TeX}{\textrm{\Oldtex}}
    \renewcommand{\LaTeX}{\textrm{\Oldlatex}}
    % Document parameters
    % Document title
    \title{Tarea\_7\_Optimizacion}
    
    
    
    
    
% Pygments definitions
\makeatletter
\def\PY@reset{\let\PY@it=\relax \let\PY@bf=\relax%
    \let\PY@ul=\relax \let\PY@tc=\relax%
    \let\PY@bc=\relax \let\PY@ff=\relax}
\def\PY@tok#1{\csname PY@tok@#1\endcsname}
\def\PY@toks#1+{\ifx\relax#1\empty\else%
    \PY@tok{#1}\expandafter\PY@toks\fi}
\def\PY@do#1{\PY@bc{\PY@tc{\PY@ul{%
    \PY@it{\PY@bf{\PY@ff{#1}}}}}}}
\def\PY#1#2{\PY@reset\PY@toks#1+\relax+\PY@do{#2}}

\expandafter\def\csname PY@tok@w\endcsname{\def\PY@tc##1{\textcolor[rgb]{0.73,0.73,0.73}{##1}}}
\expandafter\def\csname PY@tok@c\endcsname{\let\PY@it=\textit\def\PY@tc##1{\textcolor[rgb]{0.25,0.50,0.50}{##1}}}
\expandafter\def\csname PY@tok@cp\endcsname{\def\PY@tc##1{\textcolor[rgb]{0.74,0.48,0.00}{##1}}}
\expandafter\def\csname PY@tok@k\endcsname{\let\PY@bf=\textbf\def\PY@tc##1{\textcolor[rgb]{0.00,0.50,0.00}{##1}}}
\expandafter\def\csname PY@tok@kp\endcsname{\def\PY@tc##1{\textcolor[rgb]{0.00,0.50,0.00}{##1}}}
\expandafter\def\csname PY@tok@kt\endcsname{\def\PY@tc##1{\textcolor[rgb]{0.69,0.00,0.25}{##1}}}
\expandafter\def\csname PY@tok@o\endcsname{\def\PY@tc##1{\textcolor[rgb]{0.40,0.40,0.40}{##1}}}
\expandafter\def\csname PY@tok@ow\endcsname{\let\PY@bf=\textbf\def\PY@tc##1{\textcolor[rgb]{0.67,0.13,1.00}{##1}}}
\expandafter\def\csname PY@tok@nb\endcsname{\def\PY@tc##1{\textcolor[rgb]{0.00,0.50,0.00}{##1}}}
\expandafter\def\csname PY@tok@nf\endcsname{\def\PY@tc##1{\textcolor[rgb]{0.00,0.00,1.00}{##1}}}
\expandafter\def\csname PY@tok@nc\endcsname{\let\PY@bf=\textbf\def\PY@tc##1{\textcolor[rgb]{0.00,0.00,1.00}{##1}}}
\expandafter\def\csname PY@tok@nn\endcsname{\let\PY@bf=\textbf\def\PY@tc##1{\textcolor[rgb]{0.00,0.00,1.00}{##1}}}
\expandafter\def\csname PY@tok@ne\endcsname{\let\PY@bf=\textbf\def\PY@tc##1{\textcolor[rgb]{0.82,0.25,0.23}{##1}}}
\expandafter\def\csname PY@tok@nv\endcsname{\def\PY@tc##1{\textcolor[rgb]{0.10,0.09,0.49}{##1}}}
\expandafter\def\csname PY@tok@no\endcsname{\def\PY@tc##1{\textcolor[rgb]{0.53,0.00,0.00}{##1}}}
\expandafter\def\csname PY@tok@nl\endcsname{\def\PY@tc##1{\textcolor[rgb]{0.63,0.63,0.00}{##1}}}
\expandafter\def\csname PY@tok@ni\endcsname{\let\PY@bf=\textbf\def\PY@tc##1{\textcolor[rgb]{0.60,0.60,0.60}{##1}}}
\expandafter\def\csname PY@tok@na\endcsname{\def\PY@tc##1{\textcolor[rgb]{0.49,0.56,0.16}{##1}}}
\expandafter\def\csname PY@tok@nt\endcsname{\let\PY@bf=\textbf\def\PY@tc##1{\textcolor[rgb]{0.00,0.50,0.00}{##1}}}
\expandafter\def\csname PY@tok@nd\endcsname{\def\PY@tc##1{\textcolor[rgb]{0.67,0.13,1.00}{##1}}}
\expandafter\def\csname PY@tok@s\endcsname{\def\PY@tc##1{\textcolor[rgb]{0.73,0.13,0.13}{##1}}}
\expandafter\def\csname PY@tok@sd\endcsname{\let\PY@it=\textit\def\PY@tc##1{\textcolor[rgb]{0.73,0.13,0.13}{##1}}}
\expandafter\def\csname PY@tok@si\endcsname{\let\PY@bf=\textbf\def\PY@tc##1{\textcolor[rgb]{0.73,0.40,0.53}{##1}}}
\expandafter\def\csname PY@tok@se\endcsname{\let\PY@bf=\textbf\def\PY@tc##1{\textcolor[rgb]{0.73,0.40,0.13}{##1}}}
\expandafter\def\csname PY@tok@sr\endcsname{\def\PY@tc##1{\textcolor[rgb]{0.73,0.40,0.53}{##1}}}
\expandafter\def\csname PY@tok@ss\endcsname{\def\PY@tc##1{\textcolor[rgb]{0.10,0.09,0.49}{##1}}}
\expandafter\def\csname PY@tok@sx\endcsname{\def\PY@tc##1{\textcolor[rgb]{0.00,0.50,0.00}{##1}}}
\expandafter\def\csname PY@tok@m\endcsname{\def\PY@tc##1{\textcolor[rgb]{0.40,0.40,0.40}{##1}}}
\expandafter\def\csname PY@tok@gh\endcsname{\let\PY@bf=\textbf\def\PY@tc##1{\textcolor[rgb]{0.00,0.00,0.50}{##1}}}
\expandafter\def\csname PY@tok@gu\endcsname{\let\PY@bf=\textbf\def\PY@tc##1{\textcolor[rgb]{0.50,0.00,0.50}{##1}}}
\expandafter\def\csname PY@tok@gd\endcsname{\def\PY@tc##1{\textcolor[rgb]{0.63,0.00,0.00}{##1}}}
\expandafter\def\csname PY@tok@gi\endcsname{\def\PY@tc##1{\textcolor[rgb]{0.00,0.63,0.00}{##1}}}
\expandafter\def\csname PY@tok@gr\endcsname{\def\PY@tc##1{\textcolor[rgb]{1.00,0.00,0.00}{##1}}}
\expandafter\def\csname PY@tok@ge\endcsname{\let\PY@it=\textit}
\expandafter\def\csname PY@tok@gs\endcsname{\let\PY@bf=\textbf}
\expandafter\def\csname PY@tok@gp\endcsname{\let\PY@bf=\textbf\def\PY@tc##1{\textcolor[rgb]{0.00,0.00,0.50}{##1}}}
\expandafter\def\csname PY@tok@go\endcsname{\def\PY@tc##1{\textcolor[rgb]{0.53,0.53,0.53}{##1}}}
\expandafter\def\csname PY@tok@gt\endcsname{\def\PY@tc##1{\textcolor[rgb]{0.00,0.27,0.87}{##1}}}
\expandafter\def\csname PY@tok@err\endcsname{\def\PY@bc##1{\setlength{\fboxsep}{0pt}\fcolorbox[rgb]{1.00,0.00,0.00}{1,1,1}{\strut ##1}}}
\expandafter\def\csname PY@tok@kc\endcsname{\let\PY@bf=\textbf\def\PY@tc##1{\textcolor[rgb]{0.00,0.50,0.00}{##1}}}
\expandafter\def\csname PY@tok@kd\endcsname{\let\PY@bf=\textbf\def\PY@tc##1{\textcolor[rgb]{0.00,0.50,0.00}{##1}}}
\expandafter\def\csname PY@tok@kn\endcsname{\let\PY@bf=\textbf\def\PY@tc##1{\textcolor[rgb]{0.00,0.50,0.00}{##1}}}
\expandafter\def\csname PY@tok@kr\endcsname{\let\PY@bf=\textbf\def\PY@tc##1{\textcolor[rgb]{0.00,0.50,0.00}{##1}}}
\expandafter\def\csname PY@tok@bp\endcsname{\def\PY@tc##1{\textcolor[rgb]{0.00,0.50,0.00}{##1}}}
\expandafter\def\csname PY@tok@fm\endcsname{\def\PY@tc##1{\textcolor[rgb]{0.00,0.00,1.00}{##1}}}
\expandafter\def\csname PY@tok@vc\endcsname{\def\PY@tc##1{\textcolor[rgb]{0.10,0.09,0.49}{##1}}}
\expandafter\def\csname PY@tok@vg\endcsname{\def\PY@tc##1{\textcolor[rgb]{0.10,0.09,0.49}{##1}}}
\expandafter\def\csname PY@tok@vi\endcsname{\def\PY@tc##1{\textcolor[rgb]{0.10,0.09,0.49}{##1}}}
\expandafter\def\csname PY@tok@vm\endcsname{\def\PY@tc##1{\textcolor[rgb]{0.10,0.09,0.49}{##1}}}
\expandafter\def\csname PY@tok@sa\endcsname{\def\PY@tc##1{\textcolor[rgb]{0.73,0.13,0.13}{##1}}}
\expandafter\def\csname PY@tok@sb\endcsname{\def\PY@tc##1{\textcolor[rgb]{0.73,0.13,0.13}{##1}}}
\expandafter\def\csname PY@tok@sc\endcsname{\def\PY@tc##1{\textcolor[rgb]{0.73,0.13,0.13}{##1}}}
\expandafter\def\csname PY@tok@dl\endcsname{\def\PY@tc##1{\textcolor[rgb]{0.73,0.13,0.13}{##1}}}
\expandafter\def\csname PY@tok@s2\endcsname{\def\PY@tc##1{\textcolor[rgb]{0.73,0.13,0.13}{##1}}}
\expandafter\def\csname PY@tok@sh\endcsname{\def\PY@tc##1{\textcolor[rgb]{0.73,0.13,0.13}{##1}}}
\expandafter\def\csname PY@tok@s1\endcsname{\def\PY@tc##1{\textcolor[rgb]{0.73,0.13,0.13}{##1}}}
\expandafter\def\csname PY@tok@mb\endcsname{\def\PY@tc##1{\textcolor[rgb]{0.40,0.40,0.40}{##1}}}
\expandafter\def\csname PY@tok@mf\endcsname{\def\PY@tc##1{\textcolor[rgb]{0.40,0.40,0.40}{##1}}}
\expandafter\def\csname PY@tok@mh\endcsname{\def\PY@tc##1{\textcolor[rgb]{0.40,0.40,0.40}{##1}}}
\expandafter\def\csname PY@tok@mi\endcsname{\def\PY@tc##1{\textcolor[rgb]{0.40,0.40,0.40}{##1}}}
\expandafter\def\csname PY@tok@il\endcsname{\def\PY@tc##1{\textcolor[rgb]{0.40,0.40,0.40}{##1}}}
\expandafter\def\csname PY@tok@mo\endcsname{\def\PY@tc##1{\textcolor[rgb]{0.40,0.40,0.40}{##1}}}
\expandafter\def\csname PY@tok@ch\endcsname{\let\PY@it=\textit\def\PY@tc##1{\textcolor[rgb]{0.25,0.50,0.50}{##1}}}
\expandafter\def\csname PY@tok@cm\endcsname{\let\PY@it=\textit\def\PY@tc##1{\textcolor[rgb]{0.25,0.50,0.50}{##1}}}
\expandafter\def\csname PY@tok@cpf\endcsname{\let\PY@it=\textit\def\PY@tc##1{\textcolor[rgb]{0.25,0.50,0.50}{##1}}}
\expandafter\def\csname PY@tok@c1\endcsname{\let\PY@it=\textit\def\PY@tc##1{\textcolor[rgb]{0.25,0.50,0.50}{##1}}}
\expandafter\def\csname PY@tok@cs\endcsname{\let\PY@it=\textit\def\PY@tc##1{\textcolor[rgb]{0.25,0.50,0.50}{##1}}}

\def\PYZbs{\char`\\}
\def\PYZus{\char`\_}
\def\PYZob{\char`\{}
\def\PYZcb{\char`\}}
\def\PYZca{\char`\^}
\def\PYZam{\char`\&}
\def\PYZlt{\char`\<}
\def\PYZgt{\char`\>}
\def\PYZsh{\char`\#}
\def\PYZpc{\char`\%}
\def\PYZdl{\char`\$}
\def\PYZhy{\char`\-}
\def\PYZsq{\char`\'}
\def\PYZdq{\char`\"}
\def\PYZti{\char`\~}
% for compatibility with earlier versions
\def\PYZat{@}
\def\PYZlb{[}
\def\PYZrb{]}
\makeatother


    % For linebreaks inside Verbatim environment from package fancyvrb. 
    \makeatletter
        \newbox\Wrappedcontinuationbox 
        \newbox\Wrappedvisiblespacebox 
        \newcommand*\Wrappedvisiblespace {\textcolor{red}{\textvisiblespace}} 
        \newcommand*\Wrappedcontinuationsymbol {\textcolor{red}{\llap{\tiny$\m@th\hookrightarrow$}}} 
        \newcommand*\Wrappedcontinuationindent {3ex } 
        \newcommand*\Wrappedafterbreak {\kern\Wrappedcontinuationindent\copy\Wrappedcontinuationbox} 
        % Take advantage of the already applied Pygments mark-up to insert 
        % potential linebreaks for TeX processing. 
        %        {, <, #, %, $, ' and ": go to next line. 
        %        _, }, ^, &, >, - and ~: stay at end of broken line. 
        % Use of \textquotesingle for straight quote. 
        \newcommand*\Wrappedbreaksatspecials {% 
            \def\PYGZus{\discretionary{\char`\_}{\Wrappedafterbreak}{\char`\_}}% 
            \def\PYGZob{\discretionary{}{\Wrappedafterbreak\char`\{}{\char`\{}}% 
            \def\PYGZcb{\discretionary{\char`\}}{\Wrappedafterbreak}{\char`\}}}% 
            \def\PYGZca{\discretionary{\char`\^}{\Wrappedafterbreak}{\char`\^}}% 
            \def\PYGZam{\discretionary{\char`\&}{\Wrappedafterbreak}{\char`\&}}% 
            \def\PYGZlt{\discretionary{}{\Wrappedafterbreak\char`\<}{\char`\<}}% 
            \def\PYGZgt{\discretionary{\char`\>}{\Wrappedafterbreak}{\char`\>}}% 
            \def\PYGZsh{\discretionary{}{\Wrappedafterbreak\char`\#}{\char`\#}}% 
            \def\PYGZpc{\discretionary{}{\Wrappedafterbreak\char`\%}{\char`\%}}% 
            \def\PYGZdl{\discretionary{}{\Wrappedafterbreak\char`\$}{\char`\$}}% 
            \def\PYGZhy{\discretionary{\char`\-}{\Wrappedafterbreak}{\char`\-}}% 
            \def\PYGZsq{\discretionary{}{\Wrappedafterbreak\textquotesingle}{\textquotesingle}}% 
            \def\PYGZdq{\discretionary{}{\Wrappedafterbreak\char`\"}{\char`\"}}% 
            \def\PYGZti{\discretionary{\char`\~}{\Wrappedafterbreak}{\char`\~}}% 
        } 
        % Some characters . , ; ? ! / are not pygmentized. 
        % This macro makes them "active" and they will insert potential linebreaks 
        \newcommand*\Wrappedbreaksatpunct {% 
            \lccode`\~`\.\lowercase{\def~}{\discretionary{\hbox{\char`\.}}{\Wrappedafterbreak}{\hbox{\char`\.}}}% 
            \lccode`\~`\,\lowercase{\def~}{\discretionary{\hbox{\char`\,}}{\Wrappedafterbreak}{\hbox{\char`\,}}}% 
            \lccode`\~`\;\lowercase{\def~}{\discretionary{\hbox{\char`\;}}{\Wrappedafterbreak}{\hbox{\char`\;}}}% 
            \lccode`\~`\:\lowercase{\def~}{\discretionary{\hbox{\char`\:}}{\Wrappedafterbreak}{\hbox{\char`\:}}}% 
            \lccode`\~`\?\lowercase{\def~}{\discretionary{\hbox{\char`\?}}{\Wrappedafterbreak}{\hbox{\char`\?}}}% 
            \lccode`\~`\!\lowercase{\def~}{\discretionary{\hbox{\char`\!}}{\Wrappedafterbreak}{\hbox{\char`\!}}}% 
            \lccode`\~`\/\lowercase{\def~}{\discretionary{\hbox{\char`\/}}{\Wrappedafterbreak}{\hbox{\char`\/}}}% 
            \catcode`\.\active
            \catcode`\,\active 
            \catcode`\;\active
            \catcode`\:\active
            \catcode`\?\active
            \catcode`\!\active
            \catcode`\/\active 
            \lccode`\~`\~ 	
        }
    \makeatother

    \let\OriginalVerbatim=\Verbatim
    \makeatletter
    \renewcommand{\Verbatim}[1][1]{%
        %\parskip\z@skip
        \sbox\Wrappedcontinuationbox {\Wrappedcontinuationsymbol}%
        \sbox\Wrappedvisiblespacebox {\FV@SetupFont\Wrappedvisiblespace}%
        \def\FancyVerbFormatLine ##1{\hsize\linewidth
            \vtop{\raggedright\hyphenpenalty\z@\exhyphenpenalty\z@
                \doublehyphendemerits\z@\finalhyphendemerits\z@
                \strut ##1\strut}%
        }%
        % If the linebreak is at a space, the latter will be displayed as visible
        % space at end of first line, and a continuation symbol starts next line.
        % Stretch/shrink are however usually zero for typewriter font.
        \def\FV@Space {%
            \nobreak\hskip\z@ plus\fontdimen3\font minus\fontdimen4\font
            \discretionary{\copy\Wrappedvisiblespacebox}{\Wrappedafterbreak}
            {\kern\fontdimen2\font}%
        }%
        
        % Allow breaks at special characters using \PYG... macros.
        \Wrappedbreaksatspecials
        % Breaks at punctuation characters . , ; ? ! and / need catcode=\active 	
        \OriginalVerbatim[#1,codes*=\Wrappedbreaksatpunct]%
    }
    \makeatother

    % Exact colors from NB
    \definecolor{incolor}{HTML}{303F9F}
    \definecolor{outcolor}{HTML}{D84315}
    \definecolor{cellborder}{HTML}{CFCFCF}
    \definecolor{cellbackground}{HTML}{F7F7F7}
    
    % prompt
    \makeatletter
    \newcommand{\boxspacing}{\kern\kvtcb@left@rule\kern\kvtcb@boxsep}
    \makeatother
    \newcommand{\prompt}[4]{
        \ttfamily\llap{{\color{#2}[#3]:\hspace{3pt}#4}}\vspace{-\baselineskip}
    }
    

    
    % Prevent overflowing lines due to hard-to-break entities
    \sloppy 
    % Setup hyperref package
    \hypersetup{
      breaklinks=true,  % so long urls are correctly broken across lines
      colorlinks=true,
      urlcolor=urlcolor,
      linkcolor=linkcolor,
      citecolor=citecolor,
      }
    % Slightly bigger margins than the latex defaults
    
    \geometry{verbose,tmargin=1in,bmargin=1in,lmargin=1in,rmargin=1in}
    
    

\begin{document}
    
    \title{Tarea 7 Optimización}
    \author{Roberto Vásquez Martínez \\ Profesor: Joaquín Peña Acevedo}
    \date{27/Marzo/2022}
    \maketitle   

    \hypertarget{ejercicio-1-5-puntos}{%
\section{Ejercicio 1 (5 puntos)}\label{ejercicio-1-5-puntos}}

Programar el método de Gauss-Newton para resolver el problema de mínimos
cuadrados no lineales

\[ \min_z  f(z) = \frac{1}{2} \sum_{j=1}^m r_j^2(z), \]

donde \(r_j: \mathbb{R}^n \rightarrow \mathbb{R}\) para \(j=1,...,m\).
Si definimos la función \(R: \mathbb{R}^n \rightarrow \mathbb{R}^m\)
como

\[ R(z) = \left( \begin{array}{c}
            r_{1}(z) \\
            \vdots \\
            r_{m}(z) \\
        \end{array} \right),
\]

entonces

\[ \min_z  f(z) = \frac{1}{2} R(z)^\top R(z). \]

\begin{center}\rule{0.5\linewidth}{0.5pt}\end{center}

Dar la función de residuales \(R(z)\), la función Jacobiana \(J(z)\), un
punto inicial \(z_0\), un número máximo de iteraciones \(N\), y una
tolerancia \(\tau>0\).

\begin{enumerate}
\def\labelenumi{\arabic{enumi}.}
\item
  Hacer \(res=0\).
\item
  Para \(k = 0, 1, ..., N\):
\end{enumerate}

\begin{itemize}
\tightlist
\item
  Calcular \(R_k = R(z_k)\)
\item
  Calcular \(J_{k} = J(z_k)\)
\item
  Calcular la dirección de descenso \(p_k\) resolviendo el sistema
\end{itemize}

\[  J_{k}^\top J_{k} p_k = -J_{k}^\top R_{k} \]

\begin{itemize}
\tightlist
\item
  Si \(\|p_k\|<\tau\), hacer \(res=1\) y terminar el ciclo
\item
  Hacer \(z_{k+1} = z_k + p_k\).
\end{itemize}

\begin{enumerate}
\def\labelenumi{\arabic{enumi}.}
\setcounter{enumi}{2}
\tightlist
\item
  Devolver \(z_k, R_k, k, \|p_k\|\) y \(res\).
\end{enumerate}

\begin{center}\rule{0.5\linewidth}{0.5pt}\end{center}

\begin{enumerate}
\def\labelenumi{\arabic{enumi}.}
\item
  Escriba una función que implementa el algoritmo anterior usando
  arreglos de Numpy.
\item
  Leer el archivo \textbf{puntos2D\_1.npy} que contiene una matriz con
  dos columnas. La primer columna tiene los valores
  \(x_1, x_2, ..., x_m\) y en la segunda columna los valores
  \(y_1, y_2, ..., y_m\), de modo que cada par \((x_i, y_i)\) es un
  dato. Queremos ajustar al conjunto de puntos \((x_i, y_i)\) el modelo

  \[  A \sin(w x + \phi)\]

  por lo que la función \(R(\mathbf{z})=R(A, w, \phi)\) está formada por
  los residuales

  \[ r_i(z) = r_i(A, w, \phi) = A \sin(w x_i + \phi) - y_i \]

  para \(i=1,2,...,m\).

  Programe la función \(R(\mathbf{z})\) con
  \(\mathbf{z} = (A, w, \phi)\) y su Jacobiana \(J(\mathbf{z})\).

  \textbf{Nota:} Puede programar estas funciones de la forma
  \texttt{funcion(z,\ paramf)}, donde \texttt{paramf} corresponda a la
  matriz que tiene los puntos \((x_i,y_i)\). También puede pasar el
  arreglo \texttt{paramf} como arumento del algoritmo para que pueda
  evaluar las funciones.
\item
  Use el algoritmo con estas funciones \(R(\mathbf{z})\) y
  \(J(\mathbf{z})\), el punto inicial \(\mathbf{z}_0 = (15, 0.6, 0)\)
  (esto es \(A_0=15\), \(w_0=0.6\) y \(\phi_0=0\)), un número máximo de
  iteraciones \(N=5000\) y una tolerancia \(\tau =\sqrt{\epsilon_m}\)
  donde \(\epsilon_m\) es el épsilon máquina.
\end{enumerate}

\begin{itemize}
\tightlist
\item
  Imprima el valor inicial
  \(f(\mathbf{z}_0) = \frac{1}{2} R(\mathbf{z}_0)^\top R(\mathbf{z}_0)\).
\item
  Ejecute el algoritmo e imprima un mensaje que indique si el algoritmo
  converge dependiendo de la variable \(res\).
\item
  Imprima \(\mathbf{z}_k\),
  \(f(\mathbf{z}_k) = \frac{1}{2} R(\mathbf{z}_k)^\top R(\mathbf{z}_k)\),
  la norma \(\|p_k\|\), y el número de iteraciones \(k\) realizadas.
\end{itemize}

\begin{enumerate}
\def\labelenumi{\arabic{enumi}.}
\setcounter{enumi}{3}
\tightlist
\item
  Genere una gráfica que muestre a los puntos \((x_i, y_i)\) y la
  gráfica del modelo \(z_{k0} \sin(z_{k1} x + z_{k2})\), evaluando esta
  función en el intervalo
\end{enumerate}

\[x \in [\min x_i, \max x_i]\]

\begin{enumerate}
\def\labelenumi{\arabic{enumi}.}
\setcounter{enumi}{4}
\tightlist
\item
  De la gráfica de los datos, e interpretando el parámetro \(A\) como la
  amplitud de la onda, se ve que \(A_0=15\) es una buena inicialización
  para este paramétro. Para los otros parámetros también se debería usar
  su interpretación para dar buenos valores iniciales. Repita las
  pruebas con los puntos iniciales \(\mathbf{z}_0 = (15, 1, 0)\) y
  \(\mathbf{z}_0 = (15, 0.6, 1.6)\).
\end{enumerate}

    \hypertarget{soluciuxf3n}{%
\subsection{Solución}\label{soluciuxf3n}}

Importaremos el módulo \texttt{lib\_t7} que está en el mismo directorio
que este notebook. En este módulo está la función
\texttt{gauss\_newton\_nlls} que implementa el método de Gauss-Newton en
el contexto de mínimos cuadrados no lineales.

En la función \texttt{proof\_gauss\_newton\_nlls} se encuentra el ajuste
resultante de aplicar el algoritmo de Gauss-Newton al conjunto de puntos
\((x_i,y_i)\) en \texttt{puntos2D\_1.npy} suponiendo el modelo
\[ \Psi(\mathbf{z})=\Psi(A,\omega,\phi)=A\sin(\omega x+\phi)\]

A continuación, haremos el ajuste anterior para 3 condiciones iniciales
diferentes y considerando un número máximo de iteraciones \(N=5000\) y
una tolerancia \(\tau=\sqrt{\epsilon_m}\)

    \hypertarget{condiciuxf3n-inicial-mathbfz_0150.60}{%
\subsubsection{\texorpdfstring{Condición inicial
\(\mathbf{z}_0=(15,0.6,0)\)}{Condición inicial \textbackslash mathbf\{z\}\_0=(15,0.6,0)}}\label{condiciuxf3n-inicial-mathbfz_0150.60}}

El resultado para esta condición inicial es el siguiente

    \begin{tcolorbox}[breakable, size=fbox, boxrule=1pt, pad at break*=1mm,colback=cellbackground, colframe=cellborder]
\prompt{In}{incolor}{1}{\boxspacing}
\begin{Verbatim}[commandchars=\\\{\}]
\PY{k+kn}{import} \PY{n+nn}{numpy} \PY{k}{as} \PY{n+nn}{np}
\PY{k+kn}{import} \PY{n+nn}{importlib}
\PY{k+kn}{import} \PY{n+nn}{lib\PYZus{}t7}
\PY{n}{importlib}\PY{o}{.}\PY{n}{reload}\PY{p}{(}\PY{n}{lib\PYZus{}t7}\PY{p}{)}
\PY{k+kn}{from} \PY{n+nn}{lib\PYZus{}t7} \PY{k+kn}{import} \PY{o}{*}

\PY{c+c1}{\PYZsh{} Iteraciones maximas y tolerancia}
\PY{n}{N}\PY{o}{=}\PY{l+m+mi}{5000}
\PY{n}{tol}\PY{o}{=}\PY{n}{np}\PY{o}{.}\PY{n}{finfo}\PY{p}{(}\PY{n+nb}{float}\PY{p}{)}\PY{o}{.}\PY{n}{eps}\PY{o}{*}\PY{o}{*}\PY{p}{(}\PY{l+m+mi}{1}\PY{o}{/}\PY{l+m+mi}{2}\PY{p}{)}

\PY{c+c1}{\PYZsh{} Muestra}
\PY{n}{sample}\PY{o}{=}\PY{n}{np}\PY{o}{.}\PY{n}{load}\PY{p}{(}\PY{l+s+s1}{\PYZsq{}}\PY{l+s+s1}{puntos2D\PYZus{}1.npy}\PY{l+s+s1}{\PYZsq{}}\PY{p}{)}

\PY{c+c1}{\PYZsh{} Condiciones iniciales}
\PY{n}{z0}\PY{o}{=}\PY{p}{[}\PY{n}{np}\PY{o}{.}\PY{n}{array}\PY{p}{(}\PY{p}{[}\PY{l+m+mf}{15.0}\PY{p}{,}\PY{l+m+mf}{0.6}\PY{p}{,}\PY{l+m+mf}{0.0}\PY{p}{]}\PY{p}{)}\PY{p}{,}
\PY{n}{np}\PY{o}{.}\PY{n}{array}\PY{p}{(}\PY{p}{[}\PY{l+m+mf}{15.0}\PY{p}{,}\PY{l+m+mf}{1.0}\PY{p}{,}\PY{l+m+mf}{0.0}\PY{p}{]}\PY{p}{)}\PY{p}{,}
\PY{n}{np}\PY{o}{.}\PY{n}{array}\PY{p}{(}\PY{p}{[}\PY{l+m+mf}{15.0}\PY{p}{,}\PY{l+m+mf}{0.6}\PY{p}{,}\PY{l+m+mf}{1.6}\PY{p}{]}\PY{p}{)}\PY{p}{]}

\PY{c+c1}{\PYZsh{} Resultados del Algoritmo Gauss}
\PY{n}{proof\PYZus{}gauss\PYZus{}newton\PYZus{}nlls}\PY{p}{(}\PY{n}{R}\PY{p}{,}\PY{n}{J}\PY{p}{,}\PY{n}{z0}\PY{p}{[}\PY{l+m+mi}{0}\PY{p}{]}\PY{p}{,}\PY{n}{N}\PY{p}{,}\PY{n}{tol}\PY{p}{,}\PY{n}{sample}\PY{p}{)}
\end{Verbatim}
\end{tcolorbox}

    \begin{Verbatim}[commandchars=\\\{\}]
El algoritmo de Gauss-Newton CONVERGE
z0 =  [15.   0.6  0. ]
f(z0) =  45454.05280978729
zk =  [12.99606648  1.19935917 -5.67317097]
f(zk) =  457.1693612130722
|pk| =  1.4545189678275178e-08
k =  8
    \end{Verbatim}

    \begin{center}
    \adjustimage{max size={0.9\linewidth}{0.9\paperheight}}{Tarea_7_Optimizacion_files/Tarea_7_Optimizacion_3_1.png}
    \end{center}
    { \hspace*{\fill} \\}
    
    Pusimos en el título del plot el valor de la función objetivo en el
punto al que converge el algoritmo.

    \hypertarget{condiciuxf3n-inicial-mathbfz_01510}{%
\subsubsection{\texorpdfstring{Condición inicial
\(\mathbf{z}_0=(15,1,0)\)}{Condición inicial \textbackslash mathbf\{z\}\_0=(15,1,0)}}\label{condiciuxf3n-inicial-mathbfz_01510}}

Ahora variando la frecuencia del modelo \(\Psi\) en al condición inicial
obtenemos el siguiente resultado.

    \begin{tcolorbox}[breakable, size=fbox, boxrule=1pt, pad at break*=1mm,colback=cellbackground, colframe=cellborder]
\prompt{In}{incolor}{2}{\boxspacing}
\begin{Verbatim}[commandchars=\\\{\}]
\PY{c+c1}{\PYZsh{} Resultados del Algoritmo Gauss}
\PY{n}{proof\PYZus{}gauss\PYZus{}newton\PYZus{}nlls}\PY{p}{(}\PY{n}{R}\PY{p}{,}\PY{n}{J}\PY{p}{,}\PY{n}{z0}\PY{p}{[}\PY{l+m+mi}{1}\PY{p}{]}\PY{p}{,}\PY{n}{N}\PY{p}{,}\PY{n}{tol}\PY{p}{,}\PY{n}{sample}\PY{p}{)}
\end{Verbatim}
\end{tcolorbox}

    \begin{Verbatim}[commandchars=\\\{\}]
El algoritmo de Gauss-Newton CONVERGE
z0 =  [15.  1.  0.]
f(z0) =  40807.16289819636
zk =  [-1.11472532e-01  9.85781641e+01 -3.32725905e+02]
f(zk) =  18654.618220305696
|pk| =  1.348024075214489e-08
k =  66
    \end{Verbatim}

    \begin{center}
    \adjustimage{max size={0.9\linewidth}{0.9\paperheight}}{Tarea_7_Optimizacion_files/Tarea_7_Optimizacion_6_1.png}
    \end{center}
    { \hspace*{\fill} \\}
    
    \hypertarget{condiciuxf3n-inicial-mathbfz_0150.61.6}{%
\subsubsection{\texorpdfstring{Condición inicial
\(\mathbf{z}_0=(15,0.6,1.6)\)}{Condición inicial \textbackslash mathbf\{z\}\_0=(15,0.6,1.6)}}\label{condiciuxf3n-inicial-mathbfz_0150.61.6}}

Finalmente, para esta condición inicial el resultado es

    \begin{tcolorbox}[breakable, size=fbox, boxrule=1pt, pad at break*=1mm,colback=cellbackground, colframe=cellborder]
\prompt{In}{incolor}{3}{\boxspacing}
\begin{Verbatim}[commandchars=\\\{\}]
\PY{c+c1}{\PYZsh{} Resultados del Algoritmo Gauss}
\PY{n}{proof\PYZus{}gauss\PYZus{}newton\PYZus{}nlls}\PY{p}{(}\PY{n}{R}\PY{p}{,}\PY{n}{J}\PY{p}{,}\PY{n}{z0}\PY{p}{[}\PY{l+m+mi}{2}\PY{p}{]}\PY{p}{,}\PY{n}{N}\PY{p}{,}\PY{n}{tol}\PY{p}{,}\PY{n}{sample}\PY{p}{)}
\end{Verbatim}
\end{tcolorbox}

    \begin{Verbatim}[commandchars=\\\{\}]
El algoritmo de Gauss-Newton CONVERGE
z0 =  [15.   0.6  1.6]
f(z0) =  37048.62007346928
zk =  [  -0.25186599   59.17703745 -214.18352473]
f(zk) =  18649.518793671767
|pk| =  1.3802970551977972e-08
k =  68
    \end{Verbatim}

    \begin{center}
    \adjustimage{max size={0.9\linewidth}{0.9\paperheight}}{Tarea_7_Optimizacion_files/Tarea_7_Optimizacion_8_1.png}
    \end{center}
    { \hspace*{\fill} \\}
    
    Observamos que el número de condición de
\(J(\mathbf{z}_0)^TJ(\mathbf{z}_0)\) en cada una de las condiciones
iniciales es

    \begin{tcolorbox}[breakable, size=fbox, boxrule=1pt, pad at break*=1mm,colback=cellbackground, colframe=cellborder]
\prompt{In}{incolor}{4}{\boxspacing}
\begin{Verbatim}[commandchars=\\\{\}]
\PY{k}{for} \PY{n}{k}\PY{p}{,}\PY{n}{p0} \PY{o+ow}{in} \PY{n+nb}{enumerate}\PY{p}{(}\PY{n}{z0}\PY{p}{)}\PY{p}{:}
    \PY{n+nb}{print}\PY{p}{(}\PY{l+s+sa}{f}\PY{l+s+s1}{\PYZsq{}}\PY{l+s+s1}{El num de condicion de J(z0).TJ(z0) en la condicion inicial }\PY{l+s+si}{\PYZob{}}\PY{n}{k}\PY{o}{+}\PY{l+m+mi}{1}\PY{l+s+si}{\PYZcb{}}\PY{l+s+s1}{ es: }\PY{l+s+s1}{\PYZsq{}}\PY{p}{)}
    \PY{n+nb}{print}\PY{p}{(}\PY{n}{np}\PY{o}{.}\PY{n}{linalg}\PY{o}{.}\PY{n}{cond}\PY{p}{(}\PY{n}{J}\PY{p}{(}\PY{n}{p0}\PY{p}{,}\PY{n}{sample}\PY{p}{)}\PY{o}{.}\PY{n}{T}\PY{n+nd}{@J}\PY{p}{(}\PY{n}{p0}\PY{p}{,}\PY{n}{sample}\PY{p}{)}\PY{p}{)}\PY{p}{)}
\end{Verbatim}
\end{tcolorbox}

    \begin{Verbatim}[commandchars=\\\{\}]
El num de condicion de J(z0).TJ(z0) en la condicion inicial 1 es:
4319.205562634667
El num de condicion de J(z0).TJ(z0) en la condicion inicial 2 es:
5528.751104008563
El num de condicion de J(z0).TJ(z0) en la condicion inicial 3 es:
9222.650564875052
    \end{Verbatim}

    No hay diferencias tan grandes entre las tres condiciones iniciales, sin
embargo la única en la que se obtiene un buen ajuste es en la primera.

Por otro lado, el número de condición para cada condición inicial es
considerable lo que puede explicar que en el caso de las condiciones 2 y
3 no se llegue al mismo resultado que en la condición inicial 1.

Además una de las desventajas del método de Gauss-Newton es que no se
tiene convergencia al óptimo global, que fue lo que sucedió en las
condiciones iniciales 2 y 3, hubo convergencia pero a un óptimo local
probablemente.

    \hypertarget{soluciuxf3n}{%
\subsection{Solución}\label{soluciuxf3n}}

Al igual que en el ejercicio anterior, las funciones que realizan lo que
se pide están en el módulo \texttt{lib\_t7}.

La función que implementa el algoritmo de Levenberg-Marquardt con
parámetro de regularización \(\mu\) que se actualiza iterativamente
usando la razón de ganancia \(\rho\) es
\texttt{levenberg\_marquardt\_nlls}. Por otro lado, la función que
ajusta el modelo \(\Psi\) al conjunto de puntos en
\texttt{puntos2D\_1.npy} usando el algoritmo de Levenberg-Marquardt es
\texttt{proof\_levenberg\_marquardt\_nlls}.

A continuación mostramos el desempeño de este algoritmo con las 3
condiciones iniciales solicitadas.

    \hypertarget{condiciuxf3n-inicial-mathbfz_0150.60}{%
\subsubsection{\texorpdfstring{Condición inicial
\(\mathbf{z}_0=(15,0.6,0)\)}{Condición inicial \textbackslash mathbf\{z\}\_0=(15,0.6,0)}}\label{condiciuxf3n-inicial-mathbfz_0150.60}}

El resultado con esta condición inicial es

    \begin{tcolorbox}[breakable, size=fbox, boxrule=1pt, pad at break*=1mm,colback=cellbackground, colframe=cellborder]
\prompt{In}{incolor}{5}{\boxspacing}
\begin{Verbatim}[commandchars=\\\{\}]
\PY{n}{importlib}\PY{o}{.}\PY{n}{reload}\PY{p}{(}\PY{n}{lib\PYZus{}t7}\PY{p}{)}
\PY{k+kn}{from} \PY{n+nn}{lib\PYZus{}t7} \PY{k+kn}{import} \PY{o}{*}

\PY{c+c1}{\PYZsh{} Parametro de regularizacion}
\PY{n}{mu\PYZus{}ref}\PY{o}{=}\PY{l+m+mf}{0.001}

\PY{n}{proof\PYZus{}levenberg\PYZus{}marquardt\PYZus{}nlls}\PY{p}{(}\PY{n}{R}\PY{p}{,}\PY{n}{J}\PY{p}{,}\PY{n}{z0}\PY{p}{[}\PY{l+m+mi}{0}\PY{p}{]}\PY{p}{,}\PY{n}{N}\PY{p}{,}\PY{n}{tol}\PY{p}{,}\PY{n}{mu\PYZus{}ref}\PY{p}{,}\PY{n}{sample}\PY{p}{)}
\end{Verbatim}
\end{tcolorbox}

    \begin{Verbatim}[commandchars=\\\{\}]
El algoritmo de Levenberg-Marquardt CONVERGE
z0 =  [15.   0.6  0. ]
f(z0) =  45454.05280978729
zk =  [12.99606648  1.19935917 -5.67317097]
f(zk) =  457.16936121307214
|pk| =  1.4549146764407206e-08
k =  8
    \end{Verbatim}

    \begin{center}
    \adjustimage{max size={0.9\linewidth}{0.9\paperheight}}{Tarea_7_Optimizacion_files/Tarea_7_Optimizacion_14_1.png}
    \end{center}
    { \hspace*{\fill} \\}
    
    El resultado es similar al obtenido por Gauss-Newton incluso en el valor
de la función objetivo en el último punto de la trayectoria.

    \hypertarget{condiciuxf3n-inicial-mathbfz_01510}{%
\subsubsection{\texorpdfstring{Condición inicial
\(\mathbf{z}_0=(15,1,0)\)}{Condición inicial \textbackslash mathbf\{z\}\_0=(15,1,0)}}\label{condiciuxf3n-inicial-mathbfz_01510}}

Mostramos el resultado con la segunda condición inicial

    \begin{tcolorbox}[breakable, size=fbox, boxrule=1pt, pad at break*=1mm,colback=cellbackground, colframe=cellborder]
\prompt{In}{incolor}{6}{\boxspacing}
\begin{Verbatim}[commandchars=\\\{\}]
\PY{n}{proof\PYZus{}levenberg\PYZus{}marquardt\PYZus{}nlls}\PY{p}{(}\PY{n}{R}\PY{p}{,}\PY{n}{J}\PY{p}{,}\PY{n}{z0}\PY{p}{[}\PY{l+m+mi}{1}\PY{p}{]}\PY{p}{,}\PY{n}{N}\PY{p}{,}\PY{n}{tol}\PY{p}{,}\PY{n}{mu\PYZus{}ref}\PY{p}{,}\PY{n}{sample}\PY{p}{)}
\end{Verbatim}
\end{tcolorbox}

    \begin{Verbatim}[commandchars=\\\{\}]
El algoritmo de Levenberg-Marquardt CONVERGE
z0 =  [15.  1.  0.]
f(z0) =  40807.16289819636
zk =  [  0.24437787 -17.31400178  26.79331113]
f(zk) =  18649.792255409797
|pk| =  1.2596492950063504e-08
k =  37
    \end{Verbatim}

    \begin{center}
    \adjustimage{max size={0.9\linewidth}{0.9\paperheight}}{Tarea_7_Optimizacion_files/Tarea_7_Optimizacion_17_1.png}
    \end{center}
    { \hspace*{\fill} \\}
    
    \hypertarget{condiciuxf3n-inicial-mathbfz_0150.61.6}{%
\subsubsection{\texorpdfstring{Condición inicial
\(\mathbf{z}_0=(15,0.6,1.6)\)}{Condición inicial \textbackslash mathbf\{z\}\_0=(15,0.6,1.6)}}\label{condiciuxf3n-inicial-mathbfz_0150.61.6}}

Finalmente, el resultado con la condición inicial 3 es

    \begin{tcolorbox}[breakable, size=fbox, boxrule=1pt, pad at break*=1mm,colback=cellbackground, colframe=cellborder]
\prompt{In}{incolor}{7}{\boxspacing}
\begin{Verbatim}[commandchars=\\\{\}]
\PY{n}{proof\PYZus{}levenberg\PYZus{}marquardt\PYZus{}nlls}\PY{p}{(}\PY{n}{R}\PY{p}{,}\PY{n}{J}\PY{p}{,}\PY{n}{z0}\PY{p}{[}\PY{l+m+mi}{2}\PY{p}{]}\PY{p}{,}\PY{n}{N}\PY{p}{,}\PY{n}{tol}\PY{p}{,}\PY{n}{mu\PYZus{}ref}\PY{p}{,}\PY{n}{sample}\PY{p}{)}
\end{Verbatim}
\end{tcolorbox}

    \begin{Verbatim}[commandchars=\\\{\}]
El algoritmo de Levenberg-Marquardt CONVERGE
z0 =  [15.   0.6  1.6]
f(z0) =  37048.62007346928
zk =  [  0.25620933   4.08962698 -24.29530542]
f(zk) =  18620.150604673836
|pk| =  7.461281545956077e-09
k =  49
    \end{Verbatim}

    \begin{center}
    \adjustimage{max size={0.9\linewidth}{0.9\paperheight}}{Tarea_7_Optimizacion_files/Tarea_7_Optimizacion_19_1.png}
    \end{center}
    { \hspace*{\fill} \\}
    
    Aunque no se obtuvo una solución con el ajuste de la condición inicial 1
usando las condiciones iniciales 2 y 3, el algoritmo de
Levenberg-Marquardt resultó más eficiente en el número de iteraciones,
al menos en la tercer condición inicial.

Lo que se observa de estas soluciones es que las obtenidas por
Levenberg-Marquardt oscilan menos que las obtenidas por Gauss-Newton y
esto es el efecto de la regularización, ya que nos permite controlar la
norma de la solución a la que llegamos, en consecuencia, los valores de
la amplitud, la frecuencia y la fase son más pequeños en valor absoluto
que los encontrados con Gauss-Newton.


    % Add a bibliography block to the postdoc
    
    
    
\end{document}
