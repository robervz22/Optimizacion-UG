\documentclass[11pt]{article}

    \usepackage[breakable]{tcolorbox}
    \usepackage{parskip} % Stop auto-indenting (to mimic markdown behaviour)
    
    \usepackage{iftex}
    \ifPDFTeX
    	\usepackage[T1]{fontenc}
    	\usepackage{mathpazo}
    \else
    	\usepackage{fontspec}
    \fi

    % Basic figure setup, for now with no caption control since it's done
    % automatically by Pandoc (which extracts ![](path) syntax from Markdown).
    \usepackage{graphicx}
    % Maintain compatibility with old templates. Remove in nbconvert 6.0
    \let\Oldincludegraphics\includegraphics
    % Ensure that by default, figures have no caption (until we provide a
    % proper Figure object with a Caption API and a way to capture that
    % in the conversion process - todo).
    \usepackage{caption}
    \DeclareCaptionFormat{nocaption}{}
    \captionsetup{format=nocaption,aboveskip=0pt,belowskip=0pt}

    \usepackage{float}
    \floatplacement{figure}{H} % forces figures to be placed at the correct location
    \usepackage{xcolor} % Allow colors to be defined
    \usepackage{enumerate} % Needed for markdown enumerations to work
    \usepackage{geometry} % Used to adjust the document margins
    \usepackage{amsmath} % Equations
    \usepackage{amssymb} % Equations
    \usepackage{textcomp} % defines textquotesingle
    % Hack from http://tex.stackexchange.com/a/47451/13684:
    \AtBeginDocument{%
        \def\PYZsq{\textquotesingle}% Upright quotes in Pygmentized code
    }
    \usepackage{upquote} % Upright quotes for verbatim code
    \usepackage{eurosym} % defines \euro
    \usepackage[mathletters]{ucs} % Extended unicode (utf-8) support
    \usepackage{fancyvrb} % verbatim replacement that allows latex
    \usepackage{grffile} % extends the file name processing of package graphics 
                         % to support a larger range
    \makeatletter % fix for old versions of grffile with XeLaTeX
    \@ifpackagelater{grffile}{2019/11/01}
    {
      % Do nothing on new versions
    }
    {
      \def\Gread@@xetex#1{%
        \IfFileExists{"\Gin@base".bb}%
        {\Gread@eps{\Gin@base.bb}}%
        {\Gread@@xetex@aux#1}%
      }
    }
    \makeatother
    \usepackage[Export]{adjustbox} % Used to constrain images to a maximum size
    \adjustboxset{max size={0.9\linewidth}{0.9\paperheight}}

    % The hyperref package gives us a pdf with properly built
    % internal navigation ('pdf bookmarks' for the table of contents,
    % internal cross-reference links, web links for URLs, etc.)
    \usepackage{hyperref}
    % The default LaTeX title has an obnoxious amount of whitespace. By default,
    % titling removes some of it. It also provides customization options.
    \usepackage{titling}
    \usepackage{longtable} % longtable support required by pandoc >1.10
    \usepackage{booktabs}  % table support for pandoc > 1.12.2
    \usepackage[inline]{enumitem} % IRkernel/repr support (it uses the enumerate* environment)
    \usepackage[normalem]{ulem} % ulem is needed to support strikethroughs (\sout)
                                % normalem makes italics be italics, not underlines
    \usepackage{mathrsfs}
    

    
    % Colors for the hyperref package
    \definecolor{urlcolor}{rgb}{0,.145,.698}
    \definecolor{linkcolor}{rgb}{.71,0.21,0.01}
    \definecolor{citecolor}{rgb}{.12,.54,.11}

    % ANSI colors
    \definecolor{ansi-black}{HTML}{3E424D}
    \definecolor{ansi-black-intense}{HTML}{282C36}
    \definecolor{ansi-red}{HTML}{E75C58}
    \definecolor{ansi-red-intense}{HTML}{B22B31}
    \definecolor{ansi-green}{HTML}{00A250}
    \definecolor{ansi-green-intense}{HTML}{007427}
    \definecolor{ansi-yellow}{HTML}{DDB62B}
    \definecolor{ansi-yellow-intense}{HTML}{B27D12}
    \definecolor{ansi-blue}{HTML}{208FFB}
    \definecolor{ansi-blue-intense}{HTML}{0065CA}
    \definecolor{ansi-magenta}{HTML}{D160C4}
    \definecolor{ansi-magenta-intense}{HTML}{A03196}
    \definecolor{ansi-cyan}{HTML}{60C6C8}
    \definecolor{ansi-cyan-intense}{HTML}{258F8F}
    \definecolor{ansi-white}{HTML}{C5C1B4}
    \definecolor{ansi-white-intense}{HTML}{A1A6B2}
    \definecolor{ansi-default-inverse-fg}{HTML}{FFFFFF}
    \definecolor{ansi-default-inverse-bg}{HTML}{000000}

    % common color for the border for error outputs.
    \definecolor{outerrorbackground}{HTML}{FFDFDF}

    % commands and environments needed by pandoc snippets
    % extracted from the output of `pandoc -s`
    \providecommand{\tightlist}{%
      \setlength{\itemsep}{0pt}\setlength{\parskip}{0pt}}
    \DefineVerbatimEnvironment{Highlighting}{Verbatim}{commandchars=\\\{\}}
    % Add ',fontsize=\small' for more characters per line
    \newenvironment{Shaded}{}{}
    \newcommand{\KeywordTok}[1]{\textcolor[rgb]{0.00,0.44,0.13}{\textbf{{#1}}}}
    \newcommand{\DataTypeTok}[1]{\textcolor[rgb]{0.56,0.13,0.00}{{#1}}}
    \newcommand{\DecValTok}[1]{\textcolor[rgb]{0.25,0.63,0.44}{{#1}}}
    \newcommand{\BaseNTok}[1]{\textcolor[rgb]{0.25,0.63,0.44}{{#1}}}
    \newcommand{\FloatTok}[1]{\textcolor[rgb]{0.25,0.63,0.44}{{#1}}}
    \newcommand{\CharTok}[1]{\textcolor[rgb]{0.25,0.44,0.63}{{#1}}}
    \newcommand{\StringTok}[1]{\textcolor[rgb]{0.25,0.44,0.63}{{#1}}}
    \newcommand{\CommentTok}[1]{\textcolor[rgb]{0.38,0.63,0.69}{\textit{{#1}}}}
    \newcommand{\OtherTok}[1]{\textcolor[rgb]{0.00,0.44,0.13}{{#1}}}
    \newcommand{\AlertTok}[1]{\textcolor[rgb]{1.00,0.00,0.00}{\textbf{{#1}}}}
    \newcommand{\FunctionTok}[1]{\textcolor[rgb]{0.02,0.16,0.49}{{#1}}}
    \newcommand{\RegionMarkerTok}[1]{{#1}}
    \newcommand{\ErrorTok}[1]{\textcolor[rgb]{1.00,0.00,0.00}{\textbf{{#1}}}}
    \newcommand{\NormalTok}[1]{{#1}}
    
    % Additional commands for more recent versions of Pandoc
    \newcommand{\ConstantTok}[1]{\textcolor[rgb]{0.53,0.00,0.00}{{#1}}}
    \newcommand{\SpecialCharTok}[1]{\textcolor[rgb]{0.25,0.44,0.63}{{#1}}}
    \newcommand{\VerbatimStringTok}[1]{\textcolor[rgb]{0.25,0.44,0.63}{{#1}}}
    \newcommand{\SpecialStringTok}[1]{\textcolor[rgb]{0.73,0.40,0.53}{{#1}}}
    \newcommand{\ImportTok}[1]{{#1}}
    \newcommand{\DocumentationTok}[1]{\textcolor[rgb]{0.73,0.13,0.13}{\textit{{#1}}}}
    \newcommand{\AnnotationTok}[1]{\textcolor[rgb]{0.38,0.63,0.69}{\textbf{\textit{{#1}}}}}
    \newcommand{\CommentVarTok}[1]{\textcolor[rgb]{0.38,0.63,0.69}{\textbf{\textit{{#1}}}}}
    \newcommand{\VariableTok}[1]{\textcolor[rgb]{0.10,0.09,0.49}{{#1}}}
    \newcommand{\ControlFlowTok}[1]{\textcolor[rgb]{0.00,0.44,0.13}{\textbf{{#1}}}}
    \newcommand{\OperatorTok}[1]{\textcolor[rgb]{0.40,0.40,0.40}{{#1}}}
    \newcommand{\BuiltInTok}[1]{{#1}}
    \newcommand{\ExtensionTok}[1]{{#1}}
    \newcommand{\PreprocessorTok}[1]{\textcolor[rgb]{0.74,0.48,0.00}{{#1}}}
    \newcommand{\AttributeTok}[1]{\textcolor[rgb]{0.49,0.56,0.16}{{#1}}}
    \newcommand{\InformationTok}[1]{\textcolor[rgb]{0.38,0.63,0.69}{\textbf{\textit{{#1}}}}}
    \newcommand{\WarningTok}[1]{\textcolor[rgb]{0.38,0.63,0.69}{\textbf{\textit{{#1}}}}}
    
    
    % Define a nice break command that doesn't care if a line doesn't already
    % exist.
    \def\br{\hspace*{\fill} \\* }
    % Math Jax compatibility definitions
    \def\gt{>}
    \def\lt{<}
    \let\Oldtex\TeX
    \let\Oldlatex\LaTeX
    \renewcommand{\TeX}{\textrm{\Oldtex}}
    \renewcommand{\LaTeX}{\textrm{\Oldlatex}}
    % Document parameters
    % Document title
    \title{Tarea\_2\_Optimizacion}
    
    
    
    
    
% Pygments definitions
\makeatletter
\def\PY@reset{\let\PY@it=\relax \let\PY@bf=\relax%
    \let\PY@ul=\relax \let\PY@tc=\relax%
    \let\PY@bc=\relax \let\PY@ff=\relax}
\def\PY@tok#1{\csname PY@tok@#1\endcsname}
\def\PY@toks#1+{\ifx\relax#1\empty\else%
    \PY@tok{#1}\expandafter\PY@toks\fi}
\def\PY@do#1{\PY@bc{\PY@tc{\PY@ul{%
    \PY@it{\PY@bf{\PY@ff{#1}}}}}}}
\def\PY#1#2{\PY@reset\PY@toks#1+\relax+\PY@do{#2}}

\@namedef{PY@tok@w}{\def\PY@tc##1{\textcolor[rgb]{0.73,0.73,0.73}{##1}}}
\@namedef{PY@tok@c}{\let\PY@it=\textit\def\PY@tc##1{\textcolor[rgb]{0.25,0.50,0.50}{##1}}}
\@namedef{PY@tok@cp}{\def\PY@tc##1{\textcolor[rgb]{0.74,0.48,0.00}{##1}}}
\@namedef{PY@tok@k}{\let\PY@bf=\textbf\def\PY@tc##1{\textcolor[rgb]{0.00,0.50,0.00}{##1}}}
\@namedef{PY@tok@kp}{\def\PY@tc##1{\textcolor[rgb]{0.00,0.50,0.00}{##1}}}
\@namedef{PY@tok@kt}{\def\PY@tc##1{\textcolor[rgb]{0.69,0.00,0.25}{##1}}}
\@namedef{PY@tok@o}{\def\PY@tc##1{\textcolor[rgb]{0.40,0.40,0.40}{##1}}}
\@namedef{PY@tok@ow}{\let\PY@bf=\textbf\def\PY@tc##1{\textcolor[rgb]{0.67,0.13,1.00}{##1}}}
\@namedef{PY@tok@nb}{\def\PY@tc##1{\textcolor[rgb]{0.00,0.50,0.00}{##1}}}
\@namedef{PY@tok@nf}{\def\PY@tc##1{\textcolor[rgb]{0.00,0.00,1.00}{##1}}}
\@namedef{PY@tok@nc}{\let\PY@bf=\textbf\def\PY@tc##1{\textcolor[rgb]{0.00,0.00,1.00}{##1}}}
\@namedef{PY@tok@nn}{\let\PY@bf=\textbf\def\PY@tc##1{\textcolor[rgb]{0.00,0.00,1.00}{##1}}}
\@namedef{PY@tok@ne}{\let\PY@bf=\textbf\def\PY@tc##1{\textcolor[rgb]{0.82,0.25,0.23}{##1}}}
\@namedef{PY@tok@nv}{\def\PY@tc##1{\textcolor[rgb]{0.10,0.09,0.49}{##1}}}
\@namedef{PY@tok@no}{\def\PY@tc##1{\textcolor[rgb]{0.53,0.00,0.00}{##1}}}
\@namedef{PY@tok@nl}{\def\PY@tc##1{\textcolor[rgb]{0.63,0.63,0.00}{##1}}}
\@namedef{PY@tok@ni}{\let\PY@bf=\textbf\def\PY@tc##1{\textcolor[rgb]{0.60,0.60,0.60}{##1}}}
\@namedef{PY@tok@na}{\def\PY@tc##1{\textcolor[rgb]{0.49,0.56,0.16}{##1}}}
\@namedef{PY@tok@nt}{\let\PY@bf=\textbf\def\PY@tc##1{\textcolor[rgb]{0.00,0.50,0.00}{##1}}}
\@namedef{PY@tok@nd}{\def\PY@tc##1{\textcolor[rgb]{0.67,0.13,1.00}{##1}}}
\@namedef{PY@tok@s}{\def\PY@tc##1{\textcolor[rgb]{0.73,0.13,0.13}{##1}}}
\@namedef{PY@tok@sd}{\let\PY@it=\textit\def\PY@tc##1{\textcolor[rgb]{0.73,0.13,0.13}{##1}}}
\@namedef{PY@tok@si}{\let\PY@bf=\textbf\def\PY@tc##1{\textcolor[rgb]{0.73,0.40,0.53}{##1}}}
\@namedef{PY@tok@se}{\let\PY@bf=\textbf\def\PY@tc##1{\textcolor[rgb]{0.73,0.40,0.13}{##1}}}
\@namedef{PY@tok@sr}{\def\PY@tc##1{\textcolor[rgb]{0.73,0.40,0.53}{##1}}}
\@namedef{PY@tok@ss}{\def\PY@tc##1{\textcolor[rgb]{0.10,0.09,0.49}{##1}}}
\@namedef{PY@tok@sx}{\def\PY@tc##1{\textcolor[rgb]{0.00,0.50,0.00}{##1}}}
\@namedef{PY@tok@m}{\def\PY@tc##1{\textcolor[rgb]{0.40,0.40,0.40}{##1}}}
\@namedef{PY@tok@gh}{\let\PY@bf=\textbf\def\PY@tc##1{\textcolor[rgb]{0.00,0.00,0.50}{##1}}}
\@namedef{PY@tok@gu}{\let\PY@bf=\textbf\def\PY@tc##1{\textcolor[rgb]{0.50,0.00,0.50}{##1}}}
\@namedef{PY@tok@gd}{\def\PY@tc##1{\textcolor[rgb]{0.63,0.00,0.00}{##1}}}
\@namedef{PY@tok@gi}{\def\PY@tc##1{\textcolor[rgb]{0.00,0.63,0.00}{##1}}}
\@namedef{PY@tok@gr}{\def\PY@tc##1{\textcolor[rgb]{1.00,0.00,0.00}{##1}}}
\@namedef{PY@tok@ge}{\let\PY@it=\textit}
\@namedef{PY@tok@gs}{\let\PY@bf=\textbf}
\@namedef{PY@tok@gp}{\let\PY@bf=\textbf\def\PY@tc##1{\textcolor[rgb]{0.00,0.00,0.50}{##1}}}
\@namedef{PY@tok@go}{\def\PY@tc##1{\textcolor[rgb]{0.53,0.53,0.53}{##1}}}
\@namedef{PY@tok@gt}{\def\PY@tc##1{\textcolor[rgb]{0.00,0.27,0.87}{##1}}}
\@namedef{PY@tok@err}{\def\PY@bc##1{{\setlength{\fboxsep}{\string -\fboxrule}\fcolorbox[rgb]{1.00,0.00,0.00}{1,1,1}{\strut ##1}}}}
\@namedef{PY@tok@kc}{\let\PY@bf=\textbf\def\PY@tc##1{\textcolor[rgb]{0.00,0.50,0.00}{##1}}}
\@namedef{PY@tok@kd}{\let\PY@bf=\textbf\def\PY@tc##1{\textcolor[rgb]{0.00,0.50,0.00}{##1}}}
\@namedef{PY@tok@kn}{\let\PY@bf=\textbf\def\PY@tc##1{\textcolor[rgb]{0.00,0.50,0.00}{##1}}}
\@namedef{PY@tok@kr}{\let\PY@bf=\textbf\def\PY@tc##1{\textcolor[rgb]{0.00,0.50,0.00}{##1}}}
\@namedef{PY@tok@bp}{\def\PY@tc##1{\textcolor[rgb]{0.00,0.50,0.00}{##1}}}
\@namedef{PY@tok@fm}{\def\PY@tc##1{\textcolor[rgb]{0.00,0.00,1.00}{##1}}}
\@namedef{PY@tok@vc}{\def\PY@tc##1{\textcolor[rgb]{0.10,0.09,0.49}{##1}}}
\@namedef{PY@tok@vg}{\def\PY@tc##1{\textcolor[rgb]{0.10,0.09,0.49}{##1}}}
\@namedef{PY@tok@vi}{\def\PY@tc##1{\textcolor[rgb]{0.10,0.09,0.49}{##1}}}
\@namedef{PY@tok@vm}{\def\PY@tc##1{\textcolor[rgb]{0.10,0.09,0.49}{##1}}}
\@namedef{PY@tok@sa}{\def\PY@tc##1{\textcolor[rgb]{0.73,0.13,0.13}{##1}}}
\@namedef{PY@tok@sb}{\def\PY@tc##1{\textcolor[rgb]{0.73,0.13,0.13}{##1}}}
\@namedef{PY@tok@sc}{\def\PY@tc##1{\textcolor[rgb]{0.73,0.13,0.13}{##1}}}
\@namedef{PY@tok@dl}{\def\PY@tc##1{\textcolor[rgb]{0.73,0.13,0.13}{##1}}}
\@namedef{PY@tok@s2}{\def\PY@tc##1{\textcolor[rgb]{0.73,0.13,0.13}{##1}}}
\@namedef{PY@tok@sh}{\def\PY@tc##1{\textcolor[rgb]{0.73,0.13,0.13}{##1}}}
\@namedef{PY@tok@s1}{\def\PY@tc##1{\textcolor[rgb]{0.73,0.13,0.13}{##1}}}
\@namedef{PY@tok@mb}{\def\PY@tc##1{\textcolor[rgb]{0.40,0.40,0.40}{##1}}}
\@namedef{PY@tok@mf}{\def\PY@tc##1{\textcolor[rgb]{0.40,0.40,0.40}{##1}}}
\@namedef{PY@tok@mh}{\def\PY@tc##1{\textcolor[rgb]{0.40,0.40,0.40}{##1}}}
\@namedef{PY@tok@mi}{\def\PY@tc##1{\textcolor[rgb]{0.40,0.40,0.40}{##1}}}
\@namedef{PY@tok@il}{\def\PY@tc##1{\textcolor[rgb]{0.40,0.40,0.40}{##1}}}
\@namedef{PY@tok@mo}{\def\PY@tc##1{\textcolor[rgb]{0.40,0.40,0.40}{##1}}}
\@namedef{PY@tok@ch}{\let\PY@it=\textit\def\PY@tc##1{\textcolor[rgb]{0.25,0.50,0.50}{##1}}}
\@namedef{PY@tok@cm}{\let\PY@it=\textit\def\PY@tc##1{\textcolor[rgb]{0.25,0.50,0.50}{##1}}}
\@namedef{PY@tok@cpf}{\let\PY@it=\textit\def\PY@tc##1{\textcolor[rgb]{0.25,0.50,0.50}{##1}}}
\@namedef{PY@tok@c1}{\let\PY@it=\textit\def\PY@tc##1{\textcolor[rgb]{0.25,0.50,0.50}{##1}}}
\@namedef{PY@tok@cs}{\let\PY@it=\textit\def\PY@tc##1{\textcolor[rgb]{0.25,0.50,0.50}{##1}}}

\def\PYZbs{\char`\\}
\def\PYZus{\char`\_}
\def\PYZob{\char`\{}
\def\PYZcb{\char`\}}
\def\PYZca{\char`\^}
\def\PYZam{\char`\&}
\def\PYZlt{\char`\<}
\def\PYZgt{\char`\>}
\def\PYZsh{\char`\#}
\def\PYZpc{\char`\%}
\def\PYZdl{\char`\$}
\def\PYZhy{\char`\-}
\def\PYZsq{\char`\'}
\def\PYZdq{\char`\"}
\def\PYZti{\char`\~}
% for compatibility with earlier versions
\def\PYZat{@}
\def\PYZlb{[}
\def\PYZrb{]}
\makeatother


    % For linebreaks inside Verbatim environment from package fancyvrb. 
    \makeatletter
        \newbox\Wrappedcontinuationbox 
        \newbox\Wrappedvisiblespacebox 
        \newcommand*\Wrappedvisiblespace {\textcolor{red}{\textvisiblespace}} 
        \newcommand*\Wrappedcontinuationsymbol {\textcolor{red}{\llap{\tiny$\m@th\hookrightarrow$}}} 
        \newcommand*\Wrappedcontinuationindent {3ex } 
        \newcommand*\Wrappedafterbreak {\kern\Wrappedcontinuationindent\copy\Wrappedcontinuationbox} 
        % Take advantage of the already applied Pygments mark-up to insert 
        % potential linebreaks for TeX processing. 
        %        {, <, #, %, $, ' and ": go to next line. 
        %        _, }, ^, &, >, - and ~: stay at end of broken line. 
        % Use of \textquotesingle for straight quote. 
        \newcommand*\Wrappedbreaksatspecials {% 
            \def\PYGZus{\discretionary{\char`\_}{\Wrappedafterbreak}{\char`\_}}% 
            \def\PYGZob{\discretionary{}{\Wrappedafterbreak\char`\{}{\char`\{}}% 
            \def\PYGZcb{\discretionary{\char`\}}{\Wrappedafterbreak}{\char`\}}}% 
            \def\PYGZca{\discretionary{\char`\^}{\Wrappedafterbreak}{\char`\^}}% 
            \def\PYGZam{\discretionary{\char`\&}{\Wrappedafterbreak}{\char`\&}}% 
            \def\PYGZlt{\discretionary{}{\Wrappedafterbreak\char`\<}{\char`\<}}% 
            \def\PYGZgt{\discretionary{\char`\>}{\Wrappedafterbreak}{\char`\>}}% 
            \def\PYGZsh{\discretionary{}{\Wrappedafterbreak\char`\#}{\char`\#}}% 
            \def\PYGZpc{\discretionary{}{\Wrappedafterbreak\char`\%}{\char`\%}}% 
            \def\PYGZdl{\discretionary{}{\Wrappedafterbreak\char`\$}{\char`\$}}% 
            \def\PYGZhy{\discretionary{\char`\-}{\Wrappedafterbreak}{\char`\-}}% 
            \def\PYGZsq{\discretionary{}{\Wrappedafterbreak\textquotesingle}{\textquotesingle}}% 
            \def\PYGZdq{\discretionary{}{\Wrappedafterbreak\char`\"}{\char`\"}}% 
            \def\PYGZti{\discretionary{\char`\~}{\Wrappedafterbreak}{\char`\~}}% 
        } 
        % Some characters . , ; ? ! / are not pygmentized. 
        % This macro makes them "active" and they will insert potential linebreaks 
        \newcommand*\Wrappedbreaksatpunct {% 
            \lccode`\~`\.\lowercase{\def~}{\discretionary{\hbox{\char`\.}}{\Wrappedafterbreak}{\hbox{\char`\.}}}% 
            \lccode`\~`\,\lowercase{\def~}{\discretionary{\hbox{\char`\,}}{\Wrappedafterbreak}{\hbox{\char`\,}}}% 
            \lccode`\~`\;\lowercase{\def~}{\discretionary{\hbox{\char`\;}}{\Wrappedafterbreak}{\hbox{\char`\;}}}% 
            \lccode`\~`\:\lowercase{\def~}{\discretionary{\hbox{\char`\:}}{\Wrappedafterbreak}{\hbox{\char`\:}}}% 
            \lccode`\~`\?\lowercase{\def~}{\discretionary{\hbox{\char`\?}}{\Wrappedafterbreak}{\hbox{\char`\?}}}% 
            \lccode`\~`\!\lowercase{\def~}{\discretionary{\hbox{\char`\!}}{\Wrappedafterbreak}{\hbox{\char`\!}}}% 
            \lccode`\~`\/\lowercase{\def~}{\discretionary{\hbox{\char`\/}}{\Wrappedafterbreak}{\hbox{\char`\/}}}% 
            \catcode`\.\active
            \catcode`\,\active 
            \catcode`\;\active
            \catcode`\:\active
            \catcode`\?\active
            \catcode`\!\active
            \catcode`\/\active 
            \lccode`\~`\~ 	
        }
    \makeatother

    \let\OriginalVerbatim=\Verbatim
    \makeatletter
    \renewcommand{\Verbatim}[1][1]{%
        %\parskip\z@skip
        \sbox\Wrappedcontinuationbox {\Wrappedcontinuationsymbol}%
        \sbox\Wrappedvisiblespacebox {\FV@SetupFont\Wrappedvisiblespace}%
        \def\FancyVerbFormatLine ##1{\hsize\linewidth
            \vtop{\raggedright\hyphenpenalty\z@\exhyphenpenalty\z@
                \doublehyphendemerits\z@\finalhyphendemerits\z@
                \strut ##1\strut}%
        }%
        % If the linebreak is at a space, the latter will be displayed as visible
        % space at end of first line, and a continuation symbol starts next line.
        % Stretch/shrink are however usually zero for typewriter font.
        \def\FV@Space {%
            \nobreak\hskip\z@ plus\fontdimen3\font minus\fontdimen4\font
            \discretionary{\copy\Wrappedvisiblespacebox}{\Wrappedafterbreak}
            {\kern\fontdimen2\font}%
        }%
        
        % Allow breaks at special characters using \PYG... macros.
        \Wrappedbreaksatspecials
        % Breaks at punctuation characters . , ; ? ! and / need catcode=\active 	
        \OriginalVerbatim[#1,codes*=\Wrappedbreaksatpunct]%
    }
    \makeatother

    % Exact colors from NB
    \definecolor{incolor}{HTML}{303F9F}
    \definecolor{outcolor}{HTML}{D84315}
    \definecolor{cellborder}{HTML}{CFCFCF}
    \definecolor{cellbackground}{HTML}{F7F7F7}
    
    % prompt
    \makeatletter
    \newcommand{\boxspacing}{\kern\kvtcb@left@rule\kern\kvtcb@boxsep}
    \makeatother
    \newcommand{\prompt}[4]{
        {\ttfamily\llap{{\color{#2}[#3]:\hspace{3pt}#4}}\vspace{-\baselineskip}}
    }
    

    
    % Prevent overflowing lines due to hard-to-break entities
    \sloppy 
    % Setup hyperref package
    \hypersetup{
      breaklinks=true,  % so long urls are correctly broken across lines
      colorlinks=true,
      urlcolor=urlcolor,
      linkcolor=linkcolor,
      citecolor=citecolor,
      }
    % Slightly bigger margins than the latex defaults
    
    \geometry{verbose,tmargin=1in,bmargin=1in,lmargin=1in,rmargin=1in}
    
    

\begin{document}
    \title{Tarea 2 Optimización}
    \author{Roberto Vásquez Martínez \\ Profesor: Joaquín Peña Acevedo}
    \date{20/Febrero/2022}
    \maketitle    
    

    
    \hypertarget{ejercicio-1-4-puntos}{%
\section{Ejercicio 1 (4 puntos)}\label{ejercicio-1-4-puntos}}

Calcular las raíces de los polinomios dados y generar las gráficas de
los polinomos para mostrar las raíces reales encontradas.

\begin{enumerate}
\def\labelenumi{\arabic{enumi}.}
\tightlist
\item
  Escriba una función que reciba un arreglo \(c\) que contiene los
  coeficientes del polinomio, de modo que si \(c\) es un arreglo de
  longitud \(n\) el valor del polinomio de grado \(n-1\) en \(x\) se
  calcula mediante
\end{enumerate}

\[ c[0]*x**(n-1) + c[1]*x**(n-2) + ... + c[n-2]*x + c[n-1]\]

\begin{enumerate}
\def\labelenumi{\arabic{enumi}.}
\setcounter{enumi}{1}
\item
  Revise la documentación de la función
  \href{https://numpy.org/doc/stable/reference/generated/numpy.roots.html}{roots()}
  de Numpy y úsela para obtener un arreglo con las raíces del polinomio
  e imprima las raíces encontradas.
\item
  Las raíces pueden ser complejas. Obtenga un arreglo \(r\) que contenga
  sólo las raíces reales, imprimalo y calcule las raíz real \(r_{min}\)
  más pequeña y la raíz real \(r_{max}\) más grande.
\item
  Use la función \texttt{linspace()} para generar un arreglo \(x\) con
  100 valores que corresponden a una partición del intervalo
  \([r_{min}-1, r_{max}+1]\). Evalúe el polinomio en los valores de
  \(x\). Puede usar la función de Numpy
  \href{https://numpy.org/doc/stable/reference/generated/numpy.polyval.html\#numpy.polyval}{polyval()}
  para evaluar el polinomio y generar un arreglo \(y\).
\item
  Use los arreglos \(x\) y \(y\) para generar la gráfica del polinomio.
\item
  Agregue a la gráfica los puntos que representan las raíces reales
  \(r\). Para esto evalue el polinomio en \(r\) para generar un arreglo
  \(polr\) con esos valores. Use los arreglos \(r\) y \(polr\) para
  graficar como puntos en la gráfica anterior\\
  para ver que coinciden con los ceros de las gráfica.
\item
  Pruebe la función con los siguientes polinomios:
\end{enumerate}

\[ f_1(x) = -4 {{x}^{3}}+33 {{x}^{2}}+97 x-840 \]

\[ f_2(x) = -2 {{x}^{4}}+15 {{x}^{3}}-36 {{x}^{2}}+135 x-162 \]

\hypertarget{soluciuxf3n}{%
\subsection{Solución:}\label{soluciuxf3n}}

    La función que implementa los puntos del 1 al 6 es
\texttt{pol\_roots\_plot()} y se encuentra en el módulo
\texttt{lib\_t2.py}. Probamos esta función con el polinomio \(f_1\). En
primer lugar, imprimimos las raices de \(f_1\).

    \begin{tcolorbox}[breakable, size=fbox, boxrule=1pt, pad at break*=1mm,colback=cellbackground, colframe=cellborder]
\prompt{In}{incolor}{1}{\boxspacing}
\begin{Verbatim}[commandchars=\\\{\}]
\PY{k+kn}{import} \PY{n+nn}{os}
\PY{k+kn}{import} \PY{n+nn}{numpy} \PY{k}{as} \PY{n+nn}{np}
\PY{k+kn}{from} \PY{n+nn}{lib\PYZus{}t2} \PY{k+kn}{import} \PY{o}{*}
\PY{k+kn}{from} \PY{n+nn}{plotnine} \PY{k+kn}{import} \PY{o}{*}
\PY{n}{coef\PYZus{}f1}\PY{o}{=}\PY{p}{[}\PY{o}{\PYZhy{}}\PY{l+m+mf}{4.0}\PY{p}{,}\PY{l+m+mf}{33.0}\PY{p}{,}\PY{l+m+mf}{97.0}\PY{p}{,}\PY{o}{\PYZhy{}}\PY{l+m+mf}{840.0}\PY{p}{]}
\PY{n}{roots\PYZus{}f1}\PY{p}{,}\PY{n}{re\PYZus{}roots\PYZus{}f1}\PY{p}{,}\PY{n}{gg\PYZus{}f1}\PY{o}{=}\PY{n}{poly\PYZus{}roots\PYZus{}plot}\PY{p}{(}\PY{n}{coef\PYZus{}f1}\PY{p}{)}
\PY{n+nb}{print}\PY{p}{(}\PY{l+s+s1}{\PYZsq{}}\PY{l+s+s1}{Las raíces de f1 son: }\PY{l+s+s1}{\PYZsq{}}\PY{p}{,}\PY{n}{roots\PYZus{}f1}\PY{p}{)}
\PY{n+nb}{print}\PY{p}{(}\PY{l+s+s1}{\PYZsq{}}\PY{l+s+s1}{Las raíces reales son: }\PY{l+s+s1}{\PYZsq{}}\PY{p}{,}\PY{n}{re\PYZus{}roots\PYZus{}f1}\PY{p}{)}
\end{Verbatim}
\end{tcolorbox}

    \begin{Verbatim}[commandchars=\\\{\}]
Las raíces de f1 son:  [-5.    8.    5.25]
Las raíces reales son:  [-5.    8.    5.25]
    \end{Verbatim}

    La gráfica del polinomio, con dominio
\([r_{\mathrm{min}}-1,r_{\mathrm{max}}+1]\) donde
\(r_{\mathrm{min}},r_{\mathrm{max}}\) son la mínima y máxima de las
raíces reales del polinomio \(f_1\), es:

    \begin{tcolorbox}[breakable, size=fbox, boxrule=1pt, pad at break*=1mm,colback=cellbackground, colframe=cellborder]
\prompt{In}{incolor}{2}{\boxspacing}
\begin{Verbatim}[commandchars=\\\{\}]
\PY{n}{gg\PYZus{}1}\PY{o}{=}\PY{n}{gg\PYZus{}f1}\PY{o}{+}\PY{n}{xlab}\PY{p}{(}\PY{l+s+sa}{r}\PY{l+s+s1}{\PYZsq{}}\PY{l+s+s1}{\PYZdl{}x\PYZdl{}}\PY{l+s+s1}{\PYZsq{}}\PY{p}{)}\PY{o}{+}\PY{n}{ylab}\PY{p}{(}\PY{l+s+sa}{r}\PY{l+s+s1}{\PYZsq{}}\PY{l+s+s1}{\PYZdl{}f\PYZus{}1(x)\PYZdl{}}\PY{l+s+s1}{\PYZsq{}}\PY{p}{)}\PY{o}{+}\PY{n}{ggtitle}\PY{p}{(}\PY{l+s+sa}{r}\PY{l+s+s1}{\PYZsq{}}\PY{l+s+s1}{Gráfica de \PYZdl{}f\PYZus{}1\PYZdl{}}\PY{l+s+s1}{\PYZsq{}}\PY{p}{)}\PY{o}{+}\PY{n}{theme}\PY{p}{(}\PY{n}{figure\PYZus{}size}\PY{o}{=}\PY{p}{(}\PY{l+m+mi}{6}\PY{p}{,} \PY{l+m+mi}{4}\PY{p}{)}\PY{p}{)}
\PY{n}{gg\PYZus{}1}
\end{Verbatim}
\end{tcolorbox}

    \begin{center}
    \adjustimage{max size={0.9\linewidth}{0.9\paperheight}}{Tarea_2_Optimizacion_files/Tarea_2_Optimizacion_4_0.png}
    \end{center}
    { \hspace*{\fill} \\}
    
            \begin{tcolorbox}[breakable, size=fbox, boxrule=.5pt, pad at break*=1mm, opacityfill=0]
\prompt{Out}{outcolor}{2}{\boxspacing}
\begin{Verbatim}[commandchars=\\\{\}]
<ggplot: (8737991389296)>
\end{Verbatim}
\end{tcolorbox}
        
    Hacemos el mismo ejercicio para el polinomio \(f_2\). Primero,
imprimimos las raíces y las raíces reales.

    \begin{tcolorbox}[breakable, size=fbox, boxrule=1pt, pad at break*=1mm,colback=cellbackground, colframe=cellborder]
\prompt{In}{incolor}{3}{\boxspacing}
\begin{Verbatim}[commandchars=\\\{\}]
\PY{n}{coef\PYZus{}f2}\PY{o}{=}\PY{p}{[}\PY{o}{\PYZhy{}}\PY{l+m+mf}{2.0}\PY{p}{,}\PY{l+m+mf}{15.0}\PY{p}{,}\PY{o}{\PYZhy{}}\PY{l+m+mf}{36.0}\PY{p}{,}\PY{l+m+mf}{135.0}\PY{p}{,}\PY{o}{\PYZhy{}}\PY{l+m+mf}{162.0}\PY{p}{]}
\PY{n}{roots\PYZus{}f2}\PY{p}{,}\PY{n}{re\PYZus{}roots\PYZus{}f2}\PY{p}{,}\PY{n}{gg\PYZus{}f2}\PY{o}{=}\PY{n}{poly\PYZus{}roots\PYZus{}plot}\PY{p}{(}\PY{n}{coef\PYZus{}f2}\PY{p}{)}
\PY{n+nb}{print}\PY{p}{(}\PY{l+s+s1}{\PYZsq{}}\PY{l+s+s1}{Las raíces de f1 son: }\PY{l+s+s1}{\PYZsq{}}\PY{p}{,}\PY{n}{roots\PYZus{}f2}\PY{p}{)}
\PY{n+nb}{print}\PY{p}{(}\PY{l+s+s1}{\PYZsq{}}\PY{l+s+s1}{Las raíces reales son: }\PY{l+s+s1}{\PYZsq{}}\PY{p}{,}\PY{n}{re\PYZus{}roots\PYZus{}f2}\PY{p}{)}
\end{Verbatim}
\end{tcolorbox}

    \begin{Verbatim}[commandchars=\\\{\}]
Las raíces de f1 son:  [6.00000000e+00+0.j 8.32667268e-16+3.j 8.32667268e-16-3.j
 1.50000000e+00+0.j]
Las raíces reales son:  [6.  1.5]
    \end{Verbatim}

    Y graficamos en el dominio \([r_{\mathrm{min}}-1,r_{\mathrm{max}}+1]\)

    \begin{tcolorbox}[breakable, size=fbox, boxrule=1pt, pad at break*=1mm,colback=cellbackground, colframe=cellborder]
\prompt{In}{incolor}{4}{\boxspacing}
\begin{Verbatim}[commandchars=\\\{\}]
\PY{n}{gg\PYZus{}2}\PY{o}{=}\PY{n}{gg\PYZus{}f2}\PY{o}{+}\PY{n}{xlab}\PY{p}{(}\PY{l+s+sa}{r}\PY{l+s+s1}{\PYZsq{}}\PY{l+s+s1}{\PYZdl{}x\PYZdl{}}\PY{l+s+s1}{\PYZsq{}}\PY{p}{)}\PY{o}{+}\PY{n}{ylab}\PY{p}{(}\PY{l+s+sa}{r}\PY{l+s+s1}{\PYZsq{}}\PY{l+s+s1}{\PYZdl{}f\PYZus{}2(x)\PYZdl{}}\PY{l+s+s1}{\PYZsq{}}\PY{p}{)}\PY{o}{+}\PY{n}{ggtitle}\PY{p}{(}\PY{l+s+sa}{r}\PY{l+s+s1}{\PYZsq{}}\PY{l+s+s1}{Gráfica de \PYZdl{}f\PYZus{}2\PYZdl{}}\PY{l+s+s1}{\PYZsq{}}\PY{p}{)}\PY{o}{+}\PY{n}{theme}\PY{p}{(}\PY{n}{figure\PYZus{}size}\PY{o}{=}\PY{p}{(}\PY{l+m+mi}{6}\PY{p}{,} \PY{l+m+mi}{4}\PY{p}{)}\PY{p}{)}
\PY{n}{gg\PYZus{}2}
\end{Verbatim}
\end{tcolorbox}

    \begin{center}
    \adjustimage{max size={0.9\linewidth}{0.9\paperheight}}{Tarea_2_Optimizacion_files/Tarea_2_Optimizacion_8_0.png}
    \end{center}
    { \hspace*{\fill} \\}
    
            \begin{tcolorbox}[breakable, size=fbox, boxrule=.5pt, pad at break*=1mm, opacityfill=0]
\prompt{Out}{outcolor}{4}{\boxspacing}
\begin{Verbatim}[commandchars=\\\{\}]
<ggplot: (8737991242084)>
\end{Verbatim}
\end{tcolorbox}
        
    \hypertarget{ejercicio-2-6-puntos}{%
\section{Ejercicio 2 (6 puntos)}\label{ejercicio-2-6-puntos}}

Programar la función que resuelve el problema de ajustar un polinomio a
un conjunto de puntos \(\{(x_0, y_0), (x_1, y_1), ..., (x_m, y_m)\}\)
usando mínimos cuadrados lineales.

Si revisan las notas del curso de métodos numéricos, se ve que para
ajustar el polinomio de grado \(n\)
\(p(x) = c_n x^n + c_{n-1} x^{n-1} + ... + c_1 x + c_0\) mediante
mínimos cuadrados, hay que plantear el problema de minimizar la suma de
diferencias elevadas al cuadrado:

\(\sum_{i=0}^{m} (p(x_i) - y_i )^2\)

Esto nos lleva a construir la matriz \(A\) y el vector de términos
independientes

    \[ A = \left[\begin{array}{ccccc}
x_1^n  & x_1^{n-1} & \cdots & x_1 & 1 \\
x_2^n  & x_2^{n-1} & \cdots & x_2 & 1 \\
\vdots & \vdots    &        & \vdots  & \vdots \\
x_m^n  & x_m^{n-1} & \cdots & x_m & 1 
\end{array}\right], \qquad
b =\left( \begin{array}{c}
y_1 \\ y_2 \\ \vdots \\ y_m
\end{array}  
\right)\]

    Entonces el vector \(c\) con los coeficientes del polinomio se obtiene
resolviendo el sistema de ecuaciones

\(A^{\top} A c= A^{\top} y\).

\begin{enumerate}
\def\labelenumi{\arabic{enumi}.}
\tightlist
\item
  Programe la función que recibe como argumento un arreglo 2D que
  contiene los puntos \(\{(x_0, y_0), (x_1, y_1), ..., (x_m, y_m)\}\) y
  el grado del polinomio \(n\), que construya y resuelva el sistema de
  ecuaciones para obtener el vector de coeficientes \(c\) usando las
  funciones de Numpy y que devuelva este arreglo y el número de
  condición de la matriz del sistema. Este último dato lo puede obtener
  usando la función \texttt{numpy.linalg.cond()}.
\item
  Escriba una función que reciba como argumentos el nombre de un archivo
  que contiene los datos, el valor \(n\) del grado del polinomio que se
  quiere ajustar y un entero \(r>0\).
\end{enumerate}

\begin{itemize}
\tightlist
\item
  La función debe leer el archivo, cargar los datos en una matriz y usar
  la función del inciso anterior para obtener el vector de coeficientes
  \(c\) y el número de condición. El archivo contiene una matriz con dos
  columnas. La primer columna tiene las abscisas
  \(x_0, x_1, ..., x_{m}\) y la segunda columna tiene las ordenadas
  \(y_0, y_1, ..., y_{m}\) de los puntos.
\item
  Obtenga el valor mínimo \(x_{\min}\) y máximo \(x_{\max}\) de las
  abscicas \(x_i\).\\
\item
  Genere una partición \(z_0, z_1, ..., z_{r-1}\) del intervalo\\
  \([x_{\min}, x_{\min}]\) con \(r\) puntos y use la función
  \texttt{numpy.polyval()} para evaluar el polinomio \(p(x)\) en los
  puntos de la partición del intervalo.
\item
  Haga que la función imprima el grado \(n\) del polinomio, los
  coeficientes \(\mathbf{c}\) del polinomio y el número de condición.
  También haga que la función genere una gráfica que muestre los puntos
  \(\{(x_0, y_0), (x_1, y_1), ..., (x_m, y_m)\}\) (como puntos) y los
  puntos \((z_i, p(z_i))\) con un trazo continuo para comparar los datos
  con la gráfica del polinomio.
\end{itemize}

\begin{enumerate}
\def\labelenumi{\arabic{enumi}.}
\setcounter{enumi}{2}
\tightlist
\item
  Pruebe la función del inciso anterior usando los archivos \texttt{npy}
  que se encuentran dentro del archivo \texttt{datosTarea02.zip}. Para
  cada caso, use \(r=100\) y \(n=1,2,3,4,5\) y \(6\) (puede poner un
  ciclo para generar los resultados de cada caso).
\end{enumerate}

\hypertarget{soluciuxf3n}{%
\subsection{Solución:}\label{soluciuxf3n}}

    La función del numeral 1 que encuentra los coeficientes resolviendo el
sistema \(A^TAc=A^Ty\) y el número de condición de \(A^TA\) es
\texttt{pol\_solve()} y la función del punto 2 que recibe como argumento
el grado del polinomio a ajustar y el nombre del archivo para cargar los
datos desde un archivo \texttt{.npy} es \texttt{pol\_reg()}, ambas
funciones se encuentran al igual que en el ejercicio 1 en el módulo
\texttt{lib\_t2.py}. A continuación ponemos el path, tomando como
referencia el directorio donde se está ejecutando este notebook, de los
archivos con los datos.

    \begin{tcolorbox}[breakable, size=fbox, boxrule=1pt, pad at break*=1mm,colback=cellbackground, colframe=cellborder]
\prompt{In}{incolor}{5}{\boxspacing}
\begin{Verbatim}[commandchars=\\\{\}]
\PY{c+c1}{\PYZsh{} Path de los archivos con los datos}
\PY{n}{DataEj1}\PY{o}{=}\PY{l+s+s1}{\PYZsq{}}\PY{l+s+s1}{/datos Tarea 2/datos1.npy}\PY{l+s+s1}{\PYZsq{}}
\PY{n}{DataEj2}\PY{o}{=}\PY{l+s+s1}{\PYZsq{}}\PY{l+s+s1}{/datos Tarea 2/datos2.npy}\PY{l+s+s1}{\PYZsq{}}
\end{Verbatim}
\end{tcolorbox}

    En primer lugar, probamos estas funciones con el archivo de datos
\texttt{datos1.npy}. A continuación hacemos las pruebas correspondientes
con \(n=1,2,3,4,5,6\) y \(r=100\), en la siguiente celda se imprime el
vector con los coeficientes del polinomio resultante para cada grado, en
orden decreciente, es decir, del coeficiente de grado más alto hasta el
término independiente, además del número de condición.

    \begin{tcolorbox}[breakable, size=fbox, boxrule=1pt, pad at break*=1mm,colback=cellbackground, colframe=cellborder]
\prompt{In}{incolor}{6}{\boxspacing}
\begin{Verbatim}[commandchars=\\\{\}]
\PY{k+kn}{from} \PY{n+nn}{matplotlib} \PY{k+kn}{import} \PY{n}{gridspec}
\PY{k+kn}{from} \PY{n+nn}{plotnine} \PY{k+kn}{import} \PY{n}{data}
\PY{n}{np}\PY{o}{.}\PY{n}{set\PYZus{}printoptions}\PY{p}{(}\PY{n}{precision}\PY{o}{=}\PY{l+m+mi}{4}\PY{p}{)}
\PY{n}{list\PYZus{}plots}\PY{o}{=}\PY{p}{[}\PY{p}{]}
\PY{n}{nn}\PY{o}{=}\PY{p}{[}\PY{l+m+mi}{1}\PY{p}{,}\PY{l+m+mi}{2}\PY{p}{,}\PY{l+m+mi}{3}\PY{p}{,}\PY{l+m+mi}{4}\PY{p}{,}\PY{l+m+mi}{5}\PY{p}{,}\PY{l+m+mi}{6}\PY{p}{]}
\PY{n}{r}\PY{o}{=}\PY{l+m+mi}{100}
\PY{k}{for} \PY{n}{n} \PY{o+ow}{in} \PY{n}{nn}\PY{p}{:}
    \PY{n}{c}\PY{p}{,}\PY{n}{cod\PYZus{}num}\PY{p}{,}\PY{n}{gg}\PY{o}{=}\PY{n}{poly\PYZus{}reg}\PY{p}{(}\PY{n}{DataEj1}\PY{p}{,}\PY{n}{n}\PY{p}{,}\PY{n}{r}\PY{p}{)}
    \PY{n}{list\PYZus{}plots}\PY{o}{.}\PY{n}{append}\PY{p}{(}\PY{n}{gg}\PY{p}{)}
    \PY{n+nb}{print}\PY{p}{(}\PY{l+s+sa}{f}\PY{l+s+s1}{\PYZsq{}}\PY{l+s+s1}{Los coeficientes del polinomio de grado }\PY{l+s+si}{\PYZob{}}\PY{n}{n}\PY{l+s+si}{\PYZcb{}}\PY{l+s+s1}{ son: }\PY{l+s+si}{\PYZob{}}\PY{n}{c}\PY{l+s+si}{\PYZcb{}}\PY{l+s+s1}{\PYZsq{}}\PY{p}{)}
    \PY{n+nb}{print}\PY{p}{(}\PY{l+s+s1}{\PYZsq{}}\PY{l+s+s1}{El número de condición de la matriz A es: }\PY{l+s+si}{\PYZpc{}.4f}\PY{l+s+s1}{\PYZsq{}}\PY{o}{\PYZpc{}}\PY{k}{cod\PYZus{}num})
    \PY{n+nb}{print}\PY{p}{(}\PY{l+s+s1}{\PYZsq{}}\PY{l+s+se}{\PYZbs{}n}\PY{l+s+s1}{\PYZsq{}}\PY{p}{)}
\end{Verbatim}
\end{tcolorbox}

    \begin{Verbatim}[commandchars=\\\{\}]
Los coeficientes del polinomio de grado 1 son: [-2.5592 -3.5203]
El número de condición de la matriz A es: 29.5460


Los coeficientes del polinomio de grado 2 son: [-0.4416 -0.7196  3.1585]
El número de condición de la matriz A es: 2458.7290


Los coeficientes del polinomio de grado 3 son: [-0.0611 -0.0726  0.5994 -0.3879]
El número de condición de la matriz A es: 210402.5774


Los coeficientes del polinomio de grado 4 son: [ 0.017  -0.2059 -0.4706  3.6684
1.4594]
El número de condición de la matriz A es: 15816940.8189


Los coeficientes del polinomio de grado 5 son: [ 1.2995e-03  3.0530e-03
-2.3079e-01 -4.8683e-02  3.7021e+00 -4.0155e-03]
El número de condición de la matriz A es: 1627465651.9694


Los coeficientes del polinomio de grado 6 son: [-2.4132e-04  4.5146e-03
5.4331e-03 -3.5826e-01  4.2782e-02  4.7803e+00
 -2.2272e-01]
El número de condición de la matriz A es: 132393662527.0403


    \end{Verbatim}

    Finalmente, graficamos los resultados obtenidos. En línea continua
aparece la gráfica del polinomio resultante de resolver el sistema de
ecuaciones.

    \begin{tcolorbox}[breakable, size=fbox, boxrule=1pt, pad at break*=1mm,colback=cellbackground, colframe=cellborder]
\prompt{In}{incolor}{7}{\boxspacing}
\begin{Verbatim}[commandchars=\\\{\}]
\PY{n}{fig} \PY{o}{=} \PY{p}{(}\PY{n}{ggplot}\PY{p}{(}\PY{p}{)}\PY{o}{+}\PY{n}{geom\PYZus{}blank}\PY{p}{(}\PY{n}{data}\PY{o}{=}\PY{n}{data}\PY{o}{.}\PY{n}{diamonds}\PY{p}{)}\PY{o}{+}\PY{n}{theme\PYZus{}void}\PY{p}{(}\PY{p}{)}\PY{o}{+}\PY{n}{theme}\PY{p}{(}\PY{n}{figure\PYZus{}size}\PY{o}{=}\PY{p}{(}\PY{l+m+mi}{8}\PY{p}{,} \PY{l+m+mi}{6}\PY{p}{)}\PY{p}{)}\PY{p}{)}\PY{o}{.}\PY{n}{draw}\PY{p}{(}\PY{p}{)}
\PY{n}{gs} \PY{o}{=} \PY{n}{gridspec}\PY{o}{.}\PY{n}{GridSpec}\PY{p}{(}\PY{l+m+mi}{2}\PY{p}{,}\PY{l+m+mi}{3}\PY{p}{)}
\PY{n}{index}\PY{o}{=}\PY{l+m+mi}{0}
\PY{k}{for} \PY{n}{i} \PY{o+ow}{in} \PY{n+nb}{range}\PY{p}{(}\PY{l+m+mi}{2}\PY{p}{)}\PY{p}{:}
    \PY{k}{for} \PY{n}{j} \PY{o+ow}{in} \PY{n+nb}{range}\PY{p}{(}\PY{l+m+mi}{3}\PY{p}{)}\PY{p}{:}
        \PY{n}{ax}\PY{o}{=}\PY{n}{fig}\PY{o}{.}\PY{n}{add\PYZus{}subplot}\PY{p}{(}\PY{n}{gs}\PY{p}{[}\PY{n}{i}\PY{p}{,}\PY{n}{j}\PY{p}{]}\PY{p}{,}\PY{n}{title}\PY{o}{=}\PY{l+s+sa}{f}\PY{l+s+s1}{\PYZsq{}}\PY{l+s+s1}{Grado }\PY{l+s+si}{\PYZob{}}\PY{n}{index}\PY{o}{+}\PY{l+m+mi}{1}\PY{l+s+si}{\PYZcb{}}\PY{l+s+s1}{\PYZsq{}}\PY{p}{,}\PY{n}{xmargin}\PY{o}{=}\PY{l+m+mi}{20}\PY{p}{,}\PY{n}{ymargin}\PY{o}{=}\PY{l+m+mi}{20}\PY{p}{)}
        \PY{n}{\PYZus{}}\PY{o}{=}\PY{n}{list\PYZus{}plots}\PY{p}{[}\PY{n}{index}\PY{p}{]}\PY{o}{.}\PY{n}{\PYZus{}draw\PYZus{}using\PYZus{}figure}\PY{p}{(}\PY{n}{fig}\PY{p}{,}\PY{p}{[}\PY{n}{ax}\PY{p}{]}\PY{p}{)}
        \PY{n}{fig}\PY{o}{.}\PY{n}{tight\PYZus{}layout}\PY{p}{(}\PY{n}{pad}\PY{o}{=}\PY{l+m+mf}{2.0}\PY{p}{)}
        \PY{n}{index}\PY{o}{+}\PY{o}{=}\PY{l+m+mi}{1}
\end{Verbatim}
\end{tcolorbox}

    \begin{center}
    \adjustimage{max size={0.9\linewidth}{0.9\paperheight}}{Tarea_2_Optimizacion_files/Tarea_2_Optimizacion_17_0.png}
    \end{center}
    { \hspace*{\fill} \\}
    
    Observamos que el número de condición aumenta conforme aumenta el grado
del polinomio, esto se debe a que para mayor grado este ajuste es más
susceptible a la muestra que tenemos. Además, se ve que del grado 4 al 6
el ajuste es prácticamente el mismo, lo que se ve también en la magnitud
de los coeficientes de orden 5 y 6, que son números pequeños. Si
tuvieramos que escoger un polinomio para modelar los datos sería el de
grado 4, pues es el mejor condicionado y en que se observa un buen
ajuste, aunque también podrías elegir el de grado 3 ya que aunque no
tiene un gran ajuste como el polinomio de grado 4 su condicionamiento es
mucho mejor.

    Hacemos el mismo ejercicio para los datos en \texttt{datos2.npy}.
Primero imprimimos los vectores de coeficientes resultantes que definen
a los polnomios de grado \(n=1,2,3,4,5,6\) con tamaño de partición
\(r=100\).

    \begin{tcolorbox}[breakable, size=fbox, boxrule=1pt, pad at break*=1mm,colback=cellbackground, colframe=cellborder]
\prompt{In}{incolor}{8}{\boxspacing}
\begin{Verbatim}[commandchars=\\\{\}]
\PY{n}{np}\PY{o}{.}\PY{n}{set\PYZus{}printoptions}\PY{p}{(}\PY{n}{precision}\PY{o}{=}\PY{l+m+mi}{4}\PY{p}{)}
\PY{n}{list\PYZus{}plots}\PY{o}{=}\PY{p}{[}\PY{p}{]}
\PY{n}{nn}\PY{o}{=}\PY{p}{[}\PY{l+m+mi}{1}\PY{p}{,}\PY{l+m+mi}{2}\PY{p}{,}\PY{l+m+mi}{3}\PY{p}{,}\PY{l+m+mi}{4}\PY{p}{,}\PY{l+m+mi}{5}\PY{p}{,}\PY{l+m+mi}{6}\PY{p}{]}
\PY{n}{r}\PY{o}{=}\PY{l+m+mi}{100}
\PY{k}{for} \PY{n}{n} \PY{o+ow}{in} \PY{n}{nn}\PY{p}{:}
    \PY{n}{c}\PY{p}{,}\PY{n}{cod\PYZus{}num}\PY{p}{,}\PY{n}{gg}\PY{o}{=}\PY{n}{poly\PYZus{}reg}\PY{p}{(}\PY{n}{DataEj2}\PY{p}{,}\PY{n}{n}\PY{p}{,}\PY{n}{r}\PY{p}{)}
    \PY{n}{list\PYZus{}plots}\PY{o}{.}\PY{n}{append}\PY{p}{(}\PY{n}{gg}\PY{p}{)}
    \PY{n+nb}{print}\PY{p}{(}\PY{l+s+sa}{f}\PY{l+s+s1}{\PYZsq{}}\PY{l+s+s1}{Los coeficientes del polinomio de grado }\PY{l+s+si}{\PYZob{}}\PY{n}{n}\PY{l+s+si}{\PYZcb{}}\PY{l+s+s1}{ son: }\PY{l+s+si}{\PYZob{}}\PY{n}{c}\PY{l+s+si}{\PYZcb{}}\PY{l+s+s1}{\PYZsq{}}\PY{p}{)}
    \PY{n+nb}{print}\PY{p}{(}\PY{l+s+s1}{\PYZsq{}}\PY{l+s+s1}{El número de condición de la matriz A es: }\PY{l+s+si}{\PYZpc{}.4f}\PY{l+s+s1}{\PYZsq{}}\PY{o}{\PYZpc{}}\PY{k}{cod\PYZus{}num})
    \PY{n+nb}{print}\PY{p}{(}\PY{l+s+s1}{\PYZsq{}}\PY{l+s+se}{\PYZbs{}n}\PY{l+s+s1}{\PYZsq{}}\PY{p}{)}
\end{Verbatim}
\end{tcolorbox}

    \begin{Verbatim}[commandchars=\\\{\}]
Los coeficientes del polinomio de grado 1 son: [ 0.455  21.8233]
El número de condición de la matriz A es: 57.6662


Los coeficientes del polinomio de grado 2 son: [-0.0728 -0.0499 21.9262]
El número de condición de la matriz A es: 3340.5363


Los coeficientes del polinomio de grado 3 son: [ 0.0889  0.8635  0.7332 16.6059]
El número de condición de la matriz A es: 458979.9355


Los coeficientes del polinomio de grado 4 son: [ 0.0259  0.4468  1.7319 -1.8706
11.8505]
El número de condición de la matriz A es: 57617346.0545


Los coeficientes del polinomio de grado 5 son: [-0.0184 -0.307  -1.0775  2.7639
10.8231 16.0055]
El número de condición de la matriz A es: 5519382925.3796


Los coeficientes del polinomio de grado 6 son: [-2.6826e-03 -7.6110e-02
-6.6084e-01 -1.2096e+00  6.2980e+00  1.4285e+01
  1.1681e+01]
El número de condición de la matriz A es: 537838254507.8020


    \end{Verbatim}

    Y observamos el ajuste de cada uno de estos polinomios con la muestra.

    \begin{tcolorbox}[breakable, size=fbox, boxrule=1pt, pad at break*=1mm,colback=cellbackground, colframe=cellborder]
\prompt{In}{incolor}{9}{\boxspacing}
\begin{Verbatim}[commandchars=\\\{\}]
\PY{n}{fig} \PY{o}{=} \PY{p}{(}\PY{n}{ggplot}\PY{p}{(}\PY{p}{)}\PY{o}{+}\PY{n}{geom\PYZus{}blank}\PY{p}{(}\PY{n}{data}\PY{o}{=}\PY{n}{data}\PY{o}{.}\PY{n}{diamonds}\PY{p}{)}\PY{o}{+}\PY{n}{theme\PYZus{}void}\PY{p}{(}\PY{p}{)}\PY{o}{+}\PY{n}{theme}\PY{p}{(}\PY{n}{figure\PYZus{}size}\PY{o}{=}\PY{p}{(}\PY{l+m+mi}{8}\PY{p}{,} \PY{l+m+mi}{6}\PY{p}{)}\PY{p}{)}\PY{p}{)}\PY{o}{.}\PY{n}{draw}\PY{p}{(}\PY{p}{)}
\PY{n}{gs} \PY{o}{=} \PY{n}{gridspec}\PY{o}{.}\PY{n}{GridSpec}\PY{p}{(}\PY{l+m+mi}{2}\PY{p}{,}\PY{l+m+mi}{3}\PY{p}{)}
\PY{n}{index}\PY{o}{=}\PY{l+m+mi}{0}
\PY{k}{for} \PY{n}{i} \PY{o+ow}{in} \PY{n+nb}{range}\PY{p}{(}\PY{l+m+mi}{2}\PY{p}{)}\PY{p}{:}
    \PY{k}{for} \PY{n}{j} \PY{o+ow}{in} \PY{n+nb}{range}\PY{p}{(}\PY{l+m+mi}{3}\PY{p}{)}\PY{p}{:}
        \PY{n}{ax}\PY{o}{=}\PY{n}{fig}\PY{o}{.}\PY{n}{add\PYZus{}subplot}\PY{p}{(}\PY{n}{gs}\PY{p}{[}\PY{n}{i}\PY{p}{,}\PY{n}{j}\PY{p}{]}\PY{p}{,}\PY{n}{title}\PY{o}{=}\PY{l+s+sa}{f}\PY{l+s+s1}{\PYZsq{}}\PY{l+s+s1}{Grado }\PY{l+s+si}{\PYZob{}}\PY{n}{index}\PY{o}{+}\PY{l+m+mi}{1}\PY{l+s+si}{\PYZcb{}}\PY{l+s+s1}{\PYZsq{}}\PY{p}{,}\PY{n}{xmargin}\PY{o}{=}\PY{l+m+mi}{20}\PY{p}{,}\PY{n}{ymargin}\PY{o}{=}\PY{l+m+mi}{20}\PY{p}{)}
        \PY{n}{\PYZus{}}\PY{o}{=}\PY{n}{list\PYZus{}plots}\PY{p}{[}\PY{n}{index}\PY{p}{]}\PY{o}{.}\PY{n}{\PYZus{}draw\PYZus{}using\PYZus{}figure}\PY{p}{(}\PY{n}{fig}\PY{p}{,}\PY{p}{[}\PY{n}{ax}\PY{p}{]}\PY{p}{)}
        \PY{n}{fig}\PY{o}{.}\PY{n}{tight\PYZus{}layout}\PY{p}{(}\PY{n}{pad}\PY{o}{=}\PY{l+m+mf}{2.0}\PY{p}{)}
        \PY{n}{index}\PY{o}{+}\PY{o}{=}\PY{l+m+mi}{1}
\end{Verbatim}
\end{tcolorbox}

    \begin{center}
    \adjustimage{max size={0.9\linewidth}{0.9\paperheight}}{Tarea_2_Optimizacion_files/Tarea_2_Optimizacion_22_0.png}
    \end{center}
    { \hspace*{\fill} \\}
    
    Aqui observamos un ajuste mucho más pobre incluso para los grados del 4
al 6 a diferencia de la primer base de datos, incluso el
condicionamiento es peor para cada polinomio. El que tiene mejor ajuste
dado los datos es el grado \(n=6\), sin embargo debido al
condicionamiento tan alto podría pensar que la naturaleza de los datos
no es polinomial, de hecho se observa un comportamiento periodico, por
lo que podría pensarse más bien en una función seno o coseno.


    % Add a bibliography block to the postdoc
    
    
    
\end{document}
