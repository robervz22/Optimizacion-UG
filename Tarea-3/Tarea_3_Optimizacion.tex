\documentclass[11pt]{article}

    \usepackage[breakable]{tcolorbox}
    \usepackage{parskip} % Stop auto-indenting (to mimic markdown behaviour)
    
    \usepackage{iftex}
    \ifPDFTeX
    	\usepackage[T1]{fontenc}
    	\usepackage{mathpazo}
    \else
    	\usepackage{fontspec}
    \fi

    % Basic figure setup, for now with no caption control since it's done
    % automatically by Pandoc (which extracts ![](path) syntax from Markdown).
    \usepackage{graphicx}
    % Maintain compatibility with old templates. Remove in nbconvert 6.0
    \let\Oldincludegraphics\includegraphics
    % Ensure that by default, figures have no caption (until we provide a
    % proper Figure object with a Caption API and a way to capture that
    % in the conversion process - todo).
    \usepackage{caption}
    \DeclareCaptionFormat{nocaption}{}
    \captionsetup{format=nocaption,aboveskip=0pt,belowskip=0pt}

    \usepackage{float}
    \floatplacement{figure}{H} % forces figures to be placed at the correct location
    \usepackage{xcolor} % Allow colors to be defined
    \usepackage{enumerate} % Needed for markdown enumerations to work
    \usepackage{geometry} % Used to adjust the document margins
    \usepackage{amsmath} % Equations
    \usepackage{amssymb} % Equations
    \usepackage{textcomp} % defines textquotesingle
    % Hack from http://tex.stackexchange.com/a/47451/13684:
    \AtBeginDocument{%
        \def\PYZsq{\textquotesingle}% Upright quotes in Pygmentized code
    }
    \usepackage{upquote} % Upright quotes for verbatim code
    \usepackage{eurosym} % defines \euro
    \usepackage[mathletters]{ucs} % Extended unicode (utf-8) support
    \usepackage{fancyvrb} % verbatim replacement that allows latex
    \usepackage{grffile} % extends the file name processing of package graphics 
                         % to support a larger range
    \makeatletter % fix for old versions of grffile with XeLaTeX
    \@ifpackagelater{grffile}{2019/11/01}
    {
      % Do nothing on new versions
    }
    {
      \def\Gread@@xetex#1{%
        \IfFileExists{"\Gin@base".bb}%
        {\Gread@eps{\Gin@base.bb}}%
        {\Gread@@xetex@aux#1}%
      }
    }
    \makeatother
    \usepackage[Export]{adjustbox} % Used to constrain images to a maximum size
    \adjustboxset{max size={0.9\linewidth}{0.9\paperheight}}

    % The hyperref package gives us a pdf with properly built
    % internal navigation ('pdf bookmarks' for the table of contents,
    % internal cross-reference links, web links for URLs, etc.)
    \usepackage{hyperref}
    % The default LaTeX title has an obnoxious amount of whitespace. By default,
    % titling removes some of it. It also provides customization options.
    \usepackage{titling}
    \usepackage{longtable} % longtable support required by pandoc >1.10
    \usepackage{booktabs}  % table support for pandoc > 1.12.2
    \usepackage[inline]{enumitem} % IRkernel/repr support (it uses the enumerate* environment)
    \usepackage[normalem]{ulem} % ulem is needed to support strikethroughs (\sout)
                                % normalem makes italics be italics, not underlines
    \usepackage{mathrsfs}
    

    
    % Colors for the hyperref package
    \definecolor{urlcolor}{rgb}{0,.145,.698}
    \definecolor{linkcolor}{rgb}{.71,0.21,0.01}
    \definecolor{citecolor}{rgb}{.12,.54,.11}

    % ANSI colors
    \definecolor{ansi-black}{HTML}{3E424D}
    \definecolor{ansi-black-intense}{HTML}{282C36}
    \definecolor{ansi-red}{HTML}{E75C58}
    \definecolor{ansi-red-intense}{HTML}{B22B31}
    \definecolor{ansi-green}{HTML}{00A250}
    \definecolor{ansi-green-intense}{HTML}{007427}
    \definecolor{ansi-yellow}{HTML}{DDB62B}
    \definecolor{ansi-yellow-intense}{HTML}{B27D12}
    \definecolor{ansi-blue}{HTML}{208FFB}
    \definecolor{ansi-blue-intense}{HTML}{0065CA}
    \definecolor{ansi-magenta}{HTML}{D160C4}
    \definecolor{ansi-magenta-intense}{HTML}{A03196}
    \definecolor{ansi-cyan}{HTML}{60C6C8}
    \definecolor{ansi-cyan-intense}{HTML}{258F8F}
    \definecolor{ansi-white}{HTML}{C5C1B4}
    \definecolor{ansi-white-intense}{HTML}{A1A6B2}
    \definecolor{ansi-default-inverse-fg}{HTML}{FFFFFF}
    \definecolor{ansi-default-inverse-bg}{HTML}{000000}

    % common color for the border for error outputs.
    \definecolor{outerrorbackground}{HTML}{FFDFDF}

    % commands and environments needed by pandoc snippets
    % extracted from the output of `pandoc -s`
    \providecommand{\tightlist}{%
      \setlength{\itemsep}{0pt}\setlength{\parskip}{0pt}}
    \DefineVerbatimEnvironment{Highlighting}{Verbatim}{commandchars=\\\{\}}
    % Add ',fontsize=\small' for more characters per line
    \newenvironment{Shaded}{}{}
    \newcommand{\KeywordTok}[1]{\textcolor[rgb]{0.00,0.44,0.13}{\textbf{{#1}}}}
    \newcommand{\DataTypeTok}[1]{\textcolor[rgb]{0.56,0.13,0.00}{{#1}}}
    \newcommand{\DecValTok}[1]{\textcolor[rgb]{0.25,0.63,0.44}{{#1}}}
    \newcommand{\BaseNTok}[1]{\textcolor[rgb]{0.25,0.63,0.44}{{#1}}}
    \newcommand{\FloatTok}[1]{\textcolor[rgb]{0.25,0.63,0.44}{{#1}}}
    \newcommand{\CharTok}[1]{\textcolor[rgb]{0.25,0.44,0.63}{{#1}}}
    \newcommand{\StringTok}[1]{\textcolor[rgb]{0.25,0.44,0.63}{{#1}}}
    \newcommand{\CommentTok}[1]{\textcolor[rgb]{0.38,0.63,0.69}{\textit{{#1}}}}
    \newcommand{\OtherTok}[1]{\textcolor[rgb]{0.00,0.44,0.13}{{#1}}}
    \newcommand{\AlertTok}[1]{\textcolor[rgb]{1.00,0.00,0.00}{\textbf{{#1}}}}
    \newcommand{\FunctionTok}[1]{\textcolor[rgb]{0.02,0.16,0.49}{{#1}}}
    \newcommand{\RegionMarkerTok}[1]{{#1}}
    \newcommand{\ErrorTok}[1]{\textcolor[rgb]{1.00,0.00,0.00}{\textbf{{#1}}}}
    \newcommand{\NormalTok}[1]{{#1}}
    
    % Additional commands for more recent versions of Pandoc
    \newcommand{\ConstantTok}[1]{\textcolor[rgb]{0.53,0.00,0.00}{{#1}}}
    \newcommand{\SpecialCharTok}[1]{\textcolor[rgb]{0.25,0.44,0.63}{{#1}}}
    \newcommand{\VerbatimStringTok}[1]{\textcolor[rgb]{0.25,0.44,0.63}{{#1}}}
    \newcommand{\SpecialStringTok}[1]{\textcolor[rgb]{0.73,0.40,0.53}{{#1}}}
    \newcommand{\ImportTok}[1]{{#1}}
    \newcommand{\DocumentationTok}[1]{\textcolor[rgb]{0.73,0.13,0.13}{\textit{{#1}}}}
    \newcommand{\AnnotationTok}[1]{\textcolor[rgb]{0.38,0.63,0.69}{\textbf{\textit{{#1}}}}}
    \newcommand{\CommentVarTok}[1]{\textcolor[rgb]{0.38,0.63,0.69}{\textbf{\textit{{#1}}}}}
    \newcommand{\VariableTok}[1]{\textcolor[rgb]{0.10,0.09,0.49}{{#1}}}
    \newcommand{\ControlFlowTok}[1]{\textcolor[rgb]{0.00,0.44,0.13}{\textbf{{#1}}}}
    \newcommand{\OperatorTok}[1]{\textcolor[rgb]{0.40,0.40,0.40}{{#1}}}
    \newcommand{\BuiltInTok}[1]{{#1}}
    \newcommand{\ExtensionTok}[1]{{#1}}
    \newcommand{\PreprocessorTok}[1]{\textcolor[rgb]{0.74,0.48,0.00}{{#1}}}
    \newcommand{\AttributeTok}[1]{\textcolor[rgb]{0.49,0.56,0.16}{{#1}}}
    \newcommand{\InformationTok}[1]{\textcolor[rgb]{0.38,0.63,0.69}{\textbf{\textit{{#1}}}}}
    \newcommand{\WarningTok}[1]{\textcolor[rgb]{0.38,0.63,0.69}{\textbf{\textit{{#1}}}}}
    
    
    % Define a nice break command that doesn't care if a line doesn't already
    % exist.
    \def\br{\hspace*{\fill} \\* }
    % Math Jax compatibility definitions
    \def\gt{>}
    \def\lt{<}
    \let\Oldtex\TeX
    \let\Oldlatex\LaTeX
    \renewcommand{\TeX}{\textrm{\Oldtex}}
    \renewcommand{\LaTeX}{\textrm{\Oldlatex}}
    % Document parameters
    % Document title
    \title{Tarea\_3\_Optimizacion}
    
    
    
    
    
% Pygments definitions
\makeatletter
\def\PY@reset{\let\PY@it=\relax \let\PY@bf=\relax%
    \let\PY@ul=\relax \let\PY@tc=\relax%
    \let\PY@bc=\relax \let\PY@ff=\relax}
\def\PY@tok#1{\csname PY@tok@#1\endcsname}
\def\PY@toks#1+{\ifx\relax#1\empty\else%
    \PY@tok{#1}\expandafter\PY@toks\fi}
\def\PY@do#1{\PY@bc{\PY@tc{\PY@ul{%
    \PY@it{\PY@bf{\PY@ff{#1}}}}}}}
\def\PY#1#2{\PY@reset\PY@toks#1+\relax+\PY@do{#2}}

\@namedef{PY@tok@w}{\def\PY@tc##1{\textcolor[rgb]{0.73,0.73,0.73}{##1}}}
\@namedef{PY@tok@c}{\let\PY@it=\textit\def\PY@tc##1{\textcolor[rgb]{0.25,0.50,0.50}{##1}}}
\@namedef{PY@tok@cp}{\def\PY@tc##1{\textcolor[rgb]{0.74,0.48,0.00}{##1}}}
\@namedef{PY@tok@k}{\let\PY@bf=\textbf\def\PY@tc##1{\textcolor[rgb]{0.00,0.50,0.00}{##1}}}
\@namedef{PY@tok@kp}{\def\PY@tc##1{\textcolor[rgb]{0.00,0.50,0.00}{##1}}}
\@namedef{PY@tok@kt}{\def\PY@tc##1{\textcolor[rgb]{0.69,0.00,0.25}{##1}}}
\@namedef{PY@tok@o}{\def\PY@tc##1{\textcolor[rgb]{0.40,0.40,0.40}{##1}}}
\@namedef{PY@tok@ow}{\let\PY@bf=\textbf\def\PY@tc##1{\textcolor[rgb]{0.67,0.13,1.00}{##1}}}
\@namedef{PY@tok@nb}{\def\PY@tc##1{\textcolor[rgb]{0.00,0.50,0.00}{##1}}}
\@namedef{PY@tok@nf}{\def\PY@tc##1{\textcolor[rgb]{0.00,0.00,1.00}{##1}}}
\@namedef{PY@tok@nc}{\let\PY@bf=\textbf\def\PY@tc##1{\textcolor[rgb]{0.00,0.00,1.00}{##1}}}
\@namedef{PY@tok@nn}{\let\PY@bf=\textbf\def\PY@tc##1{\textcolor[rgb]{0.00,0.00,1.00}{##1}}}
\@namedef{PY@tok@ne}{\let\PY@bf=\textbf\def\PY@tc##1{\textcolor[rgb]{0.82,0.25,0.23}{##1}}}
\@namedef{PY@tok@nv}{\def\PY@tc##1{\textcolor[rgb]{0.10,0.09,0.49}{##1}}}
\@namedef{PY@tok@no}{\def\PY@tc##1{\textcolor[rgb]{0.53,0.00,0.00}{##1}}}
\@namedef{PY@tok@nl}{\def\PY@tc##1{\textcolor[rgb]{0.63,0.63,0.00}{##1}}}
\@namedef{PY@tok@ni}{\let\PY@bf=\textbf\def\PY@tc##1{\textcolor[rgb]{0.60,0.60,0.60}{##1}}}
\@namedef{PY@tok@na}{\def\PY@tc##1{\textcolor[rgb]{0.49,0.56,0.16}{##1}}}
\@namedef{PY@tok@nt}{\let\PY@bf=\textbf\def\PY@tc##1{\textcolor[rgb]{0.00,0.50,0.00}{##1}}}
\@namedef{PY@tok@nd}{\def\PY@tc##1{\textcolor[rgb]{0.67,0.13,1.00}{##1}}}
\@namedef{PY@tok@s}{\def\PY@tc##1{\textcolor[rgb]{0.73,0.13,0.13}{##1}}}
\@namedef{PY@tok@sd}{\let\PY@it=\textit\def\PY@tc##1{\textcolor[rgb]{0.73,0.13,0.13}{##1}}}
\@namedef{PY@tok@si}{\let\PY@bf=\textbf\def\PY@tc##1{\textcolor[rgb]{0.73,0.40,0.53}{##1}}}
\@namedef{PY@tok@se}{\let\PY@bf=\textbf\def\PY@tc##1{\textcolor[rgb]{0.73,0.40,0.13}{##1}}}
\@namedef{PY@tok@sr}{\def\PY@tc##1{\textcolor[rgb]{0.73,0.40,0.53}{##1}}}
\@namedef{PY@tok@ss}{\def\PY@tc##1{\textcolor[rgb]{0.10,0.09,0.49}{##1}}}
\@namedef{PY@tok@sx}{\def\PY@tc##1{\textcolor[rgb]{0.00,0.50,0.00}{##1}}}
\@namedef{PY@tok@m}{\def\PY@tc##1{\textcolor[rgb]{0.40,0.40,0.40}{##1}}}
\@namedef{PY@tok@gh}{\let\PY@bf=\textbf\def\PY@tc##1{\textcolor[rgb]{0.00,0.00,0.50}{##1}}}
\@namedef{PY@tok@gu}{\let\PY@bf=\textbf\def\PY@tc##1{\textcolor[rgb]{0.50,0.00,0.50}{##1}}}
\@namedef{PY@tok@gd}{\def\PY@tc##1{\textcolor[rgb]{0.63,0.00,0.00}{##1}}}
\@namedef{PY@tok@gi}{\def\PY@tc##1{\textcolor[rgb]{0.00,0.63,0.00}{##1}}}
\@namedef{PY@tok@gr}{\def\PY@tc##1{\textcolor[rgb]{1.00,0.00,0.00}{##1}}}
\@namedef{PY@tok@ge}{\let\PY@it=\textit}
\@namedef{PY@tok@gs}{\let\PY@bf=\textbf}
\@namedef{PY@tok@gp}{\let\PY@bf=\textbf\def\PY@tc##1{\textcolor[rgb]{0.00,0.00,0.50}{##1}}}
\@namedef{PY@tok@go}{\def\PY@tc##1{\textcolor[rgb]{0.53,0.53,0.53}{##1}}}
\@namedef{PY@tok@gt}{\def\PY@tc##1{\textcolor[rgb]{0.00,0.27,0.87}{##1}}}
\@namedef{PY@tok@err}{\def\PY@bc##1{{\setlength{\fboxsep}{\string -\fboxrule}\fcolorbox[rgb]{1.00,0.00,0.00}{1,1,1}{\strut ##1}}}}
\@namedef{PY@tok@kc}{\let\PY@bf=\textbf\def\PY@tc##1{\textcolor[rgb]{0.00,0.50,0.00}{##1}}}
\@namedef{PY@tok@kd}{\let\PY@bf=\textbf\def\PY@tc##1{\textcolor[rgb]{0.00,0.50,0.00}{##1}}}
\@namedef{PY@tok@kn}{\let\PY@bf=\textbf\def\PY@tc##1{\textcolor[rgb]{0.00,0.50,0.00}{##1}}}
\@namedef{PY@tok@kr}{\let\PY@bf=\textbf\def\PY@tc##1{\textcolor[rgb]{0.00,0.50,0.00}{##1}}}
\@namedef{PY@tok@bp}{\def\PY@tc##1{\textcolor[rgb]{0.00,0.50,0.00}{##1}}}
\@namedef{PY@tok@fm}{\def\PY@tc##1{\textcolor[rgb]{0.00,0.00,1.00}{##1}}}
\@namedef{PY@tok@vc}{\def\PY@tc##1{\textcolor[rgb]{0.10,0.09,0.49}{##1}}}
\@namedef{PY@tok@vg}{\def\PY@tc##1{\textcolor[rgb]{0.10,0.09,0.49}{##1}}}
\@namedef{PY@tok@vi}{\def\PY@tc##1{\textcolor[rgb]{0.10,0.09,0.49}{##1}}}
\@namedef{PY@tok@vm}{\def\PY@tc##1{\textcolor[rgb]{0.10,0.09,0.49}{##1}}}
\@namedef{PY@tok@sa}{\def\PY@tc##1{\textcolor[rgb]{0.73,0.13,0.13}{##1}}}
\@namedef{PY@tok@sb}{\def\PY@tc##1{\textcolor[rgb]{0.73,0.13,0.13}{##1}}}
\@namedef{PY@tok@sc}{\def\PY@tc##1{\textcolor[rgb]{0.73,0.13,0.13}{##1}}}
\@namedef{PY@tok@dl}{\def\PY@tc##1{\textcolor[rgb]{0.73,0.13,0.13}{##1}}}
\@namedef{PY@tok@s2}{\def\PY@tc##1{\textcolor[rgb]{0.73,0.13,0.13}{##1}}}
\@namedef{PY@tok@sh}{\def\PY@tc##1{\textcolor[rgb]{0.73,0.13,0.13}{##1}}}
\@namedef{PY@tok@s1}{\def\PY@tc##1{\textcolor[rgb]{0.73,0.13,0.13}{##1}}}
\@namedef{PY@tok@mb}{\def\PY@tc##1{\textcolor[rgb]{0.40,0.40,0.40}{##1}}}
\@namedef{PY@tok@mf}{\def\PY@tc##1{\textcolor[rgb]{0.40,0.40,0.40}{##1}}}
\@namedef{PY@tok@mh}{\def\PY@tc##1{\textcolor[rgb]{0.40,0.40,0.40}{##1}}}
\@namedef{PY@tok@mi}{\def\PY@tc##1{\textcolor[rgb]{0.40,0.40,0.40}{##1}}}
\@namedef{PY@tok@il}{\def\PY@tc##1{\textcolor[rgb]{0.40,0.40,0.40}{##1}}}
\@namedef{PY@tok@mo}{\def\PY@tc##1{\textcolor[rgb]{0.40,0.40,0.40}{##1}}}
\@namedef{PY@tok@ch}{\let\PY@it=\textit\def\PY@tc##1{\textcolor[rgb]{0.25,0.50,0.50}{##1}}}
\@namedef{PY@tok@cm}{\let\PY@it=\textit\def\PY@tc##1{\textcolor[rgb]{0.25,0.50,0.50}{##1}}}
\@namedef{PY@tok@cpf}{\let\PY@it=\textit\def\PY@tc##1{\textcolor[rgb]{0.25,0.50,0.50}{##1}}}
\@namedef{PY@tok@c1}{\let\PY@it=\textit\def\PY@tc##1{\textcolor[rgb]{0.25,0.50,0.50}{##1}}}
\@namedef{PY@tok@cs}{\let\PY@it=\textit\def\PY@tc##1{\textcolor[rgb]{0.25,0.50,0.50}{##1}}}

\def\PYZbs{\char`\\}
\def\PYZus{\char`\_}
\def\PYZob{\char`\{}
\def\PYZcb{\char`\}}
\def\PYZca{\char`\^}
\def\PYZam{\char`\&}
\def\PYZlt{\char`\<}
\def\PYZgt{\char`\>}
\def\PYZsh{\char`\#}
\def\PYZpc{\char`\%}
\def\PYZdl{\char`\$}
\def\PYZhy{\char`\-}
\def\PYZsq{\char`\'}
\def\PYZdq{\char`\"}
\def\PYZti{\char`\~}
% for compatibility with earlier versions
\def\PYZat{@}
\def\PYZlb{[}
\def\PYZrb{]}
\makeatother


    % For linebreaks inside Verbatim environment from package fancyvrb. 
    \makeatletter
        \newbox\Wrappedcontinuationbox 
        \newbox\Wrappedvisiblespacebox 
        \newcommand*\Wrappedvisiblespace {\textcolor{red}{\textvisiblespace}} 
        \newcommand*\Wrappedcontinuationsymbol {\textcolor{red}{\llap{\tiny$\m@th\hookrightarrow$}}} 
        \newcommand*\Wrappedcontinuationindent {3ex } 
        \newcommand*\Wrappedafterbreak {\kern\Wrappedcontinuationindent\copy\Wrappedcontinuationbox} 
        % Take advantage of the already applied Pygments mark-up to insert 
        % potential linebreaks for TeX processing. 
        %        {, <, #, %, $, ' and ": go to next line. 
        %        _, }, ^, &, >, - and ~: stay at end of broken line. 
        % Use of \textquotesingle for straight quote. 
        \newcommand*\Wrappedbreaksatspecials {% 
            \def\PYGZus{\discretionary{\char`\_}{\Wrappedafterbreak}{\char`\_}}% 
            \def\PYGZob{\discretionary{}{\Wrappedafterbreak\char`\{}{\char`\{}}% 
            \def\PYGZcb{\discretionary{\char`\}}{\Wrappedafterbreak}{\char`\}}}% 
            \def\PYGZca{\discretionary{\char`\^}{\Wrappedafterbreak}{\char`\^}}% 
            \def\PYGZam{\discretionary{\char`\&}{\Wrappedafterbreak}{\char`\&}}% 
            \def\PYGZlt{\discretionary{}{\Wrappedafterbreak\char`\<}{\char`\<}}% 
            \def\PYGZgt{\discretionary{\char`\>}{\Wrappedafterbreak}{\char`\>}}% 
            \def\PYGZsh{\discretionary{}{\Wrappedafterbreak\char`\#}{\char`\#}}% 
            \def\PYGZpc{\discretionary{}{\Wrappedafterbreak\char`\%}{\char`\%}}% 
            \def\PYGZdl{\discretionary{}{\Wrappedafterbreak\char`\$}{\char`\$}}% 
            \def\PYGZhy{\discretionary{\char`\-}{\Wrappedafterbreak}{\char`\-}}% 
            \def\PYGZsq{\discretionary{}{\Wrappedafterbreak\textquotesingle}{\textquotesingle}}% 
            \def\PYGZdq{\discretionary{}{\Wrappedafterbreak\char`\"}{\char`\"}}% 
            \def\PYGZti{\discretionary{\char`\~}{\Wrappedafterbreak}{\char`\~}}% 
        } 
        % Some characters . , ; ? ! / are not pygmentized. 
        % This macro makes them "active" and they will insert potential linebreaks 
        \newcommand*\Wrappedbreaksatpunct {% 
            \lccode`\~`\.\lowercase{\def~}{\discretionary{\hbox{\char`\.}}{\Wrappedafterbreak}{\hbox{\char`\.}}}% 
            \lccode`\~`\,\lowercase{\def~}{\discretionary{\hbox{\char`\,}}{\Wrappedafterbreak}{\hbox{\char`\,}}}% 
            \lccode`\~`\;\lowercase{\def~}{\discretionary{\hbox{\char`\;}}{\Wrappedafterbreak}{\hbox{\char`\;}}}% 
            \lccode`\~`\:\lowercase{\def~}{\discretionary{\hbox{\char`\:}}{\Wrappedafterbreak}{\hbox{\char`\:}}}% 
            \lccode`\~`\?\lowercase{\def~}{\discretionary{\hbox{\char`\?}}{\Wrappedafterbreak}{\hbox{\char`\?}}}% 
            \lccode`\~`\!\lowercase{\def~}{\discretionary{\hbox{\char`\!}}{\Wrappedafterbreak}{\hbox{\char`\!}}}% 
            \lccode`\~`\/\lowercase{\def~}{\discretionary{\hbox{\char`\/}}{\Wrappedafterbreak}{\hbox{\char`\/}}}% 
            \catcode`\.\active
            \catcode`\,\active 
            \catcode`\;\active
            \catcode`\:\active
            \catcode`\?\active
            \catcode`\!\active
            \catcode`\/\active 
            \lccode`\~`\~ 	
        }
    \makeatother

    \let\OriginalVerbatim=\Verbatim
    \makeatletter
    \renewcommand{\Verbatim}[1][1]{%
        %\parskip\z@skip
        \sbox\Wrappedcontinuationbox {\Wrappedcontinuationsymbol}%
        \sbox\Wrappedvisiblespacebox {\FV@SetupFont\Wrappedvisiblespace}%
        \def\FancyVerbFormatLine ##1{\hsize\linewidth
            \vtop{\raggedright\hyphenpenalty\z@\exhyphenpenalty\z@
                \doublehyphendemerits\z@\finalhyphendemerits\z@
                \strut ##1\strut}%
        }%
        % If the linebreak is at a space, the latter will be displayed as visible
        % space at end of first line, and a continuation symbol starts next line.
        % Stretch/shrink are however usually zero for typewriter font.
        \def\FV@Space {%
            \nobreak\hskip\z@ plus\fontdimen3\font minus\fontdimen4\font
            \discretionary{\copy\Wrappedvisiblespacebox}{\Wrappedafterbreak}
            {\kern\fontdimen2\font}%
        }%
        
        % Allow breaks at special characters using \PYG... macros.
        \Wrappedbreaksatspecials
        % Breaks at punctuation characters . , ; ? ! and / need catcode=\active 	
        \OriginalVerbatim[#1,codes*=\Wrappedbreaksatpunct]%
    }
    \makeatother

    % Exact colors from NB
    \definecolor{incolor}{HTML}{303F9F}
    \definecolor{outcolor}{HTML}{D84315}
    \definecolor{cellborder}{HTML}{CFCFCF}
    \definecolor{cellbackground}{HTML}{F7F7F7}
    
    % prompt
    \makeatletter
    \newcommand{\boxspacing}{\kern\kvtcb@left@rule\kern\kvtcb@boxsep}
    \makeatother
    \newcommand{\prompt}[4]{
        {\ttfamily\llap{{\color{#2}[#3]:\hspace{3pt}#4}}\vspace{-\baselineskip}}
    }
    

    
    % Prevent overflowing lines due to hard-to-break entities
    \sloppy 
    % Setup hyperref package
    \hypersetup{
      breaklinks=true,  % so long urls are correctly broken across lines
      colorlinks=true,
      urlcolor=urlcolor,
      linkcolor=linkcolor,
      citecolor=citecolor,
      }
    % Slightly bigger margins than the latex defaults
    
    \geometry{verbose,tmargin=1in,bmargin=1in,lmargin=1in,rmargin=1in}
    
    

\begin{document}
    \title{Tarea 3 Optimización}
    \author{Roberto Vásquez Martínez \\ Profesor: Joaquín Peña Acevedo}
    \date{27/Febrero/2022}
    \maketitle

    
    \hypertarget{ejercicio-1-4-puntos}{%
\subsection{Ejercicio 1 (4 puntos)}\label{ejercicio-1-4-puntos}}

La función de Rosenbrock se define como

\[ f(x_1, x_2) = 100\left(x_2 − x_1^2 \right)^2 + (1 − x_1)^2. \]

\begin{enumerate}
\def\labelenumi{\arabic{enumi}.}
\tightlist
\item
  Calcule las expresiones del gradiente y la Hessiana de la función de
  Rosenbrock.
\item
  Escriba las funciones en Python que evaluan la función de Rosenbrock,
  su gradiente y Hessiana.
\item
  Muestre que \(x_* = (1,1)^{\top}\) es el único punto estacionario de
  la función.
\item
  Calcule los eigenvalores de la matriz Hessiana de \(f\) en el punto
  \(x_*\) para mostrar que es definida positiva, por lo que \(x_*\)
  corresponde a un mínimo.
\item
  Grafique la función de Rosenbrock en el rectángulo
  \([-1.5, 1.5] \times [-1, 2]\). Use las funciones \texttt{surface()} e
  \texttt{imshow()} para generar la gráfica 3D y la vista 2D.
\end{enumerate}

\hypertarget{soluciuxf3n}{%
\subsubsection{Solución:}\label{soluciuxf3n}}

    El gradiente de \(f\) es
\[\bigtriangledown f(x_1,x_2)=\begin{pmatrix}-400(x_2-x_1^2)x_1-2(1-x_1)&\\ &\\200(x_2-x_1^2)&\end{pmatrix}\]
y el hessiano
\[ H(x_1,x_2)=\begin{pmatrix}1200x_1^2-400x_2+2 & -400x_1\\ -400x_1 & 200\end{pmatrix}.\]

Las definiciones de las expresiones anteriores junto a la definición de
la función de Rosenbrock se encuentran en el módulo \texttt{lib\_t3.py}.
Por lo que con esto tenemos los puntos 1 y 2.

Para el punto 3 debemos probar que \(x_\ast=(1,1)^T\) es el único punto
estacionario de la función de Rosenbrock. Entonces debemos resolver la
ecuación \[ \bigtriangledown f(\mathbf{x})=(0,0)^T,\] donde
\(\mathbf{x}=(x_1,x_2)\in\mathbb{R}^2\).

Igualando a \(0\) cada componente del gradiente tenemos que queremos
hallar \(\mathbf{x}\in\mathbb{R}^2\) tal que \[200(x_2-x_1^2)=0,\] y
\[-400(x_2-x_1^2)x_1-2(1-x_1)=0.\]

De la primer identidad se tiene que cumplir \(x_2=x_1^2\), sustituyendo
en la segunda ecuación obtenemos que se debe satisfacer \[2(1-x_1)=0,\]
lo que pasa sólo si \(x_1=1\), luego \(x_2=1\), por lo que el único
punto estacionario de \(f\) es \(x_\ast=(1,1)^T\), que es lo que
queríamos probar.

A continuación escribimos el código con el que obtenemos los
eigenvalores de la matriz \(H(x_\ast)\).

    \begin{tcolorbox}[breakable, size=fbox, boxrule=1pt, pad at break*=1mm,colback=cellbackground, colframe=cellborder]
\prompt{In}{incolor}{1}{\boxspacing}
\begin{Verbatim}[commandchars=\\\{\}]
\PY{c+c1}{\PYZsh{} Respuesta 1.4.}
\PY{k+kn}{from} \PY{n+nn}{scipy} \PY{k+kn}{import} \PY{n}{linalg}
\PY{k+kn}{import} \PY{n+nn}{lib\PYZus{}t3}
\PY{k+kn}{from} \PY{n+nn}{lib\PYZus{}t3} \PY{k+kn}{import} \PY{o}{*}
\PY{k+kn}{import} \PY{n+nn}{numpy} \PY{k}{as} \PY{n+nn}{np}

\PY{n}{np}\PY{o}{.}\PY{n}{set\PYZus{}printoptions}\PY{p}{(}\PY{n}{precision}\PY{o}{=}\PY{l+m+mi}{4}\PY{p}{)}
\PY{n}{x\PYZus{}ast}\PY{o}{=}\PY{n}{np}\PY{o}{.}\PY{n}{ones}\PY{p}{(}\PY{l+m+mi}{2}\PY{p}{)}
\PY{n}{eigen\PYZus{}val\PYZus{}x\PYZus{}ast}\PY{o}{=}\PY{n}{linalg}\PY{o}{.}\PY{n}{eigvals}\PY{p}{(}\PY{n}{hess\PYZus{}Rosenbrock}\PY{p}{(}\PY{n}{x\PYZus{}ast}\PY{p}{)}\PY{p}{)}
\PY{n+nb}{print}\PY{p}{(}\PY{l+s+s1}{\PYZsq{}}\PY{l+s+s1}{Eigenvalores de H(1,1): }\PY{l+s+s1}{\PYZsq{}}\PY{p}{,}\PY{n}{eigen\PYZus{}val\PYZus{}x\PYZus{}ast}\PY{p}{)}
\end{Verbatim}
\end{tcolorbox}

    \begin{Verbatim}[commandchars=\\\{\}]
Eigenvalores de H(1,1):  [1.0016e+03+0.j 3.9936e-01+0.j]
    \end{Verbatim}

    Como la parte imaginaria es \(0\) tenemos que \(H(x_\ast)\) es una
matriz con valores propios reales y positivos, entonces \(H(x_\ast)\) es
positiva definida, de lo que se concluye \(x_\ast\) corresponde a un
mínimo local.

Finalmente para la parte 5 graficaremos la función de Rosenbrock en el
domino \([-1.5,1.5]\times [-1,2]\). La superficia la graficamos con
ayuda de la función \texttt{plot\_surface} de la librería
\texttt{matplotlib.pyplot}

    \begin{tcolorbox}[breakable, size=fbox, boxrule=1pt, pad at break*=1mm,colback=cellbackground, colframe=cellborder]
\prompt{In}{incolor}{2}{\boxspacing}
\begin{Verbatim}[commandchars=\\\{\}]
\PY{c+c1}{\PYZsh{} Respuesta 1.5.}
\PY{k+kn}{import} \PY{n+nn}{numpy} \PY{k}{as} \PY{n+nn}{np}
\PY{k+kn}{import} \PY{n+nn}{matplotlib}\PY{n+nn}{.}\PY{n+nn}{pyplot} \PY{k}{as} \PY{n+nn}{plt}
\PY{k+kn}{import} \PY{n+nn}{importlib}

\PY{n}{importlib}\PY{o}{.}\PY{n}{reload}\PY{p}{(}\PY{n}{lib\PYZus{}t3}\PY{p}{)}
\PY{c+c1}{\PYZsh{} Setting up input values}
\PY{n}{x} \PY{o}{=} \PY{n}{np}\PY{o}{.}\PY{n}{arange}\PY{p}{(}\PY{o}{\PYZhy{}}\PY{l+m+mf}{1.5}\PY{p}{,} \PY{l+m+mf}{1.5}\PY{p}{,} \PY{l+m+mf}{0.1}\PY{p}{)}
\PY{n}{y} \PY{o}{=} \PY{n}{np}\PY{o}{.}\PY{n}{arange}\PY{p}{(}\PY{o}{\PYZhy{}}\PY{l+m+mf}{1.0}\PY{p}{,} \PY{l+m+mf}{2.0}\PY{p}{,} \PY{l+m+mf}{0.1}\PY{p}{)}
\PY{n}{X}\PY{p}{,} \PY{n}{Y} \PY{o}{=} \PY{n}{np}\PY{o}{.}\PY{n}{meshgrid}\PY{p}{(}\PY{n}{x}\PY{p}{,} \PY{n}{y}\PY{p}{)}
\PY{c+c1}{\PYZsh{} Calculating the output and storing it in the array Z}
\PY{n}{Z} \PY{o}{=} \PY{n}{f\PYZus{}Rosenbrock\PYZus{}graph}\PY{p}{(}\PY{n}{X}\PY{p}{,}\PY{n}{Y}\PY{p}{)}

\PY{n}{fig} \PY{o}{=} \PY{n}{plt}\PY{o}{.}\PY{n}{figure}\PY{p}{(}\PY{n}{figsize}\PY{o}{=}\PY{p}{(}\PY{l+m+mi}{8}\PY{p}{,}\PY{l+m+mi}{6}\PY{p}{)}\PY{p}{)}
\PY{n}{ax} \PY{o}{=} \PY{n}{plt}\PY{o}{.}\PY{n}{axes}\PY{p}{(}\PY{n}{projection}\PY{o}{=}\PY{l+s+s1}{\PYZsq{}}\PY{l+s+s1}{3d}\PY{l+s+s1}{\PYZsq{}}\PY{p}{)}
\PY{n}{ax}\PY{o}{.}\PY{n}{plot\PYZus{}surface}\PY{p}{(}\PY{n}{X}\PY{p}{,} \PY{n}{Y}\PY{p}{,} \PY{n}{Z}\PY{p}{,} \PY{n}{rstride}\PY{o}{=}\PY{l+m+mi}{1}\PY{p}{,} \PY{n}{cstride}\PY{o}{=}\PY{l+m+mi}{1}\PY{p}{,}
\PY{n}{cmap}\PY{o}{=}\PY{l+s+s1}{\PYZsq{}}\PY{l+s+s1}{coolwarm}\PY{l+s+s1}{\PYZsq{}}\PY{p}{,} \PY{n}{edgecolor}\PY{o}{=}\PY{l+s+s1}{\PYZsq{}}\PY{l+s+s1}{none}\PY{l+s+s1}{\PYZsq{}}\PY{p}{)}
\PY{n}{ax}\PY{o}{.}\PY{n}{set\PYZus{}title}\PY{p}{(}\PY{l+s+s2}{\PYZdq{}}\PY{l+s+s2}{Función de Rosenbrock}\PY{l+s+se}{\PYZbs{}n}\PY{l+s+s2}{\PYZdq{}}\PY{o}{+}\PY{l+s+sa}{r}\PY{l+s+s2}{\PYZdq{}}\PY{l+s+s2}{\PYZdl{}f(x\PYZus{}1,x\PYZus{}2)=100}\PY{l+s+s2}{\PYZbs{}}\PY{l+s+s2}{left(x\PYZus{}2 − x\PYZus{}1\PYZca{}2 }\PY{l+s+s2}{\PYZbs{}}\PY{l+s+s2}{right)\PYZca{}2 + (1 − x\PYZus{}1)\PYZca{}2\PYZdl{}}\PY{l+s+s2}{\PYZdq{}}\PY{p}{)}
\PY{n}{ax}\PY{o}{.}\PY{n}{set\PYZus{}xlabel}\PY{p}{(}\PY{l+s+s1}{\PYZsq{}}\PY{l+s+s1}{x1}\PY{l+s+s1}{\PYZsq{}}\PY{p}{)}
\PY{n}{ax}\PY{o}{.}\PY{n}{set\PYZus{}ylabel}\PY{p}{(}\PY{l+s+s1}{\PYZsq{}}\PY{l+s+s1}{x2}\PY{l+s+s1}{\PYZsq{}}\PY{p}{)}
\PY{n}{ax}\PY{o}{.}\PY{n}{view\PYZus{}init}\PY{p}{(}\PY{l+m+mi}{20}\PY{p}{,} \PY{l+m+mi}{45}\PY{p}{)}
\end{Verbatim}
\end{tcolorbox}

    \begin{center}
    \adjustimage{max size={0.9\linewidth}{0.9\paperheight}}{Tarea_3_Optimizacion_files/Tarea_3_Optimizacion_4_0.png}
    \end{center}
    { \hspace*{\fill} \\}
    
    Y la vista 2D la obtenemos con la función \texttt{imshow}

    \begin{tcolorbox}[breakable, size=fbox, boxrule=1pt, pad at break*=1mm,colback=cellbackground, colframe=cellborder]
\prompt{In}{incolor}{3}{\boxspacing}
\begin{Verbatim}[commandchars=\\\{\}]
\PY{c+c1}{\PYZsh{} Tambien se puede crear una vista 2D de la superficie}
\PY{n}{fig} \PY{o}{=} \PY{n}{plt}\PY{o}{.}\PY{n}{figure}\PY{p}{(}\PY{n}{figsize}\PY{o}{=}\PY{p}{(}\PY{l+m+mi}{8}\PY{p}{,}\PY{l+m+mi}{6}\PY{p}{)}\PY{p}{)}
\PY{n}{im} \PY{o}{=} \PY{n}{plt}\PY{o}{.}\PY{n}{imshow}\PY{p}{(}\PY{n}{Z}\PY{p}{,} \PY{n}{cmap}\PY{o}{=}\PY{l+s+s1}{\PYZsq{}}\PY{l+s+s1}{coolwarm}\PY{l+s+s1}{\PYZsq{}}\PY{p}{,} \PY{n}{extent}\PY{o}{=}\PY{p}{(}\PY{o}{\PYZhy{}}\PY{l+m+mf}{1.5}\PY{p}{,} \PY{l+m+mf}{1.5}\PY{p}{,} \PY{o}{\PYZhy{}}\PY{l+m+mf}{1.0}\PY{p}{,} \PY{l+m+mf}{2.0}\PY{p}{)}\PY{p}{,}\PY{n}{interpolation}\PY{o}{=}\PY{l+s+s1}{\PYZsq{}}\PY{l+s+s1}{bilinear}\PY{l+s+s1}{\PYZsq{}}\PY{p}{)}
\PY{n}{plt}\PY{o}{.}\PY{n}{colorbar}\PY{p}{(}\PY{n}{im}\PY{p}{)}
\PY{n}{plt}\PY{o}{.}\PY{n}{title}\PY{p}{(}\PY{l+s+s2}{\PYZdq{}}\PY{l+s+s2}{Función de Rosenbrock}\PY{l+s+se}{\PYZbs{}n}\PY{l+s+s2}{\PYZdq{}}\PY{o}{+}\PY{l+s+sa}{r}\PY{l+s+s2}{\PYZdq{}}\PY{l+s+s2}{\PYZdl{}f(x\PYZus{}1,x\PYZus{}2)=100}\PY{l+s+s2}{\PYZbs{}}\PY{l+s+s2}{left(x\PYZus{}2 − x\PYZus{}1\PYZca{}2 }\PY{l+s+s2}{\PYZbs{}}\PY{l+s+s2}{right)\PYZca{}2 + (1 − x\PYZus{}1)\PYZca{}2\PYZdl{}}\PY{l+s+s2}{\PYZdq{}}\PY{p}{)}
\PY{n}{plt}\PY{o}{.}\PY{n}{xlabel}\PY{p}{(}\PY{l+s+s2}{\PYZdq{}}\PY{l+s+s2}{x1}\PY{l+s+s2}{\PYZdq{}}\PY{p}{)}
\PY{n}{plt}\PY{o}{.}\PY{n}{ylabel}\PY{p}{(}\PY{l+s+s2}{\PYZdq{}}\PY{l+s+s2}{x2}\PY{l+s+s2}{\PYZdq{}}\PY{p}{)}
\end{Verbatim}
\end{tcolorbox}

            \begin{tcolorbox}[breakable, size=fbox, boxrule=.5pt, pad at break*=1mm, opacityfill=0]
\prompt{Out}{outcolor}{3}{\boxspacing}
\begin{Verbatim}[commandchars=\\\{\}]
Text(0, 0.5, 'x2')
\end{Verbatim}
\end{tcolorbox}
        
    \begin{center}
    \adjustimage{max size={0.9\linewidth}{0.9\paperheight}}{Tarea_3_Optimizacion_files/Tarea_3_Optimizacion_6_1.png}
    \end{center}
    { \hspace*{\fill} \\}
    
    Y se observa que el mínimo local de \(f\) es de hecho global pues
\(f(\mathbf{x})\geq 0\) para todo \(\mathbf{x}\in\mathbb{R}^2\) por ser
suma de dos reales al cuadrado y \(f(x_{\ast})=0\)

    \hypertarget{ejercicio-2-3-puntos}{%
\subsection{Ejercicio 2 (3 puntos)}\label{ejercicio-2-3-puntos}}

Programe la función que devuelve una aproximación del gradiente de una
función en un punto particular usando diferencias finitas.

\begin{enumerate}
\def\labelenumi{\arabic{enumi}.}
\tightlist
\item
  La función que calcula la aproximación debe recibir como parámetros
  una función escalar \(f\), el punto \(x\) y el incremento \(h>0\).
\end{enumerate}

\begin{itemize}
\tightlist
\item
  Si \(n\) es el tamaño del arreglo \(x\), cree un arreglo de tamaño
  \(n\) para almacenar las componentes de las aproximaciones del vector
  gradiente. Para aproximar la \(i\)-ésima derivada parcial use
\end{itemize}

\[ \frac{\partial f}{\partial x_i}(x) \approx
\frac{f(x + he_i) - f(x)}{h}, \]

donde \(e_i\) es el \(i\)-ésimo vector canónico.

\begin{enumerate}
\def\labelenumi{\arabic{enumi}.}
\setcounter{enumi}{1}
\tightlist
\item
  Pruebe la función comparando el gradiente analítico de la función de
  Rosenbrock en varios puntos y varios valores del parámetro \(h\):
\end{enumerate}

\begin{itemize}
\item
  Seleccione \(h \in \{0.001, 0.0001, 0.00001 \}\).
\item
  Tome \(x = (-1.5,2) + \lambda (2.5,-1)\) con
  \(\lambda \in \{0, 0.5, 1.0\}\). Imprima el valor \(h\), el punto
  \(x\), el gradiente \(g_{a}(x)\) obtenido con la función analítica
  programada en el Ejercicio 1, el gradiente \(g_{df}(x;h)\) obtenido
  por diferencias finitas y la norma del vector
  \(\|g_{a}(x) - g_{df}(x;h)\|\) (puede elegir la norma que quiera
  usar).
\end{itemize}

\hypertarget{soluciuxf3n}{%
\subsubsection{Solución:}\label{soluciuxf3n}}

    La función que resuelve el numeral 1 es \texttt{diff\_f},que está en el
módulo \texttt{lib\_t3.py}, la cual recibe como argumendo una función
escalar \(f\), el vector \(\mathbf{x}\) donde se calculará la derivada y
el tamaño de paso \(h\) empleado en la aproximación por diferencias
finitas.

Para \(h=0.001\) imprimimos el tamaño de paso \(h\), el valor del
gradiente analítico \(g_a(x)\) iterando sobre \(\lambda\), el valor del
gradiente obtenido por diferencias finitas \(g_{df}(x;h)\) y la norma
euclideana \(\lVert g_a(x)-g_{df}(x;h) \rVert_2\).

    \begin{tcolorbox}[breakable, size=fbox, boxrule=1pt, pad at break*=1mm,colback=cellbackground, colframe=cellborder]
\prompt{In}{incolor}{4}{\boxspacing}
\begin{Verbatim}[commandchars=\\\{\}]
\PY{c+c1}{\PYZsh{} Respuesta 2.1.}
\PY{n}{h}\PY{o}{=}\PY{n}{np}\PY{o}{.}\PY{n}{array}\PY{p}{(}\PY{p}{[}\PY{l+m+mf}{1e\PYZhy{}3}\PY{p}{,}\PY{l+m+mf}{1e\PYZhy{}4}\PY{p}{,}\PY{l+m+mf}{1e\PYZhy{}5}\PY{p}{]}\PY{p}{)}
\PY{n}{x1}\PY{p}{,}\PY{n}{x2}\PY{o}{=}\PY{n}{np}\PY{o}{.}\PY{n}{array}\PY{p}{(}\PY{p}{[}\PY{o}{\PYZhy{}}\PY{l+m+mf}{1.5}\PY{p}{,}\PY{l+m+mf}{2.0}\PY{p}{]}\PY{p}{)}\PY{p}{,}\PY{n}{np}\PY{o}{.}\PY{n}{array}\PY{p}{(}\PY{p}{[}\PY{l+m+mf}{2.5}\PY{p}{,}\PY{o}{\PYZhy{}}\PY{l+m+mf}{1.0}\PY{p}{]}\PY{p}{)}
\PY{n}{lamb}\PY{o}{=}\PY{n}{np}\PY{o}{.}\PY{n}{array}\PY{p}{(}\PY{p}{[}\PY{l+m+mf}{0.0}\PY{p}{,}\PY{l+m+mf}{0.5}\PY{p}{,}\PY{l+m+mf}{1.0}\PY{p}{]}\PY{p}{)}
\PY{n+nb}{print}\PY{p}{(}\PY{l+s+s1}{\PYZsq{}}\PY{l+s+s1}{El tamaño de paso es: }\PY{l+s+s1}{\PYZsq{}}\PY{p}{,}\PY{n}{h}\PY{p}{[}\PY{l+m+mi}{0}\PY{p}{]}\PY{p}{)}
\PY{n+nb}{print}\PY{p}{(}\PY{l+s+s1}{\PYZsq{}}\PY{l+s+se}{\PYZbs{}n}\PY{l+s+s1}{\PYZsq{}}\PY{p}{)}
\PY{k}{for} \PY{n}{l} \PY{o+ow}{in} \PY{n}{lamb}\PY{p}{:}
    \PY{n}{grad\PYZus{}a}\PY{o}{=}\PY{n}{grad\PYZus{}Rosenbrock}\PY{p}{(}\PY{n}{x1}\PY{o}{+}\PY{n}{l}\PY{o}{*}\PY{n}{x2}\PY{p}{)}
    \PY{n}{grad\PYZus{}df}\PY{o}{=}\PY{n}{diff\PYZus{}f}\PY{p}{(}\PY{n}{f\PYZus{}Rosenbrock}\PY{p}{,}\PY{n}{x1}\PY{o}{+}\PY{n}{l}\PY{o}{*}\PY{n}{x2}\PY{p}{,}\PY{n}{h}\PY{p}{[}\PY{l+m+mi}{0}\PY{p}{]}\PY{p}{)}
    \PY{n+nb}{print}\PY{p}{(}\PY{l+s+sa}{f}\PY{l+s+s1}{\PYZsq{}}\PY{l+s+s1}{El valor donde se calcula la derivada es: }\PY{l+s+si}{\PYZob{}}\PY{n}{x1}\PY{o}{+}\PY{n}{l}\PY{o}{*}\PY{n}{x2}\PY{l+s+si}{\PYZcb{}}\PY{l+s+s1}{\PYZsq{}}\PY{p}{)}
    \PY{n+nb}{print}\PY{p}{(}\PY{l+s+sa}{f}\PY{l+s+s1}{\PYZsq{}}\PY{l+s+s1}{El gradiente analítico en }\PY{l+s+si}{\PYZob{}}\PY{n}{x1}\PY{o}{+}\PY{n}{l}\PY{o}{*}\PY{n}{x2}\PY{l+s+si}{\PYZcb{}}\PY{l+s+s1}{ es:}\PY{l+s+se}{\PYZbs{}n}\PY{l+s+s1}{ }\PY{l+s+si}{\PYZob{}}\PY{n}{grad\PYZus{}a}\PY{l+s+si}{\PYZcb{}}\PY{l+s+s1}{\PYZsq{}}\PY{p}{)}
    \PY{n+nb}{print}\PY{p}{(}\PY{l+s+sa}{f}\PY{l+s+s1}{\PYZsq{}}\PY{l+s+s1}{El gradiente por diferencias finitas en }\PY{l+s+si}{\PYZob{}}\PY{n}{x1}\PY{o}{+}\PY{n}{l}\PY{o}{*}\PY{n}{x2}\PY{l+s+si}{\PYZcb{}}\PY{l+s+s1}{ es:}\PY{l+s+se}{\PYZbs{}n}\PY{l+s+s1}{ }\PY{l+s+si}{\PYZob{}}\PY{n}{grad\PYZus{}df}\PY{l+s+si}{\PYZcb{}}\PY{l+s+s1}{\PYZsq{}}\PY{p}{)}
    \PY{n+nb}{print}\PY{p}{(}\PY{l+s+sa}{f}\PY{l+s+s1}{\PYZsq{}}\PY{l+s+s1}{La norma 2 ||g\PYZus{}df\PYZhy{}g\PYZus{}a|| en }\PY{l+s+si}{\PYZob{}}\PY{n}{x1}\PY{o}{+}\PY{n}{l}\PY{o}{*}\PY{n}{x2}\PY{l+s+si}{\PYZcb{}}\PY{l+s+s1}{ es:}\PY{l+s+se}{\PYZbs{}n}\PY{l+s+s1}{ }\PY{l+s+si}{\PYZob{}}\PY{n}{np}\PY{o}{.}\PY{n}{linalg}\PY{o}{.}\PY{n}{norm}\PY{p}{(}\PY{p}{(}\PY{n}{grad\PYZus{}a}\PY{o}{\PYZhy{}}\PY{n}{grad\PYZus{}df}\PY{p}{)}\PY{p}{)}\PY{l+s+si}{\PYZcb{}}\PY{l+s+s1}{\PYZsq{}}\PY{p}{)}
    \PY{n+nb}{print}\PY{p}{(}\PY{l+s+s1}{\PYZsq{}}\PY{l+s+se}{\PYZbs{}n}\PY{l+s+se}{\PYZbs{}n}\PY{l+s+s1}{\PYZsq{}}\PY{p}{)}
\end{Verbatim}
\end{tcolorbox}

    \begin{Verbatim}[commandchars=\\\{\}]
El tamaño de paso es:  0.001


El valor donde se calcula la derivada es: [-1.5  2. ]
El gradiente analítico en [-1.5  2. ] es:
 [[-155.]
 [ -50.]]
El gradiente por diferencias finitas en [-1.5  2. ] es:
 [[-154.0496]
 [ -49.9   ]]
La norma 2 ||g\_df-g\_a|| en [-1.5  2. ] es:
 0.9556465612998835



El valor donde se calcula la derivada es: [-0.25  1.5 ]
El gradiente analítico en [-0.25  1.5 ] es:
 [[141.25]
 [287.5 ]]
El gradiente por diferencias finitas en [-0.25  1.5 ] es:
 [[140.9884]
 [287.6   ]]
La norma 2 ||g\_df-g\_a|| en [-0.25  1.5 ] es:
 0.2800616140542944



El valor donde se calcula la derivada es: [1. 1.]
El gradiente analítico en [1. 1.] es:
 [[0.]
 [0.]]
El gradiente por diferencias finitas en [1. 1.] es:
 [[0.4014]
 [0.1   ]]
La norma 2 ||g\_df-g\_a|| en [1. 1.] es:
 0.4136689984515496



    \end{Verbatim}

    Con \(h=0.001\) obtenemos un error del orden de \(10^{-1}\) para cada
valor donde obtuvimos el gradiente. Repitiendo el experimento pero ahora
con tamaño de paso \(h=0.0001\) obtenemos los siguientes resultados

    \begin{tcolorbox}[breakable, size=fbox, boxrule=1pt, pad at break*=1mm,colback=cellbackground, colframe=cellborder]
\prompt{In}{incolor}{5}{\boxspacing}
\begin{Verbatim}[commandchars=\\\{\}]
\PY{n+nb}{print}\PY{p}{(}\PY{l+s+s1}{\PYZsq{}}\PY{l+s+s1}{El tamaño de paso es: }\PY{l+s+s1}{\PYZsq{}}\PY{p}{,}\PY{n}{h}\PY{p}{[}\PY{l+m+mi}{1}\PY{p}{]}\PY{p}{)}
\PY{n+nb}{print}\PY{p}{(}\PY{l+s+s1}{\PYZsq{}}\PY{l+s+se}{\PYZbs{}n}\PY{l+s+s1}{\PYZsq{}}\PY{p}{)}
\PY{k}{for} \PY{n}{l} \PY{o+ow}{in} \PY{n}{lamb}\PY{p}{:}
    \PY{n}{grad\PYZus{}a}\PY{o}{=}\PY{n}{grad\PYZus{}Rosenbrock}\PY{p}{(}\PY{n}{x1}\PY{o}{+}\PY{n}{l}\PY{o}{*}\PY{n}{x2}\PY{p}{)}
    \PY{n}{grad\PYZus{}df}\PY{o}{=}\PY{n}{diff\PYZus{}f}\PY{p}{(}\PY{n}{f\PYZus{}Rosenbrock}\PY{p}{,}\PY{n}{x1}\PY{o}{+}\PY{n}{l}\PY{o}{*}\PY{n}{x2}\PY{p}{,}\PY{n}{h}\PY{p}{[}\PY{l+m+mi}{1}\PY{p}{]}\PY{p}{)}
    \PY{n+nb}{print}\PY{p}{(}\PY{l+s+sa}{f}\PY{l+s+s1}{\PYZsq{}}\PY{l+s+s1}{El valor donde se calcula la derivada es: }\PY{l+s+si}{\PYZob{}}\PY{n}{x1}\PY{o}{+}\PY{n}{l}\PY{o}{*}\PY{n}{x2}\PY{l+s+si}{\PYZcb{}}\PY{l+s+s1}{\PYZsq{}}\PY{p}{)}
    \PY{n+nb}{print}\PY{p}{(}\PY{l+s+sa}{f}\PY{l+s+s1}{\PYZsq{}}\PY{l+s+s1}{El gradiente analítico en }\PY{l+s+si}{\PYZob{}}\PY{n}{x1}\PY{o}{+}\PY{n}{l}\PY{o}{*}\PY{n}{x2}\PY{l+s+si}{\PYZcb{}}\PY{l+s+s1}{ es:}\PY{l+s+se}{\PYZbs{}n}\PY{l+s+s1}{ }\PY{l+s+si}{\PYZob{}}\PY{n}{grad\PYZus{}a}\PY{l+s+si}{\PYZcb{}}\PY{l+s+s1}{\PYZsq{}}\PY{p}{)}
    \PY{n+nb}{print}\PY{p}{(}\PY{l+s+sa}{f}\PY{l+s+s1}{\PYZsq{}}\PY{l+s+s1}{El gradiente por diferencias finitas en }\PY{l+s+si}{\PYZob{}}\PY{n}{x1}\PY{o}{+}\PY{n}{l}\PY{o}{*}\PY{n}{x2}\PY{l+s+si}{\PYZcb{}}\PY{l+s+s1}{ es:}\PY{l+s+se}{\PYZbs{}n}\PY{l+s+s1}{ }\PY{l+s+si}{\PYZob{}}\PY{n}{grad\PYZus{}df}\PY{l+s+si}{\PYZcb{}}\PY{l+s+s1}{\PYZsq{}}\PY{p}{)}
    \PY{n+nb}{print}\PY{p}{(}\PY{l+s+sa}{f}\PY{l+s+s1}{\PYZsq{}}\PY{l+s+s1}{La norma 2 ||g\PYZus{}df\PYZhy{}g\PYZus{}a|| en }\PY{l+s+si}{\PYZob{}}\PY{n}{x1}\PY{o}{+}\PY{n}{l}\PY{o}{*}\PY{n}{x2}\PY{l+s+si}{\PYZcb{}}\PY{l+s+s1}{ es:}\PY{l+s+se}{\PYZbs{}n}\PY{l+s+s1}{ }\PY{l+s+si}{\PYZob{}}\PY{n}{np}\PY{o}{.}\PY{n}{linalg}\PY{o}{.}\PY{n}{norm}\PY{p}{(}\PY{p}{(}\PY{n}{grad\PYZus{}a}\PY{o}{\PYZhy{}}\PY{n}{grad\PYZus{}df}\PY{p}{)}\PY{p}{)}\PY{l+s+si}{\PYZcb{}}\PY{l+s+s1}{\PYZsq{}}\PY{p}{)}
    \PY{n+nb}{print}\PY{p}{(}\PY{l+s+s1}{\PYZsq{}}\PY{l+s+se}{\PYZbs{}n}\PY{l+s+se}{\PYZbs{}n}\PY{l+s+s1}{\PYZsq{}}\PY{p}{)}
\end{Verbatim}
\end{tcolorbox}

    \begin{Verbatim}[commandchars=\\\{\}]
El tamaño de paso es:  0.0001


El valor donde se calcula la derivada es: [-1.5  2. ]
El gradiente analítico en [-1.5  2. ] es:
 [[-155.]
 [ -50.]]
El gradiente por diferencias finitas en [-1.5  2. ] es:
 [[-154.9049]
 [ -49.99  ]]
La norma 2 ||g\_df-g\_a|| en [-1.5  2. ] es:
 0.09561835007987253



El valor donde se calcula la derivada es: [-0.25  1.5 ]
El gradiente analítico en [-0.25  1.5 ] es:
 [[141.25]
 [287.5 ]]
El gradiente por diferencias finitas en [-0.25  1.5 ] es:
 [[141.2238]
 [287.51  ]]
La norma 2 ||g\_df-g\_a|| en [-0.25  1.5 ] es:
 0.027997764395229302



El valor donde se calcula la derivada es: [1. 1.]
El gradiente analítico en [1. 1.] es:
 [[0.]
 [0.]]
El gradiente por diferencias finitas en [1. 1.] es:
 [[0.0401]
 [0.01  ]]
La norma 2 ||g\_df-g\_a|| en [1. 1.] es:
 0.04133195886983942



    \end{Verbatim}

    Aquí con \(h=0.0001\) obtuvimos diferencia entre el gradiente analítico
y numérico del orden de \(10^{-2}\). Finalmente, considerando
\(h=0.00001\) obtenemos

    \begin{tcolorbox}[breakable, size=fbox, boxrule=1pt, pad at break*=1mm,colback=cellbackground, colframe=cellborder]
\prompt{In}{incolor}{6}{\boxspacing}
\begin{Verbatim}[commandchars=\\\{\}]
\PY{n+nb}{print}\PY{p}{(}\PY{l+s+s1}{\PYZsq{}}\PY{l+s+s1}{El tamaño de paso es: }\PY{l+s+s1}{\PYZsq{}}\PY{p}{,}\PY{n}{h}\PY{p}{[}\PY{l+m+mi}{2}\PY{p}{]}\PY{p}{)}
\PY{n+nb}{print}\PY{p}{(}\PY{l+s+s1}{\PYZsq{}}\PY{l+s+se}{\PYZbs{}n}\PY{l+s+s1}{\PYZsq{}}\PY{p}{)}
\PY{k}{for} \PY{n}{l} \PY{o+ow}{in} \PY{n}{lamb}\PY{p}{:}
    \PY{n}{grad\PYZus{}a}\PY{o}{=}\PY{n}{grad\PYZus{}Rosenbrock}\PY{p}{(}\PY{n}{x1}\PY{o}{+}\PY{n}{l}\PY{o}{*}\PY{n}{x2}\PY{p}{)}
    \PY{n}{grad\PYZus{}df}\PY{o}{=}\PY{n}{diff\PYZus{}f}\PY{p}{(}\PY{n}{f\PYZus{}Rosenbrock}\PY{p}{,}\PY{n}{x1}\PY{o}{+}\PY{n}{l}\PY{o}{*}\PY{n}{x2}\PY{p}{,}\PY{n}{h}\PY{p}{[}\PY{l+m+mi}{2}\PY{p}{]}\PY{p}{)}
    \PY{n+nb}{print}\PY{p}{(}\PY{l+s+sa}{f}\PY{l+s+s1}{\PYZsq{}}\PY{l+s+s1}{El valor donde se calcula la derivada es: }\PY{l+s+si}{\PYZob{}}\PY{n}{x1}\PY{o}{+}\PY{n}{l}\PY{o}{*}\PY{n}{x2}\PY{l+s+si}{\PYZcb{}}\PY{l+s+s1}{\PYZsq{}}\PY{p}{)}
    \PY{n+nb}{print}\PY{p}{(}\PY{l+s+sa}{f}\PY{l+s+s1}{\PYZsq{}}\PY{l+s+s1}{El gradiente analítico en }\PY{l+s+si}{\PYZob{}}\PY{n}{x1}\PY{o}{+}\PY{n}{l}\PY{o}{*}\PY{n}{x2}\PY{l+s+si}{\PYZcb{}}\PY{l+s+s1}{ es:}\PY{l+s+se}{\PYZbs{}n}\PY{l+s+s1}{ }\PY{l+s+si}{\PYZob{}}\PY{n}{grad\PYZus{}a}\PY{l+s+si}{\PYZcb{}}\PY{l+s+s1}{\PYZsq{}}\PY{p}{)}
    \PY{n+nb}{print}\PY{p}{(}\PY{l+s+sa}{f}\PY{l+s+s1}{\PYZsq{}}\PY{l+s+s1}{El gradiente por diferencias finitas en }\PY{l+s+si}{\PYZob{}}\PY{n}{x1}\PY{o}{+}\PY{n}{l}\PY{o}{*}\PY{n}{x2}\PY{l+s+si}{\PYZcb{}}\PY{l+s+s1}{ es:}\PY{l+s+se}{\PYZbs{}n}\PY{l+s+s1}{ }\PY{l+s+si}{\PYZob{}}\PY{n}{grad\PYZus{}df}\PY{l+s+si}{\PYZcb{}}\PY{l+s+s1}{\PYZsq{}}\PY{p}{)}
    \PY{n+nb}{print}\PY{p}{(}\PY{l+s+sa}{f}\PY{l+s+s1}{\PYZsq{}}\PY{l+s+s1}{La norma 2 ||g\PYZus{}df\PYZhy{}g\PYZus{}a|| en }\PY{l+s+si}{\PYZob{}}\PY{n}{x1}\PY{o}{+}\PY{n}{l}\PY{o}{*}\PY{n}{x2}\PY{l+s+si}{\PYZcb{}}\PY{l+s+s1}{ es:}\PY{l+s+se}{\PYZbs{}n}\PY{l+s+s1}{ }\PY{l+s+si}{\PYZob{}}\PY{n}{np}\PY{o}{.}\PY{n}{linalg}\PY{o}{.}\PY{n}{norm}\PY{p}{(}\PY{n}{grad\PYZus{}a}\PY{o}{\PYZhy{}}\PY{n}{grad\PYZus{}df}\PY{p}{)}\PY{l+s+si}{\PYZcb{}}\PY{l+s+s1}{\PYZsq{}}\PY{p}{)}
    \PY{n+nb}{print}\PY{p}{(}\PY{l+s+s1}{\PYZsq{}}\PY{l+s+se}{\PYZbs{}n}\PY{l+s+se}{\PYZbs{}n}\PY{l+s+s1}{\PYZsq{}}\PY{p}{)}
\end{Verbatim}
\end{tcolorbox}

    \begin{Verbatim}[commandchars=\\\{\}]
El tamaño de paso es:  1e-05


El valor donde se calcula la derivada es: [-1.5  2. ]
El gradiente analítico en [-1.5  2. ] es:
 [[-155.]
 [ -50.]]
El gradiente por diferencias finitas en [-1.5  2. ] es:
 [[-154.9905]
 [ -49.999 ]]
La norma 2 ||g\_df-g\_a|| en [-1.5  2. ] es:
 0.00956237097442853



El valor donde se calcula la derivada es: [-0.25  1.5 ]
El gradiente analítico en [-0.25  1.5 ] es:
 [[141.25]
 [287.5 ]]
El gradiente por diferencias finitas en [-0.25  1.5 ] es:
 [[141.2474]
 [287.501 ]]
La norma 2 ||g\_df-g\_a|| en [-0.25  1.5 ] es:
 0.0027996949222215094



El valor donde se calcula la derivada es: [1. 1.]
El gradiente analítico en [1. 1.] es:
 [[0.]
 [0.]]
El gradiente por diferencias finitas en [1. 1.] es:
 [[0.004]
 [0.001]]
La norma 2 ||g\_df-g\_a|| en [1. 1.] es:
 0.004132846573836989



    \end{Verbatim}

    Y en este caso obtuvimos una diferencia entre los gradientes del orden
de \(10^{-3}\) con respecto a la norma euclideana.

    \hypertarget{ejercicio-3-3-puntos}{%
\subsection{Ejercicio 3 (3 puntos)}\label{ejercicio-3-3-puntos}}

Programe la función que devuelve una aproximación de la Hessiana de una
función en un punto particular usando diferencias finitas.

\begin{enumerate}
\def\labelenumi{\arabic{enumi}.}
\tightlist
\item
  La función que calcula la aproximación debe recibir como parámetros
  una función escalar \(f\), el punto \(x\) y el incremento \(h>0\).
\end{enumerate}

\begin{itemize}
\tightlist
\item
  Si \(n\) es el tamaño del arreglo \(x\), cree una matriz de tamaño
  \(n \times n\) para almacenar las entradas de las aproximaciones de
  las segundas derivadas parciales. Puede usar
\end{itemize}

\[ \frac{\partial^2 f}{\partial x_i \partial x_j}(x) \approx
\frac{f(x + he_i+he_j) - f(x+he_i)- f(x+he_j) + f(x)}{h^2}, \]

donde \(e_i\) es el \(i\)-ésimo vector canónico.

\begin{enumerate}
\def\labelenumi{\arabic{enumi}.}
\setcounter{enumi}{1}
\tightlist
\item
  Pruebe la función comparando el gradiente analítico de la función de
  Rosenbrock en varios puntos y varios valores del parámetro \(h\):
\end{enumerate}

\begin{itemize}
\tightlist
\item
  Seleccione \(h \in \{0.001, 0.0001, 0.00001 \}\)
\item
  Tome \(x = (-1.5,2) + \lambda (2.5,-1)\) con
  \(\lambda \in \{0, 0.5, 1.0\}\) Imprima el valor \(h\), el punto
  \(x\), la Hessiana \(H_{a}(x)\) obtenido con la función analítica
  programada en el Ejercicio 1, la Hessiana \(H_{df}(x;h)\) obtenido por
  diferencias finitas y la norma de la matriz
  \(\|H_{a}(x) - H_{df}(x;h)\|\) (puede elegir la norma que quiera
  usar).
\end{itemize}

\hypertarget{soluciuxf3n}{%
\subsubsection{Solución:}\label{soluciuxf3n}}

    La función que resuelve el numeral 1 es \texttt{hess\_f}, que está en el
módulo \texttt{lib\_t3.py}, la cual recibe como argumendo una función
escalar \(f\), el vector \(\mathbf{x}\) donde se calculará la derivada y
el tamaño de paso \(h\) empleado en la aproximación por diferencias
finitas.

Para \(h=0.001\) imprimimos el tamaño de paso \(h\), el valor del
hessiano analítico \(H_a(x)\) iterando sobre \(\lambda\), el valor del
hessiano obtenido por diferencias finitas \(H_{df}(x;h)\) y la norma de
Fröbenius \(\lVert H_a(x)-H_{df}(x;h) \rVert_F\).

    \begin{tcolorbox}[breakable, size=fbox, boxrule=1pt, pad at break*=1mm,colback=cellbackground, colframe=cellborder]
\prompt{In}{incolor}{7}{\boxspacing}
\begin{Verbatim}[commandchars=\\\{\}]
\PY{c+c1}{\PYZsh{} Respuesta 3.1.}
\PY{n+nb}{print}\PY{p}{(}\PY{l+s+s1}{\PYZsq{}}\PY{l+s+s1}{El tamaño de paso es: }\PY{l+s+s1}{\PYZsq{}}\PY{p}{,}\PY{n}{h}\PY{p}{[}\PY{l+m+mi}{0}\PY{p}{]}\PY{p}{)}
\PY{n+nb}{print}\PY{p}{(}\PY{l+s+s1}{\PYZsq{}}\PY{l+s+se}{\PYZbs{}n}\PY{l+s+s1}{\PYZsq{}}\PY{p}{)}
\PY{k}{for} \PY{n}{l} \PY{o+ow}{in} \PY{n}{lamb}\PY{p}{:}
    \PY{n}{hess\PYZus{}a}\PY{o}{=}\PY{n}{hess\PYZus{}Rosenbrock}\PY{p}{(}\PY{n}{x1}\PY{o}{+}\PY{n}{l}\PY{o}{*}\PY{n}{x2}\PY{p}{)}
    \PY{n}{hess\PYZus{}df}\PY{o}{=}\PY{n}{hess\PYZus{}f}\PY{p}{(}\PY{n}{f\PYZus{}Rosenbrock}\PY{p}{,}\PY{n}{x1}\PY{o}{+}\PY{n}{l}\PY{o}{*}\PY{n}{x2}\PY{p}{,}\PY{n}{h}\PY{p}{[}\PY{l+m+mi}{0}\PY{p}{]}\PY{p}{)}
    \PY{n}{norm\PYZus{}hess}\PY{o}{=}\PY{n}{np}\PY{o}{.}\PY{n}{linalg}\PY{o}{.}\PY{n}{norm}\PY{p}{(}\PY{p}{(}\PY{n}{hess\PYZus{}a}\PY{o}{\PYZhy{}}\PY{n}{hess\PYZus{}df}\PY{p}{)}\PY{p}{,}\PY{n+nb}{ord}\PY{o}{=}\PY{l+s+s1}{\PYZsq{}}\PY{l+s+s1}{fro}\PY{l+s+s1}{\PYZsq{}}\PY{p}{)}
    \PY{n+nb}{print}\PY{p}{(}\PY{l+s+sa}{f}\PY{l+s+s1}{\PYZsq{}}\PY{l+s+s1}{El valor donde se calcula el hessiano es: }\PY{l+s+si}{\PYZob{}}\PY{n}{x1}\PY{o}{+}\PY{n}{l}\PY{o}{*}\PY{n}{x2}\PY{l+s+si}{\PYZcb{}}\PY{l+s+s1}{\PYZsq{}}\PY{p}{)}
    \PY{n+nb}{print}\PY{p}{(}\PY{l+s+sa}{f}\PY{l+s+s1}{\PYZsq{}}\PY{l+s+s1}{El hessiano analítico en }\PY{l+s+si}{\PYZob{}}\PY{n}{x1}\PY{o}{+}\PY{n}{l}\PY{o}{*}\PY{n}{x2}\PY{l+s+si}{\PYZcb{}}\PY{l+s+s1}{ es:}\PY{l+s+se}{\PYZbs{}n}\PY{l+s+s1}{ }\PY{l+s+si}{\PYZob{}}\PY{n}{hess\PYZus{}a}\PY{l+s+si}{\PYZcb{}}\PY{l+s+s1}{\PYZsq{}}\PY{p}{)}
    \PY{n+nb}{print}\PY{p}{(}\PY{l+s+sa}{f}\PY{l+s+s1}{\PYZsq{}}\PY{l+s+s1}{El hessiano por diferencias finitas en }\PY{l+s+si}{\PYZob{}}\PY{n}{x1}\PY{o}{+}\PY{n}{l}\PY{o}{*}\PY{n}{x2}\PY{l+s+si}{\PYZcb{}}\PY{l+s+s1}{ es:}\PY{l+s+se}{\PYZbs{}n}\PY{l+s+s1}{ }\PY{l+s+si}{\PYZob{}}\PY{n}{hess\PYZus{}df}\PY{l+s+si}{\PYZcb{}}\PY{l+s+s1}{\PYZsq{}}\PY{p}{)}
    \PY{n+nb}{print}\PY{p}{(}\PY{l+s+sa}{f}\PY{l+s+s1}{\PYZsq{}}\PY{l+s+s1}{La norma de Frobenius ||H\PYZus{}df\PYZhy{}H\PYZus{}a|| en }\PY{l+s+si}{\PYZob{}}\PY{n}{x1}\PY{o}{+}\PY{n}{l}\PY{o}{*}\PY{n}{x2}\PY{l+s+si}{\PYZcb{}}\PY{l+s+s1}{ es:}\PY{l+s+se}{\PYZbs{}n}\PY{l+s+s1}{ }\PY{l+s+si}{\PYZob{}}\PY{n}{norm\PYZus{}hess}\PY{l+s+si}{\PYZcb{}}\PY{l+s+s1}{\PYZsq{}}\PY{p}{)}
    \PY{n+nb}{print}\PY{p}{(}\PY{l+s+s1}{\PYZsq{}}\PY{l+s+se}{\PYZbs{}n}\PY{l+s+se}{\PYZbs{}n}\PY{l+s+s1}{\PYZsq{}}\PY{p}{)}
\end{Verbatim}
\end{tcolorbox}

    \begin{Verbatim}[commandchars=\\\{\}]
El tamaño de paso es:  0.001


El valor donde se calcula el hessiano es: [-1.5  2. ]
El hessiano analítico en [-1.5  2. ] es:
 [[1902.  600.]
 [ 600.  200.]]
El hessiano por diferencias finitas en [-1.5  2. ] es:
 [[1898.4014  599.8   ]
 [ 599.8     200.    ]]
La norma de Frobenius ||H\_df-H\_a|| en [-1.5  2. ] es:
 3.6096983250551644



El valor donde se calcula el hessiano es: [-0.25  1.5 ]
El hessiano analítico en [-0.25  1.5 ] es:
 [[-523.  100.]
 [ 100.  200.]]
El hessiano por diferencias finitas en [-0.25  1.5 ] es:
 [[-523.5986   99.8   ]
 [  99.8     200.    ]]
La norma de Frobenius ||H\_df-H\_a|| en [-0.25  1.5 ] es:
 0.662058898046224



El valor donde se calcula el hessiano es: [1. 1.]
El hessiano analítico en [1. 1.] es:
 [[ 802. -400.]
 [-400.  200.]]
El hessiano por diferencias finitas en [1. 1.] es:
 [[ 804.4014 -400.2   ]
 [-400.2     200.    ]]
La norma de Frobenius ||H\_df-H\_a|| en [1. 1.] es:
 2.417999577950844



    \end{Verbatim}

    Respecto a las norma de Frobenius la diferencia entre el hessiano
analítico y el hessiano obtenido por diferencias finitas ronda entre los
ordenes \(10^{-1}\) y \(1\) par aun tamaño de paso \(h=0.001\).

Considerando \(h=0.0001\) obtenemos lo siguiente

    \begin{tcolorbox}[breakable, size=fbox, boxrule=1pt, pad at break*=1mm,colback=cellbackground, colframe=cellborder]
\prompt{In}{incolor}{9}{\boxspacing}
\begin{Verbatim}[commandchars=\\\{\}]
\PY{n+nb}{print}\PY{p}{(}\PY{l+s+s1}{\PYZsq{}}\PY{l+s+s1}{El tamaño de paso es: }\PY{l+s+s1}{\PYZsq{}}\PY{p}{,}\PY{n}{h}\PY{p}{[}\PY{l+m+mi}{1}\PY{p}{]}\PY{p}{)}
\PY{n+nb}{print}\PY{p}{(}\PY{l+s+s1}{\PYZsq{}}\PY{l+s+se}{\PYZbs{}n}\PY{l+s+s1}{\PYZsq{}}\PY{p}{)}
\PY{k}{for} \PY{n}{l} \PY{o+ow}{in} \PY{n}{lamb}\PY{p}{:}
    \PY{n}{hess\PYZus{}a}\PY{o}{=}\PY{n}{hess\PYZus{}Rosenbrock}\PY{p}{(}\PY{n}{x1}\PY{o}{+}\PY{n}{l}\PY{o}{*}\PY{n}{x2}\PY{p}{)}
    \PY{n}{hess\PYZus{}df}\PY{o}{=}\PY{n}{hess\PYZus{}f}\PY{p}{(}\PY{n}{f\PYZus{}Rosenbrock}\PY{p}{,}\PY{n}{x1}\PY{o}{+}\PY{n}{l}\PY{o}{*}\PY{n}{x2}\PY{p}{,}\PY{n}{h}\PY{p}{[}\PY{l+m+mi}{1}\PY{p}{]}\PY{p}{)}
    \PY{n}{norm\PYZus{}hess}\PY{o}{=}\PY{n}{np}\PY{o}{.}\PY{n}{linalg}\PY{o}{.}\PY{n}{norm}\PY{p}{(}\PY{p}{(}\PY{n}{hess\PYZus{}a}\PY{o}{\PYZhy{}}\PY{n}{hess\PYZus{}df}\PY{p}{)}\PY{p}{,}\PY{n+nb}{ord}\PY{o}{=}\PY{l+s+s1}{\PYZsq{}}\PY{l+s+s1}{fro}\PY{l+s+s1}{\PYZsq{}}\PY{p}{)}
    \PY{n+nb}{print}\PY{p}{(}\PY{l+s+sa}{f}\PY{l+s+s1}{\PYZsq{}}\PY{l+s+s1}{El valor donde se calcula el hessiano es: }\PY{l+s+si}{\PYZob{}}\PY{n}{x1}\PY{o}{+}\PY{n}{l}\PY{o}{*}\PY{n}{x2}\PY{l+s+si}{\PYZcb{}}\PY{l+s+s1}{\PYZsq{}}\PY{p}{)}
    \PY{n+nb}{print}\PY{p}{(}\PY{l+s+sa}{f}\PY{l+s+s1}{\PYZsq{}}\PY{l+s+s1}{El hessiano analítico en }\PY{l+s+si}{\PYZob{}}\PY{n}{x1}\PY{o}{+}\PY{n}{l}\PY{o}{*}\PY{n}{x2}\PY{l+s+si}{\PYZcb{}}\PY{l+s+s1}{ es:}\PY{l+s+se}{\PYZbs{}n}\PY{l+s+s1}{ }\PY{l+s+si}{\PYZob{}}\PY{n}{hess\PYZus{}a}\PY{l+s+si}{\PYZcb{}}\PY{l+s+s1}{\PYZsq{}}\PY{p}{)}
    \PY{n+nb}{print}\PY{p}{(}\PY{l+s+sa}{f}\PY{l+s+s1}{\PYZsq{}}\PY{l+s+s1}{El hessiano por diferencias finitas en }\PY{l+s+si}{\PYZob{}}\PY{n}{x1}\PY{o}{+}\PY{n}{l}\PY{o}{*}\PY{n}{x2}\PY{l+s+si}{\PYZcb{}}\PY{l+s+s1}{ es:}\PY{l+s+se}{\PYZbs{}n}\PY{l+s+s1}{ }\PY{l+s+si}{\PYZob{}}\PY{n}{hess\PYZus{}df}\PY{l+s+si}{\PYZcb{}}\PY{l+s+s1}{\PYZsq{}}\PY{p}{)}
    \PY{n+nb}{print}\PY{p}{(}\PY{l+s+sa}{f}\PY{l+s+s1}{\PYZsq{}}\PY{l+s+s1}{La norma de Frobenius ||H\PYZus{}df\PYZhy{}H\PYZus{}a|| en }\PY{l+s+si}{\PYZob{}}\PY{n}{x1}\PY{o}{+}\PY{n}{l}\PY{o}{*}\PY{n}{x2}\PY{l+s+si}{\PYZcb{}}\PY{l+s+s1}{ es:}\PY{l+s+se}{\PYZbs{}n}\PY{l+s+s1}{ }\PY{l+s+si}{\PYZob{}}\PY{n}{norm\PYZus{}hess}\PY{l+s+si}{\PYZcb{}}\PY{l+s+s1}{\PYZsq{}}\PY{p}{)}
    \PY{n+nb}{print}\PY{p}{(}\PY{l+s+s1}{\PYZsq{}}\PY{l+s+se}{\PYZbs{}n}\PY{l+s+se}{\PYZbs{}n}\PY{l+s+s1}{\PYZsq{}}\PY{p}{)}
\end{Verbatim}
\end{tcolorbox}

    \begin{Verbatim}[commandchars=\\\{\}]
El tamaño de paso es:  0.0001


El valor donde se calcula el hessiano es: [-1.5  2. ]
El hessiano analítico en [-1.5  2. ] es:
 [[1902.  600.]
 [ 600.  200.]]
El hessiano por diferencias finitas en [-1.5  2. ] es:
 [[1901.64  599.98]
 [ 599.98  200.  ]]
La norma de Frobenius ||H\_df-H\_a|| en [-1.5  2. ] es:
 0.36109597644905883



El valor donde se calcula el hessiano es: [-0.25  1.5 ]
El hessiano analítico en [-0.25  1.5 ] es:
 [[-523.  100.]
 [ 100.  200.]]
El hessiano por diferencias finitas en [-0.25  1.5 ] es:
 [[-523.06   99.98]
 [  99.98  200.  ]]
La norma de Frobenius ||H\_df-H\_a|| en [-0.25  1.5 ] es:
 0.06632005904773697



El valor donde se calcula el hessiano es: [1. 1.]
El hessiano analítico en [1. 1.] es:
 [[ 802. -400.]
 [-400.  200.]]
El hessiano por diferencias finitas en [1. 1.] es:
 [[ 802.24 -400.02]
 [-400.02  200.  ]]
La norma de Frobenius ||H\_df-H\_a|| en [1. 1.] es:
 0.24167482336199914



    \end{Verbatim}

    En este caso la diferencia entre el hessiano analítico y numérico ronda
entre los ordenes \(10^{-2}\) y \(10^{-1}\) para cada valor de
\(\mathbf{x}\).

Finalmente, para \(h=0.00001\) los resultados son los siguientes

    \begin{tcolorbox}[breakable, size=fbox, boxrule=1pt, pad at break*=1mm,colback=cellbackground, colframe=cellborder]
\prompt{In}{incolor}{10}{\boxspacing}
\begin{Verbatim}[commandchars=\\\{\}]
\PY{n+nb}{print}\PY{p}{(}\PY{l+s+s1}{\PYZsq{}}\PY{l+s+s1}{El tamaño de paso es: }\PY{l+s+s1}{\PYZsq{}}\PY{p}{,}\PY{n}{h}\PY{p}{[}\PY{l+m+mi}{2}\PY{p}{]}\PY{p}{)}
\PY{n+nb}{print}\PY{p}{(}\PY{l+s+s1}{\PYZsq{}}\PY{l+s+se}{\PYZbs{}n}\PY{l+s+s1}{\PYZsq{}}\PY{p}{)}
\PY{k}{for} \PY{n}{l} \PY{o+ow}{in} \PY{n}{lamb}\PY{p}{:}
    \PY{n}{hess\PYZus{}a}\PY{o}{=}\PY{n}{hess\PYZus{}Rosenbrock}\PY{p}{(}\PY{n}{x1}\PY{o}{+}\PY{n}{l}\PY{o}{*}\PY{n}{x2}\PY{p}{)}
    \PY{n}{hess\PYZus{}df}\PY{o}{=}\PY{n}{hess\PYZus{}f}\PY{p}{(}\PY{n}{f\PYZus{}Rosenbrock}\PY{p}{,}\PY{n}{x1}\PY{o}{+}\PY{n}{l}\PY{o}{*}\PY{n}{x2}\PY{p}{,}\PY{n}{h}\PY{p}{[}\PY{l+m+mi}{2}\PY{p}{]}\PY{p}{)}
    \PY{n}{norm\PYZus{}hess}\PY{o}{=}\PY{n}{np}\PY{o}{.}\PY{n}{linalg}\PY{o}{.}\PY{n}{norm}\PY{p}{(}\PY{p}{(}\PY{n}{hess\PYZus{}a}\PY{o}{\PYZhy{}}\PY{n}{hess\PYZus{}df}\PY{p}{)}\PY{p}{,}\PY{n+nb}{ord}\PY{o}{=}\PY{l+s+s1}{\PYZsq{}}\PY{l+s+s1}{fro}\PY{l+s+s1}{\PYZsq{}}\PY{p}{)}
    \PY{n+nb}{print}\PY{p}{(}\PY{l+s+sa}{f}\PY{l+s+s1}{\PYZsq{}}\PY{l+s+s1}{El valor donde se calcula el hessiano es: }\PY{l+s+si}{\PYZob{}}\PY{n}{x1}\PY{o}{+}\PY{n}{l}\PY{o}{*}\PY{n}{x2}\PY{l+s+si}{\PYZcb{}}\PY{l+s+s1}{\PYZsq{}}\PY{p}{)}
    \PY{n+nb}{print}\PY{p}{(}\PY{l+s+sa}{f}\PY{l+s+s1}{\PYZsq{}}\PY{l+s+s1}{El hessiano analítico en }\PY{l+s+si}{\PYZob{}}\PY{n}{x1}\PY{o}{+}\PY{n}{l}\PY{o}{*}\PY{n}{x2}\PY{l+s+si}{\PYZcb{}}\PY{l+s+s1}{ es:}\PY{l+s+se}{\PYZbs{}n}\PY{l+s+s1}{ }\PY{l+s+si}{\PYZob{}}\PY{n}{hess\PYZus{}a}\PY{l+s+si}{\PYZcb{}}\PY{l+s+s1}{\PYZsq{}}\PY{p}{)}
    \PY{n+nb}{print}\PY{p}{(}\PY{l+s+sa}{f}\PY{l+s+s1}{\PYZsq{}}\PY{l+s+s1}{El hessiano por diferencias finitas en }\PY{l+s+si}{\PYZob{}}\PY{n}{x1}\PY{o}{+}\PY{n}{l}\PY{o}{*}\PY{n}{x2}\PY{l+s+si}{\PYZcb{}}\PY{l+s+s1}{ es:}\PY{l+s+se}{\PYZbs{}n}\PY{l+s+s1}{ }\PY{l+s+si}{\PYZob{}}\PY{n}{hess\PYZus{}df}\PY{l+s+si}{\PYZcb{}}\PY{l+s+s1}{\PYZsq{}}\PY{p}{)}
    \PY{n+nb}{print}\PY{p}{(}\PY{l+s+sa}{f}\PY{l+s+s1}{\PYZsq{}}\PY{l+s+s1}{La norma de Frobenius ||H\PYZus{}df\PYZhy{}H\PYZus{}a|| en }\PY{l+s+si}{\PYZob{}}\PY{n}{x1}\PY{o}{+}\PY{n}{l}\PY{o}{*}\PY{n}{x2}\PY{l+s+si}{\PYZcb{}}\PY{l+s+s1}{ es:}\PY{l+s+se}{\PYZbs{}n}\PY{l+s+s1}{ }\PY{l+s+si}{\PYZob{}}\PY{n}{norm\PYZus{}hess}\PY{l+s+si}{\PYZcb{}}\PY{l+s+s1}{\PYZsq{}}\PY{p}{)}
    \PY{n+nb}{print}\PY{p}{(}\PY{l+s+s1}{\PYZsq{}}\PY{l+s+se}{\PYZbs{}n}\PY{l+s+se}{\PYZbs{}n}\PY{l+s+s1}{\PYZsq{}}\PY{p}{)}
\end{Verbatim}
\end{tcolorbox}

    \begin{Verbatim}[commandchars=\\\{\}]
El tamaño de paso es:  1e-05


El valor donde se calcula el hessiano es: [-1.5  2. ]
El hessiano analítico en [-1.5  2. ] es:
 [[1902.  600.]
 [ 600.  200.]]
El hessiano por diferencias finitas en [-1.5  2. ] es:
 [[1901.964  599.998]
 [ 599.998  200.   ]]
La norma de Frobenius ||H\_df-H\_a|| en [-1.5  2. ] es:
 0.03608399064911253



El valor donde se calcula el hessiano es: [-0.25  1.5 ]
El hessiano analítico en [-0.25  1.5 ] es:
 [[-523.  100.]
 [ 100.  200.]]
El hessiano por diferencias finitas en [-0.25  1.5 ] es:
 [[-523.0055   99.9984]
 [  99.9984  200.0002]]
La norma de Frobenius ||H\_df-H\_a|| en [-0.25  1.5 ] es:
 0.005960397911606387



El valor donde se calcula el hessiano es: [1. 1.]
El hessiano analítico en [1. 1.] es:
 [[ 802. -400.]
 [-400.  200.]]
El hessiano por diferencias finitas en [1. 1.] es:
 [[ 802.024 -400.002]
 [-400.002  200.   ]]
La norma de Frobenius ||H\_df-H\_a|| en [1. 1.] es:
 0.024166244287936563



    \end{Verbatim}

    Y aquí la norma de Frobenius de la diferencia entre el hessiano
analítico y númerico ronda entre los ordenes \(10^{-3}\) y \(10^{-2}\).

Podemos observar que en general, la aproximación del gradiente como del
hessiano es mejor en \((-0.25,1.5)\), luego en \((1,1)\) y al final la
peor aproximación se obtiene en \((-1.5,2)\) para cada valor de \(h\).


    % Add a bibliography block to the postdoc
    
    
    
\end{document}
