\documentclass[11pt]{article}

    \usepackage[breakable]{tcolorbox}
    \usepackage{parskip} % Stop auto-indenting (to mimic markdown behaviour)
    
    \usepackage{iftex}
    \ifPDFTeX
    	\usepackage[T1]{fontenc}
    	\usepackage{mathpazo}
    \else
    	\usepackage{fontspec}
    \fi

    % Basic figure setup, for now with no caption control since it's done
    % automatically by Pandoc (which extracts ![](path) syntax from Markdown).
    \usepackage{graphicx}
    % Maintain compatibility with old templates. Remove in nbconvert 6.0
    \let\Oldincludegraphics\includegraphics
    % Ensure that by default, figures have no caption (until we provide a
    % proper Figure object with a Caption API and a way to capture that
    % in the conversion process - todo).
    \usepackage{caption}
    \DeclareCaptionFormat{nocaption}{}
    \captionsetup{format=nocaption,aboveskip=0pt,belowskip=0pt}

    \usepackage{float}
    \floatplacement{figure}{H} % forces figures to be placed at the correct location
    \usepackage{xcolor} % Allow colors to be defined
    \usepackage{enumerate} % Needed for markdown enumerations to work
    \usepackage{geometry} % Used to adjust the document margins
    \usepackage{amsmath} % Equations
    \usepackage{amssymb} % Equations
    \usepackage{textcomp} % defines textquotesingle
    % Hack from http://tex.stackexchange.com/a/47451/13684:
    \AtBeginDocument{%
        \def\PYZsq{\textquotesingle}% Upright quotes in Pygmentized code
    }
    \usepackage{upquote} % Upright quotes for verbatim code
    \usepackage{eurosym} % defines \euro
    \usepackage[mathletters]{ucs} % Extended unicode (utf-8) support
    \usepackage{fancyvrb} % verbatim replacement that allows latex
    \usepackage{grffile} % extends the file name processing of package graphics 
                         % to support a larger range
    \makeatletter % fix for old versions of grffile with XeLaTeX
    \@ifpackagelater{grffile}{2019/11/01}
    {
      % Do nothing on new versions
    }
    {
      \def\Gread@@xetex#1{%
        \IfFileExists{"\Gin@base".bb}%
        {\Gread@eps{\Gin@base.bb}}%
        {\Gread@@xetex@aux#1}%
      }
    }
    \makeatother
    \usepackage[Export]{adjustbox} % Used to constrain images to a maximum size
    \adjustboxset{max size={0.9\linewidth}{0.9\paperheight}}

    % The hyperref package gives us a pdf with properly built
    % internal navigation ('pdf bookmarks' for the table of contents,
    % internal cross-reference links, web links for URLs, etc.)
    \usepackage{hyperref}
    % The default LaTeX title has an obnoxious amount of whitespace. By default,
    % titling removes some of it. It also provides customization options.
    \usepackage{titling}
    \usepackage{longtable} % longtable support required by pandoc >1.10
    \usepackage{booktabs}  % table support for pandoc > 1.12.2
    \usepackage[inline]{enumitem} % IRkernel/repr support (it uses the enumerate* environment)
    \usepackage[normalem]{ulem} % ulem is needed to support strikethroughs (\sout)
                                % normalem makes italics be italics, not underlines
    \usepackage{mathrsfs}
    

    
    % Colors for the hyperref package
    \definecolor{urlcolor}{rgb}{0,.145,.698}
    \definecolor{linkcolor}{rgb}{.71,0.21,0.01}
    \definecolor{citecolor}{rgb}{.12,.54,.11}

    % ANSI colors
    \definecolor{ansi-black}{HTML}{3E424D}
    \definecolor{ansi-black-intense}{HTML}{282C36}
    \definecolor{ansi-red}{HTML}{E75C58}
    \definecolor{ansi-red-intense}{HTML}{B22B31}
    \definecolor{ansi-green}{HTML}{00A250}
    \definecolor{ansi-green-intense}{HTML}{007427}
    \definecolor{ansi-yellow}{HTML}{DDB62B}
    \definecolor{ansi-yellow-intense}{HTML}{B27D12}
    \definecolor{ansi-blue}{HTML}{208FFB}
    \definecolor{ansi-blue-intense}{HTML}{0065CA}
    \definecolor{ansi-magenta}{HTML}{D160C4}
    \definecolor{ansi-magenta-intense}{HTML}{A03196}
    \definecolor{ansi-cyan}{HTML}{60C6C8}
    \definecolor{ansi-cyan-intense}{HTML}{258F8F}
    \definecolor{ansi-white}{HTML}{C5C1B4}
    \definecolor{ansi-white-intense}{HTML}{A1A6B2}
    \definecolor{ansi-default-inverse-fg}{HTML}{FFFFFF}
    \definecolor{ansi-default-inverse-bg}{HTML}{000000}

    % common color for the border for error outputs.
    \definecolor{outerrorbackground}{HTML}{FFDFDF}

    % commands and environments needed by pandoc snippets
    % extracted from the output of `pandoc -s`
    \providecommand{\tightlist}{%
      \setlength{\itemsep}{0pt}\setlength{\parskip}{0pt}}
    \DefineVerbatimEnvironment{Highlighting}{Verbatim}{commandchars=\\\{\}}
    % Add ',fontsize=\small' for more characters per line
    \newenvironment{Shaded}{}{}
    \newcommand{\KeywordTok}[1]{\textcolor[rgb]{0.00,0.44,0.13}{\textbf{{#1}}}}
    \newcommand{\DataTypeTok}[1]{\textcolor[rgb]{0.56,0.13,0.00}{{#1}}}
    \newcommand{\DecValTok}[1]{\textcolor[rgb]{0.25,0.63,0.44}{{#1}}}
    \newcommand{\BaseNTok}[1]{\textcolor[rgb]{0.25,0.63,0.44}{{#1}}}
    \newcommand{\FloatTok}[1]{\textcolor[rgb]{0.25,0.63,0.44}{{#1}}}
    \newcommand{\CharTok}[1]{\textcolor[rgb]{0.25,0.44,0.63}{{#1}}}
    \newcommand{\StringTok}[1]{\textcolor[rgb]{0.25,0.44,0.63}{{#1}}}
    \newcommand{\CommentTok}[1]{\textcolor[rgb]{0.38,0.63,0.69}{\textit{{#1}}}}
    \newcommand{\OtherTok}[1]{\textcolor[rgb]{0.00,0.44,0.13}{{#1}}}
    \newcommand{\AlertTok}[1]{\textcolor[rgb]{1.00,0.00,0.00}{\textbf{{#1}}}}
    \newcommand{\FunctionTok}[1]{\textcolor[rgb]{0.02,0.16,0.49}{{#1}}}
    \newcommand{\RegionMarkerTok}[1]{{#1}}
    \newcommand{\ErrorTok}[1]{\textcolor[rgb]{1.00,0.00,0.00}{\textbf{{#1}}}}
    \newcommand{\NormalTok}[1]{{#1}}
    
    % Additional commands for more recent versions of Pandoc
    \newcommand{\ConstantTok}[1]{\textcolor[rgb]{0.53,0.00,0.00}{{#1}}}
    \newcommand{\SpecialCharTok}[1]{\textcolor[rgb]{0.25,0.44,0.63}{{#1}}}
    \newcommand{\VerbatimStringTok}[1]{\textcolor[rgb]{0.25,0.44,0.63}{{#1}}}
    \newcommand{\SpecialStringTok}[1]{\textcolor[rgb]{0.73,0.40,0.53}{{#1}}}
    \newcommand{\ImportTok}[1]{{#1}}
    \newcommand{\DocumentationTok}[1]{\textcolor[rgb]{0.73,0.13,0.13}{\textit{{#1}}}}
    \newcommand{\AnnotationTok}[1]{\textcolor[rgb]{0.38,0.63,0.69}{\textbf{\textit{{#1}}}}}
    \newcommand{\CommentVarTok}[1]{\textcolor[rgb]{0.38,0.63,0.69}{\textbf{\textit{{#1}}}}}
    \newcommand{\VariableTok}[1]{\textcolor[rgb]{0.10,0.09,0.49}{{#1}}}
    \newcommand{\ControlFlowTok}[1]{\textcolor[rgb]{0.00,0.44,0.13}{\textbf{{#1}}}}
    \newcommand{\OperatorTok}[1]{\textcolor[rgb]{0.40,0.40,0.40}{{#1}}}
    \newcommand{\BuiltInTok}[1]{{#1}}
    \newcommand{\ExtensionTok}[1]{{#1}}
    \newcommand{\PreprocessorTok}[1]{\textcolor[rgb]{0.74,0.48,0.00}{{#1}}}
    \newcommand{\AttributeTok}[1]{\textcolor[rgb]{0.49,0.56,0.16}{{#1}}}
    \newcommand{\InformationTok}[1]{\textcolor[rgb]{0.38,0.63,0.69}{\textbf{\textit{{#1}}}}}
    \newcommand{\WarningTok}[1]{\textcolor[rgb]{0.38,0.63,0.69}{\textbf{\textit{{#1}}}}}
    
    
    % Define a nice break command that doesn't care if a line doesn't already
    % exist.
    \def\br{\hspace*{\fill} \\* }
    % Math Jax compatibility definitions
    \def\gt{>}
    \def\lt{<}
    \let\Oldtex\TeX
    \let\Oldlatex\LaTeX
    \renewcommand{\TeX}{\textrm{\Oldtex}}
    \renewcommand{\LaTeX}{\textrm{\Oldlatex}}
    % Document parameters
    % Document title
    \title{Tarea\_11\_Optimizacion}
    
    
    
    
    
% Pygments definitions
\makeatletter
\def\PY@reset{\let\PY@it=\relax \let\PY@bf=\relax%
    \let\PY@ul=\relax \let\PY@tc=\relax%
    \let\PY@bc=\relax \let\PY@ff=\relax}
\def\PY@tok#1{\csname PY@tok@#1\endcsname}
\def\PY@toks#1+{\ifx\relax#1\empty\else%
    \PY@tok{#1}\expandafter\PY@toks\fi}
\def\PY@do#1{\PY@bc{\PY@tc{\PY@ul{%
    \PY@it{\PY@bf{\PY@ff{#1}}}}}}}
\def\PY#1#2{\PY@reset\PY@toks#1+\relax+\PY@do{#2}}

\@namedef{PY@tok@w}{\def\PY@tc##1{\textcolor[rgb]{0.73,0.73,0.73}{##1}}}
\@namedef{PY@tok@c}{\let\PY@it=\textit\def\PY@tc##1{\textcolor[rgb]{0.25,0.50,0.50}{##1}}}
\@namedef{PY@tok@cp}{\def\PY@tc##1{\textcolor[rgb]{0.74,0.48,0.00}{##1}}}
\@namedef{PY@tok@k}{\let\PY@bf=\textbf\def\PY@tc##1{\textcolor[rgb]{0.00,0.50,0.00}{##1}}}
\@namedef{PY@tok@kp}{\def\PY@tc##1{\textcolor[rgb]{0.00,0.50,0.00}{##1}}}
\@namedef{PY@tok@kt}{\def\PY@tc##1{\textcolor[rgb]{0.69,0.00,0.25}{##1}}}
\@namedef{PY@tok@o}{\def\PY@tc##1{\textcolor[rgb]{0.40,0.40,0.40}{##1}}}
\@namedef{PY@tok@ow}{\let\PY@bf=\textbf\def\PY@tc##1{\textcolor[rgb]{0.67,0.13,1.00}{##1}}}
\@namedef{PY@tok@nb}{\def\PY@tc##1{\textcolor[rgb]{0.00,0.50,0.00}{##1}}}
\@namedef{PY@tok@nf}{\def\PY@tc##1{\textcolor[rgb]{0.00,0.00,1.00}{##1}}}
\@namedef{PY@tok@nc}{\let\PY@bf=\textbf\def\PY@tc##1{\textcolor[rgb]{0.00,0.00,1.00}{##1}}}
\@namedef{PY@tok@nn}{\let\PY@bf=\textbf\def\PY@tc##1{\textcolor[rgb]{0.00,0.00,1.00}{##1}}}
\@namedef{PY@tok@ne}{\let\PY@bf=\textbf\def\PY@tc##1{\textcolor[rgb]{0.82,0.25,0.23}{##1}}}
\@namedef{PY@tok@nv}{\def\PY@tc##1{\textcolor[rgb]{0.10,0.09,0.49}{##1}}}
\@namedef{PY@tok@no}{\def\PY@tc##1{\textcolor[rgb]{0.53,0.00,0.00}{##1}}}
\@namedef{PY@tok@nl}{\def\PY@tc##1{\textcolor[rgb]{0.63,0.63,0.00}{##1}}}
\@namedef{PY@tok@ni}{\let\PY@bf=\textbf\def\PY@tc##1{\textcolor[rgb]{0.60,0.60,0.60}{##1}}}
\@namedef{PY@tok@na}{\def\PY@tc##1{\textcolor[rgb]{0.49,0.56,0.16}{##1}}}
\@namedef{PY@tok@nt}{\let\PY@bf=\textbf\def\PY@tc##1{\textcolor[rgb]{0.00,0.50,0.00}{##1}}}
\@namedef{PY@tok@nd}{\def\PY@tc##1{\textcolor[rgb]{0.67,0.13,1.00}{##1}}}
\@namedef{PY@tok@s}{\def\PY@tc##1{\textcolor[rgb]{0.73,0.13,0.13}{##1}}}
\@namedef{PY@tok@sd}{\let\PY@it=\textit\def\PY@tc##1{\textcolor[rgb]{0.73,0.13,0.13}{##1}}}
\@namedef{PY@tok@si}{\let\PY@bf=\textbf\def\PY@tc##1{\textcolor[rgb]{0.73,0.40,0.53}{##1}}}
\@namedef{PY@tok@se}{\let\PY@bf=\textbf\def\PY@tc##1{\textcolor[rgb]{0.73,0.40,0.13}{##1}}}
\@namedef{PY@tok@sr}{\def\PY@tc##1{\textcolor[rgb]{0.73,0.40,0.53}{##1}}}
\@namedef{PY@tok@ss}{\def\PY@tc##1{\textcolor[rgb]{0.10,0.09,0.49}{##1}}}
\@namedef{PY@tok@sx}{\def\PY@tc##1{\textcolor[rgb]{0.00,0.50,0.00}{##1}}}
\@namedef{PY@tok@m}{\def\PY@tc##1{\textcolor[rgb]{0.40,0.40,0.40}{##1}}}
\@namedef{PY@tok@gh}{\let\PY@bf=\textbf\def\PY@tc##1{\textcolor[rgb]{0.00,0.00,0.50}{##1}}}
\@namedef{PY@tok@gu}{\let\PY@bf=\textbf\def\PY@tc##1{\textcolor[rgb]{0.50,0.00,0.50}{##1}}}
\@namedef{PY@tok@gd}{\def\PY@tc##1{\textcolor[rgb]{0.63,0.00,0.00}{##1}}}
\@namedef{PY@tok@gi}{\def\PY@tc##1{\textcolor[rgb]{0.00,0.63,0.00}{##1}}}
\@namedef{PY@tok@gr}{\def\PY@tc##1{\textcolor[rgb]{1.00,0.00,0.00}{##1}}}
\@namedef{PY@tok@ge}{\let\PY@it=\textit}
\@namedef{PY@tok@gs}{\let\PY@bf=\textbf}
\@namedef{PY@tok@gp}{\let\PY@bf=\textbf\def\PY@tc##1{\textcolor[rgb]{0.00,0.00,0.50}{##1}}}
\@namedef{PY@tok@go}{\def\PY@tc##1{\textcolor[rgb]{0.53,0.53,0.53}{##1}}}
\@namedef{PY@tok@gt}{\def\PY@tc##1{\textcolor[rgb]{0.00,0.27,0.87}{##1}}}
\@namedef{PY@tok@err}{\def\PY@bc##1{{\setlength{\fboxsep}{\string -\fboxrule}\fcolorbox[rgb]{1.00,0.00,0.00}{1,1,1}{\strut ##1}}}}
\@namedef{PY@tok@kc}{\let\PY@bf=\textbf\def\PY@tc##1{\textcolor[rgb]{0.00,0.50,0.00}{##1}}}
\@namedef{PY@tok@kd}{\let\PY@bf=\textbf\def\PY@tc##1{\textcolor[rgb]{0.00,0.50,0.00}{##1}}}
\@namedef{PY@tok@kn}{\let\PY@bf=\textbf\def\PY@tc##1{\textcolor[rgb]{0.00,0.50,0.00}{##1}}}
\@namedef{PY@tok@kr}{\let\PY@bf=\textbf\def\PY@tc##1{\textcolor[rgb]{0.00,0.50,0.00}{##1}}}
\@namedef{PY@tok@bp}{\def\PY@tc##1{\textcolor[rgb]{0.00,0.50,0.00}{##1}}}
\@namedef{PY@tok@fm}{\def\PY@tc##1{\textcolor[rgb]{0.00,0.00,1.00}{##1}}}
\@namedef{PY@tok@vc}{\def\PY@tc##1{\textcolor[rgb]{0.10,0.09,0.49}{##1}}}
\@namedef{PY@tok@vg}{\def\PY@tc##1{\textcolor[rgb]{0.10,0.09,0.49}{##1}}}
\@namedef{PY@tok@vi}{\def\PY@tc##1{\textcolor[rgb]{0.10,0.09,0.49}{##1}}}
\@namedef{PY@tok@vm}{\def\PY@tc##1{\textcolor[rgb]{0.10,0.09,0.49}{##1}}}
\@namedef{PY@tok@sa}{\def\PY@tc##1{\textcolor[rgb]{0.73,0.13,0.13}{##1}}}
\@namedef{PY@tok@sb}{\def\PY@tc##1{\textcolor[rgb]{0.73,0.13,0.13}{##1}}}
\@namedef{PY@tok@sc}{\def\PY@tc##1{\textcolor[rgb]{0.73,0.13,0.13}{##1}}}
\@namedef{PY@tok@dl}{\def\PY@tc##1{\textcolor[rgb]{0.73,0.13,0.13}{##1}}}
\@namedef{PY@tok@s2}{\def\PY@tc##1{\textcolor[rgb]{0.73,0.13,0.13}{##1}}}
\@namedef{PY@tok@sh}{\def\PY@tc##1{\textcolor[rgb]{0.73,0.13,0.13}{##1}}}
\@namedef{PY@tok@s1}{\def\PY@tc##1{\textcolor[rgb]{0.73,0.13,0.13}{##1}}}
\@namedef{PY@tok@mb}{\def\PY@tc##1{\textcolor[rgb]{0.40,0.40,0.40}{##1}}}
\@namedef{PY@tok@mf}{\def\PY@tc##1{\textcolor[rgb]{0.40,0.40,0.40}{##1}}}
\@namedef{PY@tok@mh}{\def\PY@tc##1{\textcolor[rgb]{0.40,0.40,0.40}{##1}}}
\@namedef{PY@tok@mi}{\def\PY@tc##1{\textcolor[rgb]{0.40,0.40,0.40}{##1}}}
\@namedef{PY@tok@il}{\def\PY@tc##1{\textcolor[rgb]{0.40,0.40,0.40}{##1}}}
\@namedef{PY@tok@mo}{\def\PY@tc##1{\textcolor[rgb]{0.40,0.40,0.40}{##1}}}
\@namedef{PY@tok@ch}{\let\PY@it=\textit\def\PY@tc##1{\textcolor[rgb]{0.25,0.50,0.50}{##1}}}
\@namedef{PY@tok@cm}{\let\PY@it=\textit\def\PY@tc##1{\textcolor[rgb]{0.25,0.50,0.50}{##1}}}
\@namedef{PY@tok@cpf}{\let\PY@it=\textit\def\PY@tc##1{\textcolor[rgb]{0.25,0.50,0.50}{##1}}}
\@namedef{PY@tok@c1}{\let\PY@it=\textit\def\PY@tc##1{\textcolor[rgb]{0.25,0.50,0.50}{##1}}}
\@namedef{PY@tok@cs}{\let\PY@it=\textit\def\PY@tc##1{\textcolor[rgb]{0.25,0.50,0.50}{##1}}}

\def\PYZbs{\char`\\}
\def\PYZus{\char`\_}
\def\PYZob{\char`\{}
\def\PYZcb{\char`\}}
\def\PYZca{\char`\^}
\def\PYZam{\char`\&}
\def\PYZlt{\char`\<}
\def\PYZgt{\char`\>}
\def\PYZsh{\char`\#}
\def\PYZpc{\char`\%}
\def\PYZdl{\char`\$}
\def\PYZhy{\char`\-}
\def\PYZsq{\char`\'}
\def\PYZdq{\char`\"}
\def\PYZti{\char`\~}
% for compatibility with earlier versions
\def\PYZat{@}
\def\PYZlb{[}
\def\PYZrb{]}
\makeatother


    % For linebreaks inside Verbatim environment from package fancyvrb. 
    \makeatletter
        \newbox\Wrappedcontinuationbox 
        \newbox\Wrappedvisiblespacebox 
        \newcommand*\Wrappedvisiblespace {\textcolor{red}{\textvisiblespace}} 
        \newcommand*\Wrappedcontinuationsymbol {\textcolor{red}{\llap{\tiny$\m@th\hookrightarrow$}}} 
        \newcommand*\Wrappedcontinuationindent {3ex } 
        \newcommand*\Wrappedafterbreak {\kern\Wrappedcontinuationindent\copy\Wrappedcontinuationbox} 
        % Take advantage of the already applied Pygments mark-up to insert 
        % potential linebreaks for TeX processing. 
        %        {, <, #, %, $, ' and ": go to next line. 
        %        _, }, ^, &, >, - and ~: stay at end of broken line. 
        % Use of \textquotesingle for straight quote. 
        \newcommand*\Wrappedbreaksatspecials {% 
            \def\PYGZus{\discretionary{\char`\_}{\Wrappedafterbreak}{\char`\_}}% 
            \def\PYGZob{\discretionary{}{\Wrappedafterbreak\char`\{}{\char`\{}}% 
            \def\PYGZcb{\discretionary{\char`\}}{\Wrappedafterbreak}{\char`\}}}% 
            \def\PYGZca{\discretionary{\char`\^}{\Wrappedafterbreak}{\char`\^}}% 
            \def\PYGZam{\discretionary{\char`\&}{\Wrappedafterbreak}{\char`\&}}% 
            \def\PYGZlt{\discretionary{}{\Wrappedafterbreak\char`\<}{\char`\<}}% 
            \def\PYGZgt{\discretionary{\char`\>}{\Wrappedafterbreak}{\char`\>}}% 
            \def\PYGZsh{\discretionary{}{\Wrappedafterbreak\char`\#}{\char`\#}}% 
            \def\PYGZpc{\discretionary{}{\Wrappedafterbreak\char`\%}{\char`\%}}% 
            \def\PYGZdl{\discretionary{}{\Wrappedafterbreak\char`\$}{\char`\$}}% 
            \def\PYGZhy{\discretionary{\char`\-}{\Wrappedafterbreak}{\char`\-}}% 
            \def\PYGZsq{\discretionary{}{\Wrappedafterbreak\textquotesingle}{\textquotesingle}}% 
            \def\PYGZdq{\discretionary{}{\Wrappedafterbreak\char`\"}{\char`\"}}% 
            \def\PYGZti{\discretionary{\char`\~}{\Wrappedafterbreak}{\char`\~}}% 
        } 
        % Some characters . , ; ? ! / are not pygmentized. 
        % This macro makes them "active" and they will insert potential linebreaks 
        \newcommand*\Wrappedbreaksatpunct {% 
            \lccode`\~`\.\lowercase{\def~}{\discretionary{\hbox{\char`\.}}{\Wrappedafterbreak}{\hbox{\char`\.}}}% 
            \lccode`\~`\,\lowercase{\def~}{\discretionary{\hbox{\char`\,}}{\Wrappedafterbreak}{\hbox{\char`\,}}}% 
            \lccode`\~`\;\lowercase{\def~}{\discretionary{\hbox{\char`\;}}{\Wrappedafterbreak}{\hbox{\char`\;}}}% 
            \lccode`\~`\:\lowercase{\def~}{\discretionary{\hbox{\char`\:}}{\Wrappedafterbreak}{\hbox{\char`\:}}}% 
            \lccode`\~`\?\lowercase{\def~}{\discretionary{\hbox{\char`\?}}{\Wrappedafterbreak}{\hbox{\char`\?}}}% 
            \lccode`\~`\!\lowercase{\def~}{\discretionary{\hbox{\char`\!}}{\Wrappedafterbreak}{\hbox{\char`\!}}}% 
            \lccode`\~`\/\lowercase{\def~}{\discretionary{\hbox{\char`\/}}{\Wrappedafterbreak}{\hbox{\char`\/}}}% 
            \catcode`\.\active
            \catcode`\,\active 
            \catcode`\;\active
            \catcode`\:\active
            \catcode`\?\active
            \catcode`\!\active
            \catcode`\/\active 
            \lccode`\~`\~ 	
        }
    \makeatother

    \let\OriginalVerbatim=\Verbatim
    \makeatletter
    \renewcommand{\Verbatim}[1][1]{%
        %\parskip\z@skip
        \sbox\Wrappedcontinuationbox {\Wrappedcontinuationsymbol}%
        \sbox\Wrappedvisiblespacebox {\FV@SetupFont\Wrappedvisiblespace}%
        \def\FancyVerbFormatLine ##1{\hsize\linewidth
            \vtop{\raggedright\hyphenpenalty\z@\exhyphenpenalty\z@
                \doublehyphendemerits\z@\finalhyphendemerits\z@
                \strut ##1\strut}%
        }%
        % If the linebreak is at a space, the latter will be displayed as visible
        % space at end of first line, and a continuation symbol starts next line.
        % Stretch/shrink are however usually zero for typewriter font.
        \def\FV@Space {%
            \nobreak\hskip\z@ plus\fontdimen3\font minus\fontdimen4\font
            \discretionary{\copy\Wrappedvisiblespacebox}{\Wrappedafterbreak}
            {\kern\fontdimen2\font}%
        }%
        
        % Allow breaks at special characters using \PYG... macros.
        \Wrappedbreaksatspecials
        % Breaks at punctuation characters . , ; ? ! and / need catcode=\active 	
        \OriginalVerbatim[#1,codes*=\Wrappedbreaksatpunct]%
    }
    \makeatother

    % Exact colors from NB
    \definecolor{incolor}{HTML}{303F9F}
    \definecolor{outcolor}{HTML}{D84315}
    \definecolor{cellborder}{HTML}{CFCFCF}
    \definecolor{cellbackground}{HTML}{F7F7F7}
    
    % prompt
    \makeatletter
    \newcommand{\boxspacing}{\kern\kvtcb@left@rule\kern\kvtcb@boxsep}
    \makeatother
    \newcommand{\prompt}[4]{
        {\ttfamily\llap{{\color{#2}[#3]:\hspace{3pt}#4}}\vspace{-\baselineskip}}
    }
    

    
    % Prevent overflowing lines due to hard-to-break entities
    \sloppy 
    % Setup hyperref package
    \hypersetup{
      breaklinks=true,  % so long urls are correctly broken across lines
      colorlinks=true,
      urlcolor=urlcolor,
      linkcolor=linkcolor,
      citecolor=citecolor,
      }
    % Slightly bigger margins than the latex defaults
    
    \geometry{verbose,tmargin=1in,bmargin=1in,lmargin=1in,rmargin=1in}
    
    

\begin{document}
    
\title{Tarea 11 Optimización}
\author{Roberto Vásquez Martínez \\ Profesor: Joaquín Peña Acevedo}
\date{22/Mayo/2022}
\maketitle 
    
    
    

    
    \hypertarget{ejercicio-1-3-puntos}{%
\section{Ejercicio 1 (3 puntos)}\label{ejercicio-1-3-puntos}}

Usando alguna librería de Python para resolver problemas de programación
lineal, escriba y resuelva el problema de la Tarea 10:

\[
\begin{array}{rl}
\max & x_1 + x_2\\
     & 50x_1 + 24x_2 \leq 2400 \\
     & 30x_1 + 33x_2 \leq 2100 \\
     & x_1 \geq 45 \\
     & x_2 \geq 5
\end{array}
\]

\begin{enumerate}
\def\labelenumi{\arabic{enumi}.}
\tightlist
\item
  Cambie el problema para que todas las desigualdes sean de la forma
\end{enumerate}

\[\mathbf{A}\mathbf{x}\leq \mathbf{b}.\]

\begin{enumerate}
\def\labelenumi{\arabic{enumi}.}
\setcounter{enumi}{1}
\item
  Construya los vectores \(\mathbf{b},\mathbf{c}\) y la matriz
  \(\mathbf{A}\) y resuelva el problema con la librería.
\item
  Imprima un mensaje que indique si se encontró la solución, y en ese
  caso imprima :
\end{enumerate}

\begin{itemize}
\tightlist
\item
  la solución \(\mathbf{x}\),
\item
  el valor de la función objetivo,
\item
  las variables de holgura,
\end{itemize}

\begin{enumerate}
\def\labelenumi{\arabic{enumi}.}
\setcounter{enumi}{3}
\tightlist
\item
  Calcule los errores
\end{enumerate}

\[ E_x = \sum_{x_i<0} |x_i|. \]
\[ E_{b-Ax} = \sum_{(b-Ax)_i<0} |(b-Ax)_i|\]

Es decir, se suman las componentes de \(\mathbf{x}\) que no cumplen la
condición \(\mathbf{x}\geq \mathbf{0}\) y las componentes que no cumplen
con \(\mathbf{A}\mathbf{x}\leq \mathbf{b}\).

\begin{enumerate}
\def\labelenumi{\arabic{enumi}.}
\setcounter{enumi}{4}
\tightlist
\item
  Defina la tolerancia \(\tau=\sqrt{\epsilon_m}\), donde \(\epsilon_m\)
  es el épsilon de la máquina. Si \(E_x<\tau\) imprima un mensaje de que
  se cumple la condición de no negatividad, y si \(E_{b-Ax}<\tau\)
  imprima un mensaje de que se cumplen las restricciones de desigualdad.
\end{enumerate}

\hypertarget{soluciuxf3n}{%
\subsection{Solución:}\label{soluciuxf3n}}

    En primer lugar, hallaremos \(\mathbf{A},\mathbf{b}\) de forma que las
restricciones son de la forma \(\mathbf{A}\mathbf{x}\leq\mathbf{b}\).
Notemos que las restricciones son equivalentes a la siguiente lista de
desigualdades \[
\begin{array}{rl}
     & 50x_1 + 24x_2 \leq 2400 \\
     & 30x_1 + 33x_2 \leq 2100 \\
     & -x_1 \leq 45 \\
     & -x_2 \leq 5
\end{array}
\]

Si \(\mathbf{c}=(-1,-1)^T\), \(\mathbf{b}=(2400,2100,-45,-5)^T\) y \[ 
\mathbf{A}=\begin{pmatrix}
50 & 24\\
30 & 33\\
-1 & 0\\
 0 & -1
\end{pmatrix},
\]

entonces el problema de optimización que queremos resolver es \[ 
\min_{\mathbf{x}}\mathbf{c}^T\mathbf{x}\quad\text{ sujeto a }\quad \mathbf{A}\mathbf{x}\leq \mathbf{b},
\]

que es lo que queríamos obtener el numeral 1.

Para el numeral 2 y 3 construimos los vectores y matrices
correspondiente para posteriormente aplicar la librería \texttt{linprog}
de \texttt{scipy} obtener la solución así como los demás datos
requeridos. Esto lo hacemos en la siguiente celda de código.

    \begin{tcolorbox}[breakable, size=fbox, boxrule=1pt, pad at break*=1mm,colback=cellbackground, colframe=cellborder]
\prompt{In}{incolor}{1}{\boxspacing}
\begin{Verbatim}[commandchars=\\\{\}]
\PY{k+kn}{from} \PY{n+nn}{scipy}\PY{n+nn}{.}\PY{n+nn}{optimize} \PY{k+kn}{import} \PY{n}{linprog}
\PY{k+kn}{import} \PY{n+nn}{scipy}


\PY{c+c1}{\PYZsh{} Coeficientes de la funcion objetivo}
\PY{n}{obj} \PY{o}{=} \PY{p}{[}\PY{o}{\PYZhy{}}\PY{l+m+mf}{1.0}\PY{p}{,} \PY{o}{\PYZhy{}}\PY{l+m+mf}{1.0}\PY{p}{]}

\PY{c+c1}{\PYZsh{} Coeficientes del lado izquierdo de las desigualdades del tipo \PYZdq{}menor o igual a\PYZdq{}}
\PY{n}{lhs\PYZus{}ineq} \PY{o}{=} \PY{p}{[}\PY{p}{[}\PY{l+m+mf}{50.0}\PY{p}{,} \PY{l+m+mf}{24.0}\PY{p}{]}\PY{p}{,}
            \PY{p}{[}\PY{l+m+mf}{30.0}\PY{p}{,} \PY{l+m+mf}{33.0}\PY{p}{]}\PY{p}{,}
            \PY{p}{[}\PY{o}{\PYZhy{}}\PY{l+m+mf}{1.0}\PY{p}{,} \PY{l+m+mf}{0.0}\PY{p}{]}\PY{p}{,}
            \PY{p}{[}\PY{l+m+mf}{0.0}\PY{p}{,}\PY{o}{\PYZhy{}}\PY{l+m+mf}{1.0}\PY{p}{]}\PY{p}{]}

\PY{c+c1}{\PYZsh{} Coeficientes del vector del lado derecho de las desigualdades del tipo \PYZdq{}menor o igual a\PYZdq{}}
\PY{n}{rhs\PYZus{}ineq} \PY{o}{=} \PY{p}{[}\PY{l+m+mi}{2400}\PY{p}{,} \PY{l+m+mi}{2100}\PY{p}{,} \PY{o}{\PYZhy{}}\PY{l+m+mi}{45}\PY{p}{,}\PY{o}{\PYZhy{}}\PY{l+m+mi}{5}\PY{p}{]}


\PY{c+c1}{\PYZsh{} Cotas de las variables}
\PY{n}{bnd} \PY{o}{=} \PY{p}{[}\PY{p}{(}\PY{l+m+mi}{0}\PY{p}{,} \PY{n}{scipy}\PY{o}{.}\PY{n}{inf}\PY{p}{)}\PY{p}{,}  \PY{c+c1}{\PYZsh{} cotas para x1}
       \PY{p}{(}\PY{l+m+mi}{0}\PY{p}{,} \PY{n}{scipy}\PY{o}{.}\PY{n}{inf}\PY{p}{)}\PY{p}{]}  \PY{c+c1}{\PYZsh{} cotas para x2}

\PY{n}{opt\PYZus{}ineq} \PY{o}{=} \PY{n}{linprog}\PY{p}{(}\PY{n}{c}\PY{o}{=}\PY{n}{obj}\PY{p}{,} \PY{n}{A\PYZus{}ub}\PY{o}{=}\PY{n}{lhs\PYZus{}ineq}\PY{p}{,} \PY{n}{b\PYZus{}ub}\PY{o}{=}\PY{n}{rhs\PYZus{}ineq}\PY{p}{,} \PY{n}{bounds}\PY{o}{=}\PY{n}{bnd}\PY{p}{,}
              \PY{n}{method}\PY{o}{=}\PY{l+s+s2}{\PYZdq{}}\PY{l+s+s2}{simplex}\PY{l+s+s2}{\PYZdq{}}\PY{p}{)}

\PY{n+nb}{print}\PY{p}{(}\PY{l+s+s1}{\PYZsq{}}\PY{l+s+se}{\PYZbs{}n}\PY{l+s+s1}{Resultado del proceso:}\PY{l+s+s1}{\PYZsq{}}\PY{p}{,} \PY{n}{opt\PYZus{}ineq}\PY{o}{.}\PY{n}{message}\PY{p}{)}
\PY{k}{if} \PY{n}{opt\PYZus{}ineq}\PY{o}{.}\PY{n}{success}\PY{p}{:}
    \PY{n+nb}{print}\PY{p}{(}\PY{l+s+s1}{\PYZsq{}}\PY{l+s+s1}{Valor de la función objetivo:}\PY{l+s+s1}{\PYZsq{}}\PY{p}{,} \PY{n}{opt\PYZus{}ineq}\PY{o}{.}\PY{n}{fun}\PY{p}{)}
    \PY{n+nb}{print}\PY{p}{(}\PY{l+s+s1}{\PYZsq{}}\PY{l+s+s1}{Solución:}\PY{l+s+se}{\PYZbs{}n}\PY{l+s+s1}{\PYZsq{}}\PY{p}{,} \PY{n}{opt\PYZus{}ineq}\PY{o}{.}\PY{n}{x}\PY{p}{)}
    \PY{n+nb}{print}\PY{p}{(}\PY{l+s+s1}{\PYZsq{}}\PY{l+s+se}{\PYZbs{}n}\PY{l+s+s1}{Variables de holgura:}\PY{l+s+se}{\PYZbs{}n}\PY{l+s+s1}{\PYZsq{}}\PY{p}{,} \PY{n}{opt\PYZus{}ineq}\PY{o}{.}\PY{n}{slack}\PY{p}{)}
\end{Verbatim}
\end{tcolorbox}

    \begin{Verbatim}[commandchars=\\\{\}]

Resultado del proceso: Optimization terminated successfully.
Valor de la función objetivo: -51.25
Solución:
 [45.    6.25]

Variables de holgura:
 [  0.   543.75   0.     1.25]
    \end{Verbatim}

    Y esta solución es la misma que la encontrada en el Ejercicio 2 de la
Tarea 10 usando la forma estándar y hallando los puntos básicos
factibles.

    Finalmente, importamos el modulo \texttt{lib\_t11} donde implementamos
la funciones \texttt{positive\_cond} y \texttt{restriction\_cond}, las
cuales verifican bajo la tolerancia seleccionada si se cumplen la
condición de no negatividad de las variables y las reestricciones de
desigualdad, respectivamente.

Resolvemos el numeral 4 y 5 con estas funciones en la siguiente celda.

    \begin{tcolorbox}[breakable, size=fbox, boxrule=1pt, pad at break*=1mm,colback=cellbackground, colframe=cellborder]
\prompt{In}{incolor}{2}{\boxspacing}
\begin{Verbatim}[commandchars=\\\{\}]
\PY{k+kn}{import} \PY{n+nn}{numpy} \PY{k}{as} \PY{n+nn}{np}
\PY{k+kn}{from} \PY{n+nn}{lib\PYZus{}t11} \PY{k+kn}{import} \PY{o}{*}

\PY{n}{tol}\PY{o}{=}\PY{n}{np}\PY{o}{.}\PY{n}{finfo}\PY{p}{(}\PY{n+nb}{float}\PY{p}{)}\PY{o}{.}\PY{n}{eps}\PY{o}{*}\PY{o}{*}\PY{p}{(}\PY{l+m+mi}{1}\PY{o}{/}\PY{l+m+mi}{2}\PY{p}{)}

\PY{c+c1}{\PYZsh{} Condicion de no negatividad}
\PY{n}{positive\PYZus{}cond}\PY{p}{(}\PY{n}{tol}\PY{p}{,}\PY{n}{opt\PYZus{}ineq}\PY{o}{.}\PY{n}{x}\PY{p}{)}

\PY{c+c1}{\PYZsh{} Restricciones de desigualdad}
\PY{n}{restriction\PYZus{}cond}\PY{p}{(}\PY{n}{tol}\PY{p}{,}\PY{n}{opt\PYZus{}ineq}\PY{o}{.}\PY{n}{x}\PY{p}{,}\PY{n}{lhs\PYZus{}ineq}\PY{p}{,}\PY{n}{rhs\PYZus{}ineq}\PY{p}{)}
\end{Verbatim}
\end{tcolorbox}

    \begin{Verbatim}[commandchars=\\\{\}]
Se cumplen la condicion de no negatividad
Se cumple las restricciones de desigualdad
    \end{Verbatim}

    \hypertarget{ejercicio-2-3-puntos}{%
\section{Ejercicio 2 (3 puntos)}\label{ejercicio-2-3-puntos}}

\begin{enumerate}
\def\labelenumi{\arabic{enumi}.}
\tightlist
\item
  Escriba el problema anterior en su forma estándar.
\item
  Construya los vectores \(\mathbf{b},\mathbf{c}\) y la matriz
  \(\mathbf{A}\) y resuelva este problema con la librería.
\item
  Imprima un mensaje que indique si se encontró la solución, y en ese
  caso imprima la solución, el valor de la función objetivo, las
  variables de holgura y el error
\end{enumerate}

\[ \|\mathbf{A}\mathbf{x}-\mathbf{b}\|. \]

\begin{enumerate}
\def\labelenumi{\arabic{enumi}.}
\setcounter{enumi}{3}
\tightlist
\item
  Calcule el error \(E_x\) como en el Ejercicio 1 y si \(E_x<\tau\)
  imprima un mensaje de que se cumple la condición de no negatividad.
\end{enumerate}

\hypertarget{soluciuxf3n}{%
\subsection{Solución:}\label{soluciuxf3n}}

    En la Tarea 10 se nos proporciona la forma estándar considerando
\(\mathbf{x}=(x_1,x_2,x_3,x_4,x_5,x_6)^T\), aquí las variables del
problema son \(x_1\) y \(x_2\), las restantes son las variables de
holgura.

En este caso, consideramos \(\mathbf{c}=(-1,-1,0,0,0,0)^T\),
\(\mathbf{b}=(2400,2100,45,5)\) y \[ 
\mathbf{A}=\begin{pmatrix}
50 & 24 & 1 & 0 & 0 & 0\\
30 & 33 & 0 & 1 & 0 & 0\\
1  &  0 & 0 & 0 &-1 & 0\\
0  &  1 & 0 & 0 & 0 &-1
\end{pmatrix}.
\]

Por lo tanto, el problema en forma estándar es \[ 
\min_{\mathbf{x}} \mathbf{c}^T\mathbf{x}\quad\text{ sujeto a }\quad \mathbf{A}\mathbf{x}=\mathbf{b}\quad\text{ y }\quad \mathbf{x}\geq \mathbf{0}
\]

En forma estándar, el problema de programación lineal lo resolvemos
utilizando los argumentos \texttt{A\_eq} y \texttt{b\_eq} de la función
\texttt{linprog} de la librería \texttt{scipy}

    \begin{tcolorbox}[breakable, size=fbox, boxrule=1pt, pad at break*=1mm,colback=cellbackground, colframe=cellborder]
\prompt{In}{incolor}{3}{\boxspacing}
\begin{Verbatim}[commandchars=\\\{\}]
\PY{n}{c} \PY{o}{=} \PY{n}{np}\PY{o}{.}\PY{n}{array}\PY{p}{(}\PY{p}{[}\PY{o}{\PYZhy{}}\PY{l+m+mi}{1}\PY{p}{,} \PY{o}{\PYZhy{}}\PY{l+m+mi}{1}\PY{p}{,} \PY{l+m+mi}{0}\PY{p}{,} \PY{l+m+mi}{0}\PY{p}{,} \PY{l+m+mi}{0}\PY{p}{,} \PY{l+m+mi}{0} \PY{p}{]}\PY{p}{)}
\PY{n}{b} \PY{o}{=} \PY{n}{np}\PY{o}{.}\PY{n}{array}\PY{p}{(}\PY{p}{[}\PY{l+m+mi}{2400}\PY{p}{,} \PY{l+m+mi}{2100}\PY{p}{,} \PY{l+m+mi}{45}\PY{p}{,} \PY{l+m+mi}{5}\PY{p}{]}\PY{p}{)}
\PY{n}{A} \PY{o}{=} \PY{n}{np}\PY{o}{.}\PY{n}{array}\PY{p}{(}\PY{p}{[}\PY{p}{[}\PY{l+m+mi}{50}\PY{p}{,} \PY{l+m+mi}{24}\PY{p}{,} \PY{l+m+mi}{1}\PY{p}{,} \PY{l+m+mi}{0}\PY{p}{,} \PY{l+m+mi}{0}\PY{p}{,} \PY{l+m+mi}{0}\PY{p}{]}\PY{p}{,}
              \PY{p}{[}\PY{l+m+mi}{30}\PY{p}{,} \PY{l+m+mi}{33}\PY{p}{,} \PY{l+m+mi}{0}\PY{p}{,} \PY{l+m+mi}{1}\PY{p}{,} \PY{l+m+mi}{0}\PY{p}{,} \PY{l+m+mi}{0}\PY{p}{]}\PY{p}{,}
              \PY{p}{[} \PY{l+m+mi}{1}\PY{p}{,}  \PY{l+m+mi}{0}\PY{p}{,} \PY{l+m+mi}{0}\PY{p}{,} \PY{l+m+mi}{0}\PY{p}{,}\PY{o}{\PYZhy{}}\PY{l+m+mi}{1}\PY{p}{,} \PY{l+m+mi}{0}\PY{p}{]}\PY{p}{,}
              \PY{p}{[} \PY{l+m+mi}{0}\PY{p}{,}  \PY{l+m+mi}{1}\PY{p}{,} \PY{l+m+mi}{0}\PY{p}{,} \PY{l+m+mi}{0}\PY{p}{,} \PY{l+m+mi}{0}\PY{p}{,}\PY{o}{\PYZhy{}}\PY{l+m+mi}{1}\PY{p}{]} \PY{p}{]}\PY{p}{)}

\PY{c+c1}{\PYZsh{} Cotas de las variables}
\PY{n}{bnd} \PY{o}{=} \PY{p}{[}\PY{p}{(}\PY{l+m+mi}{0}\PY{p}{,} \PY{n}{scipy}\PY{o}{.}\PY{n}{inf}\PY{p}{)}\PY{p}{,}  \PY{c+c1}{\PYZsh{} cotas para x1}
       \PY{p}{(}\PY{l+m+mi}{0}\PY{p}{,} \PY{n}{scipy}\PY{o}{.}\PY{n}{inf}\PY{p}{)}\PY{p}{,}  \PY{c+c1}{\PYZsh{} cotas para x2}
       \PY{p}{(}\PY{l+m+mi}{0}\PY{p}{,} \PY{n}{scipy}\PY{o}{.}\PY{n}{inf}\PY{p}{)}\PY{p}{,}  \PY{c+c1}{\PYZsh{} cotas para x3}
       \PY{p}{(}\PY{l+m+mi}{0}\PY{p}{,} \PY{n}{scipy}\PY{o}{.}\PY{n}{inf}\PY{p}{)}\PY{p}{,}  \PY{c+c1}{\PYZsh{} cotas para x4}
       \PY{p}{(}\PY{l+m+mi}{0}\PY{p}{,} \PY{n}{scipy}\PY{o}{.}\PY{n}{inf}\PY{p}{)}\PY{p}{,}  \PY{c+c1}{\PYZsh{} cotas para x5}
       \PY{p}{(}\PY{l+m+mi}{0}\PY{p}{,} \PY{n}{scipy}\PY{o}{.}\PY{n}{inf}\PY{p}{)}\PY{p}{]}  \PY{c+c1}{\PYZsh{} cotas para x6}

\PY{n}{opt\PYZus{}eq} \PY{o}{=} \PY{n}{linprog}\PY{p}{(}\PY{n}{c}\PY{o}{=}\PY{n}{c}\PY{p}{,} \PY{n}{A\PYZus{}eq}\PY{o}{=}\PY{n}{A}\PY{p}{,} \PY{n}{b\PYZus{}eq}\PY{o}{=}\PY{n}{b}\PY{p}{,} \PY{n}{bounds}\PY{o}{=}\PY{n}{bnd}\PY{p}{,}
              \PY{n}{method}\PY{o}{=}\PY{l+s+s2}{\PYZdq{}}\PY{l+s+s2}{simplex}\PY{l+s+s2}{\PYZdq{}}\PY{p}{)}

\PY{n+nb}{print}\PY{p}{(}\PY{l+s+s1}{\PYZsq{}}\PY{l+s+se}{\PYZbs{}n}\PY{l+s+s1}{Resultado del proceso:}\PY{l+s+s1}{\PYZsq{}}\PY{p}{,} \PY{n}{opt\PYZus{}eq}\PY{o}{.}\PY{n}{message}\PY{p}{)}
\PY{k}{if} \PY{n}{opt\PYZus{}eq}\PY{o}{.}\PY{n}{success}\PY{p}{:}
    \PY{n+nb}{print}\PY{p}{(}\PY{l+s+s1}{\PYZsq{}}\PY{l+s+s1}{Valor de la función objetivo:}\PY{l+s+s1}{\PYZsq{}}\PY{p}{,} \PY{n}{opt\PYZus{}eq}\PY{o}{.}\PY{n}{fun}\PY{p}{)}
    \PY{n+nb}{print}\PY{p}{(}\PY{l+s+s1}{\PYZsq{}}\PY{l+s+se}{\PYZbs{}n}\PY{l+s+s1}{Solución:}\PY{l+s+se}{\PYZbs{}n}\PY{l+s+s1}{\PYZsq{}}\PY{p}{,} \PY{n}{opt\PYZus{}eq}\PY{o}{.}\PY{n}{x}\PY{p}{)}
    \PY{n+nb}{print}\PY{p}{(}\PY{l+s+s1}{\PYZsq{}}\PY{l+s+se}{\PYZbs{}n}\PY{l+s+s1}{|Ax\PYZhy{}b|: }\PY{l+s+s1}{\PYZsq{}}\PY{p}{,} \PY{n}{np}\PY{o}{.}\PY{n}{linalg}\PY{o}{.}\PY{n}{norm}\PY{p}{(}\PY{n}{opt\PYZus{}eq}\PY{o}{.}\PY{n}{con}\PY{p}{)}\PY{p}{)}
\end{Verbatim}
\end{tcolorbox}

    \begin{Verbatim}[commandchars=\\\{\}]

Resultado del proceso: Optimization terminated successfully.
Valor de la función objetivo: -51.25

Solución:
 [ 45.     6.25   0.   543.75   0.     1.25]

|Ax-b|:  3.552713678800501e-15
    \end{Verbatim}

    Las variables de holgura, como previamente dijimos, son las coordenadas
del vector solución correspondiente a las variables \(x_3,x_4,x_5,x_6\).

    Finalmente, checamos la condición de no negatividad

    \begin{tcolorbox}[breakable, size=fbox, boxrule=1pt, pad at break*=1mm,colback=cellbackground, colframe=cellborder]
\prompt{In}{incolor}{4}{\boxspacing}
\begin{Verbatim}[commandchars=\\\{\}]
\PY{n}{positive\PYZus{}cond}\PY{p}{(}\PY{n}{tol}\PY{p}{,}\PY{n}{opt\PYZus{}eq}\PY{o}{.}\PY{n}{x}\PY{p}{)}
\end{Verbatim}
\end{tcolorbox}

    \begin{Verbatim}[commandchars=\\\{\}]
Se cumplen la condicion de no negatividad
    \end{Verbatim}

    \hypertarget{ejercicio-3-4-puntos}{%
\section{Ejercicio 3 (4 puntos)}\label{ejercicio-3-4-puntos}}

\begin{enumerate}
\def\labelenumi{\arabic{enumi}.}
\tightlist
\item
  Escriba el problema dual del Ejercicio 2.
\item
  Resuelva el problema dual con la librería. Esto debería devolver el
  vector \(\lambda\) que son los multiplicadores de Lagrange de la
  restricciones de igualdad del problema primal.
\item
  Imprima un mensaje que indique si se encontró la solución, y de ser
  así, imprima \(\lambda\), el valor de la función objetivo y las
  variables de holgura.
\item
  Usando el valor \(\mathbf{x}\) del Ejercicio 2, imprima el error
  relativo
\end{enumerate}

\[\frac{|\mathbf{c}^\top\mathbf{x} - \mathbf{b}^\top\lambda|}
{|\mathbf{c}^\top\mathbf{x}|}.\]

\begin{enumerate}
\def\labelenumi{\arabic{enumi}.}
\setcounter{enumi}{3}
\tightlist
\item
  Defina el vector \(\mathbf{s}\) como las variables de holgura.
\item
  Programe una función que reciba los vectores
  \(\mathbf{b}, \mathbf{c}\), \(\mathbf{x}, \lambda, \mathbf{s}\), la
  matriz \(\mathbf{A}\) y una tolerancia \(\tau\), y verique que se
  cumplen las condiciones KKT:
\end{enumerate}

\[
\begin{array}{rclc}
  \mathbf{A}^\top \lambda + \mathbf{s} &=& \mathbf{c}, & (1) \\
  \mathbf{A}\mathbf{x} &=& \mathbf{b}, & (2) \\
  \mathbf{x} & \geq & \mathbf{0}, & (3)  \\
  \mathbf{s} & \geq & \mathbf{0}, & (4)  \\
  x_i s_i &=& 0, \qquad i=1,2,...,n. & (5)
\end{array}
\]

Calcule los errores \(E_x\) y \(E_{s}\) como en el Ejercicio 1, para
saber que tanto se violan las restricciones
\(\mathbf{x}\geq \mathbf{0}\) y \(\mathbf{s}\geq \mathbf{0}\).

La función debe imprimir - El error
\(\|\mathbf{A}^\top \lambda + \mathbf{s}- \mathbf{c}\|\). - El
error \(\|\mathbf{A}\mathbf{x} - \mathbf{b}\|\). - Si \(E_x<\tau\),
imprima que se cumple las restricciones de no negatividad de
\(\mathbf{x}\). - Si \(E_s<\tau\), imprima que se cumple las
restricciones de no negatividad de \(\mathbf{s}\). - Calcule el valor de
la suma \(\sum_i |x_i s_i|\) y si es menor que \(\tau\), imprima un
mensaje que indique que se cumple la condición de complementariedad.

\begin{enumerate}
\def\labelenumi{\arabic{enumi}.}
\setcounter{enumi}{5}
\tightlist
\item
  Use la función anterior en el problema para reportar los resultados.
\end{enumerate}

\begin{quote}
\textbf{Nota}: En el problema dual las variables en \(\lambda\) no
tienen restricciones de cota. Si usa, por ejemplo, la función
\texttt{linprog} para resolver el problema, ponga explícitamente que las
cotas de las variables son \(-\infty\) e \(\infty\) para que la función
no use las cotas que tiene fijas de manera predeterminada.
\end{quote}

\hypertarget{soluciuxf3n}{%
\subsection{Solución:}\label{soluciuxf3n}}

    El problema dual del problema primal en forma estándar descrito en el
Ejercicio 2 es \[
\min_{\lambda}-\mathbf{b}^T\lambda\quad\text{  sujeto a  }\quad \mathbf{A}^T\lambda\leq \mathbf{c},
\]

con \(\mathbf{A},\mathbf{b}\) y \(\mathbf{c}\) como en el Ejercicio 2.

Utilizamos la función \texttt{linprog} con los argumentos
correspondientes las restricciones de desigualdad del tipo \emph{menor o
igual}. La solución al problema dual es la siguiente.

    \begin{tcolorbox}[breakable, size=fbox, boxrule=1pt, pad at break*=1mm,colback=cellbackground, colframe=cellborder]
\prompt{In}{incolor}{5}{\boxspacing}
\begin{Verbatim}[commandchars=\\\{\}]
\PY{c+c1}{\PYZsh{} Coeficientes de la funcion objetivo}
\PY{n}{c\PYZus{}dual} \PY{o}{=} \PY{o}{\PYZhy{}}\PY{n}{b}

\PY{c+c1}{\PYZsh{} Coeficientes del lado izquierdo de las desigualdades del tipo \PYZdq{}menor o igual a\PYZdq{}}
\PY{n}{A\PYZus{}dual} \PY{o}{=} \PY{n}{A}\PY{o}{.}\PY{n}{T}

\PY{c+c1}{\PYZsh{} Coeficientes del vector del lado derecho de las desigualdades del tipo \PYZdq{}menor o igual a\PYZdq{}}
\PY{n}{b\PYZus{}dual} \PY{o}{=} \PY{n}{c}

\PY{c+c1}{\PYZsh{} Cotas de las variables}
\PY{n}{bnd} \PY{o}{=} \PY{p}{[}\PY{p}{(}\PY{o}{\PYZhy{}}\PY{n}{scipy}\PY{o}{.}\PY{n}{inf}\PY{p}{,} \PY{n}{scipy}\PY{o}{.}\PY{n}{inf}\PY{p}{)}\PY{p}{,}  \PY{c+c1}{\PYZsh{} cotas para lamb1}
       \PY{p}{(}\PY{o}{\PYZhy{}}\PY{n}{scipy}\PY{o}{.}\PY{n}{inf}\PY{p}{,} \PY{n}{scipy}\PY{o}{.}\PY{n}{inf}\PY{p}{)}\PY{p}{,}  \PY{c+c1}{\PYZsh{} cotas para lamb2}
       \PY{p}{(}\PY{o}{\PYZhy{}}\PY{n}{scipy}\PY{o}{.}\PY{n}{inf}\PY{p}{,} \PY{n}{scipy}\PY{o}{.}\PY{n}{inf}\PY{p}{)}\PY{p}{,}  \PY{c+c1}{\PYZsh{} cotas para lamb3}
       \PY{p}{(}\PY{o}{\PYZhy{}}\PY{n}{scipy}\PY{o}{.}\PY{n}{inf}\PY{p}{,} \PY{n}{scipy}\PY{o}{.}\PY{n}{inf}\PY{p}{)}\PY{p}{]}  \PY{c+c1}{\PYZsh{} cotas para lamb4}

\PY{n}{opt\PYZus{}dual} \PY{o}{=} \PY{n}{linprog}\PY{p}{(}\PY{n}{c}\PY{o}{=}\PY{n}{c\PYZus{}dual}\PY{p}{,} \PY{n}{A\PYZus{}ub}\PY{o}{=}\PY{n}{A\PYZus{}dual}\PY{p}{,} \PY{n}{b\PYZus{}ub}\PY{o}{=}\PY{n}{b\PYZus{}dual}\PY{p}{,} \PY{n}{bounds}\PY{o}{=}\PY{n}{bnd}\PY{p}{,}
              \PY{n}{method}\PY{o}{=}\PY{l+s+s2}{\PYZdq{}}\PY{l+s+s2}{simplex}\PY{l+s+s2}{\PYZdq{}}\PY{p}{)}

\PY{n+nb}{print}\PY{p}{(}\PY{l+s+s1}{\PYZsq{}}\PY{l+s+se}{\PYZbs{}n}\PY{l+s+s1}{Resultado del proceso:}\PY{l+s+s1}{\PYZsq{}}\PY{p}{,} \PY{n}{opt\PYZus{}dual}\PY{o}{.}\PY{n}{message}\PY{p}{)}
\PY{k}{if} \PY{n}{opt\PYZus{}dual}\PY{o}{.}\PY{n}{success}\PY{p}{:}
    \PY{n+nb}{print}\PY{p}{(}\PY{l+s+s1}{\PYZsq{}}\PY{l+s+s1}{Valor de la función objetivo:}\PY{l+s+s1}{\PYZsq{}}\PY{p}{,} \PY{n}{opt\PYZus{}dual}\PY{o}{.}\PY{n}{fun}\PY{p}{)}
    \PY{n+nb}{print}\PY{p}{(}\PY{l+s+s1}{\PYZsq{}}\PY{l+s+s1}{Solución:}\PY{l+s+se}{\PYZbs{}n}\PY{l+s+s1}{\PYZsq{}}\PY{p}{,} \PY{n}{opt\PYZus{}dual}\PY{o}{.}\PY{n}{x}\PY{p}{)}
    \PY{n+nb}{print}\PY{p}{(}\PY{l+s+s1}{\PYZsq{}}\PY{l+s+se}{\PYZbs{}n}\PY{l+s+s1}{Variables de holgura:}\PY{l+s+se}{\PYZbs{}n}\PY{l+s+s1}{\PYZsq{}}\PY{p}{,} \PY{n}{opt\PYZus{}dual}\PY{o}{.}\PY{n}{slack}\PY{p}{)}
\end{Verbatim}
\end{tcolorbox}

    \begin{Verbatim}[commandchars=\\\{\}]

Resultado del proceso: Optimization terminated successfully.
Valor de la función objetivo: 51.24999999999999
Solución:
 [-0.04166667  0.          1.08333333  0.        ]

Variables de holgura:
 [-4.44089210e-16  0.00000000e+00  4.16666667e-02  0.00000000e+00
  1.08333333e+00  0.00000000e+00]
    \end{Verbatim}

    Que coincide con la vectores de multiplicadores de Lagrange \(\lambda\)
y \(\mathbf{s}\), salvo por error numérico, que se obtuvieron en la Tarea 10 a partir de la
solución del problema primal y las condiciones KKT.

Por otro lado, por la relación existente entre el problema primal y su
dual tienen el mismo valor para la función objetivo (considerando la
función objetivo del problema dual correspondiente a una maximización).

A continuación, mostramos el error relativo entre los valores óptimos
hayados para el para el problema primal y el dual.

    \begin{tcolorbox}[breakable, size=fbox, boxrule=1pt, pad at break*=1mm,colback=cellbackground, colframe=cellborder]
\prompt{In}{incolor}{6}{\boxspacing}
\begin{Verbatim}[commandchars=\\\{\}]
\PY{n}{lamb}\PY{o}{=}\PY{n}{opt\PYZus{}dual}\PY{o}{.}\PY{n}{x}  \PY{c+c1}{\PYZsh{} Multiplicadores de Lagrange condiciones de igualdad.}
\PY{n}{s}\PY{o}{=}\PY{n}{opt\PYZus{}dual}\PY{o}{.}\PY{n}{slack} \PY{c+c1}{\PYZsh{} Variables de holgura}

\PY{n}{rel\PYZus{}err}\PY{o}{=}\PY{n}{np}\PY{o}{.}\PY{n}{abs}\PY{p}{(}\PY{n}{opt\PYZus{}eq}\PY{o}{.}\PY{n}{fun}\PY{o}{+}\PY{n}{opt\PYZus{}dual}\PY{o}{.}\PY{n}{fun}\PY{p}{)}\PY{o}{/}\PY{n}{np}\PY{o}{.}\PY{n}{abs}\PY{p}{(}\PY{n}{opt\PYZus{}eq}\PY{o}{.}\PY{n}{fun}\PY{p}{)} \PY{c+c1}{\PYZsh{} Error relativo}

\PY{n+nb}{print}\PY{p}{(}\PY{l+s+s1}{\PYZsq{}}\PY{l+s+s1}{Error relativo entre las funciones objetivos del problema prima y dual es: }\PY{l+s+s1}{\PYZsq{}}\PY{p}{)}
\PY{n+nb}{print}\PY{p}{(}\PY{n}{rel\PYZus{}err}\PY{p}{)}
\end{Verbatim}
\end{tcolorbox}

    \begin{Verbatim}[commandchars=\\\{\}]
Error relativo entre las funciones objetivos del problema prima y dual es:
1.38642485026361e-16
    \end{Verbatim}

    Que es un valor pequeño y de hecho es más pequeño que la tolerancia
\(\tau=\sqrt{\epsilon_m}\), por lo que podríamos afirmar que el valor
óptimo de la función objetivo para el problema primal y dual es el
mismo.

    Finalmente, prograremos la función que verificará las condiciones KKT
con la solución del problema obtenida y las variables duales óptimas.
Esta función quedará definida en el módulo \texttt{lib\_t11} y llevará
por nombre \texttt{KKT\_cond}.

Usamos esta última función para obtener los resultados solicitados

    \begin{tcolorbox}[breakable, size=fbox, boxrule=1pt, pad at break*=1mm,colback=cellbackground, colframe=cellborder]
\prompt{In}{incolor}{7}{\boxspacing}
\begin{Verbatim}[commandchars=\\\{\}]
\PY{n}{KKT\PYZus{}cond}\PY{p}{(}\PY{n}{tol}\PY{p}{,}\PY{n}{b}\PY{p}{,}\PY{n}{c}\PY{p}{,}\PY{n}{opt\PYZus{}eq}\PY{o}{.}\PY{n}{x}\PY{p}{,}\PY{n}{lamb}\PY{p}{,}\PY{n}{s}\PY{p}{,}\PY{n}{A}\PY{p}{)}
\end{Verbatim}
\end{tcolorbox}

    \begin{Verbatim}[commandchars=\\\{\}]
Condicion 1: |AT*lamb+s-c| = 0.0
Condicion 2: |Ax-b| =  3.552713678800501e-15
SI se cumple la condicion de no negatividad de x
SI se cumple la condicion de no negatividad de s
SI se cumple la condicion de complentariedad
    \end{Verbatim}

    De los resultados anterior concluimos que los valores hallados para el
problema primal como el dual son óptimos de sus respectivos contextos.


    % Add a bibliography block to the postdoc
    
    
    
\end{document}
