\documentclass[11pt]{article}

    \usepackage[breakable]{tcolorbox}
    \usepackage{parskip} % Stop auto-indenting (to mimic markdown behaviour)
    
    \usepackage{iftex}
    \ifPDFTeX
    	\usepackage[T1]{fontenc}
    	\usepackage{mathpazo}
    \else
    	\usepackage{fontspec}
    \fi

    % Basic figure setup, for now with no caption control since it's done
    % automatically by Pandoc (which extracts ![](path) syntax from Markdown).
    \usepackage{graphicx}
    % Maintain compatibility with old templates. Remove in nbconvert 6.0
    \let\Oldincludegraphics\includegraphics
    % Ensure that by default, figures have no caption (until we provide a
    % proper Figure object with a Caption API and a way to capture that
    % in the conversion process - todo).
    \usepackage{caption}
    \DeclareCaptionFormat{nocaption}{}
    \captionsetup{format=nocaption,aboveskip=0pt,belowskip=0pt}

    \usepackage[Export]{adjustbox} % Used to constrain images to a maximum size
    \adjustboxset{max size={0.9\linewidth}{0.9\paperheight}}
    \usepackage{float}
    \floatplacement{figure}{H} % forces figures to be placed at the correct location
    \usepackage{xcolor} % Allow colors to be defined
    \usepackage{enumerate} % Needed for markdown enumerations to work
    \usepackage{geometry} % Used to adjust the document margins
    \usepackage{amsmath} % Equations
    \usepackage{amssymb} % Equations
    \usepackage{textcomp} % defines textquotesingle
    % Hack from http://tex.stackexchange.com/a/47451/13684:
    \AtBeginDocument{%
        \def\PYZsq{\textquotesingle}% Upright quotes in Pygmentized code
    }
    \usepackage{upquote} % Upright quotes for verbatim code
    \usepackage{eurosym} % defines \euro
    \usepackage[mathletters]{ucs} % Extended unicode (utf-8) support
    \usepackage{fancyvrb} % verbatim replacement that allows latex
    \usepackage{grffile} % extends the file name processing of package graphics 
                         % to support a larger range
    \makeatletter % fix for grffile with XeLaTeX
    \def\Gread@@xetex#1{%
      \IfFileExists{"\Gin@base".bb}%
      {\Gread@eps{\Gin@base.bb}}%
      {\Gread@@xetex@aux#1}%
    }
    \makeatother

    % The hyperref package gives us a pdf with properly built
    % internal navigation ('pdf bookmarks' for the table of contents,
    % internal cross-reference links, web links for URLs, etc.)
    \usepackage{hyperref}
    % The default LaTeX title has an obnoxious amount of whitespace. By default,
    % titling removes some of it. It also provides customization options.
    \usepackage{titling}
    \usepackage{longtable} % longtable support required by pandoc >1.10
    \usepackage{booktabs}  % table support for pandoc > 1.12.2
    \usepackage[inline]{enumitem} % IRkernel/repr support (it uses the enumerate* environment)
    \usepackage[normalem]{ulem} % ulem is needed to support strikethroughs (\sout)
                                % normalem makes italics be italics, not underlines
    \usepackage{mathrsfs}
    

    
    % Colors for the hyperref package
    \definecolor{urlcolor}{rgb}{0,.145,.698}
    \definecolor{linkcolor}{rgb}{.71,0.21,0.01}
    \definecolor{citecolor}{rgb}{.12,.54,.11}

    % ANSI colors
    \definecolor{ansi-black}{HTML}{3E424D}
    \definecolor{ansi-black-intense}{HTML}{282C36}
    \definecolor{ansi-red}{HTML}{E75C58}
    \definecolor{ansi-red-intense}{HTML}{B22B31}
    \definecolor{ansi-green}{HTML}{00A250}
    \definecolor{ansi-green-intense}{HTML}{007427}
    \definecolor{ansi-yellow}{HTML}{DDB62B}
    \definecolor{ansi-yellow-intense}{HTML}{B27D12}
    \definecolor{ansi-blue}{HTML}{208FFB}
    \definecolor{ansi-blue-intense}{HTML}{0065CA}
    \definecolor{ansi-magenta}{HTML}{D160C4}
    \definecolor{ansi-magenta-intense}{HTML}{A03196}
    \definecolor{ansi-cyan}{HTML}{60C6C8}
    \definecolor{ansi-cyan-intense}{HTML}{258F8F}
    \definecolor{ansi-white}{HTML}{C5C1B4}
    \definecolor{ansi-white-intense}{HTML}{A1A6B2}
    \definecolor{ansi-default-inverse-fg}{HTML}{FFFFFF}
    \definecolor{ansi-default-inverse-bg}{HTML}{000000}

    % commands and environments needed by pandoc snippets
    % extracted from the output of `pandoc -s`
    \providecommand{\tightlist}{%
      \setlength{\itemsep}{0pt}\setlength{\parskip}{0pt}}
    \DefineVerbatimEnvironment{Highlighting}{Verbatim}{commandchars=\\\{\}}
    % Add ',fontsize=\small' for more characters per line
    \newenvironment{Shaded}{}{}
    \newcommand{\KeywordTok}[1]{\textcolor[rgb]{0.00,0.44,0.13}{\textbf{{#1}}}}
    \newcommand{\DataTypeTok}[1]{\textcolor[rgb]{0.56,0.13,0.00}{{#1}}}
    \newcommand{\DecValTok}[1]{\textcolor[rgb]{0.25,0.63,0.44}{{#1}}}
    \newcommand{\BaseNTok}[1]{\textcolor[rgb]{0.25,0.63,0.44}{{#1}}}
    \newcommand{\FloatTok}[1]{\textcolor[rgb]{0.25,0.63,0.44}{{#1}}}
    \newcommand{\CharTok}[1]{\textcolor[rgb]{0.25,0.44,0.63}{{#1}}}
    \newcommand{\StringTok}[1]{\textcolor[rgb]{0.25,0.44,0.63}{{#1}}}
    \newcommand{\CommentTok}[1]{\textcolor[rgb]{0.38,0.63,0.69}{\textit{{#1}}}}
    \newcommand{\OtherTok}[1]{\textcolor[rgb]{0.00,0.44,0.13}{{#1}}}
    \newcommand{\AlertTok}[1]{\textcolor[rgb]{1.00,0.00,0.00}{\textbf{{#1}}}}
    \newcommand{\FunctionTok}[1]{\textcolor[rgb]{0.02,0.16,0.49}{{#1}}}
    \newcommand{\RegionMarkerTok}[1]{{#1}}
    \newcommand{\ErrorTok}[1]{\textcolor[rgb]{1.00,0.00,0.00}{\textbf{{#1}}}}
    \newcommand{\NormalTok}[1]{{#1}}
    
    % Additional commands for more recent versions of Pandoc
    \newcommand{\ConstantTok}[1]{\textcolor[rgb]{0.53,0.00,0.00}{{#1}}}
    \newcommand{\SpecialCharTok}[1]{\textcolor[rgb]{0.25,0.44,0.63}{{#1}}}
    \newcommand{\VerbatimStringTok}[1]{\textcolor[rgb]{0.25,0.44,0.63}{{#1}}}
    \newcommand{\SpecialStringTok}[1]{\textcolor[rgb]{0.73,0.40,0.53}{{#1}}}
    \newcommand{\ImportTok}[1]{{#1}}
    \newcommand{\DocumentationTok}[1]{\textcolor[rgb]{0.73,0.13,0.13}{\textit{{#1}}}}
    \newcommand{\AnnotationTok}[1]{\textcolor[rgb]{0.38,0.63,0.69}{\textbf{\textit{{#1}}}}}
    \newcommand{\CommentVarTok}[1]{\textcolor[rgb]{0.38,0.63,0.69}{\textbf{\textit{{#1}}}}}
    \newcommand{\VariableTok}[1]{\textcolor[rgb]{0.10,0.09,0.49}{{#1}}}
    \newcommand{\ControlFlowTok}[1]{\textcolor[rgb]{0.00,0.44,0.13}{\textbf{{#1}}}}
    \newcommand{\OperatorTok}[1]{\textcolor[rgb]{0.40,0.40,0.40}{{#1}}}
    \newcommand{\BuiltInTok}[1]{{#1}}
    \newcommand{\ExtensionTok}[1]{{#1}}
    \newcommand{\PreprocessorTok}[1]{\textcolor[rgb]{0.74,0.48,0.00}{{#1}}}
    \newcommand{\AttributeTok}[1]{\textcolor[rgb]{0.49,0.56,0.16}{{#1}}}
    \newcommand{\InformationTok}[1]{\textcolor[rgb]{0.38,0.63,0.69}{\textbf{\textit{{#1}}}}}
    \newcommand{\WarningTok}[1]{\textcolor[rgb]{0.38,0.63,0.69}{\textbf{\textit{{#1}}}}}
    
    
    % Define a nice break command that doesn't care if a line doesn't already
    % exist.
    \def\br{\hspace*{\fill} \\* }
    % Math Jax compatibility definitions
    \def\gt{>}
    \def\lt{<}
    \let\Oldtex\TeX
    \let\Oldlatex\LaTeX
    \renewcommand{\TeX}{\textrm{\Oldtex}}
    \renewcommand{\LaTeX}{\textrm{\Oldlatex}}
    % Document parameters
    % Document title
    \title{Tarea\_10\_Optimizacion}
    
    
    
    
    
% Pygments definitions
\makeatletter
\def\PY@reset{\let\PY@it=\relax \let\PY@bf=\relax%
    \let\PY@ul=\relax \let\PY@tc=\relax%
    \let\PY@bc=\relax \let\PY@ff=\relax}
\def\PY@tok#1{\csname PY@tok@#1\endcsname}
\def\PY@toks#1+{\ifx\relax#1\empty\else%
    \PY@tok{#1}\expandafter\PY@toks\fi}
\def\PY@do#1{\PY@bc{\PY@tc{\PY@ul{%
    \PY@it{\PY@bf{\PY@ff{#1}}}}}}}
\def\PY#1#2{\PY@reset\PY@toks#1+\relax+\PY@do{#2}}

\expandafter\def\csname PY@tok@w\endcsname{\def\PY@tc##1{\textcolor[rgb]{0.73,0.73,0.73}{##1}}}
\expandafter\def\csname PY@tok@c\endcsname{\let\PY@it=\textit\def\PY@tc##1{\textcolor[rgb]{0.25,0.50,0.50}{##1}}}
\expandafter\def\csname PY@tok@cp\endcsname{\def\PY@tc##1{\textcolor[rgb]{0.74,0.48,0.00}{##1}}}
\expandafter\def\csname PY@tok@k\endcsname{\let\PY@bf=\textbf\def\PY@tc##1{\textcolor[rgb]{0.00,0.50,0.00}{##1}}}
\expandafter\def\csname PY@tok@kp\endcsname{\def\PY@tc##1{\textcolor[rgb]{0.00,0.50,0.00}{##1}}}
\expandafter\def\csname PY@tok@kt\endcsname{\def\PY@tc##1{\textcolor[rgb]{0.69,0.00,0.25}{##1}}}
\expandafter\def\csname PY@tok@o\endcsname{\def\PY@tc##1{\textcolor[rgb]{0.40,0.40,0.40}{##1}}}
\expandafter\def\csname PY@tok@ow\endcsname{\let\PY@bf=\textbf\def\PY@tc##1{\textcolor[rgb]{0.67,0.13,1.00}{##1}}}
\expandafter\def\csname PY@tok@nb\endcsname{\def\PY@tc##1{\textcolor[rgb]{0.00,0.50,0.00}{##1}}}
\expandafter\def\csname PY@tok@nf\endcsname{\def\PY@tc##1{\textcolor[rgb]{0.00,0.00,1.00}{##1}}}
\expandafter\def\csname PY@tok@nc\endcsname{\let\PY@bf=\textbf\def\PY@tc##1{\textcolor[rgb]{0.00,0.00,1.00}{##1}}}
\expandafter\def\csname PY@tok@nn\endcsname{\let\PY@bf=\textbf\def\PY@tc##1{\textcolor[rgb]{0.00,0.00,1.00}{##1}}}
\expandafter\def\csname PY@tok@ne\endcsname{\let\PY@bf=\textbf\def\PY@tc##1{\textcolor[rgb]{0.82,0.25,0.23}{##1}}}
\expandafter\def\csname PY@tok@nv\endcsname{\def\PY@tc##1{\textcolor[rgb]{0.10,0.09,0.49}{##1}}}
\expandafter\def\csname PY@tok@no\endcsname{\def\PY@tc##1{\textcolor[rgb]{0.53,0.00,0.00}{##1}}}
\expandafter\def\csname PY@tok@nl\endcsname{\def\PY@tc##1{\textcolor[rgb]{0.63,0.63,0.00}{##1}}}
\expandafter\def\csname PY@tok@ni\endcsname{\let\PY@bf=\textbf\def\PY@tc##1{\textcolor[rgb]{0.60,0.60,0.60}{##1}}}
\expandafter\def\csname PY@tok@na\endcsname{\def\PY@tc##1{\textcolor[rgb]{0.49,0.56,0.16}{##1}}}
\expandafter\def\csname PY@tok@nt\endcsname{\let\PY@bf=\textbf\def\PY@tc##1{\textcolor[rgb]{0.00,0.50,0.00}{##1}}}
\expandafter\def\csname PY@tok@nd\endcsname{\def\PY@tc##1{\textcolor[rgb]{0.67,0.13,1.00}{##1}}}
\expandafter\def\csname PY@tok@s\endcsname{\def\PY@tc##1{\textcolor[rgb]{0.73,0.13,0.13}{##1}}}
\expandafter\def\csname PY@tok@sd\endcsname{\let\PY@it=\textit\def\PY@tc##1{\textcolor[rgb]{0.73,0.13,0.13}{##1}}}
\expandafter\def\csname PY@tok@si\endcsname{\let\PY@bf=\textbf\def\PY@tc##1{\textcolor[rgb]{0.73,0.40,0.53}{##1}}}
\expandafter\def\csname PY@tok@se\endcsname{\let\PY@bf=\textbf\def\PY@tc##1{\textcolor[rgb]{0.73,0.40,0.13}{##1}}}
\expandafter\def\csname PY@tok@sr\endcsname{\def\PY@tc##1{\textcolor[rgb]{0.73,0.40,0.53}{##1}}}
\expandafter\def\csname PY@tok@ss\endcsname{\def\PY@tc##1{\textcolor[rgb]{0.10,0.09,0.49}{##1}}}
\expandafter\def\csname PY@tok@sx\endcsname{\def\PY@tc##1{\textcolor[rgb]{0.00,0.50,0.00}{##1}}}
\expandafter\def\csname PY@tok@m\endcsname{\def\PY@tc##1{\textcolor[rgb]{0.40,0.40,0.40}{##1}}}
\expandafter\def\csname PY@tok@gh\endcsname{\let\PY@bf=\textbf\def\PY@tc##1{\textcolor[rgb]{0.00,0.00,0.50}{##1}}}
\expandafter\def\csname PY@tok@gu\endcsname{\let\PY@bf=\textbf\def\PY@tc##1{\textcolor[rgb]{0.50,0.00,0.50}{##1}}}
\expandafter\def\csname PY@tok@gd\endcsname{\def\PY@tc##1{\textcolor[rgb]{0.63,0.00,0.00}{##1}}}
\expandafter\def\csname PY@tok@gi\endcsname{\def\PY@tc##1{\textcolor[rgb]{0.00,0.63,0.00}{##1}}}
\expandafter\def\csname PY@tok@gr\endcsname{\def\PY@tc##1{\textcolor[rgb]{1.00,0.00,0.00}{##1}}}
\expandafter\def\csname PY@tok@ge\endcsname{\let\PY@it=\textit}
\expandafter\def\csname PY@tok@gs\endcsname{\let\PY@bf=\textbf}
\expandafter\def\csname PY@tok@gp\endcsname{\let\PY@bf=\textbf\def\PY@tc##1{\textcolor[rgb]{0.00,0.00,0.50}{##1}}}
\expandafter\def\csname PY@tok@go\endcsname{\def\PY@tc##1{\textcolor[rgb]{0.53,0.53,0.53}{##1}}}
\expandafter\def\csname PY@tok@gt\endcsname{\def\PY@tc##1{\textcolor[rgb]{0.00,0.27,0.87}{##1}}}
\expandafter\def\csname PY@tok@err\endcsname{\def\PY@bc##1{\setlength{\fboxsep}{0pt}\fcolorbox[rgb]{1.00,0.00,0.00}{1,1,1}{\strut ##1}}}
\expandafter\def\csname PY@tok@kc\endcsname{\let\PY@bf=\textbf\def\PY@tc##1{\textcolor[rgb]{0.00,0.50,0.00}{##1}}}
\expandafter\def\csname PY@tok@kd\endcsname{\let\PY@bf=\textbf\def\PY@tc##1{\textcolor[rgb]{0.00,0.50,0.00}{##1}}}
\expandafter\def\csname PY@tok@kn\endcsname{\let\PY@bf=\textbf\def\PY@tc##1{\textcolor[rgb]{0.00,0.50,0.00}{##1}}}
\expandafter\def\csname PY@tok@kr\endcsname{\let\PY@bf=\textbf\def\PY@tc##1{\textcolor[rgb]{0.00,0.50,0.00}{##1}}}
\expandafter\def\csname PY@tok@bp\endcsname{\def\PY@tc##1{\textcolor[rgb]{0.00,0.50,0.00}{##1}}}
\expandafter\def\csname PY@tok@fm\endcsname{\def\PY@tc##1{\textcolor[rgb]{0.00,0.00,1.00}{##1}}}
\expandafter\def\csname PY@tok@vc\endcsname{\def\PY@tc##1{\textcolor[rgb]{0.10,0.09,0.49}{##1}}}
\expandafter\def\csname PY@tok@vg\endcsname{\def\PY@tc##1{\textcolor[rgb]{0.10,0.09,0.49}{##1}}}
\expandafter\def\csname PY@tok@vi\endcsname{\def\PY@tc##1{\textcolor[rgb]{0.10,0.09,0.49}{##1}}}
\expandafter\def\csname PY@tok@vm\endcsname{\def\PY@tc##1{\textcolor[rgb]{0.10,0.09,0.49}{##1}}}
\expandafter\def\csname PY@tok@sa\endcsname{\def\PY@tc##1{\textcolor[rgb]{0.73,0.13,0.13}{##1}}}
\expandafter\def\csname PY@tok@sb\endcsname{\def\PY@tc##1{\textcolor[rgb]{0.73,0.13,0.13}{##1}}}
\expandafter\def\csname PY@tok@sc\endcsname{\def\PY@tc##1{\textcolor[rgb]{0.73,0.13,0.13}{##1}}}
\expandafter\def\csname PY@tok@dl\endcsname{\def\PY@tc##1{\textcolor[rgb]{0.73,0.13,0.13}{##1}}}
\expandafter\def\csname PY@tok@s2\endcsname{\def\PY@tc##1{\textcolor[rgb]{0.73,0.13,0.13}{##1}}}
\expandafter\def\csname PY@tok@sh\endcsname{\def\PY@tc##1{\textcolor[rgb]{0.73,0.13,0.13}{##1}}}
\expandafter\def\csname PY@tok@s1\endcsname{\def\PY@tc##1{\textcolor[rgb]{0.73,0.13,0.13}{##1}}}
\expandafter\def\csname PY@tok@mb\endcsname{\def\PY@tc##1{\textcolor[rgb]{0.40,0.40,0.40}{##1}}}
\expandafter\def\csname PY@tok@mf\endcsname{\def\PY@tc##1{\textcolor[rgb]{0.40,0.40,0.40}{##1}}}
\expandafter\def\csname PY@tok@mh\endcsname{\def\PY@tc##1{\textcolor[rgb]{0.40,0.40,0.40}{##1}}}
\expandafter\def\csname PY@tok@mi\endcsname{\def\PY@tc##1{\textcolor[rgb]{0.40,0.40,0.40}{##1}}}
\expandafter\def\csname PY@tok@il\endcsname{\def\PY@tc##1{\textcolor[rgb]{0.40,0.40,0.40}{##1}}}
\expandafter\def\csname PY@tok@mo\endcsname{\def\PY@tc##1{\textcolor[rgb]{0.40,0.40,0.40}{##1}}}
\expandafter\def\csname PY@tok@ch\endcsname{\let\PY@it=\textit\def\PY@tc##1{\textcolor[rgb]{0.25,0.50,0.50}{##1}}}
\expandafter\def\csname PY@tok@cm\endcsname{\let\PY@it=\textit\def\PY@tc##1{\textcolor[rgb]{0.25,0.50,0.50}{##1}}}
\expandafter\def\csname PY@tok@cpf\endcsname{\let\PY@it=\textit\def\PY@tc##1{\textcolor[rgb]{0.25,0.50,0.50}{##1}}}
\expandafter\def\csname PY@tok@c1\endcsname{\let\PY@it=\textit\def\PY@tc##1{\textcolor[rgb]{0.25,0.50,0.50}{##1}}}
\expandafter\def\csname PY@tok@cs\endcsname{\let\PY@it=\textit\def\PY@tc##1{\textcolor[rgb]{0.25,0.50,0.50}{##1}}}

\def\PYZbs{\char`\\}
\def\PYZus{\char`\_}
\def\PYZob{\char`\{}
\def\PYZcb{\char`\}}
\def\PYZca{\char`\^}
\def\PYZam{\char`\&}
\def\PYZlt{\char`\<}
\def\PYZgt{\char`\>}
\def\PYZsh{\char`\#}
\def\PYZpc{\char`\%}
\def\PYZdl{\char`\$}
\def\PYZhy{\char`\-}
\def\PYZsq{\char`\'}
\def\PYZdq{\char`\"}
\def\PYZti{\char`\~}
% for compatibility with earlier versions
\def\PYZat{@}
\def\PYZlb{[}
\def\PYZrb{]}
\makeatother


    % For linebreaks inside Verbatim environment from package fancyvrb. 
    \makeatletter
        \newbox\Wrappedcontinuationbox 
        \newbox\Wrappedvisiblespacebox 
        \newcommand*\Wrappedvisiblespace {\textcolor{red}{\textvisiblespace}} 
        \newcommand*\Wrappedcontinuationsymbol {\textcolor{red}{\llap{\tiny$\m@th\hookrightarrow$}}} 
        \newcommand*\Wrappedcontinuationindent {3ex } 
        \newcommand*\Wrappedafterbreak {\kern\Wrappedcontinuationindent\copy\Wrappedcontinuationbox} 
        % Take advantage of the already applied Pygments mark-up to insert 
        % potential linebreaks for TeX processing. 
        %        {, <, #, %, $, ' and ": go to next line. 
        %        _, }, ^, &, >, - and ~: stay at end of broken line. 
        % Use of \textquotesingle for straight quote. 
        \newcommand*\Wrappedbreaksatspecials {% 
            \def\PYGZus{\discretionary{\char`\_}{\Wrappedafterbreak}{\char`\_}}% 
            \def\PYGZob{\discretionary{}{\Wrappedafterbreak\char`\{}{\char`\{}}% 
            \def\PYGZcb{\discretionary{\char`\}}{\Wrappedafterbreak}{\char`\}}}% 
            \def\PYGZca{\discretionary{\char`\^}{\Wrappedafterbreak}{\char`\^}}% 
            \def\PYGZam{\discretionary{\char`\&}{\Wrappedafterbreak}{\char`\&}}% 
            \def\PYGZlt{\discretionary{}{\Wrappedafterbreak\char`\<}{\char`\<}}% 
            \def\PYGZgt{\discretionary{\char`\>}{\Wrappedafterbreak}{\char`\>}}% 
            \def\PYGZsh{\discretionary{}{\Wrappedafterbreak\char`\#}{\char`\#}}% 
            \def\PYGZpc{\discretionary{}{\Wrappedafterbreak\char`\%}{\char`\%}}% 
            \def\PYGZdl{\discretionary{}{\Wrappedafterbreak\char`\$}{\char`\$}}% 
            \def\PYGZhy{\discretionary{\char`\-}{\Wrappedafterbreak}{\char`\-}}% 
            \def\PYGZsq{\discretionary{}{\Wrappedafterbreak\textquotesingle}{\textquotesingle}}% 
            \def\PYGZdq{\discretionary{}{\Wrappedafterbreak\char`\"}{\char`\"}}% 
            \def\PYGZti{\discretionary{\char`\~}{\Wrappedafterbreak}{\char`\~}}% 
        } 
        % Some characters . , ; ? ! / are not pygmentized. 
        % This macro makes them "active" and they will insert potential linebreaks 
        \newcommand*\Wrappedbreaksatpunct {% 
            \lccode`\~`\.\lowercase{\def~}{\discretionary{\hbox{\char`\.}}{\Wrappedafterbreak}{\hbox{\char`\.}}}% 
            \lccode`\~`\,\lowercase{\def~}{\discretionary{\hbox{\char`\,}}{\Wrappedafterbreak}{\hbox{\char`\,}}}% 
            \lccode`\~`\;\lowercase{\def~}{\discretionary{\hbox{\char`\;}}{\Wrappedafterbreak}{\hbox{\char`\;}}}% 
            \lccode`\~`\:\lowercase{\def~}{\discretionary{\hbox{\char`\:}}{\Wrappedafterbreak}{\hbox{\char`\:}}}% 
            \lccode`\~`\?\lowercase{\def~}{\discretionary{\hbox{\char`\?}}{\Wrappedafterbreak}{\hbox{\char`\?}}}% 
            \lccode`\~`\!\lowercase{\def~}{\discretionary{\hbox{\char`\!}}{\Wrappedafterbreak}{\hbox{\char`\!}}}% 
            \lccode`\~`\/\lowercase{\def~}{\discretionary{\hbox{\char`\/}}{\Wrappedafterbreak}{\hbox{\char`\/}}}% 
            \catcode`\.\active
            \catcode`\,\active 
            \catcode`\;\active
            \catcode`\:\active
            \catcode`\?\active
            \catcode`\!\active
            \catcode`\/\active 
            \lccode`\~`\~ 	
        }
    \makeatother

    \let\OriginalVerbatim=\Verbatim
    \makeatletter
    \renewcommand{\Verbatim}[1][1]{%
        %\parskip\z@skip
        \sbox\Wrappedcontinuationbox {\Wrappedcontinuationsymbol}%
        \sbox\Wrappedvisiblespacebox {\FV@SetupFont\Wrappedvisiblespace}%
        \def\FancyVerbFormatLine ##1{\hsize\linewidth
            \vtop{\raggedright\hyphenpenalty\z@\exhyphenpenalty\z@
                \doublehyphendemerits\z@\finalhyphendemerits\z@
                \strut ##1\strut}%
        }%
        % If the linebreak is at a space, the latter will be displayed as visible
        % space at end of first line, and a continuation symbol starts next line.
        % Stretch/shrink are however usually zero for typewriter font.
        \def\FV@Space {%
            \nobreak\hskip\z@ plus\fontdimen3\font minus\fontdimen4\font
            \discretionary{\copy\Wrappedvisiblespacebox}{\Wrappedafterbreak}
            {\kern\fontdimen2\font}%
        }%
        
        % Allow breaks at special characters using \PYG... macros.
        \Wrappedbreaksatspecials
        % Breaks at punctuation characters . , ; ? ! and / need catcode=\active 	
        \OriginalVerbatim[#1,codes*=\Wrappedbreaksatpunct]%
    }
    \makeatother

    % Exact colors from NB
    \definecolor{incolor}{HTML}{303F9F}
    \definecolor{outcolor}{HTML}{D84315}
    \definecolor{cellborder}{HTML}{CFCFCF}
    \definecolor{cellbackground}{HTML}{F7F7F7}
    
    % prompt
    \makeatletter
    \newcommand{\boxspacing}{\kern\kvtcb@left@rule\kern\kvtcb@boxsep}
    \makeatother
    \newcommand{\prompt}[4]{
        \ttfamily\llap{{\color{#2}[#3]:\hspace{3pt}#4}}\vspace{-\baselineskip}
    }
    

    
    % Prevent overflowing lines due to hard-to-break entities
    \sloppy 
    % Setup hyperref package
    \hypersetup{
      breaklinks=true,  % so long urls are correctly broken across lines
      colorlinks=true,
      urlcolor=urlcolor,
      linkcolor=linkcolor,
      citecolor=citecolor,
      }
    % Slightly bigger margins than the latex defaults
    
    \geometry{verbose,tmargin=1in,bmargin=1in,lmargin=1in,rmargin=1in}
    
    

\begin{document}
    
\title{Tarea 10 Optimización}
\author{Roberto Vásquez Martínez \\ Profesor: Joaquín Peña Acevedo}
\date{17/Mayo/2022}
\maketitle 
    
    

    
    \hypertarget{ejercicio-1-4-puntos}{%
\section{Ejercicio 1 (4 puntos)}\label{ejercicio-1-4-puntos}}

\begin{enumerate}
\def\labelenumi{\arabic{enumi}.}
\tightlist
\item
  Escriba la descripción del tema para el proyecto final del curso.
\item
  La descripción no tiene que ser detallada. Sólo debe quedar claro cual
  el problema que quiere resolver, si ya cuentan con la información para
  resolver el problema (datos, referencia bibliográfica, etc.)
\item
  Mencione el tipo de pruebas que va a realizar y la manera en que va a
  validar los resultados.
\item
  En la semana de 16 de mayo recibirá un mensaje que indica si el tema
  fue aceptado o necesita precisar algo o cambiarlo.
\end{enumerate}

    \hypertarget{descripciuxf3n-del-proyecto}{%
\subsection{Descripción del
proyecto}\label{descripciuxf3n-del-proyecto}}

A continuación, describiré los componentes principales de mi propuesta
de Proyecto Final para el curso de Optimización.

\textbf{Título:} Optimización en el caso no diferenciable:
Subdiferenciales.

El propósito del proyecto será dar una introducción al caso de
optimización cuando la función objetivo \(f\) es continua pero no
diferenciable. El caso en el cual \(f\in C^1\) o \(f\in C^2\) se
discutió en el curso con los métodos Newton/Quasi-Newton y de búsqueda
en línea.

En primer lugar, generalizaremos el concepto de gradiente al de
\emph{subgradiente} o subdiferencial en el caso cuando \(f\) es una
función convexa. Se pretende desarrollar un poco de teoría de esta
generalización y probar la condición de optimalidad en términos de
subdiferenciales.

Para ilustrar el problema de optimización en el caso no diferenciable
proponemos resolver el problema de regresión LASSO. Este problema se
puede resolver utilizando \emph{descenso por coordenadas} y se pueden
verificar las condiciones de optimalidad usando el subdiferencial de la
función objetivo LASSO. Además, presentaremos el caso particular con un
solo predictor para determinar la solución LASSO a través de
subdiferenciales.

Para el caso multipredictores utilizaremos el conjunto de datos
\emph{crime data: crime rate and five predictors} tomado del libro
\emph{Statistical Learning with Sparsity: The LASSO and
Generalizations}.

La teoría sobre cálculo subdiferencial será tomada del libro
\emph{Convex Analysis and Minimization Algorithms I: Fundamentals}.

    \hypertarget{ejercicio-2-3-puntos}{%
\section{Ejercicio 2 (3 puntos)}\label{ejercicio-2-3-puntos}}

Considere el ejemplo visto en clase:

\[
\begin{array}{rl}
\max & x_1 + x_2\\
     & 50x_1 + 24x_2 \leq 2400 \\
     & 30x_1 + 33x_2 \leq 2100 \\
     & x_1 \geq 45 \\
     & x_2 \geq 5
\end{array}
\]

Vimos que se puede escribir en forma estándar como:

\[
\begin{array}{rl}
\min & -x_1 - x_2\\
     & 50x_1 + 24x_2 + x_3 =  2400 \\
     & 30x_1 + 33x_2 + x_4 =  2100 \\
     & x_1 - x_5 = 45 \\
     & x_2 - x_6 =  5 \\
     & x_1, x_2, x_3, x_4, x_5, x_6 \geq 0
\end{array}
\]

Puede usar el código del Notebook
\textbf{ejemploPuntosBasicosFactibles.ipynb} para obtener la solución
\(\mathbf{x}_*\) del problema en forma estándar.

Las condiciones KKT son:

\[
\begin{array}{rclc}
  \mathbf{A}^\top \lambda + \mathbf{s} &=& \mathbf{c}, & (1) \\
  \mathbf{A}\mathbf{x} &=& \mathbf{b}, & (2) \\
  \mathbf{x} & \geq & \mathbf{0}, & (3)  \\
  \mathbf{s} & \geq & \mathbf{0}, & (4)  \\
  x_i s_i &=& 0, \qquad i=1,2,...,n. & (5)
\end{array}
\]

Debe ser claro que por la manera en que se calculó \(\mathbf{x}_*\) en
el ejemplo de la clase, se cumplen las condiciones (2) y (3).

\begin{enumerate}
\def\labelenumi{\arabic{enumi}.}
\tightlist
\item
  Dado \(\mathbf{x}_*\) y por la condición de complementaridad (5),
  sabemos cuáles son las componentes de \(\mathbf{s}\) que son cero y
  cuáles deben ser calculadas. Use eso y la condición (1) para calcular
  \(\lambda\) y las componentes de \(\mathbf{s}\) desconocidas.
  Imprima los vectores \(\lambda\) y \(\mathbf{s}\).
\item
  Verique que se cumplen la condiciones (4) y (5), y con esto se
  comprueba que \(\mathbf{x}_*\) es solución del problema estándar.
\item
  Calcule el valor
\end{enumerate}

\[\mathbf{b}^\top \lambda \]

y compare este valor con el valor de la función objetivo
\(\mathbf{c}^\top \mathbf{x}_*\).

\hypertarget{soluciuxf3n}{%
\subsubsection{Solución:}\label{soluciuxf3n}}

    \begin{tcolorbox}[breakable, size=fbox, boxrule=1pt, pad at break*=1mm,colback=cellbackground, colframe=cellborder]
\prompt{In}{incolor}{14}{\boxspacing}
\begin{Verbatim}[commandchars=\\\{\}]
\PY{c+c1}{\PYZsh{} Pongo el codigo para obtener la solución x que calculamos en clase}

\PY{k+kn}{import} \PY{n+nn}{numpy} \PY{k}{as} \PY{n+nn}{np}
\PY{k+kn}{from} \PY{n+nn}{itertools} \PY{k+kn}{import} \PY{n}{combinations}

\PY{n}{c} \PY{o}{=} \PY{n}{np}\PY{o}{.}\PY{n}{array}\PY{p}{(}\PY{p}{[}\PY{o}{\PYZhy{}}\PY{l+m+mi}{1}\PY{p}{,} \PY{o}{\PYZhy{}}\PY{l+m+mi}{1}\PY{p}{,} \PY{l+m+mi}{0}\PY{p}{,} \PY{l+m+mi}{0}\PY{p}{,} \PY{l+m+mi}{0}\PY{p}{,} \PY{l+m+mi}{0} \PY{p}{]}\PY{p}{)}
\PY{n}{b} \PY{o}{=} \PY{n}{np}\PY{o}{.}\PY{n}{array}\PY{p}{(}\PY{p}{[}\PY{l+m+mi}{2400}\PY{p}{,} \PY{l+m+mi}{2100}\PY{p}{,} \PY{l+m+mi}{45}\PY{p}{,} \PY{l+m+mi}{5}\PY{p}{]}\PY{p}{)}
\PY{n}{A} \PY{o}{=} \PY{n}{np}\PY{o}{.}\PY{n}{array}\PY{p}{(}\PY{p}{[}\PY{p}{[}\PY{l+m+mi}{50}\PY{p}{,} \PY{l+m+mi}{24}\PY{p}{,} \PY{l+m+mi}{1}\PY{p}{,} \PY{l+m+mi}{0}\PY{p}{,} \PY{l+m+mi}{0}\PY{p}{,} \PY{l+m+mi}{0}\PY{p}{]}\PY{p}{,}
              \PY{p}{[}\PY{l+m+mi}{30}\PY{p}{,} \PY{l+m+mi}{33}\PY{p}{,} \PY{l+m+mi}{0}\PY{p}{,} \PY{l+m+mi}{1}\PY{p}{,} \PY{l+m+mi}{0}\PY{p}{,} \PY{l+m+mi}{0}\PY{p}{]}\PY{p}{,}
              \PY{p}{[} \PY{l+m+mi}{1}\PY{p}{,}  \PY{l+m+mi}{0}\PY{p}{,} \PY{l+m+mi}{0}\PY{p}{,} \PY{l+m+mi}{0}\PY{p}{,}\PY{o}{\PYZhy{}}\PY{l+m+mi}{1}\PY{p}{,} \PY{l+m+mi}{0}\PY{p}{]}\PY{p}{,}
              \PY{p}{[} \PY{l+m+mi}{0}\PY{p}{,}  \PY{l+m+mi}{1}\PY{p}{,} \PY{l+m+mi}{0}\PY{p}{,} \PY{l+m+mi}{0}\PY{p}{,} \PY{l+m+mi}{0}\PY{p}{,}\PY{o}{\PYZhy{}}\PY{l+m+mi}{1}\PY{p}{]} \PY{p}{]}\PY{p}{)}

\PY{n}{m}\PY{p}{,}\PY{n}{n} \PY{o}{=} \PY{n}{A}\PY{o}{.}\PY{n}{shape}
\PY{n}{comb} \PY{o}{=} \PY{n+nb}{list}\PY{p}{(}\PY{n}{combinations}\PY{p}{(}\PY{n+nb}{list}\PY{p}{(}\PY{n+nb}{range}\PY{p}{(}\PY{n}{n}\PY{p}{)}\PY{p}{)}\PY{p}{,} \PY{n}{m}\PY{p}{)}\PY{p}{)}
\PY{n+nb}{print}\PY{p}{(}\PY{l+s+s1}{\PYZsq{}}\PY{l+s+s1}{Número de combinaciones:}\PY{l+s+s1}{\PYZsq{}}\PY{p}{,} \PY{n+nb}{len}\PY{p}{(}\PY{n}{comb}\PY{p}{)}\PY{p}{)}

\PY{n}{dmin} \PY{o}{=} \PY{k+kc}{None}
\PY{k}{for} \PY{n}{icols} \PY{o+ow}{in} \PY{n}{comb}\PY{p}{:}
    \PY{c+c1}{\PYZsh{} Indices de las columnas seleccionadas}
    \PY{n}{jj} \PY{o}{=} \PY{n+nb}{list}\PY{p}{(}\PY{n}{icols}\PY{p}{)}
    \PY{c+c1}{\PYZsh{} Matriz básica}
    \PY{n}{B}     \PY{o}{=} \PY{n}{A}\PY{p}{[}\PY{p}{:}\PY{p}{,} \PY{n}{jj}\PY{p}{]}
    \PY{n}{condB} \PY{o}{=} \PY{n}{np}\PY{o}{.}\PY{n}{linalg}\PY{o}{.}\PY{n}{cond}\PY{p}{(}\PY{n}{B}\PY{p}{)}
    \PY{k}{if} \PY{n}{condB}\PY{o}{\PYZgt{}}\PY{l+m+mf}{1.0e14}\PY{p}{:}
        \PY{n+nb}{print}\PY{p}{(}\PY{l+s+s1}{\PYZsq{}}\PY{l+s+s1}{Es casi singular la matriz B con columnas}\PY{l+s+s1}{\PYZsq{}}\PY{p}{,} \PY{n}{jj}\PY{p}{)}
    \PY{k}{else}\PY{p}{:}
        \PY{c+c1}{\PYZsh{} Solucion del sistema B*x=b}
        \PY{n}{xb} \PY{o}{=} \PY{n}{np}\PY{o}{.}\PY{n}{linalg}\PY{o}{.}\PY{n}{solve}\PY{p}{(}\PY{n}{B}\PY{p}{,} \PY{n}{b}\PY{p}{)}
        \PY{c+c1}{\PYZsh{} Solucion del problema en forma estándar}
        \PY{n}{x}     \PY{o}{=} \PY{n}{np}\PY{o}{.}\PY{n}{zeros}\PY{p}{(}\PY{n}{n}\PY{p}{)}
        \PY{n}{x}\PY{p}{[}\PY{n}{jj}\PY{p}{]} \PY{o}{=} \PY{n}{xb}
        \PY{c+c1}{\PYZsh{} Evaluación de la función objetivo}
        \PY{n}{f} \PY{o}{=} \PY{n}{np}\PY{o}{.}\PY{n}{vdot}\PY{p}{(}\PY{n}{c}\PY{p}{,} \PY{n}{x}\PY{p}{)}
        \PY{c+c1}{\PYZsh{} Se revisa si es vector x es factible. Claramente se cumple que A*x=b,}
        \PY{c+c1}{\PYZsh{} pero hay que verificar que x\PYZgt{}=0.}
        \PY{n}{smsg}  \PY{o}{=} \PY{l+s+s1}{\PYZsq{}}\PY{l+s+s1}{No factible}\PY{l+s+s1}{\PYZsq{}}
        \PY{n}{bfact} \PY{o}{=} \PY{k+kc}{False} 
        \PY{k}{if} \PY{n+nb}{sum}\PY{p}{(}\PY{n}{x}\PY{o}{\PYZgt{}}\PY{o}{=}\PY{l+m+mi}{0}\PY{p}{)}\PY{o}{==}\PY{n+nb}{len}\PY{p}{(}\PY{n}{x}\PY{p}{)}\PY{p}{:}
            \PY{n}{bfact} \PY{o}{=} \PY{k+kc}{True}
            \PY{n}{smsg}  \PY{o}{=} \PY{l+s+s1}{\PYZsq{}}\PY{l+s+s1}{Factible}\PY{l+s+s1}{\PYZsq{}}
        \PY{k}{if} \PY{n}{bfact}\PY{p}{:}
            \PY{c+c1}{\PYZsh{} Si x es factible, almacenamos en xsol el punto x donde f es mínima}
            \PY{k}{if} \PY{n}{dmin}\PY{o}{==}\PY{k+kc}{None}\PY{p}{:}
                \PY{n}{dmin} \PY{o}{=} \PY{n}{f}
                \PY{n}{xsol} \PY{o}{=} \PY{n}{x}\PY{o}{.}\PY{n}{copy}\PY{p}{(}\PY{p}{)}
            \PY{k}{elif} \PY{n}{dmin}\PY{o}{\PYZgt{}}\PY{n}{f}\PY{p}{:}
                \PY{n}{dmin} \PY{o}{=} \PY{n}{f}
                \PY{n}{xsol} \PY{o}{=} \PY{n}{x}\PY{o}{.}\PY{n}{copy}\PY{p}{(}\PY{p}{)}
        \PY{n+nb}{print}\PY{p}{(}\PY{l+s+s2}{\PYZdq{}}\PY{l+s+si}{\PYZpc{}6.1f}\PY{l+s+s2}{ }\PY{l+s+si}{\PYZpc{}7.1f}\PY{l+s+s2}{| }\PY{l+s+si}{\PYZpc{} 8.2f}\PY{l+s+s2}{ }\PY{l+s+si}{\PYZpc{} 8.2f}\PY{l+s+s2}{ }\PY{l+s+si}{\PYZpc{} 8.2f}\PY{l+s+s2}{ }\PY{l+s+si}{\PYZpc{} 8.2f}\PY{l+s+s2}{ }\PY{l+s+si}{\PYZpc{} 8.2f}\PY{l+s+s2}{ }\PY{l+s+si}{\PYZpc{} 8.2f}\PY{l+s+s2}{| }\PY{l+s+si}{\PYZpc{}s}\PY{l+s+s2}{\PYZdq{}} \PY{o}{\PYZpc{}} \PY{p}{(}\PY{n}{condB}\PY{p}{,} 
                          \PY{n}{f}\PY{p}{,}  \PY{n}{x}\PY{p}{[}\PY{l+m+mi}{0}\PY{p}{]}\PY{p}{,} \PY{n}{x}\PY{p}{[}\PY{l+m+mi}{1}\PY{p}{]}\PY{p}{,} \PY{n}{x}\PY{p}{[}\PY{l+m+mi}{2}\PY{p}{]}\PY{p}{,} \PY{n}{x}\PY{p}{[}\PY{l+m+mi}{3}\PY{p}{]}\PY{p}{,} \PY{n}{x}\PY{p}{[}\PY{l+m+mi}{4}\PY{p}{]}\PY{p}{,} \PY{n}{x}\PY{p}{[}\PY{l+m+mi}{5}\PY{p}{]}\PY{p}{,} \PY{n}{smsg}\PY{p}{)}\PY{p}{)}
        
\PY{c+c1}{\PYZsh{} Fijamos una tolerancia y hacemos cero las componentes de x que son menores que la tolerancia        }
\PY{n}{tol} \PY{o}{=} \PY{p}{(}\PY{n}{np}\PY{o}{.}\PY{n}{finfo}\PY{p}{(}\PY{n+nb}{float}\PY{p}{)}\PY{o}{.}\PY{n}{eps}\PY{p}{)}\PY{o}{*}\PY{o}{*}\PY{p}{(}\PY{l+m+mf}{3.0}\PY{o}{/}\PY{l+m+mi}{4}\PY{p}{)}
\PY{n}{ii}  \PY{o}{=} \PY{n}{np}\PY{o}{.}\PY{n}{where}\PY{p}{(}\PY{n}{xsol}\PY{o}{\PYZlt{}}\PY{n}{tol}\PY{p}{)}\PY{p}{[}\PY{l+m+mi}{0}\PY{p}{]}
\PY{n}{xsol}\PY{p}{[}\PY{n}{ii}\PY{p}{]} \PY{o}{=} \PY{l+m+mf}{0.0}

\PY{n+nb}{print}\PY{p}{(}\PY{l+s+s1}{\PYZsq{}}\PY{l+s+se}{\PYZbs{}n}\PY{l+s+s1}{Solución del problema estándar:}\PY{l+s+s1}{\PYZsq{}}\PY{p}{)}
\PY{n+nb}{print}\PY{p}{(}\PY{l+s+s1}{\PYZsq{}}\PY{l+s+s1}{x*=}\PY{l+s+s1}{\PYZsq{}}\PY{p}{,} \PY{n}{xsol}\PY{p}{)}
\PY{n+nb}{print}\PY{p}{(}\PY{l+s+s1}{\PYZsq{}}\PY{l+s+s1}{Valor de la función objetivo en x*=}\PY{l+s+s1}{\PYZsq{}}\PY{p}{,} \PY{n}{dmin}\PY{p}{)}
\end{Verbatim}
\end{tcolorbox}

    \begin{Verbatim}[commandchars=\\\{\}]
Número de combinaciones: 15
4890.1   -50.0|    45.00     5.00    30.00   585.00     0.00     0.00| Factible
2175.8   -69.5|    64.50     5.00  -945.00     0.00    19.50     0.00| No
factible
1975.8   -67.7|    45.00    22.73  -395.45     0.00     0.00    17.73| No
factible
1305.9   -50.6|    45.60     5.00     0.00   567.00     0.60     0.00| Factible
2720.9   -51.2|    45.00     6.25     0.00   543.75     0.00     1.25| Factible
  70.1   -66.5|    30.97    35.48     0.00     0.00   -14.03    30.48| No
factible
Es casi singular la matriz B con columnas [0, 2, 3, 4]
3402.0   -45.0|    45.00     0.00   150.00   750.00     0.00    -5.00| No
factible
 113.4   -70.0|    70.00     0.00 -1100.00     0.00    25.00    -5.00| No
factible
  68.0   -48.0|    48.00     0.00     0.00   660.00     3.00    -5.00| No
factible
1667.0    -5.0|     0.00     5.00  2280.00  1935.00   -45.00     0.00| No
factible
Es casi singular la matriz B con columnas [1, 2, 3, 5]
  50.5   -63.6|     0.00    63.64   872.73     0.00   -45.00    58.64| No
factible
  69.4  -100.0|     0.00   100.00     0.00 -1200.00   -45.00    95.00| No
factible
   1.0     0.0|     0.00     0.00  2400.00  2100.00   -45.00    -5.00| No
factible

Solución del problema estándar:
x*= [ 45.     6.25   0.   543.75   0.     1.25]
Valor de la función objetivo en x*= -51.25
    \end{Verbatim}

    Para hallar \(\lambda \in\mathbb{R}^m\) y
\(\mathbf{s}\in\lambda\mathbb{R}^n\) donde
\(\mathbf{A}=[A_1,\dots,A_n]\in\mathbb{R}^{n\times m}\) debemos recurrir
a las condiciones KKT. Hemos visto en clase que estas son necesarias y
suficientes para que \(\mathbf{x}_\ast\) sea óptimo local.

Como se deben satisfacer las condiciones de (1) a (5) para la solución
óptima \(\mathbf{x}_\ast\) encontrada en la celda de código anterior, si
\(\mathbf{B}\) es la matriz básica asociada a la base \(\mathcal{B}\) de
la solución encontrada entonces

\[\mathbf{B}^T\lambda=\widetilde{\mathbf{c}},\]

donde \(\widetilde{\mathbf{c}}=[c_i]_{i\in\mathcal{B}}\), pues por la
condición de complementariedad \(s_i=0\) para \(i\in\mathcal{B}\).

Por lo tanto, para calcular \(\lambda\) basta resolver el
sistema de ecuaciones
\(\mathbf{B}^T\lambda=\widetilde{\mathbf{c}}\).

Por otro lado, para hallar \(\mathbf{s}\), otra vez por la condición de
complementariedad basta con encontrar
\(\widehat{\mathbf{s}}=[s_i]_{i\not\in\mathcal{B}}\). Si
\(\mathbf{N}=[A_i]_{i\not\in \mathcal{B}}\) entonces por (1) se debe
cumplir que

\[ \widehat{\mathbf{s}}=\widehat{\mathbf{c}}-\mathbf{N}^T\lambda,\]

donde \(\widehat{\mathbf{c}}=[c_i]_{i\not\in\mathcal{B}}\).

A continuación, calculamos \(\lambda\) y \(\mathbf{s}\) con lo
antes descrito.

    \begin{tcolorbox}[breakable, size=fbox, boxrule=1pt, pad at break*=1mm,colback=cellbackground, colframe=cellborder]
\prompt{In}{incolor}{15}{\boxspacing}
\begin{Verbatim}[commandchars=\\\{\}]
\PY{c+c1}{\PYZsh{} Solucion del ejercicio}

\PY{n}{np}\PY{o}{.}\PY{n}{set\PYZus{}printoptions}\PY{p}{(}\PY{n}{precision}\PY{o}{=}\PY{l+m+mi}{4}\PY{p}{)}


\PY{n}{jj}\PY{o}{=}\PY{n+nb}{list}\PY{p}{(}\PY{n}{np}\PY{o}{.}\PY{n}{squeeze}\PY{p}{(}\PY{n}{np}\PY{o}{.}\PY{n}{where}\PY{p}{(}\PY{n}{xsol}\PY{o}{\PYZgt{}}\PY{l+m+mi}{0}\PY{p}{)}\PY{p}{)}\PY{p}{)}
\PY{n}{B}\PY{o}{=}\PY{n}{A}\PY{p}{[}\PY{p}{:}\PY{p}{,}\PY{n}{jj}\PY{p}{]}
\PY{n}{tilde\PYZus{}c}\PY{o}{=}\PY{n}{c}\PY{p}{[}\PY{n}{jj}\PY{p}{]}
\PY{c+c1}{\PYZsh{} B.T*lamb=tilde\PYZus{}c}
\PY{n}{lamb}\PY{o}{=}\PY{n}{np}\PY{o}{.}\PY{n}{linalg}\PY{o}{.}\PY{n}{solve}\PY{p}{(}\PY{n}{B}\PY{o}{.}\PY{n}{T}\PY{p}{,}\PY{n}{tilde\PYZus{}c}\PY{p}{)}
\PY{n+nb}{print}\PY{p}{(}\PY{l+s+s1}{\PYZsq{}}\PY{l+s+s1}{El vecto lambda es: }\PY{l+s+s1}{\PYZsq{}}\PY{p}{,}\PY{n}{lamb}\PY{p}{)}
\end{Verbatim}
\end{tcolorbox}

    \begin{Verbatim}[commandchars=\\\{\}]
El vecto lambda es:  [-0.0417  0.      1.0833 -0.    ]
    \end{Verbatim}

    Como \(\mathbf{s}\) corresponde a los multiplicadores de Lagrange de las
reestricciones de desigualdad \((\mathbf{x}\geq 0)\) entonces esperamos
que \(\mathbf{s}\geq 0\). Sabemos \(s_i=0\) para \(i\in\mathcal{B}\), en
la siguiente celda de código calculamos \(\hat{\mathbf{s}}\) que
corresponde a las entradas del vector \(\mathbf{s}\) no básicas.

    \begin{tcolorbox}[breakable, size=fbox, boxrule=1pt, pad at break*=1mm,colback=cellbackground, colframe=cellborder]
\prompt{In}{incolor}{16}{\boxspacing}
\begin{Verbatim}[commandchars=\\\{\}]
\PY{n}{s}\PY{o}{=}\PY{n}{np}\PY{o}{.}\PY{n}{zeros}\PY{p}{(}\PY{n}{A}\PY{o}{.}\PY{n}{shape}\PY{p}{[}\PY{l+m+mi}{1}\PY{p}{]}\PY{p}{)}
\PY{n}{jj\PYZus{}complement}\PY{o}{=}\PY{n+nb}{list}\PY{p}{(}\PY{n}{np}\PY{o}{.}\PY{n}{squeeze}\PY{p}{(}\PY{n}{np}\PY{o}{.}\PY{n}{where}\PY{p}{(}\PY{n}{xsol}\PY{o}{==}\PY{l+m+mi}{0}\PY{p}{)}\PY{p}{)}\PY{p}{)}
\PY{n}{N}\PY{o}{=}\PY{n}{A}\PY{p}{[}\PY{p}{:}\PY{p}{,}\PY{n}{jj\PYZus{}complement}\PY{p}{]}
\PY{n}{hat\PYZus{}c}\PY{o}{=}\PY{n}{c}\PY{p}{[}\PY{n}{jj\PYZus{}complement}\PY{p}{]}
\PY{c+c1}{\PYZsh{} hat\PYZus{}s=c\PYZus{}hat\PYZhy{}N.T*lamb}
\PY{n}{hat\PYZus{}s}\PY{o}{=}\PY{n}{hat\PYZus{}c}\PY{o}{\PYZhy{}}\PY{n}{N}\PY{o}{.}\PY{n}{T}\PY{n+nd}{@lamb}
\PY{n}{s}\PY{p}{[}\PY{n}{jj\PYZus{}complement}\PY{p}{]}\PY{o}{=}\PY{n}{hat\PYZus{}s}
\PY{n+nb}{print}\PY{p}{(}\PY{l+s+s1}{\PYZsq{}}\PY{l+s+s1}{El valor del vector s es: }\PY{l+s+s1}{\PYZsq{}}\PY{p}{,}\PY{n}{s}\PY{p}{)}
\end{Verbatim}
\end{tcolorbox}

    \begin{Verbatim}[commandchars=\\\{\}]
El valor del vector s es:  [0.     0.     0.0417 0.     1.0833 0.    ]
    \end{Verbatim}

    A partir de estos resultados es claro que se cumplen las condiciones
KKT.

Finalmente, para el númeral 3 ejecutamos la siguiente celda de código.

    \begin{tcolorbox}[breakable, size=fbox, boxrule=1pt, pad at break*=1mm,colback=cellbackground, colframe=cellborder]
\prompt{In}{incolor}{17}{\boxspacing}
\begin{Verbatim}[commandchars=\\\{\}]
\PY{n+nb}{print}\PY{p}{(}\PY{l+s+s1}{\PYZsq{}}\PY{l+s+s1}{Valor de b.T lambda: }\PY{l+s+s1}{\PYZsq{}}\PY{p}{,}\PY{n}{np}\PY{o}{.}\PY{n}{vdot}\PY{p}{(}\PY{n}{b}\PY{p}{,}\PY{n}{lamb}\PY{p}{)}\PY{p}{)}
\PY{n+nb}{print}\PY{p}{(}\PY{l+s+s1}{\PYZsq{}}\PY{l+s+s1}{Valor de c.T x\PYZus{}ast: }\PY{l+s+s1}{\PYZsq{}}\PY{p}{,}\PY{n}{np}\PY{o}{.}\PY{n}{vdot}\PY{p}{(}\PY{n}{c}\PY{p}{,}\PY{n}{xsol}\PY{p}{)}\PY{p}{)}
\end{Verbatim}
\end{tcolorbox}

    \begin{Verbatim}[commandchars=\\\{\}]
Valor de b.T lambda:  -51.25000000000001
Valor de c.T x\_ast:  -51.25
    \end{Verbatim}

    Vemos que el valor es el mismo, debido a que el valor \(\mathbf{b}^T\lambda\)
corresponde a la función objetivo del problema dual.

    Por otro lado, probaremos la librería \texttt{pulp} para resolver el
mismo de problema de programación lineal. El modelo descrito
matemáticamente los describimos es el siguiente

    \begin{tcolorbox}[breakable, size=fbox, boxrule=1pt, pad at break*=1mm,colback=cellbackground, colframe=cellborder]
\prompt{In}{incolor}{18}{\boxspacing}
\begin{Verbatim}[commandchars=\\\{\}]
\PY{k+kn}{from} \PY{n+nn}{pulp} \PY{k+kn}{import} \PY{n}{LpMaximize}\PY{p}{,} \PY{n}{LpProblem}\PY{p}{,} \PY{n}{LpStatus}\PY{p}{,} \PY{n}{lpSum}\PY{p}{,} \PY{n}{LpVariable}

\PY{c+c1}{\PYZsh{} Creacción de una instancia de la  clase }
\PY{n}{model} \PY{o}{=} \PY{n}{LpProblem}\PY{p}{(}\PY{n}{name}\PY{o}{=}\PY{l+s+s2}{\PYZdq{}}\PY{l+s+s2}{small\PYZhy{}problem}\PY{l+s+s2}{\PYZdq{}}\PY{p}{,} \PY{n}{sense}\PY{o}{=}\PY{n}{LpMaximize}\PY{p}{)}

\PY{c+c1}{\PYZsh{} Variables de decisión}
\PY{n}{x1} \PY{o}{=} \PY{n}{LpVariable}\PY{p}{(}\PY{n}{name}\PY{o}{=}\PY{l+s+s2}{\PYZdq{}}\PY{l+s+s2}{x1}\PY{l+s+s2}{\PYZdq{}}\PY{p}{,} \PY{n}{lowBound}\PY{o}{=}\PY{l+m+mi}{45}\PY{p}{)}
\PY{n}{x2} \PY{o}{=} \PY{n}{LpVariable}\PY{p}{(}\PY{n}{name}\PY{o}{=}\PY{l+s+s2}{\PYZdq{}}\PY{l+s+s2}{x2}\PY{l+s+s2}{\PYZdq{}}\PY{p}{,} \PY{n}{lowBound}\PY{o}{=}\PY{l+m+mi}{5}\PY{p}{)}

\PY{c+c1}{\PYZsh{} Se agregan las restricciones del modelo}
\PY{n}{model} \PY{o}{+}\PY{o}{=} \PY{p}{(}\PY{l+m+mi}{50}\PY{o}{*}\PY{n}{x1} \PY{o}{+} \PY{l+m+mi}{24}\PY{o}{*}\PY{n}{x2}  \PY{o}{\PYZlt{}}\PY{o}{=} \PY{l+m+mi}{2400}\PY{p}{,} \PY{l+s+s2}{\PYZdq{}}\PY{l+s+s2}{R1}\PY{l+s+s2}{\PYZdq{}}\PY{p}{)}
\PY{n}{model} \PY{o}{+}\PY{o}{=} \PY{p}{(}\PY{l+m+mi}{30}\PY{o}{*}\PY{n}{x1} \PY{o}{+} \PY{l+m+mi}{33}\PY{o}{*}\PY{n}{x2}  \PY{o}{\PYZlt{}}\PY{o}{=} \PY{l+m+mi}{2100}\PY{p}{,} \PY{l+s+s2}{\PYZdq{}}\PY{l+s+s2}{R2}\PY{l+s+s2}{\PYZdq{}}\PY{p}{)}

\PY{c+c1}{\PYZsh{} Función objetivo}
\PY{n}{model} \PY{o}{+}\PY{o}{=} \PY{n}{x1}\PY{o}{+}\PY{n}{x2}

\PY{n+nb}{print}\PY{p}{(}\PY{n}{model}\PY{p}{)}
\end{Verbatim}
\end{tcolorbox}

    \begin{Verbatim}[commandchars=\\\{\}]
small-problem:
MAXIMIZE
1*x1 + 1*x2 + 0
SUBJECT TO
R1: 50 x1 + 24 x2 <= 2400

R2: 30 x1 + 33 x2 <= 2100

VARIABLES
45 <= x1 Continuous
5 <= x2 Continuous

    \end{Verbatim}

    Y la solución es

    \begin{tcolorbox}[breakable, size=fbox, boxrule=1pt, pad at break*=1mm,colback=cellbackground, colframe=cellborder]
\prompt{In}{incolor}{19}{\boxspacing}
\begin{Verbatim}[commandchars=\\\{\}]
\PY{c+c1}{\PYZsh{} Cálculo de la solución}
\PY{k+kn}{from} \PY{n+nn}{pulp} \PY{k+kn}{import} \PY{n}{PULP\PYZus{}CBC\PYZus{}CMD}

\PY{n}{status} \PY{o}{=} \PY{n}{model}\PY{o}{.}\PY{n}{solve}\PY{p}{(}\PY{n}{PULP\PYZus{}CBC\PYZus{}CMD}\PY{p}{(}\PY{n}{msg}\PY{o}{=}\PY{k+kc}{False}\PY{p}{)}\PY{p}{)}

\PY{n+nb}{print}\PY{p}{(}\PY{l+s+s2}{\PYZdq{}}\PY{l+s+s2}{Resultado: }\PY{l+s+s2}{\PYZdq{}}\PY{p}{,} \PY{n}{model}\PY{o}{.}\PY{n}{status}\PY{p}{,} \PY{l+s+s2}{\PYZdq{}}\PY{l+s+s2}{ | }\PY{l+s+s2}{\PYZdq{}}\PY{p}{,} \PY{n}{LpStatus}\PY{p}{[}\PY{n}{model}\PY{o}{.}\PY{n}{status}\PY{p}{]}\PY{p}{)}

\PY{n+nb}{print}\PY{p}{(}\PY{l+s+s2}{\PYZdq{}}\PY{l+s+s2}{Valor de la funciónn objetivo: }\PY{l+s+s2}{\PYZdq{}} \PY{p}{,} \PY{n}{model}\PY{o}{.}\PY{n}{objective}\PY{o}{.}\PY{n}{value}\PY{p}{(}\PY{p}{)}\PY{p}{)}


\PY{n+nb}{print}\PY{p}{(}\PY{l+s+s1}{\PYZsq{}}\PY{l+s+s1}{Solución:}\PY{l+s+s1}{\PYZsq{}}\PY{p}{)}
\PY{k}{for} \PY{n}{var} \PY{o+ow}{in} \PY{n}{model}\PY{o}{.}\PY{n}{variables}\PY{p}{(}\PY{p}{)}\PY{p}{:}
    \PY{n+nb}{print}\PY{p}{(}\PY{l+s+s2}{\PYZdq{}}\PY{l+s+si}{\PYZpc{}10s}\PY{l+s+s2}{: }\PY{l+s+si}{\PYZpc{}f}\PY{l+s+s2}{\PYZdq{}}  \PY{o}{\PYZpc{}} \PY{p}{(}\PY{n}{var}\PY{o}{.}\PY{n}{name}\PY{p}{,} \PY{n}{var}\PY{o}{.}\PY{n}{value}\PY{p}{(}\PY{p}{)}\PY{p}{)} \PY{p}{)}

\PY{n+nb}{print}\PY{p}{(}\PY{l+s+s1}{\PYZsq{}}\PY{l+s+se}{\PYZbs{}n}\PY{l+s+s1}{Variables de holgura:}\PY{l+s+s1}{\PYZsq{}}\PY{p}{)}
\PY{k}{for} \PY{n}{name}\PY{p}{,} \PY{n}{constraint} \PY{o+ow}{in} \PY{n}{model}\PY{o}{.}\PY{n}{constraints}\PY{o}{.}\PY{n}{items}\PY{p}{(}\PY{p}{)}\PY{p}{:}
    \PY{n+nb}{print}\PY{p}{(}\PY{l+s+s2}{\PYZdq{}}\PY{l+s+si}{\PYZpc{}10s}\PY{l+s+s2}{: }\PY{l+s+si}{\PYZpc{}f}\PY{l+s+s2}{\PYZdq{}} \PY{o}{\PYZpc{}} \PY{p}{(}\PY{n}{name}\PY{p}{,} \PY{n}{constraint}\PY{o}{.}\PY{n}{value}\PY{p}{(}\PY{p}{)}\PY{p}{)} \PY{p}{)}
\end{Verbatim}
\end{tcolorbox}

    \begin{Verbatim}[commandchars=\\\{\}]
Resultado:  1  |  Optimal
Valor de la funciónn objetivo:  51.25
Solución:
        x1: 45.000000
        x2: 6.250000

Variables de holgura:
        R1: 0.000000
        R2: -543.750000
    \end{Verbatim}

    Que coincide con el resultado obtenido con el código visto en clase

    \hypertarget{ejercicio-3-3-puntos}{%
\section{Ejercicio 3 (3 puntos)}\label{ejercicio-3-3-puntos}}

Considere el problema

\[
\begin{array}{rl}
\min & x_1 + 2x_2 + x_3 + x_4\\
\text{sujeto a}  & 2x_1 +  x_2 + 3 x_3 +  x_4 \leq 8  \\
                 & 2x_1 + 3x_2 +         4x_4 \leq 12 \\
                 & 3x_1 +  x_2 + 2 x_3        \leq 18 \\
                 & x_1, x_2, x_3, x_4 \geq 0
\end{array}
\]

\begin{enumerate}
\def\labelenumi{\arabic{enumi}.}
\tightlist
\item
  Escriba el problema en su forma estándar.
\item
  Construya los vectores \(\mathbf{c}, \mathbf{b}\) y la matriz
  \(\mathbf{A}\) del problema estándar y calcule la solución
  \(\mathbf{x}_*\) del problema. Puede usar el código anterior.
\item
  Calcule los vectores \(\lambda\) y \(\mathbf{s}\) y verique
  que se cumplen la condiciones (4) y (5), y con esto se comprueba que
  \(\mathbf{x}_*\) es solución del problema estándar.
\item
  Calcule el valor
\end{enumerate}

\[\mathbf{b}^\top \lambda \]

y compare este valor con el valor de la función objetivo
\(\mathbf{c}^\top \mathbf{x}_*\).

\hypertarget{soluciuxf3n}{%
\subsubsection{Solución:}\label{soluciuxf3n}}

En este caso, lo que haremos será resolver el problema de optimización

\[\max\{x_1+2x_2+x_3+x_4\},\]

pues el problema de minimización con las restricciones proporcionadas
tiene como solución la solución trivial trivial \(x_1=x_2=x_3=x_4=0\).

El problema de maximización sujeto a las mismas reestricciones equivale
a resolver \(\min\{-x_1-2x_2-x_3-x_4\}\) con las mismas restricciones.

    En primer lugar, escribiremos el problema en forma estándar. Como
\(x_i\geq 0\) para \(i=1,2,3,4\) para obtener la forma estándar podemos
ignorar el vector con componentes \(x_i^{-}\) (la parte negativa) pues
esta es \(0\) bajo estas restricciones de positividad.

Por lo tanto, solo consideramos las variables de holgura no negativas
\(x_5,x_6,x_7\) de forma que se quiere resoler el siguiente problema de
optimización

\[
\begin{array}{rl}
\min & -x_1 - 2x_2 - x_3 - x_4\\
\text{sujeto a}  & 2x_1 +  x_2 + 3 x_3 +  x_4 +x_5 = 8  \\
                 & 2x_1 + 3x_2 +         4x_4 +x_6 = 12 \\
                 & 3x_1 +  x_2 + 2 x_3        +x_7 =  18 \\
                 & x_1, x_2, x_3, x_4,x_5,x_6,x_7\geq 0
\end{array}
\]

Si \[A=\begin{pmatrix}
2 & 1& 3& 1& 1& 0 &0 \\
2 & 3& 0& 4& 0& 1& 0 \\
3 & 1& 2& 0& 0& 0& 1
\end{pmatrix},\]

\(\mathbf{x}=(x_1,x_2,x_3,x_4,x_5,x_6,x_7)^T\),
\(\mathbf{b}=(8,12,18)^T\) y \[
 \mathbf{c}=(-1,-2,-1,-1,0,0,0),
\]

entonces el problema de programación lineal en forma estándar que
debemos resolver es
\[ \min \mathbf{c}^T\mathbf{x}\text{  sujeto a }A\mathbf{x}=\mathbf{b}\text{ y }\mathbf{x}\geq 0\]

    Para poder obtener los vectores \(\lambda\) y \(\mathbf{s}\) de
multiplicadores de Lagrange es necesario tener la matriz básica
\(\mathbf{B}\) y por supuesto, tener identificadas las variables básicas
y no básicas del problema de programación lineal.

Utilizamos el mismo código que en el ejercicio 2 para obtener la
solución de este problema ya que tenemos la forma estándar.

    \begin{tcolorbox}[breakable, size=fbox, boxrule=1pt, pad at break*=1mm,colback=cellbackground, colframe=cellborder]
\prompt{In}{incolor}{21}{\boxspacing}
\begin{Verbatim}[commandchars=\\\{\}]
\PY{n}{c} \PY{o}{=} \PY{n}{np}\PY{o}{.}\PY{n}{array}\PY{p}{(}\PY{p}{[}\PY{o}{\PYZhy{}}\PY{l+m+mi}{1}\PY{p}{,} \PY{o}{\PYZhy{}}\PY{l+m+mi}{2}\PY{p}{,} \PY{o}{\PYZhy{}}\PY{l+m+mi}{1}\PY{p}{,} \PY{o}{\PYZhy{}}\PY{l+m+mi}{1}\PY{p}{,} \PY{l+m+mi}{0}\PY{p}{,} \PY{l+m+mi}{0} \PY{p}{,}\PY{l+m+mi}{0}\PY{p}{]}\PY{p}{)}
\PY{n}{b} \PY{o}{=} \PY{n}{np}\PY{o}{.}\PY{n}{array}\PY{p}{(}\PY{p}{[}\PY{l+m+mi}{8}\PY{p}{,} \PY{l+m+mi}{12}\PY{p}{,} \PY{l+m+mi}{18}\PY{p}{]}\PY{p}{)}
\PY{n}{A} \PY{o}{=} \PY{n}{np}\PY{o}{.}\PY{n}{array}\PY{p}{(}\PY{p}{[}\PY{p}{[}\PY{l+m+mi}{2}\PY{p}{,} \PY{l+m+mi}{1}\PY{p}{,} \PY{l+m+mi}{3}\PY{p}{,} \PY{l+m+mi}{1}\PY{p}{,} \PY{l+m+mi}{1}\PY{p}{,} \PY{l+m+mi}{0}\PY{p}{,} \PY{l+m+mi}{0}\PY{p}{]}\PY{p}{,}
              \PY{p}{[}\PY{l+m+mi}{2}\PY{p}{,} \PY{l+m+mi}{3}\PY{p}{,} \PY{l+m+mi}{0}\PY{p}{,} \PY{l+m+mi}{4}\PY{p}{,} \PY{l+m+mi}{0}\PY{p}{,} \PY{l+m+mi}{1}\PY{p}{,} \PY{l+m+mi}{0}\PY{p}{]}\PY{p}{,}
              \PY{p}{[}\PY{l+m+mi}{3}\PY{p}{,} \PY{l+m+mi}{1}\PY{p}{,} \PY{l+m+mi}{2}\PY{p}{,} \PY{l+m+mi}{0}\PY{p}{,} \PY{l+m+mi}{0}\PY{p}{,} \PY{l+m+mi}{0}\PY{p}{,} \PY{l+m+mi}{1}\PY{p}{]}\PY{p}{]}\PY{p}{)}

\PY{n}{m}\PY{p}{,}\PY{n}{n} \PY{o}{=} \PY{n}{A}\PY{o}{.}\PY{n}{shape}
\PY{n}{comb} \PY{o}{=} \PY{n+nb}{list}\PY{p}{(}\PY{n}{combinations}\PY{p}{(}\PY{n+nb}{list}\PY{p}{(}\PY{n+nb}{range}\PY{p}{(}\PY{n}{n}\PY{p}{)}\PY{p}{)}\PY{p}{,} \PY{n}{m}\PY{p}{)}\PY{p}{)}
\PY{n+nb}{print}\PY{p}{(}\PY{l+s+s1}{\PYZsq{}}\PY{l+s+s1}{Número de combinaciones:}\PY{l+s+s1}{\PYZsq{}}\PY{p}{,} \PY{n+nb}{len}\PY{p}{(}\PY{n}{comb}\PY{p}{)}\PY{p}{)}

\PY{n}{dmin} \PY{o}{=} \PY{k+kc}{None}
\PY{k}{for} \PY{n}{icols} \PY{o+ow}{in} \PY{n}{comb}\PY{p}{:}
    \PY{c+c1}{\PYZsh{} Indices de las columnas seleccionadas}
    \PY{n}{jj} \PY{o}{=} \PY{n+nb}{list}\PY{p}{(}\PY{n}{icols}\PY{p}{)}
    \PY{c+c1}{\PYZsh{} Matriz básica}
    \PY{n}{B}     \PY{o}{=} \PY{n}{A}\PY{p}{[}\PY{p}{:}\PY{p}{,} \PY{n}{jj}\PY{p}{]}
    \PY{n}{condB} \PY{o}{=} \PY{n}{np}\PY{o}{.}\PY{n}{linalg}\PY{o}{.}\PY{n}{cond}\PY{p}{(}\PY{n}{B}\PY{p}{)}
    \PY{k}{if} \PY{n}{condB}\PY{o}{\PYZgt{}}\PY{l+m+mf}{1.0e14}\PY{p}{:}
        \PY{n+nb}{print}\PY{p}{(}\PY{l+s+s1}{\PYZsq{}}\PY{l+s+s1}{Es casi singular la matriz B con columnas}\PY{l+s+s1}{\PYZsq{}}\PY{p}{,} \PY{n}{jj}\PY{p}{)}
    \PY{k}{else}\PY{p}{:}
        \PY{c+c1}{\PYZsh{} Solucion del sistema B*x=b}
        \PY{n}{xb} \PY{o}{=} \PY{n}{np}\PY{o}{.}\PY{n}{linalg}\PY{o}{.}\PY{n}{solve}\PY{p}{(}\PY{n}{B}\PY{p}{,} \PY{n}{b}\PY{p}{)}
        \PY{c+c1}{\PYZsh{} Solucion del problema en forma estándar}
        \PY{n}{x}     \PY{o}{=} \PY{n}{np}\PY{o}{.}\PY{n}{zeros}\PY{p}{(}\PY{n}{n}\PY{p}{)}
        \PY{n}{x}\PY{p}{[}\PY{n}{jj}\PY{p}{]} \PY{o}{=} \PY{n}{xb}
        \PY{c+c1}{\PYZsh{} Evaluación de la función objetivo}
        \PY{n}{f} \PY{o}{=} \PY{n}{np}\PY{o}{.}\PY{n}{vdot}\PY{p}{(}\PY{n}{c}\PY{p}{,} \PY{n}{x}\PY{p}{)}
        \PY{c+c1}{\PYZsh{} Se revisa si es vector x es factible. Claramente se cumple que A*x=b,}
        \PY{c+c1}{\PYZsh{} pero hay que verificar que x\PYZgt{}=0.}
        \PY{n}{smsg}  \PY{o}{=} \PY{l+s+s1}{\PYZsq{}}\PY{l+s+s1}{No factible}\PY{l+s+s1}{\PYZsq{}}
        \PY{n}{bfact} \PY{o}{=} \PY{k+kc}{False} 
        \PY{k}{if} \PY{n+nb}{sum}\PY{p}{(}\PY{n}{x}\PY{o}{\PYZgt{}}\PY{o}{=}\PY{l+m+mi}{0}\PY{p}{)}\PY{o}{==}\PY{n+nb}{len}\PY{p}{(}\PY{n}{x}\PY{p}{)}\PY{p}{:}
            \PY{n}{bfact} \PY{o}{=} \PY{k+kc}{True}
            \PY{n}{smsg}  \PY{o}{=} \PY{l+s+s1}{\PYZsq{}}\PY{l+s+s1}{Factible}\PY{l+s+s1}{\PYZsq{}}
        \PY{k}{if} \PY{n}{bfact}\PY{p}{:}
            \PY{c+c1}{\PYZsh{} Si x es factible, almacenamos en xsol el punto x donde f es mínima}
            \PY{k}{if} \PY{n}{dmin}\PY{o}{==}\PY{k+kc}{None}\PY{p}{:}
                \PY{n}{dmin} \PY{o}{=} \PY{n}{f}
                \PY{n}{xsol} \PY{o}{=} \PY{n}{x}\PY{o}{.}\PY{n}{copy}\PY{p}{(}\PY{p}{)}
            \PY{k}{elif} \PY{n}{dmin}\PY{o}{\PYZgt{}}\PY{n}{f}\PY{p}{:}
                \PY{n}{dmin} \PY{o}{=} \PY{n}{f}
                \PY{n}{xsol} \PY{o}{=} \PY{n}{x}\PY{o}{.}\PY{n}{copy}\PY{p}{(}\PY{p}{)}
        \PY{n+nb}{print}\PY{p}{(}\PY{l+s+s2}{\PYZdq{}}\PY{l+s+si}{\PYZpc{}6.1f}\PY{l+s+s2}{ }\PY{l+s+si}{\PYZpc{}7.1f}\PY{l+s+s2}{| }\PY{l+s+si}{\PYZpc{} 8.2f}\PY{l+s+s2}{ }\PY{l+s+si}{\PYZpc{} 8.2f}\PY{l+s+s2}{ }\PY{l+s+si}{\PYZpc{} 8.2f}\PY{l+s+s2}{ }\PY{l+s+si}{\PYZpc{} 8.2f}\PY{l+s+s2}{ }\PY{l+s+si}{\PYZpc{} 8.2f}\PY{l+s+s2}{ }\PY{l+s+si}{\PYZpc{} 8.2f}\PY{l+s+s2}{ }\PY{l+s+si}{\PYZpc{} 8.2f}\PY{l+s+s2}{| }\PY{l+s+si}{\PYZpc{}s}\PY{l+s+s2}{\PYZdq{}} \PY{o}{\PYZpc{}} \PY{p}{(}\PY{n}{condB}\PY{p}{,} 
                          \PY{n}{f}\PY{p}{,}  \PY{n}{x}\PY{p}{[}\PY{l+m+mi}{0}\PY{p}{]}\PY{p}{,} \PY{n}{x}\PY{p}{[}\PY{l+m+mi}{1}\PY{p}{]}\PY{p}{,} \PY{n}{x}\PY{p}{[}\PY{l+m+mi}{2}\PY{p}{]}\PY{p}{,} \PY{n}{x}\PY{p}{[}\PY{l+m+mi}{3}\PY{p}{]}\PY{p}{,} \PY{n}{x}\PY{p}{[}\PY{l+m+mi}{4}\PY{p}{]}\PY{p}{,} \PY{n}{x}\PY{p}{[}\PY{l+m+mi}{5}\PY{p}{]}\PY{p}{,} \PY{n}{x}\PY{p}{[}\PY{l+m+mi}{6}\PY{p}{]}\PY{p}{,} \PY{n}{smsg}\PY{p}{)}\PY{p}{)}
        
\PY{c+c1}{\PYZsh{} Fijamos una tolerancia y hacemos cero las componentes de x que son menores que la tolerancia        }
\PY{n}{tol} \PY{o}{=} \PY{p}{(}\PY{n}{np}\PY{o}{.}\PY{n}{finfo}\PY{p}{(}\PY{n+nb}{float}\PY{p}{)}\PY{o}{.}\PY{n}{eps}\PY{p}{)}\PY{o}{*}\PY{o}{*}\PY{p}{(}\PY{l+m+mf}{3.0}\PY{o}{/}\PY{l+m+mi}{4}\PY{p}{)}
\PY{n}{ii}  \PY{o}{=} \PY{n}{np}\PY{o}{.}\PY{n}{where}\PY{p}{(}\PY{n}{xsol}\PY{o}{\PYZlt{}}\PY{n}{tol}\PY{p}{)}\PY{p}{[}\PY{l+m+mi}{0}\PY{p}{]}
\PY{n}{xsol}\PY{p}{[}\PY{n}{ii}\PY{p}{]} \PY{o}{=} \PY{l+m+mf}{0.0}

\PY{n+nb}{print}\PY{p}{(}\PY{l+s+s1}{\PYZsq{}}\PY{l+s+se}{\PYZbs{}n}\PY{l+s+s1}{Solución del problema estándar:}\PY{l+s+s1}{\PYZsq{}}\PY{p}{)}
\PY{n+nb}{print}\PY{p}{(}\PY{l+s+s1}{\PYZsq{}}\PY{l+s+s1}{x*=}\PY{l+s+s1}{\PYZsq{}}\PY{p}{,} \PY{n}{xsol}\PY{p}{)}
\PY{n+nb}{print}\PY{p}{(}\PY{l+s+s1}{\PYZsq{}}\PY{l+s+s1}{Valor de la función objetivo en x*=}\PY{l+s+s1}{\PYZsq{}}\PY{p}{,} \PY{n}{dmin}\PY{p}{)}
\end{Verbatim}
\end{tcolorbox}

    \begin{Verbatim}[commandchars=\\\{\}]
Número de combinaciones: 35
   6.8    -3.2|     7.85    -1.23    -2.15     0.00     0.00     0.00     0.00|
No factible
  33.8   -23.3|     0.67    16.00     0.00    -9.33     0.00     0.00     0.00|
No factible
   6.0    -6.0|     6.00     0.00     0.00     0.00    -4.00     0.00     0.00|
No factible
  45.7    14.0|    10.00   -12.00     0.00     0.00     0.00    28.00     0.00|
No factible
  11.2    -7.0|     3.00     2.00     0.00     0.00     0.00     0.00     7.00|
Factible
   4.9    -4.7|     7.33     0.00    -2.00    -0.67     0.00     0.00     0.00|
No factible
  12.4    -6.0|     6.00     0.00     0.00     0.00    -4.00     0.00     0.00|
No factible
  10.2    -5.2|     7.60     0.00    -2.40     0.00     0.00    -3.20     0.00|
No factible
   7.9    -4.7|     6.00     0.00    -1.33     0.00     0.00     0.00     2.67|
No factible
   6.0    -6.0|     6.00     0.00     0.00     0.00    -4.00     0.00     0.00|
No factible
  24.9    -2.0|     6.00     0.00     0.00    -4.00     0.00    16.00     0.00|
No factible
  12.5    -4.7|     3.33     0.00     0.00     1.33     0.00     0.00     8.00|
Factible
   5.8    -6.0|     6.00     0.00     0.00     0.00    -4.00     0.00     0.00|
No factible
   8.9    -6.0|     6.00     0.00     0.00     0.00    -4.00     0.00    -0.00|
No factible
   8.9    -4.0|     4.00     0.00     0.00     0.00     0.00     4.00     6.00|
Factible
   9.9   -25.2|     0.00    17.60     0.20   -10.20     0.00     0.00     0.00|
No factible
   7.8   -15.0|     0.00     4.00     7.00     0.00   -17.00     0.00     0.00|
No factible
  48.0   -66.0|     0.00    38.00   -10.00     0.00     0.00  -102.00     0.00|
No factible
   5.1    -9.3|     0.00     4.00     1.33     0.00     0.00     0.00    11.33|
Factible
   6.9   -25.5|     0.00    18.00     0.00   -10.50     0.50     0.00     0.00|
No factible
  23.5   -26.0|     0.00    18.00     0.00   -10.00     0.00    -2.00     0.00|
No factible
  34.8   -28.0|     0.00    20.00     0.00   -12.00     0.00     0.00    -2.00|
No factible
  11.9   -36.0|     0.00    18.00     0.00     0.00   -10.00   -42.00     0.00|
No factible
   3.7    -8.0|     0.00     4.00     0.00     0.00     4.00     0.00    14.00|
Factible
  11.9   -16.0|     0.00     8.00     0.00     0.00     0.00   -12.00    10.00|
No factible
   8.1   -12.0|     0.00     0.00     9.00     3.00   -22.00     0.00     0.00|
No factible
  33.0    10.0|     0.00     0.00     9.00   -19.00     0.00    88.00     0.00|
No factible
   5.3    -4.7|     0.00     0.00     1.67     3.00     0.00     0.00    14.67|
Factible
   6.9    -9.0|     0.00     0.00     9.00     0.00   -19.00    12.00     0.00|
No factible
Es casi singular la matriz B con columnas [2, 4, 6]
   4.4    -2.7|     0.00     0.00     2.67     0.00     0.00    12.00    12.67|
Factible
Es casi singular la matriz B con columnas [3, 4, 5]
   4.3    -3.0|     0.00     0.00     0.00     3.00     5.00     0.00    18.00|
Factible
  17.9    -8.0|     0.00     0.00     0.00     8.00     0.00   -20.00    18.00|
No factible
   1.0     0.0|     0.00     0.00     0.00     0.00     8.00    12.00    18.00|
Factible

Solución del problema estándar:
x*= [ 0.      4.      1.3333  0.      0.      0.     11.3333]
Valor de la función objetivo en x*= -9.333333333333334
    \end{Verbatim}

    Calculamos el vector \(\lambda\) de las condiciones KKT para
programación lineal como en el ejercicio anterior

    \begin{tcolorbox}[breakable, size=fbox, boxrule=1pt, pad at break*=1mm,colback=cellbackground, colframe=cellborder]
\prompt{In}{incolor}{22}{\boxspacing}
\begin{Verbatim}[commandchars=\\\{\}]
\PY{n}{jj}\PY{o}{=}\PY{n+nb}{list}\PY{p}{(}\PY{n}{np}\PY{o}{.}\PY{n}{squeeze}\PY{p}{(}\PY{n}{np}\PY{o}{.}\PY{n}{where}\PY{p}{(}\PY{n}{xsol}\PY{o}{\PYZgt{}}\PY{l+m+mi}{0}\PY{p}{)}\PY{p}{)}\PY{p}{)}
\PY{n}{B}\PY{o}{=}\PY{n}{A}\PY{p}{[}\PY{p}{:}\PY{p}{,}\PY{n}{jj}\PY{p}{]}
\PY{n}{tilde\PYZus{}c}\PY{o}{=}\PY{n}{c}\PY{p}{[}\PY{n}{jj}\PY{p}{]}
\PY{c+c1}{\PYZsh{} B.T*lamb=tilde\PYZus{}c}
\PY{n}{lamb}\PY{o}{=}\PY{n}{np}\PY{o}{.}\PY{n}{linalg}\PY{o}{.}\PY{n}{solve}\PY{p}{(}\PY{n}{B}\PY{o}{.}\PY{n}{T}\PY{p}{,}\PY{n}{tilde\PYZus{}c}\PY{p}{)}
\PY{n+nb}{print}\PY{p}{(}\PY{l+s+s1}{\PYZsq{}}\PY{l+s+s1}{El vecto lambda es: }\PY{l+s+s1}{\PYZsq{}}\PY{p}{,}\PY{n}{lamb}\PY{p}{)}
\end{Verbatim}
\end{tcolorbox}

    \begin{Verbatim}[commandchars=\\\{\}]
El vecto lambda es:  [-0.3333 -0.5556  0.    ]
    \end{Verbatim}

    Y a partir de este vector \(\lambda\) calculamos el vector
\(\mathbf{s}\), que son los multiplicadores de Lagrange asociados a la
restricción de desigualdad \(\mathbf{x}\geq 0\) del problema de
programación lineal en forma estándar.

    \begin{tcolorbox}[breakable, size=fbox, boxrule=1pt, pad at break*=1mm,colback=cellbackground, colframe=cellborder]
\prompt{In}{incolor}{23}{\boxspacing}
\begin{Verbatim}[commandchars=\\\{\}]
\PY{n}{s}\PY{o}{=}\PY{n}{np}\PY{o}{.}\PY{n}{zeros}\PY{p}{(}\PY{n}{A}\PY{o}{.}\PY{n}{shape}\PY{p}{[}\PY{l+m+mi}{1}\PY{p}{]}\PY{p}{)}
\PY{n}{jj\PYZus{}complement}\PY{o}{=}\PY{n+nb}{list}\PY{p}{(}\PY{n}{np}\PY{o}{.}\PY{n}{squeeze}\PY{p}{(}\PY{n}{np}\PY{o}{.}\PY{n}{where}\PY{p}{(}\PY{n}{xsol}\PY{o}{==}\PY{l+m+mi}{0}\PY{p}{)}\PY{p}{)}\PY{p}{)}
\PY{n}{N}\PY{o}{=}\PY{n}{A}\PY{p}{[}\PY{p}{:}\PY{p}{,}\PY{n}{jj\PYZus{}complement}\PY{p}{]}
\PY{n}{hat\PYZus{}c}\PY{o}{=}\PY{n}{c}\PY{p}{[}\PY{n}{jj\PYZus{}complement}\PY{p}{]}
\PY{c+c1}{\PYZsh{} hat\PYZus{}s=c\PYZus{}hat\PYZhy{}N.T*lamb}
\PY{n}{hat\PYZus{}s}\PY{o}{=}\PY{n}{hat\PYZus{}c}\PY{o}{\PYZhy{}}\PY{n}{N}\PY{o}{.}\PY{n}{T}\PY{n+nd}{@lamb}
\PY{n}{s}\PY{p}{[}\PY{n}{jj\PYZus{}complement}\PY{p}{]}\PY{o}{=}\PY{n}{hat\PYZus{}s}
\PY{n+nb}{print}\PY{p}{(}\PY{l+s+s1}{\PYZsq{}}\PY{l+s+s1}{El valor del vector s es: }\PY{l+s+s1}{\PYZsq{}}\PY{p}{,}\PY{n}{s}\PY{p}{)}
\end{Verbatim}
\end{tcolorbox}

    \begin{Verbatim}[commandchars=\\\{\}]
El valor del vector s es:  [0.7778 0.     0.     1.5556 0.3333 0.5556 0.    ]
    \end{Verbatim}

    Finalmente, calculamos \(\mathbf{b}^T\lambda\) y lo comparamos
con el valor \(\mathbf{c}^T\mathbf{x}\)

    \begin{tcolorbox}[breakable, size=fbox, boxrule=1pt, pad at break*=1mm,colback=cellbackground, colframe=cellborder]
\prompt{In}{incolor}{24}{\boxspacing}
\begin{Verbatim}[commandchars=\\\{\}]
\PY{n+nb}{print}\PY{p}{(}\PY{l+s+s1}{\PYZsq{}}\PY{l+s+s1}{Valor de b.T lambda: }\PY{l+s+s1}{\PYZsq{}}\PY{p}{,}\PY{n}{np}\PY{o}{.}\PY{n}{vdot}\PY{p}{(}\PY{n}{b}\PY{p}{,}\PY{n}{lamb}\PY{p}{)}\PY{p}{)}
\PY{n+nb}{print}\PY{p}{(}\PY{l+s+s1}{\PYZsq{}}\PY{l+s+s1}{Valor de c.T x\PYZus{}ast: }\PY{l+s+s1}{\PYZsq{}}\PY{p}{,}\PY{n}{np}\PY{o}{.}\PY{n}{vdot}\PY{p}{(}\PY{n}{c}\PY{p}{,}\PY{n}{xsol}\PY{p}{)}\PY{p}{)}
\end{Verbatim}
\end{tcolorbox}

    \begin{Verbatim}[commandchars=\\\{\}]
Valor de b.T lambda:  -9.333333333333334
Valor de c.T x\_ast:  -9.333333333333334
    \end{Verbatim}

    Ambos valores coinciden tratándose de las funciones objetivos del
problema primal y dual del problema de programación lineal en forma
estándar.

    Ahora para obtener validar la solución del problema usamos también la
librería \texttt{pulp}. Primero declaramos el modelo inicial

    \begin{tcolorbox}[breakable, size=fbox, boxrule=1pt, pad at break*=1mm,colback=cellbackground, colframe=cellborder]
\prompt{In}{incolor}{11}{\boxspacing}
\begin{Verbatim}[commandchars=\\\{\}]
\PY{c+c1}{\PYZsh{} Creacción de una instancia de la  clase }
\PY{n}{model} \PY{o}{=} \PY{n}{LpProblem}\PY{p}{(}\PY{n}{name}\PY{o}{=}\PY{l+s+s2}{\PYZdq{}}\PY{l+s+s2}{small\PYZhy{}problem}\PY{l+s+s2}{\PYZdq{}}\PY{p}{,} \PY{n}{sense}\PY{o}{=}\PY{n}{LpMaximize}\PY{p}{)}

\PY{c+c1}{\PYZsh{} Variables de decisión}
\PY{n}{x1} \PY{o}{=} \PY{n}{LpVariable}\PY{p}{(}\PY{n}{name}\PY{o}{=}\PY{l+s+s2}{\PYZdq{}}\PY{l+s+s2}{x1}\PY{l+s+s2}{\PYZdq{}}\PY{p}{,} \PY{n}{lowBound}\PY{o}{=}\PY{l+m+mi}{0}\PY{p}{)}
\PY{n}{x2} \PY{o}{=} \PY{n}{LpVariable}\PY{p}{(}\PY{n}{name}\PY{o}{=}\PY{l+s+s2}{\PYZdq{}}\PY{l+s+s2}{x2}\PY{l+s+s2}{\PYZdq{}}\PY{p}{,} \PY{n}{lowBound}\PY{o}{=}\PY{l+m+mi}{0}\PY{p}{)}
\PY{n}{x3} \PY{o}{=} \PY{n}{LpVariable}\PY{p}{(}\PY{n}{name}\PY{o}{=}\PY{l+s+s2}{\PYZdq{}}\PY{l+s+s2}{x3}\PY{l+s+s2}{\PYZdq{}}\PY{p}{,} \PY{n}{lowBound}\PY{o}{=}\PY{l+m+mi}{0}\PY{p}{)}
\PY{n}{x4} \PY{o}{=} \PY{n}{LpVariable}\PY{p}{(}\PY{n}{name}\PY{o}{=}\PY{l+s+s2}{\PYZdq{}}\PY{l+s+s2}{x4}\PY{l+s+s2}{\PYZdq{}}\PY{p}{,} \PY{n}{lowBound}\PY{o}{=}\PY{l+m+mi}{0}\PY{p}{)}




\PY{c+c1}{\PYZsh{} Se agregan las restricciones del modelo}
\PY{n}{model} \PY{o}{+}\PY{o}{=} \PY{p}{(}\PY{l+m+mi}{2}\PY{o}{*}\PY{n}{x1} \PY{o}{+} \PY{n}{x2} \PY{o}{+} \PY{l+m+mi}{3}\PY{o}{*}\PY{n}{x3} \PY{o}{+} \PY{n}{x4} \PY{o}{\PYZlt{}}\PY{o}{=} \PY{l+m+mi}{8}\PY{p}{,} \PY{l+s+s2}{\PYZdq{}}\PY{l+s+s2}{R1}\PY{l+s+s2}{\PYZdq{}}\PY{p}{)}
\PY{n}{model} \PY{o}{+}\PY{o}{=} \PY{p}{(}\PY{l+m+mi}{2}\PY{o}{*}\PY{n}{x1} \PY{o}{+} \PY{l+m+mi}{3}\PY{o}{*}\PY{n}{x2} \PY{o}{+} \PY{l+m+mi}{4}\PY{o}{*}\PY{n}{x4}  \PY{o}{\PYZlt{}}\PY{o}{=} \PY{l+m+mi}{12}\PY{p}{,} \PY{l+s+s2}{\PYZdq{}}\PY{l+s+s2}{R2}\PY{l+s+s2}{\PYZdq{}}\PY{p}{)}
\PY{n}{model} \PY{o}{+}\PY{o}{=} \PY{p}{(}\PY{l+m+mi}{3}\PY{o}{*}\PY{n}{x1} \PY{o}{+} \PY{n}{x2} \PY{o}{+} \PY{l+m+mi}{2}\PY{o}{*}\PY{n}{x3}  \PY{o}{\PYZlt{}}\PY{o}{=} \PY{l+m+mi}{18}\PY{p}{,} \PY{l+s+s2}{\PYZdq{}}\PY{l+s+s2}{R3}\PY{l+s+s2}{\PYZdq{}}\PY{p}{)}

\PY{c+c1}{\PYZsh{} Función objetivo}
\PY{n}{model} \PY{o}{+}\PY{o}{=} \PY{n}{x1}\PY{o}{+}\PY{l+m+mi}{2}\PY{o}{*}\PY{n}{x2}\PY{o}{+}\PY{n}{x3}\PY{o}{+}\PY{n}{x4}

\PY{n+nb}{print}\PY{p}{(}\PY{n}{model}\PY{p}{)}
\end{Verbatim}
\end{tcolorbox}

    \begin{Verbatim}[commandchars=\\\{\}]
small-problem:
MAXIMIZE
1*x1 + 2*x2 + 1*x3 + 1*x4 + 0
SUBJECT TO
R1: 2 x1 + x2 + 3 x3 + x4 <= 8

R2: 2 x1 + 3 x2 + 4 x4 <= 12

R3: 3 x1 + x2 + 2 x3 <= 18

VARIABLES
x1 Continuous
x2 Continuous
x3 Continuous
x4 Continuous

    \end{Verbatim}

    Posteriormente calculamos la solución

    \begin{tcolorbox}[breakable, size=fbox, boxrule=1pt, pad at break*=1mm,colback=cellbackground, colframe=cellborder]
\prompt{In}{incolor}{12}{\boxspacing}
\begin{Verbatim}[commandchars=\\\{\}]
\PY{n}{status} \PY{o}{=} \PY{n}{model}\PY{o}{.}\PY{n}{solve}\PY{p}{(}\PY{n}{PULP\PYZus{}CBC\PYZus{}CMD}\PY{p}{(}\PY{n}{msg}\PY{o}{=}\PY{k+kc}{False}\PY{p}{)}\PY{p}{)}

\PY{n+nb}{print}\PY{p}{(}\PY{l+s+s2}{\PYZdq{}}\PY{l+s+s2}{Resultado: }\PY{l+s+s2}{\PYZdq{}}\PY{p}{,} \PY{n}{model}\PY{o}{.}\PY{n}{status}\PY{p}{,} \PY{l+s+s2}{\PYZdq{}}\PY{l+s+s2}{ | }\PY{l+s+s2}{\PYZdq{}}\PY{p}{,} \PY{n}{LpStatus}\PY{p}{[}\PY{n}{model}\PY{o}{.}\PY{n}{status}\PY{p}{]}\PY{p}{)}

\PY{n+nb}{print}\PY{p}{(}\PY{l+s+s2}{\PYZdq{}}\PY{l+s+s2}{Valor de la funciónn objetivo: }\PY{l+s+s2}{\PYZdq{}} \PY{p}{,} \PY{n}{model}\PY{o}{.}\PY{n}{objective}\PY{o}{.}\PY{n}{value}\PY{p}{(}\PY{p}{)}\PY{p}{)}


\PY{n+nb}{print}\PY{p}{(}\PY{l+s+s1}{\PYZsq{}}\PY{l+s+s1}{Solución:}\PY{l+s+s1}{\PYZsq{}}\PY{p}{)}
\PY{k}{for} \PY{n}{var} \PY{o+ow}{in} \PY{n}{model}\PY{o}{.}\PY{n}{variables}\PY{p}{(}\PY{p}{)}\PY{p}{:}
    \PY{n+nb}{print}\PY{p}{(}\PY{l+s+s2}{\PYZdq{}}\PY{l+s+si}{\PYZpc{}10s}\PY{l+s+s2}{: }\PY{l+s+si}{\PYZpc{}f}\PY{l+s+s2}{\PYZdq{}}  \PY{o}{\PYZpc{}} \PY{p}{(}\PY{n}{var}\PY{o}{.}\PY{n}{name}\PY{p}{,} \PY{n}{var}\PY{o}{.}\PY{n}{value}\PY{p}{(}\PY{p}{)}\PY{p}{)} \PY{p}{)}

\PY{n+nb}{print}\PY{p}{(}\PY{l+s+s1}{\PYZsq{}}\PY{l+s+se}{\PYZbs{}n}\PY{l+s+s1}{Variables de holgura:}\PY{l+s+s1}{\PYZsq{}}\PY{p}{)}
\PY{k}{for} \PY{n}{name}\PY{p}{,} \PY{n}{constraint} \PY{o+ow}{in} \PY{n}{model}\PY{o}{.}\PY{n}{constraints}\PY{o}{.}\PY{n}{items}\PY{p}{(}\PY{p}{)}\PY{p}{:}
    \PY{n+nb}{print}\PY{p}{(}\PY{l+s+s2}{\PYZdq{}}\PY{l+s+si}{\PYZpc{}10s}\PY{l+s+s2}{: }\PY{l+s+si}{\PYZpc{}f}\PY{l+s+s2}{\PYZdq{}} \PY{o}{\PYZpc{}} \PY{p}{(}\PY{n}{name}\PY{p}{,} \PY{n}{constraint}\PY{o}{.}\PY{n}{value}\PY{p}{(}\PY{p}{)}\PY{p}{)} \PY{p}{)}
\end{Verbatim}
\end{tcolorbox}

    \begin{Verbatim}[commandchars=\\\{\}]
Resultado:  1  |  Optimal
Valor de la funciónn objetivo:  9.3333333
Solución:
        x1: 0.000000
        x2: 4.000000
        x3: 1.333333
        x4: 0.000000

Variables de holgura:
        R1: -0.000000
        R2: 0.000000
        R3: -11.333333
    \end{Verbatim}

    Y la solución obtenida por esta librería coincide con la que hemos
obtenido hallando las soluciones básicas factibles.


    % Add a bibliography block to the postdoc
    
    
    
\end{document}
