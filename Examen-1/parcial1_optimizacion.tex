\documentclass[11pt]{article}

    \usepackage[breakable]{tcolorbox}
    \usepackage{parskip} % Stop auto-indenting (to mimic markdown behaviour)
    
    \usepackage{iftex}
    \ifPDFTeX
    	\usepackage[T1]{fontenc}
    	\usepackage{mathpazo}
    \else
    	\usepackage{fontspec}
    \fi

    % Basic figure setup, for now with no caption control since it's done
    % automatically by Pandoc (which extracts ![](path) syntax from Markdown).
    \usepackage{graphicx}
    % Maintain compatibility with old templates. Remove in nbconvert 6.0
    \let\Oldincludegraphics\includegraphics
    % Ensure that by default, figures have no caption (until we provide a
    % proper Figure object with a Caption API and a way to capture that
    % in the conversion process - todo).
    \usepackage{caption}
    \DeclareCaptionFormat{nocaption}{}
    \captionsetup{format=nocaption,aboveskip=0pt,belowskip=0pt}

    \usepackage{float}
    \floatplacement{figure}{H} % forces figures to be placed at the correct location
    \usepackage{xcolor} % Allow colors to be defined
    \usepackage{enumerate} % Needed for markdown enumerations to work
    \usepackage{geometry} % Used to adjust the document margins
    \usepackage{amsmath} % Equations
    \usepackage{amssymb} % Equations
    \usepackage{textcomp} % defines textquotesingle
    % Hack from http://tex.stackexchange.com/a/47451/13684:
    \AtBeginDocument{%
        \def\PYZsq{\textquotesingle}% Upright quotes in Pygmentized code
    }
    \usepackage{upquote} % Upright quotes for verbatim code
    \usepackage{eurosym} % defines \euro
    \usepackage[mathletters]{ucs} % Extended unicode (utf-8) support
    \usepackage{fancyvrb} % verbatim replacement that allows latex
    \usepackage{grffile} % extends the file name processing of package graphics 
                         % to support a larger range
    \makeatletter % fix for old versions of grffile with XeLaTeX
    \@ifpackagelater{grffile}{2019/11/01}
    {
      % Do nothing on new versions
    }
    {
      \def\Gread@@xetex#1{%
        \IfFileExists{"\Gin@base".bb}%
        {\Gread@eps{\Gin@base.bb}}%
        {\Gread@@xetex@aux#1}%
      }
    }
    \makeatother
    \usepackage[Export]{adjustbox} % Used to constrain images to a maximum size
    \adjustboxset{max size={0.9\linewidth}{0.9\paperheight}}

    % The hyperref package gives us a pdf with properly built
    % internal navigation ('pdf bookmarks' for the table of contents,
    % internal cross-reference links, web links for URLs, etc.)
    \usepackage{hyperref}
    % The default LaTeX title has an obnoxious amount of whitespace. By default,
    % titling removes some of it. It also provides customization options.
    \usepackage{titling}
    \usepackage{longtable} % longtable support required by pandoc >1.10
    \usepackage{booktabs}  % table support for pandoc > 1.12.2
    \usepackage[inline]{enumitem} % IRkernel/repr support (it uses the enumerate* environment)
    \usepackage[normalem]{ulem} % ulem is needed to support strikethroughs (\sout)
                                % normalem makes italics be italics, not underlines
    \usepackage{mathrsfs}
    

    
    % Colors for the hyperref package
    \definecolor{urlcolor}{rgb}{0,.145,.698}
    \definecolor{linkcolor}{rgb}{.71,0.21,0.01}
    \definecolor{citecolor}{rgb}{.12,.54,.11}

    % ANSI colors
    \definecolor{ansi-black}{HTML}{3E424D}
    \definecolor{ansi-black-intense}{HTML}{282C36}
    \definecolor{ansi-red}{HTML}{E75C58}
    \definecolor{ansi-red-intense}{HTML}{B22B31}
    \definecolor{ansi-green}{HTML}{00A250}
    \definecolor{ansi-green-intense}{HTML}{007427}
    \definecolor{ansi-yellow}{HTML}{DDB62B}
    \definecolor{ansi-yellow-intense}{HTML}{B27D12}
    \definecolor{ansi-blue}{HTML}{208FFB}
    \definecolor{ansi-blue-intense}{HTML}{0065CA}
    \definecolor{ansi-magenta}{HTML}{D160C4}
    \definecolor{ansi-magenta-intense}{HTML}{A03196}
    \definecolor{ansi-cyan}{HTML}{60C6C8}
    \definecolor{ansi-cyan-intense}{HTML}{258F8F}
    \definecolor{ansi-white}{HTML}{C5C1B4}
    \definecolor{ansi-white-intense}{HTML}{A1A6B2}
    \definecolor{ansi-default-inverse-fg}{HTML}{FFFFFF}
    \definecolor{ansi-default-inverse-bg}{HTML}{000000}

    % common color for the border for error outputs.
    \definecolor{outerrorbackground}{HTML}{FFDFDF}

    % commands and environments needed by pandoc snippets
    % extracted from the output of `pandoc -s`
    \providecommand{\tightlist}{%
      \setlength{\itemsep}{0pt}\setlength{\parskip}{0pt}}
    \DefineVerbatimEnvironment{Highlighting}{Verbatim}{commandchars=\\\{\}}
    % Add ',fontsize=\small' for more characters per line
    \newenvironment{Shaded}{}{}
    \newcommand{\KeywordTok}[1]{\textcolor[rgb]{0.00,0.44,0.13}{\textbf{{#1}}}}
    \newcommand{\DataTypeTok}[1]{\textcolor[rgb]{0.56,0.13,0.00}{{#1}}}
    \newcommand{\DecValTok}[1]{\textcolor[rgb]{0.25,0.63,0.44}{{#1}}}
    \newcommand{\BaseNTok}[1]{\textcolor[rgb]{0.25,0.63,0.44}{{#1}}}
    \newcommand{\FloatTok}[1]{\textcolor[rgb]{0.25,0.63,0.44}{{#1}}}
    \newcommand{\CharTok}[1]{\textcolor[rgb]{0.25,0.44,0.63}{{#1}}}
    \newcommand{\StringTok}[1]{\textcolor[rgb]{0.25,0.44,0.63}{{#1}}}
    \newcommand{\CommentTok}[1]{\textcolor[rgb]{0.38,0.63,0.69}{\textit{{#1}}}}
    \newcommand{\OtherTok}[1]{\textcolor[rgb]{0.00,0.44,0.13}{{#1}}}
    \newcommand{\AlertTok}[1]{\textcolor[rgb]{1.00,0.00,0.00}{\textbf{{#1}}}}
    \newcommand{\FunctionTok}[1]{\textcolor[rgb]{0.02,0.16,0.49}{{#1}}}
    \newcommand{\RegionMarkerTok}[1]{{#1}}
    \newcommand{\ErrorTok}[1]{\textcolor[rgb]{1.00,0.00,0.00}{\textbf{{#1}}}}
    \newcommand{\NormalTok}[1]{{#1}}
    
    % Additional commands for more recent versions of Pandoc
    \newcommand{\ConstantTok}[1]{\textcolor[rgb]{0.53,0.00,0.00}{{#1}}}
    \newcommand{\SpecialCharTok}[1]{\textcolor[rgb]{0.25,0.44,0.63}{{#1}}}
    \newcommand{\VerbatimStringTok}[1]{\textcolor[rgb]{0.25,0.44,0.63}{{#1}}}
    \newcommand{\SpecialStringTok}[1]{\textcolor[rgb]{0.73,0.40,0.53}{{#1}}}
    \newcommand{\ImportTok}[1]{{#1}}
    \newcommand{\DocumentationTok}[1]{\textcolor[rgb]{0.73,0.13,0.13}{\textit{{#1}}}}
    \newcommand{\AnnotationTok}[1]{\textcolor[rgb]{0.38,0.63,0.69}{\textbf{\textit{{#1}}}}}
    \newcommand{\CommentVarTok}[1]{\textcolor[rgb]{0.38,0.63,0.69}{\textbf{\textit{{#1}}}}}
    \newcommand{\VariableTok}[1]{\textcolor[rgb]{0.10,0.09,0.49}{{#1}}}
    \newcommand{\ControlFlowTok}[1]{\textcolor[rgb]{0.00,0.44,0.13}{\textbf{{#1}}}}
    \newcommand{\OperatorTok}[1]{\textcolor[rgb]{0.40,0.40,0.40}{{#1}}}
    \newcommand{\BuiltInTok}[1]{{#1}}
    \newcommand{\ExtensionTok}[1]{{#1}}
    \newcommand{\PreprocessorTok}[1]{\textcolor[rgb]{0.74,0.48,0.00}{{#1}}}
    \newcommand{\AttributeTok}[1]{\textcolor[rgb]{0.49,0.56,0.16}{{#1}}}
    \newcommand{\InformationTok}[1]{\textcolor[rgb]{0.38,0.63,0.69}{\textbf{\textit{{#1}}}}}
    \newcommand{\WarningTok}[1]{\textcolor[rgb]{0.38,0.63,0.69}{\textbf{\textit{{#1}}}}}
    
    
    % Define a nice break command that doesn't care if a line doesn't already
    % exist.
    \def\br{\hspace*{\fill} \\* }
    % Math Jax compatibility definitions
    \def\gt{>}
    \def\lt{<}
    \let\Oldtex\TeX
    \let\Oldlatex\LaTeX
    \renewcommand{\TeX}{\textrm{\Oldtex}}
    \renewcommand{\LaTeX}{\textrm{\Oldlatex}}
    % Document parameters
    % Document title
    \title{parcial1\_optimizacion}
    
    
    
    
    
% Pygments definitions
\makeatletter
\def\PY@reset{\let\PY@it=\relax \let\PY@bf=\relax%
    \let\PY@ul=\relax \let\PY@tc=\relax%
    \let\PY@bc=\relax \let\PY@ff=\relax}
\def\PY@tok#1{\csname PY@tok@#1\endcsname}
\def\PY@toks#1+{\ifx\relax#1\empty\else%
    \PY@tok{#1}\expandafter\PY@toks\fi}
\def\PY@do#1{\PY@bc{\PY@tc{\PY@ul{%
    \PY@it{\PY@bf{\PY@ff{#1}}}}}}}
\def\PY#1#2{\PY@reset\PY@toks#1+\relax+\PY@do{#2}}

\@namedef{PY@tok@w}{\def\PY@tc##1{\textcolor[rgb]{0.73,0.73,0.73}{##1}}}
\@namedef{PY@tok@c}{\let\PY@it=\textit\def\PY@tc##1{\textcolor[rgb]{0.25,0.50,0.50}{##1}}}
\@namedef{PY@tok@cp}{\def\PY@tc##1{\textcolor[rgb]{0.74,0.48,0.00}{##1}}}
\@namedef{PY@tok@k}{\let\PY@bf=\textbf\def\PY@tc##1{\textcolor[rgb]{0.00,0.50,0.00}{##1}}}
\@namedef{PY@tok@kp}{\def\PY@tc##1{\textcolor[rgb]{0.00,0.50,0.00}{##1}}}
\@namedef{PY@tok@kt}{\def\PY@tc##1{\textcolor[rgb]{0.69,0.00,0.25}{##1}}}
\@namedef{PY@tok@o}{\def\PY@tc##1{\textcolor[rgb]{0.40,0.40,0.40}{##1}}}
\@namedef{PY@tok@ow}{\let\PY@bf=\textbf\def\PY@tc##1{\textcolor[rgb]{0.67,0.13,1.00}{##1}}}
\@namedef{PY@tok@nb}{\def\PY@tc##1{\textcolor[rgb]{0.00,0.50,0.00}{##1}}}
\@namedef{PY@tok@nf}{\def\PY@tc##1{\textcolor[rgb]{0.00,0.00,1.00}{##1}}}
\@namedef{PY@tok@nc}{\let\PY@bf=\textbf\def\PY@tc##1{\textcolor[rgb]{0.00,0.00,1.00}{##1}}}
\@namedef{PY@tok@nn}{\let\PY@bf=\textbf\def\PY@tc##1{\textcolor[rgb]{0.00,0.00,1.00}{##1}}}
\@namedef{PY@tok@ne}{\let\PY@bf=\textbf\def\PY@tc##1{\textcolor[rgb]{0.82,0.25,0.23}{##1}}}
\@namedef{PY@tok@nv}{\def\PY@tc##1{\textcolor[rgb]{0.10,0.09,0.49}{##1}}}
\@namedef{PY@tok@no}{\def\PY@tc##1{\textcolor[rgb]{0.53,0.00,0.00}{##1}}}
\@namedef{PY@tok@nl}{\def\PY@tc##1{\textcolor[rgb]{0.63,0.63,0.00}{##1}}}
\@namedef{PY@tok@ni}{\let\PY@bf=\textbf\def\PY@tc##1{\textcolor[rgb]{0.60,0.60,0.60}{##1}}}
\@namedef{PY@tok@na}{\def\PY@tc##1{\textcolor[rgb]{0.49,0.56,0.16}{##1}}}
\@namedef{PY@tok@nt}{\let\PY@bf=\textbf\def\PY@tc##1{\textcolor[rgb]{0.00,0.50,0.00}{##1}}}
\@namedef{PY@tok@nd}{\def\PY@tc##1{\textcolor[rgb]{0.67,0.13,1.00}{##1}}}
\@namedef{PY@tok@s}{\def\PY@tc##1{\textcolor[rgb]{0.73,0.13,0.13}{##1}}}
\@namedef{PY@tok@sd}{\let\PY@it=\textit\def\PY@tc##1{\textcolor[rgb]{0.73,0.13,0.13}{##1}}}
\@namedef{PY@tok@si}{\let\PY@bf=\textbf\def\PY@tc##1{\textcolor[rgb]{0.73,0.40,0.53}{##1}}}
\@namedef{PY@tok@se}{\let\PY@bf=\textbf\def\PY@tc##1{\textcolor[rgb]{0.73,0.40,0.13}{##1}}}
\@namedef{PY@tok@sr}{\def\PY@tc##1{\textcolor[rgb]{0.73,0.40,0.53}{##1}}}
\@namedef{PY@tok@ss}{\def\PY@tc##1{\textcolor[rgb]{0.10,0.09,0.49}{##1}}}
\@namedef{PY@tok@sx}{\def\PY@tc##1{\textcolor[rgb]{0.00,0.50,0.00}{##1}}}
\@namedef{PY@tok@m}{\def\PY@tc##1{\textcolor[rgb]{0.40,0.40,0.40}{##1}}}
\@namedef{PY@tok@gh}{\let\PY@bf=\textbf\def\PY@tc##1{\textcolor[rgb]{0.00,0.00,0.50}{##1}}}
\@namedef{PY@tok@gu}{\let\PY@bf=\textbf\def\PY@tc##1{\textcolor[rgb]{0.50,0.00,0.50}{##1}}}
\@namedef{PY@tok@gd}{\def\PY@tc##1{\textcolor[rgb]{0.63,0.00,0.00}{##1}}}
\@namedef{PY@tok@gi}{\def\PY@tc##1{\textcolor[rgb]{0.00,0.63,0.00}{##1}}}
\@namedef{PY@tok@gr}{\def\PY@tc##1{\textcolor[rgb]{1.00,0.00,0.00}{##1}}}
\@namedef{PY@tok@ge}{\let\PY@it=\textit}
\@namedef{PY@tok@gs}{\let\PY@bf=\textbf}
\@namedef{PY@tok@gp}{\let\PY@bf=\textbf\def\PY@tc##1{\textcolor[rgb]{0.00,0.00,0.50}{##1}}}
\@namedef{PY@tok@go}{\def\PY@tc##1{\textcolor[rgb]{0.53,0.53,0.53}{##1}}}
\@namedef{PY@tok@gt}{\def\PY@tc##1{\textcolor[rgb]{0.00,0.27,0.87}{##1}}}
\@namedef{PY@tok@err}{\def\PY@bc##1{{\setlength{\fboxsep}{\string -\fboxrule}\fcolorbox[rgb]{1.00,0.00,0.00}{1,1,1}{\strut ##1}}}}
\@namedef{PY@tok@kc}{\let\PY@bf=\textbf\def\PY@tc##1{\textcolor[rgb]{0.00,0.50,0.00}{##1}}}
\@namedef{PY@tok@kd}{\let\PY@bf=\textbf\def\PY@tc##1{\textcolor[rgb]{0.00,0.50,0.00}{##1}}}
\@namedef{PY@tok@kn}{\let\PY@bf=\textbf\def\PY@tc##1{\textcolor[rgb]{0.00,0.50,0.00}{##1}}}
\@namedef{PY@tok@kr}{\let\PY@bf=\textbf\def\PY@tc##1{\textcolor[rgb]{0.00,0.50,0.00}{##1}}}
\@namedef{PY@tok@bp}{\def\PY@tc##1{\textcolor[rgb]{0.00,0.50,0.00}{##1}}}
\@namedef{PY@tok@fm}{\def\PY@tc##1{\textcolor[rgb]{0.00,0.00,1.00}{##1}}}
\@namedef{PY@tok@vc}{\def\PY@tc##1{\textcolor[rgb]{0.10,0.09,0.49}{##1}}}
\@namedef{PY@tok@vg}{\def\PY@tc##1{\textcolor[rgb]{0.10,0.09,0.49}{##1}}}
\@namedef{PY@tok@vi}{\def\PY@tc##1{\textcolor[rgb]{0.10,0.09,0.49}{##1}}}
\@namedef{PY@tok@vm}{\def\PY@tc##1{\textcolor[rgb]{0.10,0.09,0.49}{##1}}}
\@namedef{PY@tok@sa}{\def\PY@tc##1{\textcolor[rgb]{0.73,0.13,0.13}{##1}}}
\@namedef{PY@tok@sb}{\def\PY@tc##1{\textcolor[rgb]{0.73,0.13,0.13}{##1}}}
\@namedef{PY@tok@sc}{\def\PY@tc##1{\textcolor[rgb]{0.73,0.13,0.13}{##1}}}
\@namedef{PY@tok@dl}{\def\PY@tc##1{\textcolor[rgb]{0.73,0.13,0.13}{##1}}}
\@namedef{PY@tok@s2}{\def\PY@tc##1{\textcolor[rgb]{0.73,0.13,0.13}{##1}}}
\@namedef{PY@tok@sh}{\def\PY@tc##1{\textcolor[rgb]{0.73,0.13,0.13}{##1}}}
\@namedef{PY@tok@s1}{\def\PY@tc##1{\textcolor[rgb]{0.73,0.13,0.13}{##1}}}
\@namedef{PY@tok@mb}{\def\PY@tc##1{\textcolor[rgb]{0.40,0.40,0.40}{##1}}}
\@namedef{PY@tok@mf}{\def\PY@tc##1{\textcolor[rgb]{0.40,0.40,0.40}{##1}}}
\@namedef{PY@tok@mh}{\def\PY@tc##1{\textcolor[rgb]{0.40,0.40,0.40}{##1}}}
\@namedef{PY@tok@mi}{\def\PY@tc##1{\textcolor[rgb]{0.40,0.40,0.40}{##1}}}
\@namedef{PY@tok@il}{\def\PY@tc##1{\textcolor[rgb]{0.40,0.40,0.40}{##1}}}
\@namedef{PY@tok@mo}{\def\PY@tc##1{\textcolor[rgb]{0.40,0.40,0.40}{##1}}}
\@namedef{PY@tok@ch}{\let\PY@it=\textit\def\PY@tc##1{\textcolor[rgb]{0.25,0.50,0.50}{##1}}}
\@namedef{PY@tok@cm}{\let\PY@it=\textit\def\PY@tc##1{\textcolor[rgb]{0.25,0.50,0.50}{##1}}}
\@namedef{PY@tok@cpf}{\let\PY@it=\textit\def\PY@tc##1{\textcolor[rgb]{0.25,0.50,0.50}{##1}}}
\@namedef{PY@tok@c1}{\let\PY@it=\textit\def\PY@tc##1{\textcolor[rgb]{0.25,0.50,0.50}{##1}}}
\@namedef{PY@tok@cs}{\let\PY@it=\textit\def\PY@tc##1{\textcolor[rgb]{0.25,0.50,0.50}{##1}}}

\def\PYZbs{\char`\\}
\def\PYZus{\char`\_}
\def\PYZob{\char`\{}
\def\PYZcb{\char`\}}
\def\PYZca{\char`\^}
\def\PYZam{\char`\&}
\def\PYZlt{\char`\<}
\def\PYZgt{\char`\>}
\def\PYZsh{\char`\#}
\def\PYZpc{\char`\%}
\def\PYZdl{\char`\$}
\def\PYZhy{\char`\-}
\def\PYZsq{\char`\'}
\def\PYZdq{\char`\"}
\def\PYZti{\char`\~}
% for compatibility with earlier versions
\def\PYZat{@}
\def\PYZlb{[}
\def\PYZrb{]}
\makeatother


    % For linebreaks inside Verbatim environment from package fancyvrb. 
    \makeatletter
        \newbox\Wrappedcontinuationbox 
        \newbox\Wrappedvisiblespacebox 
        \newcommand*\Wrappedvisiblespace {\textcolor{red}{\textvisiblespace}} 
        \newcommand*\Wrappedcontinuationsymbol {\textcolor{red}{\llap{\tiny$\m@th\hookrightarrow$}}} 
        \newcommand*\Wrappedcontinuationindent {3ex } 
        \newcommand*\Wrappedafterbreak {\kern\Wrappedcontinuationindent\copy\Wrappedcontinuationbox} 
        % Take advantage of the already applied Pygments mark-up to insert 
        % potential linebreaks for TeX processing. 
        %        {, <, #, %, $, ' and ": go to next line. 
        %        _, }, ^, &, >, - and ~: stay at end of broken line. 
        % Use of \textquotesingle for straight quote. 
        \newcommand*\Wrappedbreaksatspecials {% 
            \def\PYGZus{\discretionary{\char`\_}{\Wrappedafterbreak}{\char`\_}}% 
            \def\PYGZob{\discretionary{}{\Wrappedafterbreak\char`\{}{\char`\{}}% 
            \def\PYGZcb{\discretionary{\char`\}}{\Wrappedafterbreak}{\char`\}}}% 
            \def\PYGZca{\discretionary{\char`\^}{\Wrappedafterbreak}{\char`\^}}% 
            \def\PYGZam{\discretionary{\char`\&}{\Wrappedafterbreak}{\char`\&}}% 
            \def\PYGZlt{\discretionary{}{\Wrappedafterbreak\char`\<}{\char`\<}}% 
            \def\PYGZgt{\discretionary{\char`\>}{\Wrappedafterbreak}{\char`\>}}% 
            \def\PYGZsh{\discretionary{}{\Wrappedafterbreak\char`\#}{\char`\#}}% 
            \def\PYGZpc{\discretionary{}{\Wrappedafterbreak\char`\%}{\char`\%}}% 
            \def\PYGZdl{\discretionary{}{\Wrappedafterbreak\char`\$}{\char`\$}}% 
            \def\PYGZhy{\discretionary{\char`\-}{\Wrappedafterbreak}{\char`\-}}% 
            \def\PYGZsq{\discretionary{}{\Wrappedafterbreak\textquotesingle}{\textquotesingle}}% 
            \def\PYGZdq{\discretionary{}{\Wrappedafterbreak\char`\"}{\char`\"}}% 
            \def\PYGZti{\discretionary{\char`\~}{\Wrappedafterbreak}{\char`\~}}% 
        } 
        % Some characters . , ; ? ! / are not pygmentized. 
        % This macro makes them "active" and they will insert potential linebreaks 
        \newcommand*\Wrappedbreaksatpunct {% 
            \lccode`\~`\.\lowercase{\def~}{\discretionary{\hbox{\char`\.}}{\Wrappedafterbreak}{\hbox{\char`\.}}}% 
            \lccode`\~`\,\lowercase{\def~}{\discretionary{\hbox{\char`\,}}{\Wrappedafterbreak}{\hbox{\char`\,}}}% 
            \lccode`\~`\;\lowercase{\def~}{\discretionary{\hbox{\char`\;}}{\Wrappedafterbreak}{\hbox{\char`\;}}}% 
            \lccode`\~`\:\lowercase{\def~}{\discretionary{\hbox{\char`\:}}{\Wrappedafterbreak}{\hbox{\char`\:}}}% 
            \lccode`\~`\?\lowercase{\def~}{\discretionary{\hbox{\char`\?}}{\Wrappedafterbreak}{\hbox{\char`\?}}}% 
            \lccode`\~`\!\lowercase{\def~}{\discretionary{\hbox{\char`\!}}{\Wrappedafterbreak}{\hbox{\char`\!}}}% 
            \lccode`\~`\/\lowercase{\def~}{\discretionary{\hbox{\char`\/}}{\Wrappedafterbreak}{\hbox{\char`\/}}}% 
            \catcode`\.\active
            \catcode`\,\active 
            \catcode`\;\active
            \catcode`\:\active
            \catcode`\?\active
            \catcode`\!\active
            \catcode`\/\active 
            \lccode`\~`\~ 	
        }
    \makeatother

    \let\OriginalVerbatim=\Verbatim
    \makeatletter
    \renewcommand{\Verbatim}[1][1]{%
        %\parskip\z@skip
        \sbox\Wrappedcontinuationbox {\Wrappedcontinuationsymbol}%
        \sbox\Wrappedvisiblespacebox {\FV@SetupFont\Wrappedvisiblespace}%
        \def\FancyVerbFormatLine ##1{\hsize\linewidth
            \vtop{\raggedright\hyphenpenalty\z@\exhyphenpenalty\z@
                \doublehyphendemerits\z@\finalhyphendemerits\z@
                \strut ##1\strut}%
        }%
        % If the linebreak is at a space, the latter will be displayed as visible
        % space at end of first line, and a continuation symbol starts next line.
        % Stretch/shrink are however usually zero for typewriter font.
        \def\FV@Space {%
            \nobreak\hskip\z@ plus\fontdimen3\font minus\fontdimen4\font
            \discretionary{\copy\Wrappedvisiblespacebox}{\Wrappedafterbreak}
            {\kern\fontdimen2\font}%
        }%
        
        % Allow breaks at special characters using \PYG... macros.
        \Wrappedbreaksatspecials
        % Breaks at punctuation characters . , ; ? ! and / need catcode=\active 	
        \OriginalVerbatim[#1,codes*=\Wrappedbreaksatpunct]%
    }
    \makeatother

    % Exact colors from NB
    \definecolor{incolor}{HTML}{303F9F}
    \definecolor{outcolor}{HTML}{D84315}
    \definecolor{cellborder}{HTML}{CFCFCF}
    \definecolor{cellbackground}{HTML}{F7F7F7}
    
    % prompt
    \makeatletter
    \newcommand{\boxspacing}{\kern\kvtcb@left@rule\kern\kvtcb@boxsep}
    \makeatother
    \newcommand{\prompt}[4]{
        {\ttfamily\llap{{\color{#2}[#3]:\hspace{3pt}#4}}\vspace{-\baselineskip}}
    }
    

    
    % Prevent overflowing lines due to hard-to-break entities
    \sloppy 
    % Setup hyperref package
    \hypersetup{
      breaklinks=true,  % so long urls are correctly broken across lines
      colorlinks=true,
      urlcolor=urlcolor,
      linkcolor=linkcolor,
      citecolor=citecolor,
      }
    % Slightly bigger margins than the latex defaults
    
    \geometry{verbose,tmargin=1in,bmargin=1in,lmargin=1in,rmargin=1in}
    
    

\begin{document}
    
\title{Examen Parcial 1 Optimización}
\author{Roberto Vásquez Martínez \\ Profesor: Joaquín Peña Acevedo}
\date{06/Abril/2022}
\maketitle  
    
    

    
    \hypertarget{curso-de-optimizaciuxf3n-demat}{%
\section{Curso de Optimización
(DEMAT)}\label{curso-de-optimizaciuxf3n-demat}}

\hypertarget{parcial-1}{%
\subsection{Parcial 1}\label{parcial-1}}

\begin{longtable}[]{@{}ll@{}}
\toprule
Descripción: & Fechas \\
\midrule
\endhead
Fecha de publicación del documento: & \textbf{Abril 6, 2022} \\
Hora de inicio: & \textbf{15:00} \\
Hora límite de entrega: & \textbf{18:00} \\
\bottomrule
\end{longtable}

\hypertarget{indicaciones}{%
\subsubsection{Indicaciones}\label{indicaciones}}

Lea con cuidado los ejercicios.

Puede usar las notas de clase y las tareas hechas para resolver el
examen.

Al final, entregue el notebook con sus respuestas, junto con los códigos
que hagan falta para reproducir los resultados. Si es más de un archivo,
genere un archivo ZIP que contenga el notebook y los scripts
adicionales.

A partir del notebook genere un archivo PDF con las respuestas y envíelo
por separado antes de la hora límite.

    \hypertarget{ejercicio-1.-3-puntos}{%
\subsection{Ejercicio 1. (3 puntos)}\label{ejercicio-1.-3-puntos}}

Considere la función \(f: \mathbb{R}^n \rightarrow \mathbb{R}\) definida
como

\[ f(\mathbf{x}) = \sum_{i=1}^n x_i^2  
+ \left(0.5 \sum_{i=1}^n i x_i  \right)^2
+ \left(0.5 \sum_{i=1}^n i x_i  \right)^4
\]

\begin{enumerate}
\def\labelenumi{\arabic{enumi}.}
\tightlist
\item
  Para \(n=2, 4, 8, 16, 32\), aplique el método de descenso máximo
  encontrar un mínimo de la función \(f\), usando como punto inicial
  \[ \mathbf{x}_0 = (5, -5, 5, -5, ..., 5, -5), \] y una tolerancia
  \(\tau = \epsilon_m^{1/3}\) para terminar el algoritmo cuando
  \(\|\nabla f(\mathbf{x}_k)\|<\tau\).
\end{enumerate}

\begin{itemize}
\tightlist
\item
  Para calcular el tamaño de paso \(\alpha_k\) use el algoritmo de
  backtracking.
\item
  Reporte el valor \(n\), el punto \(\mathbf{x}_k\) que devuelve el
  algoritmo, el valor de \(k\), la magnitud
  \(\|\nabla f(\mathbf{x}_k)\|\) y un mensaje que indique si cumplió el
  criterio de convergencia del algoritmo.
\end{itemize}

\begin{enumerate}
\def\labelenumi{\arabic{enumi}.}
\setcounter{enumi}{1}
\item
  Repita el paso ejercicio anterior usando el método de Newton, con
  tamaño de paso \(\alpha_k=1\).
\item
  Escriba un comentario sobre el desempeño de estos métodos de
  optimización cuando la dimensión \(n\) aumenta.
\end{enumerate}

\begin{quote}
\textbf{Nota:} Puede calcular el gradiente y Hessiana de la función de
manera analítica o usar aproximaciones numéricas.
\end{quote}

\hypertarget{respuesta}{%
\subsubsection{Respuesta}\label{respuesta}}

    Utilizaremos el módulo \texttt{lib\_examen\_1} para importar desde ahí
las funciones correspondientes a cada algoritmo de optimización. En este
caso usaremos \texttt{grad\_max} para el método de descenso máximo y
\texttt{newton\_fix\_step} para el método de Newton

    \hypertarget{muxe9todo-de-descenso-muxe1ximo}{%
\subsubsection{Método de descenso
máximo}\label{muxe9todo-de-descenso-muxe1ximo}}

    \begin{tcolorbox}[breakable, size=fbox, boxrule=1pt, pad at break*=1mm,colback=cellbackground, colframe=cellborder]
\prompt{In}{incolor}{1}{\boxspacing}
\begin{Verbatim}[commandchars=\\\{\}]
\PY{k+kn}{import} \PY{n+nn}{numpy} \PY{k}{as} \PY{n+nn}{np}
\PY{k+kn}{import} \PY{n+nn}{importlib}
\PY{k+kn}{import} \PY{n+nn}{lib\PYZus{}examen\PYZus{}1}
\PY{n}{importlib}\PY{o}{.}\PY{n}{reload}\PY{p}{(}\PY{n}{lib\PYZus{}examen\PYZus{}1}\PY{p}{)}
\PY{k+kn}{from} \PY{n+nn}{lib\PYZus{}examen\PYZus{}1} \PY{k+kn}{import} \PY{n}{proof\PYZus{}grad\PYZus{}max}\PY{p}{,} \PY{n}{f\PYZus{}eje1}\PY{p}{,} \PY{n}{grad\PYZus{}f\PYZus{}eje1}

\PY{n}{tol}\PY{o}{=}\PY{n}{np}\PY{o}{.}\PY{n}{finfo}\PY{p}{(}\PY{n+nb}{float}\PY{p}{)}\PY{o}{.}\PY{n}{eps}\PY{o}{*}\PY{o}{*}\PY{p}{(}\PY{l+m+mi}{1}\PY{o}{/}\PY{l+m+mi}{3}\PY{p}{)}
\PY{n}{N}\PY{o}{=}\PY{l+m+mi}{1000}
\PY{n}{rho}\PY{o}{=}\PY{l+m+mf}{0.8}

\PY{n}{n\PYZus{}values}\PY{o}{=}\PY{p}{[}\PY{l+m+mi}{2}\PY{p}{,}\PY{l+m+mi}{4}\PY{p}{,}\PY{l+m+mi}{8}\PY{p}{,}\PY{l+m+mi}{16}\PY{p}{,}\PY{l+m+mi}{32}\PY{p}{]}
\PY{k}{for} \PY{n}{n} \PY{o+ow}{in} \PY{n}{n\PYZus{}values}\PY{p}{:}
    \PY{n}{x0}\PY{o}{=}\PY{n}{np}\PY{o}{.}\PY{n}{array}\PY{p}{(}\PY{p}{[}\PY{l+m+mf}{5.0}\PY{o}{*}\PY{p}{(}\PY{o}{\PYZhy{}}\PY{l+m+mi}{1}\PY{p}{)}\PY{o}{*}\PY{o}{*}\PY{n}{i} \PY{k}{for} \PY{n}{i} \PY{o+ow}{in} \PY{n+nb}{range}\PY{p}{(}\PY{n}{n}\PY{p}{)}\PY{p}{]}\PY{p}{)}
    \PY{n+nb}{print}\PY{p}{(}\PY{l+s+s1}{\PYZsq{}}\PY{l+s+s1}{El valor de n es: }\PY{l+s+s1}{\PYZsq{}}\PY{p}{,}\PY{n}{n}\PY{p}{)}
    \PY{n}{proof\PYZus{}grad\PYZus{}max}\PY{p}{(}\PY{n}{f\PYZus{}eje1}\PY{p}{,}\PY{n}{grad\PYZus{}f\PYZus{}eje1}\PY{p}{,}\PY{n}{x0}\PY{p}{,}\PY{n}{N}\PY{p}{,}\PY{n}{tol}\PY{p}{,}\PY{n}{rho}\PY{p}{)}
    \PY{n+nb}{print}\PY{p}{(}\PY{l+s+s1}{\PYZsq{}}\PY{l+s+se}{\PYZbs{}n}\PY{l+s+se}{\PYZbs{}n}\PY{l+s+s1}{\PYZsq{}}\PY{p}{)}
\end{Verbatim}
\end{tcolorbox}

    \begin{Verbatim}[commandchars=\\\{\}]
El valor de n es:  2
El algoritmo de descenso máximo con backtracking CONVERGE
k =  118
xk =  [-5.75245424e-07 -1.15049085e-06]
fk =  3.722707097928575e-12
||gk|| =  5.788295421060735e-06



El valor de n es:  4
El algoritmo de descenso máximo con backtracking CONVERGE
k =  141
xk =  [5.78189256e-08 1.15621241e-07 1.73445703e-07 2.31248018e-07]
fk =  8.52298038012629e-13
||gk|| =  5.383134151117816e-06



El valor de n es:  8
El algoritmo de descenso máximo con backtracking CONVERGE
k =  514
xk =  [ 4.18702547e-07 -4.04817040e-07  4.27959552e-07 -3.95560035e-07
  4.37216557e-07 -3.86303030e-07  4.46473562e-07 -3.77046025e-07]
fk =  1.487461552736672e-12
||gk|| =  5.684802804595089e-06



El valor de n es:  16
El algoritmo de descenso máximo con backtracking NO CONVERGE
k =  1000
xk =  [ 0.14996026 -0.14786586  0.15135652 -0.1464696   0.15275279 -0.14507334
  0.15414905 -0.14367707  0.15554531 -0.14228081  0.15694158 -0.14088455
  0.15833784 -0.13948828  0.1597341  -0.13809202]
fk =  0.36116207576615017
||gk|| =  3.5413958228494193



El valor de n es:  32
El algoritmo de descenso máximo con backtracking NO CONVERGE
k =  1000
xk =  [ 3.45740741 -3.44320504  3.46687566 -3.43373679  3.4763439  -3.42426855
  3.48581215 -3.4148003   3.4952804  -3.40533205  3.50474864 -3.39586381
  3.51421689 -3.38639556  3.52368514 -3.37692732  3.53315338 -3.36745907
  3.54262163 -3.35799082  3.55208988 -3.34852258  3.56155812 -3.33905433
  3.57102637 -3.32958608  3.58049462 -3.32011784  3.58996286 -3.31064959
  3.59943111 -3.30118134]
fk =  381.58423821301477
||gk|| =  104.27115345729494



    \end{Verbatim}

    \hypertarget{muxe9todo-de-newton}{%
\subsubsection{Método de Newton}\label{muxe9todo-de-newton}}

Con el mismo número de iteraciones máxima y tolerancia ejecutamos el
método de Newton

    \begin{tcolorbox}[breakable, size=fbox, boxrule=1pt, pad at break*=1mm,colback=cellbackground, colframe=cellborder]
\prompt{In}{incolor}{2}{\boxspacing}
\begin{Verbatim}[commandchars=\\\{\}]
\PY{n}{importlib}\PY{o}{.}\PY{n}{reload}\PY{p}{(}\PY{n}{lib\PYZus{}examen\PYZus{}1}\PY{p}{)}
\PY{k+kn}{from} \PY{n+nn}{lib\PYZus{}examen\PYZus{}1} \PY{k+kn}{import} \PY{n}{newton\PYZus{}fix\PYZus{}step}\PY{p}{,} \PY{n}{hess\PYZus{}f\PYZus{}eje1}

\PY{k}{for} \PY{n}{n} \PY{o+ow}{in} \PY{n}{n\PYZus{}values}\PY{p}{:}
    \PY{n}{x0}\PY{o}{=}\PY{n}{np}\PY{o}{.}\PY{n}{array}\PY{p}{(}\PY{p}{[}\PY{l+m+mf}{5.0}\PY{o}{*}\PY{p}{(}\PY{o}{\PYZhy{}}\PY{l+m+mi}{1}\PY{p}{)}\PY{o}{*}\PY{o}{*}\PY{n}{i} \PY{k}{for} \PY{n}{i} \PY{o+ow}{in} \PY{n+nb}{range}\PY{p}{(}\PY{n}{n}\PY{p}{)}\PY{p}{]}\PY{p}{)}
    \PY{n}{dic\PYZus{}results}\PY{p}{,} \PY{n}{trayectory}\PY{o}{=}\PY{n}{newton\PYZus{}fix\PYZus{}step}\PY{p}{(}\PY{n}{f\PYZus{}eje1}\PY{p}{,}\PY{n}{grad\PYZus{}f\PYZus{}eje1}\PY{p}{,}
    \PY{n}{hess\PYZus{}f\PYZus{}eje1}\PY{p}{,}\PY{n}{x0}\PY{p}{,}\PY{n}{N}\PY{p}{,}\PY{n}{tol}\PY{p}{)}
    \PY{n+nb}{print}\PY{p}{(}\PY{l+s+s1}{\PYZsq{}}\PY{l+s+s1}{El valor de n es: }\PY{l+s+s1}{\PYZsq{}}\PY{p}{,}\PY{n}{n}\PY{p}{)}
    \PY{c+c1}{\PYZsh{} Punto crítico}
    \PY{k}{if} \PY{n}{dic\PYZus{}results}\PY{p}{[}\PY{l+s+s1}{\PYZsq{}}\PY{l+s+s1}{res}\PY{l+s+s1}{\PYZsq{}}\PY{p}{]}\PY{o}{==}\PY{l+m+mi}{1}\PY{p}{:}
        \PY{n+nb}{print}\PY{p}{(}\PY{l+s+s1}{\PYZsq{}}\PY{l+s+s1}{res = }\PY{l+s+s1}{\PYZsq{}}\PY{p}{,}\PY{n}{dic\PYZus{}results}\PY{p}{[}\PY{l+s+s1}{\PYZsq{}}\PY{l+s+s1}{res}\PY{l+s+s1}{\PYZsq{}}\PY{p}{]}\PY{p}{)}
        \PY{n+nb}{print}\PY{p}{(}\PY{l+s+s1}{\PYZsq{}}\PY{l+s+s1}{El método de descenso máximo con paso fijo CONVERGE}\PY{l+s+s1}{\PYZsq{}}\PY{p}{)}
        \PY{n+nb}{print}\PY{p}{(}\PY{l+s+s1}{\PYZsq{}}\PY{l+s+s1}{k = }\PY{l+s+s1}{\PYZsq{}}\PY{p}{,}\PY{n}{dic\PYZus{}results}\PY{p}{[}\PY{l+s+s1}{\PYZsq{}}\PY{l+s+s1}{k}\PY{l+s+s1}{\PYZsq{}}\PY{p}{]}\PY{p}{)}
        \PY{n+nb}{print}\PY{p}{(}\PY{l+s+s1}{\PYZsq{}}\PY{l+s+s1}{fk = }\PY{l+s+s1}{\PYZsq{}}\PY{p}{,}\PY{n}{dic\PYZus{}results}\PY{p}{[}\PY{l+s+s1}{\PYZsq{}}\PY{l+s+s1}{fk}\PY{l+s+s1}{\PYZsq{}}\PY{p}{]}\PY{p}{)}
        \PY{n+nb}{print}\PY{p}{(}\PY{l+s+s1}{\PYZsq{}}\PY{l+s+s1}{||gk|| = }\PY{l+s+s1}{\PYZsq{}}\PY{p}{,}\PY{n}{np}\PY{o}{.}\PY{n}{linalg}\PY{o}{.}\PY{n}{norm}\PY{p}{(}\PY{n}{dic\PYZus{}results}\PY{p}{[}\PY{l+s+s1}{\PYZsq{}}\PY{l+s+s1}{gk}\PY{l+s+s1}{\PYZsq{}}\PY{p}{]}\PY{p}{)}\PY{p}{)}
        \PY{n}{xk}\PY{o}{=}\PY{n}{np}\PY{o}{.}\PY{n}{squeeze}\PY{p}{(}\PY{n}{dic\PYZus{}results}\PY{p}{[}\PY{l+s+s1}{\PYZsq{}}\PY{l+s+s1}{xk}\PY{l+s+s1}{\PYZsq{}}\PY{p}{]}\PY{p}{)}
        \PY{n+nb}{print}\PY{p}{(}\PY{l+s+s1}{\PYZsq{}}\PY{l+s+s1}{xk = }\PY{l+s+s1}{\PYZsq{}}\PY{p}{,}\PY{n}{xk}\PY{p}{)}
    \PY{k}{elif} \PY{n}{dic\PYZus{}results}\PY{p}{[}\PY{l+s+s1}{\PYZsq{}}\PY{l+s+s1}{res}\PY{l+s+s1}{\PYZsq{}}\PY{p}{]}\PY{o}{==}\PY{l+m+mi}{0}\PY{p}{:}
        \PY{n+nb}{print}\PY{p}{(}\PY{l+s+s1}{\PYZsq{}}\PY{l+s+s1}{res = }\PY{l+s+s1}{\PYZsq{}}\PY{p}{,}\PY{n}{dic\PYZus{}results}\PY{p}{[}\PY{l+s+s1}{\PYZsq{}}\PY{l+s+s1}{res}\PY{l+s+s1}{\PYZsq{}}\PY{p}{]}\PY{p}{)}
        \PY{n+nb}{print}\PY{p}{(}\PY{l+s+s1}{\PYZsq{}}\PY{l+s+s1}{El método de descenso máximo con paso exacto NO CONVERGE}\PY{l+s+s1}{\PYZsq{}}\PY{p}{)}
        \PY{n+nb}{print}\PY{p}{(}\PY{l+s+s1}{\PYZsq{}}\PY{l+s+s1}{k = }\PY{l+s+s1}{\PYZsq{}}\PY{p}{,}\PY{n}{dic\PYZus{}results}\PY{p}{[}\PY{l+s+s1}{\PYZsq{}}\PY{l+s+s1}{k}\PY{l+s+s1}{\PYZsq{}}\PY{p}{]}\PY{p}{)}
        \PY{n+nb}{print}\PY{p}{(}\PY{l+s+s1}{\PYZsq{}}\PY{l+s+s1}{fk = }\PY{l+s+s1}{\PYZsq{}}\PY{p}{,}\PY{n}{dic\PYZus{}results}\PY{p}{[}\PY{l+s+s1}{\PYZsq{}}\PY{l+s+s1}{fk}\PY{l+s+s1}{\PYZsq{}}\PY{p}{]}\PY{p}{)}
        \PY{n+nb}{print}\PY{p}{(}\PY{l+s+s1}{\PYZsq{}}\PY{l+s+s1}{||gk|| = }\PY{l+s+s1}{\PYZsq{}}\PY{p}{,}\PY{n}{np}\PY{o}{.}\PY{n}{linalg}\PY{o}{.}\PY{n}{norm}\PY{p}{(}\PY{n}{dic\PYZus{}results}\PY{p}{[}\PY{l+s+s1}{\PYZsq{}}\PY{l+s+s1}{gk}\PY{l+s+s1}{\PYZsq{}}\PY{p}{]}\PY{p}{)}\PY{p}{)}
        \PY{n}{xk}\PY{o}{=}\PY{n}{np}\PY{o}{.}\PY{n}{squeeze}\PY{p}{(}\PY{n}{dic\PYZus{}results}\PY{p}{[}\PY{l+s+s1}{\PYZsq{}}\PY{l+s+s1}{xk}\PY{l+s+s1}{\PYZsq{}}\PY{p}{]}\PY{p}{)}
        \PY{n+nb}{print}\PY{p}{(}\PY{l+s+s1}{\PYZsq{}}\PY{l+s+s1}{xk = }\PY{l+s+s1}{\PYZsq{}}\PY{p}{,}\PY{n}{xk}\PY{p}{)}
    \PY{n+nb}{print}\PY{p}{(}\PY{l+s+s1}{\PYZsq{}}\PY{l+s+se}{\PYZbs{}n}\PY{l+s+s1}{\PYZsq{}}\PY{p}{)}
    
\end{Verbatim}
\end{tcolorbox}

    \begin{Verbatim}[commandchars=\\\{\}]
El valor de n es:  2
res =  1
El método de descenso máximo con paso fijo CONVERGE
k =  6
fk =  1.692176464050546e-13
||gk|| =  1.2340821762126418e-06
xk =  [-1.22644073e-07 -2.45288145e-07]


El valor de n es:  4
res =  1
El método de descenso máximo con paso fijo CONVERGE
k =  9
fk =  1.4140683764719477e-31
||gk|| =  2.1926770122390276e-15
xk =  [-2.35483100e-17 -4.70972553e-17 -7.06457242e-17 -9.41945107e-17]


El valor de n es:  8
res =  1
El método de descenso máximo con paso fijo CONVERGE
k =  10
fk =  1.575705360411473e-13
||gk|| =  5.724916723985908e-06
xk =  [-3.85408023e-09 -7.70816045e-09 -1.15622407e-08 -1.54163209e-08
 -1.92704011e-08 -2.31244814e-08 -2.69785616e-08 -3.08326418e-08]


El valor de n es:  16
res =  1
El método de descenso máximo con paso fijo CONVERGE
k =  12
fk =  8.338488006148036e-18
||gk|| =  1.1183797212584844e-07
xk =  [-3.85533594e-12 -7.71067184e-12 -1.15660078e-11 -1.54213436e-11
 -1.92766790e-11 -2.31320155e-11 -2.69873515e-11 -3.08426873e-11
 -3.46980232e-11 -3.85533579e-11 -4.24086951e-11 -4.62640313e-11
 -5.01193683e-11 -5.39747013e-11 -5.78300391e-11 -6.16853748e-11]


El valor de n es:  32
res =  1
El método de descenso máximo con paso fijo CONVERGE
k =  14
fk =  6.043580290887581e-24
||gk|| =  2.629880849941907e-10
xk =  [-4.29706225e-16 -8.59427107e-16 -1.28913959e-15 -1.71884058e-15
 -2.14855689e-15 -2.57827941e-15 -3.00793880e-15 -3.43773248e-15
 -3.86731281e-15 -4.29710516e-15 -4.72681458e-15 -5.15661188e-15
 -5.58620225e-15 -6.01597800e-15 -6.44567218e-15 -6.87547943e-15
 -7.30511781e-15 -7.73464880e-15 -8.16429965e-15 -8.59412074e-15
 -9.02395995e-15 -9.45361026e-15 -9.88331945e-15 -1.03131330e-14
 -1.07428232e-14 -1.11723172e-14 -1.16020575e-14 -1.20316877e-14
 -1.24620341e-14 -1.28912341e-14 -1.33207195e-14 -1.37509120e-14]


    \end{Verbatim}

    La convergencia del método de Newton es mucho mejor, porque de hecho la
función es casi cuadrática y pues el método de Newton esta basado en una
aproximación de orden 2. El número de iteraciones aumenta como aumenta
la dimensión esto se debe al tiempo se tarde en evaluar las expresiones
pero intuitivamente es más porque la suma \(\sum rx_r\) tiene a aumentar
en unos ejes más que en otros, por lo que tenemos el problema de las
elipses alargadas quizas en el método de descenso máximo, entonces
descenso máximo con el algoritmo de backtracking el tamaño de paso que
se tiene que dar es pequeño lo que aumenta el número de iteraciones aun
más.

    \hypertarget{ejercicio-2.-4-puntos}{%
\subsection{Ejercicio 2. (4 puntos)}\label{ejercicio-2.-4-puntos}}

Ajustar el modelo

\[ g(x; z_1, z_2, z_3, z_4)
= z_1 - z_2\exp(-\exp(z_3+z_4\log(x)))
\]

al conjunto de puntos \(\{(x_i, y_i)\}\) que se muestran a continuación:

    \begin{tcolorbox}[breakable, size=fbox, boxrule=1pt, pad at break*=1mm,colback=cellbackground, colframe=cellborder]
\prompt{In}{incolor}{3}{\boxspacing}
\begin{Verbatim}[commandchars=\\\{\}]
\PY{k+kn}{import} \PY{n+nn}{matplotlib}\PY{n+nn}{.}\PY{n+nn}{pyplot} \PY{k}{as} \PY{n+nn}{plt}
\PY{k+kn}{import} \PY{n+nn}{numpy} \PY{k}{as} \PY{n+nn}{np}

\PY{n}{puntos} \PY{o}{=} \PY{n}{np}\PY{o}{.}\PY{n}{array}\PY{p}{(}\PY{p}{[}\PY{p}{[} \PY{l+m+mf}{9.0}\PY{p}{,}  \PY{l+m+mf}{8.93}\PY{p}{]}\PY{p}{,}
                   \PY{p}{[}\PY{l+m+mf}{14.0}\PY{p}{,} \PY{l+m+mf}{10.80}\PY{p}{]}\PY{p}{,}
                   \PY{p}{[}\PY{l+m+mf}{21.0}\PY{p}{,} \PY{l+m+mf}{18.59}\PY{p}{]}\PY{p}{,}
                   \PY{p}{[}\PY{l+m+mf}{28.0}\PY{p}{,} \PY{l+m+mf}{22.33}\PY{p}{]}\PY{p}{,}
                   \PY{p}{[}\PY{l+m+mf}{42.0}\PY{p}{,} \PY{l+m+mf}{39.35}\PY{p}{]}\PY{p}{,}
                   \PY{p}{[}\PY{l+m+mf}{57.0}\PY{p}{,} \PY{l+m+mf}{56.11}\PY{p}{]}\PY{p}{,}
                   \PY{p}{[}\PY{l+m+mf}{63.0}\PY{p}{,} \PY{l+m+mf}{61.73}\PY{p}{]}\PY{p}{,}
                   \PY{p}{[}\PY{l+m+mf}{70.0}\PY{p}{,} \PY{l+m+mf}{64.62}\PY{p}{]}\PY{p}{,}
                   \PY{p}{[}\PY{l+m+mf}{79.0}\PY{p}{,} \PY{l+m+mf}{67.08}\PY{p}{]}\PY{p}{]}\PY{p}{)}

\PY{n}{plt}\PY{o}{.}\PY{n}{plot}\PY{p}{(}\PY{n}{puntos}\PY{p}{[}\PY{p}{:}\PY{p}{,}\PY{l+m+mi}{0}\PY{p}{]}\PY{p}{,} \PY{n}{puntos}\PY{p}{[}\PY{p}{:}\PY{p}{,}\PY{l+m+mi}{1}\PY{p}{]}\PY{p}{,} \PY{l+s+s1}{\PYZsq{}}\PY{l+s+s1}{r*}\PY{l+s+s1}{\PYZsq{}}\PY{p}{)}
\PY{n}{plt}\PY{o}{.}\PY{n}{xlabel}\PY{p}{(}\PY{l+s+s1}{\PYZsq{}}\PY{l+s+s1}{x}\PY{l+s+s1}{\PYZsq{}}\PY{p}{)}
\PY{n}{plt}\PY{o}{.}\PY{n}{ylabel}\PY{p}{(}\PY{l+s+s1}{\PYZsq{}}\PY{l+s+s1}{y}\PY{l+s+s1}{\PYZsq{}}\PY{p}{)}
\end{Verbatim}
\end{tcolorbox}

            \begin{tcolorbox}[breakable, size=fbox, boxrule=.5pt, pad at break*=1mm, opacityfill=0]
\prompt{Out}{outcolor}{3}{\boxspacing}
\begin{Verbatim}[commandchars=\\\{\}]
Text(0, 0.5, 'y')
\end{Verbatim}
\end{tcolorbox}
        
    \begin{center}
    \adjustimage{max size={0.9\linewidth}{0.9\paperheight}}{parcial1_optimizacion_files/parcial1_optimizacion_9_1.png}
    \end{center}
    { \hspace*{\fill} \\}
    
    Considere los residuales

\[ r_i(\mathbf{z}) = r_i(z_1, z_2, z_3, z_4) = g(x_1; z_1, z_2, z_3, z_4) - y_i \]

Calcule los parámetros \(\mathbf{z} = (z_1, z_2, z_3, z_4)\) resolviendo
el problema de mínimos cuadrados

\[ \min_{z} \; f(\mathbf{z}) = \frac{1}{2}\sum_{i=1}^9 r_i^2(\mathbf{z}). \]

\begin{enumerate}
\def\labelenumi{\arabic{enumi}.}
\tightlist
\item
  Aplique el método de Levenberg-Marquart partiendo del punto inicial
  \(\mathbf{z}_0 = (75, 50.0, -5, 1.1)\), una tolerancia
  \(\tau=\sqrt{\epsilon_m}\) y \(\mu_{ref}=0.001\) (ver Tarea 7).
  Reporte el punto \(\mathbf{z}_k\) que devuelve el algoritmo, el valor
  \(f(\mathbf{z}_k)\), el número de iteraciones \(k\) y la variable
  \(res\) que indica si el algoritmo terminó porque se cumplió la
  tolerancia dada.
\item
  Grafique los datos y la curva del modelo usando los valores del punto
  inicial \(\mathbf{z}_0\) y del punto \(\mathbf{z}_k\) que devuelve el
  algoritmo.
\item
  En algunos casos puede ser complicado encontrar un buen punto inicial
  para el algoritmo. Pruebe con diferentes puntos iniciales generados de
  manera aleatoria y conserve la mejor solución encontrada:
\end{enumerate}

\begin{itemize}
\tightlist
\item
  Para \(i=1, 2, ..., 25\), genere el punto inicial
  \(\mathbf{z}_0 = (z_{0,1}, z_{0,2}, z_{0,3}, z_{0,4})\)
  aleatoriamente, de modo que
\end{itemize}

\[ z_{0,1} \in [50, 80], \quad z_{0,2} \in [50, 80], 
\quad z_{0,3} \in [-10, -5], \quad  z_{0,4} \in [1, 2] \] - Para cada
punto inicial, ejecute el método de Levenberg-Marquart. - En cada
iteración, imprima los valores

\[i, \quad \mathbf{z}_{k_i}, \quad  k_i,  \quad  f(\mathbf{z}_{k_i}), \quad res\]

\begin{itemize}
\tightlist
\item
  Defina \(\mathbf{z}_{min}\) como el punto \(\mathbf{z}_{k_i}\) en el
  que se obtuvo el menor valor de \(f(\mathbf{z})\) para
  \(i=1,2,...25\), y defina \(\mathbf{z}_{ini}\) como el punto inicial
  \(\mathbf{z}_0\) con el cual se obtuvo \(\mathbf{z}_{min}\).
\item
  Grafique los datos y la curva del modelo usando los valores del punto
  \(\mathbf{z}_{ini}\) y los valores del mejor punto
  \(\mathbf{z}_{min}\).
\item
  Revisando los valores de \(f(\mathbf{z}_{k_i})\), escriba un
  comentario sobre sobre si es fácil o no alcanzar el mejor valor
  \(f(\mathbf{z}_{min})\) o que se mantengan los valores de los
  parámetros \(z_1, z_2, z_3, z_4\) en los intervalos que se
  consideraron para dar los valores iniciales.
\end{itemize}

\begin{quote}
\textbf{Nota 1:} Debido a las exponenciales que aparecen en el modelo
\(g(x; \mathbf{z})\) puede ocurrir un desbordamiento de los valores o en
la solución del sistema de ecuaciones, etc. Puede agregar el manejo de
excepciones para que no se interrumpa el ciclo que genera a los puntos
\(\mathbf{z}_{k_i}\). Como el objetivo es quedarse con el mejor punto,
no importa si en\\
algunos casos el algoritmo falla o se interrumpe por generarse una
excepción.
\end{quote}

\begin{quote}
\textbf{Nota 2:} Para generar un número aleatorio con distribución
uniforme en el intervalo \([a,b]\) puede usar
\(a + np.random.rand(1)*(b-a)\).
\end{quote}

    \hypertarget{respuesta}{%
\subsubsection{Respuesta}\label{respuesta}}

    La función que ejecuta el algoritmo de Levenberg-Marquardt y grafica el
modelo paramétrico con \(z_0\) y \(z_k\) es
\texttt{proof\_levenberg\_marquardt\_nlls}. A continuación ejecutamos
esta función con \(z_0=(75,50,-5,1.1)\)

    \begin{tcolorbox}[breakable, size=fbox, boxrule=1pt, pad at break*=1mm,colback=cellbackground, colframe=cellborder]
\prompt{In}{incolor}{4}{\boxspacing}
\begin{Verbatim}[commandchars=\\\{\}]
\PY{n}{importlib}\PY{o}{.}\PY{n}{reload}\PY{p}{(}\PY{n}{lib\PYZus{}examen\PYZus{}1}\PY{p}{)}
\PY{k+kn}{from} \PY{n+nn}{lib\PYZus{}examen\PYZus{}1} \PY{k+kn}{import} \PY{o}{*}

\PY{n}{N}\PY{o}{=}\PY{l+m+mi}{1000}
\PY{n}{tol}\PY{o}{=}\PY{n}{np}\PY{o}{.}\PY{n}{finfo}\PY{p}{(}\PY{n+nb}{float}\PY{p}{)}\PY{o}{.}\PY{n}{eps}\PY{o}{*}\PY{o}{*}\PY{p}{(}\PY{l+m+mi}{1}\PY{o}{/}\PY{l+m+mi}{2}\PY{p}{)}
\PY{n}{mu\PYZus{}ref}\PY{o}{=}\PY{l+m+mf}{0.001}

\PY{n}{z0}\PY{o}{=}\PY{n}{np}\PY{o}{.}\PY{n}{array}\PY{p}{(}\PY{p}{[}\PY{l+m+mf}{75.0}\PY{p}{,}\PY{l+m+mf}{50.0}\PY{p}{,}\PY{o}{\PYZhy{}}\PY{l+m+mf}{5.0}\PY{p}{,}\PY{l+m+mf}{1.1}\PY{p}{]}\PY{p}{)}

\PY{n}{proof\PYZus{}levenberg\PYZus{}marquardt\PYZus{}nlls}\PY{p}{(}\PY{n}{R}\PY{p}{,}\PY{n}{J}\PY{p}{,}\PY{n}{z0}\PY{p}{,}\PY{n}{N}\PY{p}{,}\PY{n}{tol}\PY{p}{,}\PY{n}{mu\PYZus{}ref}\PY{p}{,}\PY{n}{puntos}\PY{p}{)}
\end{Verbatim}
\end{tcolorbox}

    \begin{Verbatim}[commandchars=\\\{\}]
El algoritmo de Levenberg-Marquardt CONVERGE
z0 =  [75.  50.  -5.   1.1]
f(z0) =  935.6411376104923
zk =  [ -50.54719015  -89.38496793 -427.67921188 -986.93518848]
f(zk) =  2324.031677777778
|pk| =  4.976683322014578e-10
k =  8
    \end{Verbatim}

    \begin{center}
    \adjustimage{max size={0.9\linewidth}{0.9\paperheight}}{parcial1_optimizacion_files/parcial1_optimizacion_13_1.png}
    \end{center}
    { \hspace*{\fill} \\}
    
    Ahora lo que haremos será buscar una condición inicial apropiada
aleatoriamente

    \begin{tcolorbox}[breakable, size=fbox, boxrule=1pt, pad at break*=1mm,colback=cellbackground, colframe=cellborder]
\prompt{In}{incolor}{24}{\boxspacing}
\begin{Verbatim}[commandchars=\\\{\}]
\PY{n}{importlib}\PY{o}{.}\PY{n}{reload}\PY{p}{(}\PY{n}{lib\PYZus{}examen\PYZus{}1}\PY{p}{)}
\PY{k+kn}{from} \PY{n+nn}{lib\PYZus{}examen\PYZus{}1} \PY{k+kn}{import} \PY{o}{*}
\PY{n}{z\PYZus{}ini}\PY{o}{=}\PY{n}{np}\PY{o}{.}\PY{n}{empty}\PY{p}{(}\PY{p}{(}\PY{l+m+mi}{25}\PY{p}{,}\PY{l+m+mi}{4}\PY{p}{)}\PY{p}{)}
\PY{k}{for} \PY{n}{i} \PY{o+ow}{in} \PY{n+nb}{range}\PY{p}{(}\PY{l+m+mi}{25}\PY{p}{)}\PY{p}{:}
    \PY{n}{z\PYZus{}ini}\PY{p}{[}\PY{n}{i}\PY{p}{,}\PY{l+m+mi}{0}\PY{p}{]}\PY{o}{=}\PY{l+m+mf}{50.0} \PY{o}{+} \PY{n}{np}\PY{o}{.}\PY{n}{random}\PY{o}{.}\PY{n}{rand}\PY{p}{(}\PY{l+m+mi}{1}\PY{p}{)}\PY{o}{*}\PY{p}{(}\PY{l+m+mf}{30.0}\PY{p}{)}
    \PY{n}{z\PYZus{}ini}\PY{p}{[}\PY{n}{i}\PY{p}{,}\PY{l+m+mi}{1}\PY{p}{]}\PY{o}{=}\PY{l+m+mf}{50.0} \PY{o}{+} \PY{n}{np}\PY{o}{.}\PY{n}{random}\PY{o}{.}\PY{n}{rand}\PY{p}{(}\PY{l+m+mi}{1}\PY{p}{)}\PY{o}{*}\PY{p}{(}\PY{l+m+mf}{30.0}\PY{p}{)}
    \PY{n}{z\PYZus{}ini}\PY{p}{[}\PY{n}{i}\PY{p}{,}\PY{l+m+mi}{2}\PY{p}{]}\PY{o}{=}\PY{o}{\PYZhy{}}\PY{l+m+mf}{10.0} \PY{o}{+} \PY{n}{np}\PY{o}{.}\PY{n}{random}\PY{o}{.}\PY{n}{rand}\PY{p}{(}\PY{l+m+mi}{1}\PY{p}{)}\PY{o}{*}\PY{p}{(}\PY{l+m+mf}{5.0}\PY{p}{)}
    \PY{n}{z\PYZus{}ini}\PY{p}{[}\PY{n}{i}\PY{p}{,}\PY{l+m+mi}{3}\PY{p}{]}\PY{o}{=}\PY{l+m+mf}{1.0} \PY{o}{+} \PY{n}{np}\PY{o}{.}\PY{n}{random}\PY{o}{.}\PY{n}{rand}\PY{p}{(}\PY{l+m+mi}{1}\PY{p}{)}\PY{o}{*}\PY{p}{(}\PY{l+m+mf}{1.0}\PY{p}{)}

\PY{n}{array\PYZus{}zki}\PY{o}{=}\PY{p}{[}\PY{p}{]}
\PY{n}{array\PYZus{}fki}\PY{o}{=}\PY{n}{np}\PY{o}{.}\PY{n}{empty}\PY{p}{(}\PY{l+m+mi}{25}\PY{p}{)}

\PY{k}{for} \PY{n}{i}\PY{p}{,}\PY{n}{z0} \PY{o+ow}{in} \PY{n+nb}{enumerate}\PY{p}{(}\PY{n}{z\PYZus{}ini}\PY{p}{)}\PY{p}{:}
    \PY{n+nb}{print}\PY{p}{(}\PY{l+s+s1}{\PYZsq{}}\PY{l+s+s1}{i = }\PY{l+s+s1}{\PYZsq{}}\PY{p}{,}\PY{n}{i}\PY{o}{+}\PY{l+m+mi}{1}\PY{p}{)}
    \PY{n}{dic\PYZus{}results}\PY{o}{=}\PY{n}{levenberg\PYZus{}marquardt\PYZus{}nlls}\PY{p}{(}\PY{n}{R}\PY{p}{,}\PY{n}{J}\PY{p}{,}\PY{n}{z0}\PY{p}{,}\PY{n}{N}\PY{p}{,}\PY{n}{tol}\PY{p}{,}\PY{n}{mu\PYZus{}ref}\PY{p}{,}\PY{n}{puntos}\PY{p}{)}
    \PY{n}{array\PYZus{}zki}\PY{o}{.}\PY{n}{append}\PY{p}{(}\PY{n}{dic\PYZus{}results}\PY{p}{[}\PY{l+s+s1}{\PYZsq{}}\PY{l+s+s1}{zk}\PY{l+s+s1}{\PYZsq{}}\PY{p}{]}\PY{p}{)}
    \PY{n}{array\PYZus{}fki}\PY{p}{[}\PY{n}{i}\PY{p}{]}\PY{o}{=}\PY{n}{dic\PYZus{}results}\PY{p}{[}\PY{l+s+s1}{\PYZsq{}}\PY{l+s+s1}{fk}\PY{l+s+s1}{\PYZsq{}}\PY{p}{]}
    \PY{n+nb}{print}\PY{p}{(}\PY{l+s+s1}{\PYZsq{}}\PY{l+s+s1}{zki = }\PY{l+s+s1}{\PYZsq{}}\PY{p}{,}\PY{n}{array\PYZus{}zki}\PY{p}{[}\PY{o}{\PYZhy{}}\PY{l+m+mi}{1}\PY{p}{]}\PY{p}{)}
    \PY{n+nb}{print}\PY{p}{(}\PY{l+s+s1}{\PYZsq{}}\PY{l+s+s1}{ki = }\PY{l+s+s1}{\PYZsq{}}\PY{p}{,}\PY{n}{dic\PYZus{}results}\PY{p}{[}\PY{l+s+s1}{\PYZsq{}}\PY{l+s+s1}{k}\PY{l+s+s1}{\PYZsq{}}\PY{p}{]}\PY{p}{)}
    \PY{n+nb}{print}\PY{p}{(}\PY{l+s+s1}{\PYZsq{}}\PY{l+s+s1}{f(zki) = }\PY{l+s+s1}{\PYZsq{}}\PY{p}{,}\PY{n}{array\PYZus{}fki}\PY{p}{[}\PY{n}{i}\PY{p}{]}\PY{p}{)}
    \PY{n+nb}{print}\PY{p}{(}\PY{l+s+s1}{\PYZsq{}}\PY{l+s+s1}{res = }\PY{l+s+s1}{\PYZsq{}}\PY{p}{,}\PY{n}{dic\PYZus{}results}\PY{p}{[}\PY{l+s+s1}{\PYZsq{}}\PY{l+s+s1}{res}\PY{l+s+s1}{\PYZsq{}}\PY{p}{]}\PY{p}{)}
    \PY{n+nb}{print}\PY{p}{(}\PY{l+s+s1}{\PYZsq{}}\PY{l+s+se}{\PYZbs{}n}\PY{l+s+s1}{\PYZsq{}}\PY{p}{)}
\end{Verbatim}
\end{tcolorbox}

    \begin{Verbatim}[commandchars=\\\{\}]
i =  1
zki =  [  71.94111955   33.10334177  -63.03041231 -200.04611535]
ki =  6
f(zki) =  2324.031677777778
res =  1


i =  2
zki =  [   9.865       -37.25071429 1809.2036215  -680.69563835]
ki =  11
f(zki) =  1244.7750107142858
res =  1


i =  3
zki =  [ 38.83777778  65.58176544 118.32812078  -4.21941313]
ki =  4
f(zki) =  2324.031677777778
res =  1


i =  4
zki =  [38.83777778 67.98585674 15.31416772 -2.76621659]
ki =  19
f(zki) =  2324.0316777993457
res =  1


i =  5
zki =  [nan nan nan nan]
ki =  1000
f(zki) =  nan
res =  0


i =  6
zki =  [ 73.34652313  34.50874535  56.91884315 -66.29523699]
ki =  6
f(zki) =  2324.031677777778
res =  1


i =  7
zki =  [  71.66973927   32.83196149  260.63670462 -211.35476575]
ki =  5
f(zki) =  2324.031677777778
res =  1


i =  8
zki =  [ 7.41880244e+01  3.53502466e+01 -4.97345184e-01 -5.75970769e+02]
ki =  6
f(zki) =  2324.0316777777775
res =  1


i =  9
zki =  [ 38.83777778  69.3175003  100.81607275  12.81430231]
ki =  4
f(zki) =  2324.0316777777775
res =  1


i =  10
zki =  [38.83777778 66.57341045 25.66284646 -3.28984079]
ki =  4
f(zki) =  2324.0316777777775
res =  1


i =  11
zki =  [  47.11571429   38.18571429 -188.55991947   70.31616736]
ki =  16
f(zki) =  1243.9007857142856
res =  1


i =  12
zki =  [    57.53311869     18.69534092  -6571.64732747 -16087.20250785]
ki =  17
f(zki) =  2324.031677777778
res =  1


i =  13
zki =  [69.95519317 61.6814639  -9.20893083  2.37781753]
ki =  18
f(zki) =  4.1879417795020135
res =  1


i =  14
zki =  [69.95517822 61.68144258 -9.20893559  2.37781887]
ki =  14
f(zki) =  4.187941779469073
res =  1


i =  15
zki =  [38.83777778 58.15169674 -3.84675145  3.6016864 ]
ki =  6
f(zki) =  2324.031677777778
res =  1


i =  16
zki =  [  38.83777778   64.06524696  613.37847684 -133.89070539]
ki =  5
f(zki) =  2324.031677777778
res =  1


i =  17
zki =  [38.83777778 63.74312558 -3.56967269  6.91024702]
ki =  9
f(zki) =  2324.0316777777775
res =  1


i =  18
zki =  [69.95518683 61.68145486 -9.20893285  2.3778181 ]
ki =  17
f(zki) =  4.187941779479113
res =  1


i =  19
zki =  [38.83777777 54.4012894  35.19426525 79.89900385]
ki =  5
f(zki) =  2324.031677777778
res =  1


i =  20
zki =  [69.95517934 61.68144418 -9.20893523  2.37781877]
ki =  13
f(zki) =  4.187941779469004
res =  1


i =  21
zki =  [69.9551818  61.68144768 -9.20893445  2.37781855]
ki =  16
f(zki) =  4.187941779470297
res =  1


i =  22
zki =  [ 38.83777778  54.49677033 156.48464742 -22.71276163]
ki =  4
f(zki) =  2324.031677777778
res =  1


i =  23
zki =  [   64.47666667    38.45833333 -3246.83231872   792.17100575]
ki =  18
f(zki) =  844.9882749999999
res =  1


i =  24
zki =  [  42.57625      33.64626336 -107.62394722   42.27377725]
ki =  9
f(zki) =  1820.8893937500002
res =  1


i =  25
zki =  [69.95518371 61.68145041 -9.20893384  2.37781838]
ki =  16
f(zki) =  4.187941779472666
res =  1


    \end{Verbatim}

    La salida no se puede mostrar completa, sin embargo obtenemos el
\(\mathbf{z}_{min}\) y hacemos la grafica correspondiente

    \begin{tcolorbox}[breakable, size=fbox, boxrule=1pt, pad at break*=1mm,colback=cellbackground, colframe=cellborder]
\prompt{In}{incolor}{25}{\boxspacing}
\begin{Verbatim}[commandchars=\\\{\}]
\PY{n}{index\PYZus{}min}\PY{o}{=}\PY{n}{np}\PY{o}{.}\PY{n}{nanargmin}\PY{p}{(}\PY{n}{array\PYZus{}fki}\PY{p}{)}
\PY{n}{z\PYZus{}min}\PY{o}{=}\PY{n}{array\PYZus{}zki}\PY{p}{[}\PY{n}{index\PYZus{}min}\PY{p}{]}
\PY{n}{z\PYZus{}0min}\PY{o}{=}\PY{n}{z\PYZus{}ini}\PY{p}{[}\PY{n}{i}\PY{p}{,}\PY{p}{:}\PY{p}{]}

\PY{l+s+sd}{\PYZsq{}\PYZsq{}\PYZsq{}}
\PY{l+s+sd}{Gráfica del ajuste obtenido}
\PY{l+s+sd}{\PYZsq{}\PYZsq{}\PYZsq{}}
\PY{n}{x}\PY{p}{,}\PY{n}{y}\PY{o}{=}\PY{n}{puntos}\PY{o}{.}\PY{n}{T}
\PY{n}{x\PYZus{}linspace}\PY{o}{=}\PY{n}{np}\PY{o}{.}\PY{n}{linspace}\PY{p}{(}\PY{n}{np}\PY{o}{.}\PY{n}{min}\PY{p}{(}\PY{n}{x}\PY{p}{)}\PY{p}{,}\PY{n}{np}\PY{o}{.}\PY{n}{max}\PY{p}{(}\PY{n}{x}\PY{p}{)}\PY{p}{,}\PY{n}{num}\PY{o}{=}\PY{l+m+mi}{1000}\PY{p}{)}
\PY{n}{y\PYZus{}model\PYZus{}k}\PY{o}{=}\PY{n}{z\PYZus{}min}\PY{p}{[}\PY{l+m+mi}{0}\PY{p}{]}\PY{o}{\PYZhy{}}\PY{n}{z\PYZus{}min}\PY{p}{[}\PY{l+m+mi}{1}\PY{p}{]}\PY{o}{*}\PY{n}{np}\PY{o}{.}\PY{n}{exp}\PY{p}{(}\PY{o}{\PYZhy{}}\PY{n}{np}\PY{o}{.}\PY{n}{exp}\PY{p}{(}\PY{n}{z\PYZus{}min}\PY{p}{[}\PY{l+m+mi}{2}\PY{p}{]}\PY{o}{+}\PY{n}{z\PYZus{}min}\PY{p}{[}\PY{l+m+mi}{3}\PY{p}{]}\PY{o}{*}\PY{n}{np}\PY{o}{.}\PY{n}{log}\PY{p}{(}\PY{n}{x\PYZus{}linspace}\PY{p}{)}\PY{p}{)}\PY{p}{)}
\PY{n}{y\PYZus{}model\PYZus{}ini}\PY{o}{=}\PY{n}{z\PYZus{}0min}\PY{p}{[}\PY{l+m+mi}{0}\PY{p}{]}\PY{o}{\PYZhy{}}\PY{n}{z\PYZus{}0min}\PY{p}{[}\PY{l+m+mi}{1}\PY{p}{]}\PY{o}{*}\PY{n}{np}\PY{o}{.}\PY{n}{exp}\PY{p}{(}\PY{o}{\PYZhy{}}\PY{n}{np}\PY{o}{.}\PY{n}{exp}\PY{p}{(}\PY{n}{z\PYZus{}0min}\PY{p}{[}\PY{l+m+mi}{2}\PY{p}{]}\PY{o}{+}\PY{n}{z\PYZus{}0min}\PY{p}{[}\PY{l+m+mi}{3}\PY{p}{]}\PY{o}{*}\PY{n}{np}\PY{o}{.}\PY{n}{log}\PY{p}{(}\PY{n}{x\PYZus{}linspace}\PY{p}{)}\PY{p}{)}\PY{p}{)}

\PY{n}{plt}\PY{o}{.}\PY{n}{style}\PY{o}{.}\PY{n}{use}\PY{p}{(}\PY{l+s+s1}{\PYZsq{}}\PY{l+s+s1}{seaborn}\PY{l+s+s1}{\PYZsq{}}\PY{p}{)}
\PY{n}{plt}\PY{o}{.}\PY{n}{plot}\PY{p}{(}\PY{n}{x}\PY{p}{,}\PY{n}{y}\PY{p}{,}\PY{l+s+s1}{\PYZsq{}}\PY{l+s+s1}{ro}\PY{l+s+s1}{\PYZsq{}}\PY{p}{,}\PY{n}{label}\PY{o}{=}\PY{l+s+s1}{\PYZsq{}}\PY{l+s+s1}{muestra}\PY{l+s+s1}{\PYZsq{}}\PY{p}{)}
\PY{n}{plt}\PY{o}{.}\PY{n}{plot}\PY{p}{(}\PY{n}{x\PYZus{}linspace}\PY{p}{,}\PY{n}{y\PYZus{}model\PYZus{}k}\PY{p}{,}\PY{l+s+s1}{\PYZsq{}}\PY{l+s+s1}{b\PYZhy{}}\PY{l+s+s1}{\PYZsq{}}\PY{p}{,}\PY{n}{label}\PY{o}{=}\PY{l+s+s1}{\PYZsq{}}\PY{l+s+s1}{punto final}\PY{l+s+s1}{\PYZsq{}}\PY{p}{)}
\PY{n}{plt}\PY{o}{.}\PY{n}{plot}\PY{p}{(}\PY{n}{x\PYZus{}linspace}\PY{p}{,}\PY{n}{y\PYZus{}model\PYZus{}ini}\PY{p}{,}\PY{l+s+s1}{\PYZsq{}}\PY{l+s+s1}{g\PYZhy{}}\PY{l+s+s1}{\PYZsq{}}\PY{p}{,}\PY{n}{label}\PY{o}{=}\PY{l+s+s1}{\PYZsq{}}\PY{l+s+s1}{punto inicial}\PY{l+s+s1}{\PYZsq{}}\PY{p}{)}
\PY{n}{plt}\PY{o}{.}\PY{n}{title}\PY{p}{(}\PY{l+s+s1}{\PYZsq{}}\PY{l+s+s1}{Levenberg\PYZhy{}Marquardt}\PY{l+s+se}{\PYZbs{}n}\PY{l+s+s1}{\PYZsq{}}\PY{o}{+}\PY{l+s+sa}{r}\PY{l+s+s1}{\PYZsq{}}\PY{l+s+s1}{\PYZdl{}f(}\PY{l+s+s1}{\PYZbs{}}\PY{l+s+s1}{mathbf}\PY{l+s+si}{\PYZob{}z\PYZcb{}}\PY{l+s+s1}{\PYZus{}k)\PYZdl{}=}\PY{l+s+si}{\PYZpc{}.4f}\PY{l+s+s1}{\PYZsq{}}\PY{o}{\PYZpc{}}\PY{p}{(}\PY{n}{array\PYZus{}fki}\PY{p}{[}\PY{n}{index\PYZus{}min}\PY{p}{]}\PY{p}{)}\PY{p}{)}
\PY{n}{plt}\PY{o}{.}\PY{n}{legend}\PY{p}{(}\PY{p}{)}
\PY{n}{plt}\PY{o}{.}\PY{n}{show}\PY{p}{(}\PY{p}{)}
\end{Verbatim}
\end{tcolorbox}

    \begin{center}
    \adjustimage{max size={0.9\linewidth}{0.9\paperheight}}{parcial1_optimizacion_files/parcial1_optimizacion_17_0.png}
    \end{center}
    { \hspace*{\fill} \\}
    
    En el que tenemos un ajuste aceptable. Revisando los valores de la
función objetivo finales en cada iteración tenemos

    \begin{tcolorbox}[breakable, size=fbox, boxrule=1pt, pad at break*=1mm,colback=cellbackground, colframe=cellborder]
\prompt{In}{incolor}{26}{\boxspacing}
\begin{Verbatim}[commandchars=\\\{\}]
\PY{n}{array\PYZus{}fki}
\end{Verbatim}
\end{tcolorbox}

            \begin{tcolorbox}[breakable, size=fbox, boxrule=.5pt, pad at break*=1mm, opacityfill=0]
\prompt{Out}{outcolor}{26}{\boxspacing}
\begin{Verbatim}[commandchars=\\\{\}]
array([2324.03167778, 1244.77501071, 2324.03167778, 2324.0316778 ,
                 nan, 2324.03167778, 2324.03167778, 2324.03167778,
       2324.03167778, 2324.03167778, 1243.90078571, 2324.03167778,
          4.18794178,    4.18794178, 2324.03167778, 2324.03167778,
       2324.03167778,    4.18794178, 2324.03167778,    4.18794178,
          4.18794178, 2324.03167778,  844.988275  , 1820.88939375,
          4.18794178])
\end{Verbatim}
\end{tcolorbox}
        
    Vemos que realmente no es tan fácil obtener un valor cercano al óptimo
de la función objetivo \(f\), pues en su mayoría los valores son
bastante más grandes. Podriamos ahora generar intervalos alrededor de
las condiciones iniciales en las que se obtuvo un buen valor de la
función objetivo.

    \hypertarget{ejercicio-3-3-puntos}{%
\subsection{Ejercicio 3 (3 puntos)}\label{ejercicio-3-3-puntos}}

Queremos encontrar una solución del problema de optimización con
restricciones

\[ \min_{x_1,x_2} f(x_1,x_2) = 2(x_1^2 + x_2^2 - 1) - x_1 \]

sujeto a que el punto \((x_1,x_2)\) esté sobre el círculo unitario, es
decir,

\[ x_1^2 + x_2^2 - 1 = 0.\]

Una manera de hallar una aproximación es convertir este problema de
optimización sin restricciones. Un enfoque es el de penalización
cuadrática en el cual se construye la función

\[ Q(x_1, x_2; \mu) = f(x_1,x_2) + \frac{\mu}{2} g(x_1, x_2),\]

donde \(g(x_1, x_2) = (x_1^2 + x_2^2 - 1)^2\) es el cuadrado de la
restricción que queremos que se cumpla.

La idea es:

\begin{enumerate}
\def\labelenumi{\arabic{enumi}.}
\tightlist
\item
  Dar un punto inicial \(\mathbf{x}_0 =(10, 8)\) y un valor del
  parámetro \(\mu_0= 1\).
\item
  Para \(r=0, 1, 2, 3\):
\end{enumerate}

\begin{itemize}
\tightlist
\item
  Calcular el mínimo de la función \(Q(\mathbf{x}; \mu_r)\) usando el
  punto inicial \(\mathbf{x}_0\)
\item
  Imprimir \(\mu_r\), el óptimo \(\mathbf{x}_k\) que devuelve el
  algoritmo, los valores \(Q(\mathbf{x}_k; \mu_k)\), \(f(\mathbf{x}_k)\)
  y \(g(\mathbf{x}_k)\)
\item
  Hacer \(\mu_{r+1} = 10\mu_r\) y \(\mathbf{x}_0=\mathbf{x}_k\).
\end{itemize}

El último punto \(\mathbf{x}_k\) generado debería ser una aproximación
de la solución del problema original porque
\(Q(\mathbf{x}; \mu_r)= f(\mathbf{x})\) si \(\mathbf{x}\) cumple la
restricción. Si no, \(Q(\mathbf{x}; \mu_r)> f(\mathbf{x})\) y el valor
de \(Q\) aumenta conforme aumenta el valor de \(\mu_r\). De este modo,
al ir incrementando gradualmente el valor \(\mu\), se penaliza cada vez
más a los puntos que no satisfacen la restricción y eso hace que en cada
iteración \(r\) se obtenga un punto que está más cerca de cumplir la
restricción.

Note que en cada iteración \(r\), se usa como punto inicial para el
algoritmo de optimización el punto que se obtuvo en la iteración
anterior. Esto ayuda a que los óptimos que se obtienen en cada iteración
gradualmente se vayan cumpliendo la restricción.

\hypertarget{respuesta}{%
\subsubsection{Respuesta}\label{respuesta}}

    A continuación haremos presentamos el esquema anterior optimizando la
función \(Q(\mathbf{x};\mu)\) dado \(\mu\) vía Newton Raphson

    \begin{tcolorbox}[breakable, size=fbox, boxrule=1pt, pad at break*=1mm,colback=cellbackground, colframe=cellborder]
\prompt{In}{incolor}{35}{\boxspacing}
\begin{Verbatim}[commandchars=\\\{\}]
\PY{n}{importlib}\PY{o}{.}\PY{n}{reload}\PY{p}{(}\PY{n}{lib\PYZus{}examen\PYZus{}1}\PY{p}{)}
\PY{k+kn}{from} \PY{n+nn}{lib\PYZus{}examen\PYZus{}1} \PY{k+kn}{import} \PY{o}{*}

\PY{n}{N}\PY{o}{=}\PY{l+m+mi}{1000}
\PY{n}{tol}\PY{o}{=}\PY{n}{np}\PY{o}{.}\PY{n}{finfo}\PY{p}{(}\PY{n+nb}{float}\PY{p}{)}\PY{o}{.}\PY{n}{eps}\PY{o}{*}\PY{o}{*}\PY{p}{(}\PY{l+m+mi}{1}\PY{o}{/}\PY{l+m+mi}{3}\PY{p}{)}
\PY{n}{mu}\PY{o}{=}\PY{l+m+mf}{1.0}
\PY{n}{x0}\PY{o}{=}\PY{n}{np}\PY{o}{.}\PY{n}{array}\PY{p}{(}\PY{p}{[}\PY{l+m+mf}{10.0}\PY{p}{,}\PY{l+m+mf}{8.0}\PY{p}{]}\PY{p}{)}
\PY{n}{a0}\PY{p}{,}\PY{n}{c}\PY{o}{=}\PY{l+m+mf}{2.0}\PY{p}{,}\PY{l+m+mf}{1e\PYZhy{}4} \PY{c+c1}{\PYZsh{} Tamaño inicial y factor de proporción fijo para las pruebas}
\PY{n}{rho}\PY{o}{=}\PY{l+m+mf}{0.5}

\PY{c+c1}{\PYZsh{} Optimizacion sin reestriccion iterativa}
\PY{k}{for} \PY{n}{r} \PY{o+ow}{in} \PY{n+nb}{range}\PY{p}{(}\PY{l+m+mi}{4}\PY{p}{)}\PY{p}{:}
    \PY{n}{dic\PYZus{}results}\PY{o}{=}\PY{n}{grad\PYZus{}max\PYZus{}eje3}\PY{p}{(}\PY{n}{f\PYZus{}opt\PYZus{}eje3}\PY{p}{,}\PY{n}{grad\PYZus{}f\PYZus{}opt\PYZus{}eje3}\PY{p}{,}\PY{n}{x0}\PY{p}{,}\PY{n}{N}\PY{p}{,}\PY{n}{tol}\PY{p}{,}\PY{n}{a0}\PY{p}{,}\PY{n}{rho}\PY{p}{,}\PY{n}{c}\PY{p}{,}\PY{n}{mu}\PY{p}{)}
    \PY{k}{if} \PY{n}{dic\PYZus{}results}\PY{p}{[}\PY{l+s+s1}{\PYZsq{}}\PY{l+s+s1}{res}\PY{l+s+s1}{\PYZsq{}}\PY{p}{]}\PY{o}{==}\PY{l+m+mi}{1}\PY{p}{:}
        \PY{n+nb}{print}\PY{p}{(}\PY{l+s+s1}{\PYZsq{}}\PY{l+s+s1}{El algoritmo CONVERGE}\PY{l+s+s1}{\PYZsq{}}\PY{p}{)}
        \PY{n+nb}{print}\PY{p}{(}\PY{l+s+s1}{\PYZsq{}}\PY{l+s+s1}{mu\PYZus{}r = }\PY{l+s+s1}{\PYZsq{}}\PY{p}{,}\PY{n}{mu}\PY{p}{)}
        \PY{n}{xk}\PY{o}{=}\PY{n}{dic\PYZus{}results}\PY{p}{[}\PY{l+s+s1}{\PYZsq{}}\PY{l+s+s1}{xk}\PY{l+s+s1}{\PYZsq{}}\PY{p}{]}
        \PY{n+nb}{print}\PY{p}{(}\PY{l+s+s1}{\PYZsq{}}\PY{l+s+s1}{xk = }\PY{l+s+s1}{\PYZsq{}}\PY{p}{,}\PY{n}{xk}\PY{p}{)}
        \PY{n+nb}{print}\PY{p}{(}\PY{l+s+s1}{\PYZsq{}}\PY{l+s+s1}{Qk = }\PY{l+s+s1}{\PYZsq{}}\PY{p}{,}\PY{n}{dic\PYZus{}results}\PY{p}{[}\PY{l+s+s1}{\PYZsq{}}\PY{l+s+s1}{fk}\PY{l+s+s1}{\PYZsq{}}\PY{p}{]}\PY{p}{)}
        \PY{n+nb}{print}\PY{p}{(}\PY{l+s+s1}{\PYZsq{}}\PY{l+s+s1}{fk = }\PY{l+s+s1}{\PYZsq{}}\PY{p}{,}\PY{n}{f\PYZus{}eje3}\PY{p}{(}\PY{n}{xk}\PY{p}{)}\PY{p}{)}
        \PY{n+nb}{print}\PY{p}{(}\PY{l+s+s1}{\PYZsq{}}\PY{l+s+s1}{gk = }\PY{l+s+s1}{\PYZsq{}}\PY{p}{,}\PY{n}{g\PYZus{}eje3}\PY{p}{(}\PY{n}{xk}\PY{p}{)}\PY{p}{)}
        \PY{n+nb}{print}\PY{p}{(}\PY{l+s+s1}{\PYZsq{}}\PY{l+s+se}{\PYZbs{}n}\PY{l+s+s1}{\PYZsq{}}\PY{p}{)}
        \PY{n}{x0}\PY{o}{=}\PY{n}{xk}
        \PY{n}{mu}\PY{o}{=}\PY{l+m+mf}{10.0}\PY{o}{*}\PY{n}{mu}
    \PY{k}{else}\PY{p}{:} 
        \PY{n+nb}{print}\PY{p}{(}\PY{l+s+s1}{\PYZsq{}}\PY{l+s+s1}{El algoritmo NO CONVERGE}\PY{l+s+s1}{\PYZsq{}}\PY{p}{)}
        \PY{k}{break}
\end{Verbatim}
\end{tcolorbox}

    \begin{Verbatim}[commandchars=\\\{\}]
El algoritmo CONVERGE
mu\_r =  1.0
xk =  [4.23852450e-01 1.10217923e-15]
Qk =  -1.7280643278065917
fk =  -2.0645506514225485
gk =  0.6729726472319135


El algoritmo CONVERGE
mu\_r =  10.0
xk =  [9.24176668e-01 1.57548970e-15]
Qk =  -1.1095412580699155
fk =  -1.2159716398975506
gk =  0.021286076365527005


El algoritmo CONVERGE
mu\_r =  100.0
xk =  [9.92490733e-01 1.53318783e-15]
Qk =  -1.0112217338041674
fk =  -1.0224150235300673
gk =  0.00022386579451799847


El algoritmo CONVERGE
mu\_r =  1000.0
xk =  [9.99249908e-01 1.33114696e-15]
Qk =  -1.0011247186093135
fk =  -1.0022491512681861
gk =  2.248865317745207e-06


    \end{Verbatim}

    En efecto vemos como el valor de \(Q\) y \(f\) en cada iteración se
aproxima pues al penalizar forzamos a \(g\) a hacerse casi \(0\). Al
final obtenemos el punto optimo en la circunferencia unitaria \((1,0)\).


    % Add a bibliography block to the postdoc
    
    
    
\end{document}
