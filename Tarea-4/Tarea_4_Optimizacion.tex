\documentclass[11pt]{article}

    \usepackage[breakable]{tcolorbox}
    \usepackage{parskip} % Stop auto-indenting (to mimic markdown behaviour)
    
    \usepackage{iftex}
    \ifPDFTeX
    	\usepackage[T1]{fontenc}
    	\usepackage{mathpazo}
    \else
    	\usepackage{fontspec}
    \fi

    % Basic figure setup, for now with no caption control since it's done
    % automatically by Pandoc (which extracts ![](path) syntax from Markdown).
    \usepackage{graphicx}
    % Maintain compatibility with old templates. Remove in nbconvert 6.0
    \let\Oldincludegraphics\includegraphics
    % Ensure that by default, figures have no caption (until we provide a
    % proper Figure object with a Caption API and a way to capture that
    % in the conversion process - todo).
    \usepackage{caption}
    \DeclareCaptionFormat{nocaption}{}
    \captionsetup{format=nocaption,aboveskip=0pt,belowskip=0pt}

    \usepackage{float}
    \floatplacement{figure}{H} % forces figures to be placed at the correct location
    \usepackage{xcolor} % Allow colors to be defined
    \usepackage{enumerate} % Needed for markdown enumerations to work
    \usepackage{geometry} % Used to adjust the document margins
    \usepackage{amsmath} % Equations
    \usepackage{amssymb} % Equations
    \usepackage{textcomp} % defines textquotesingle
    % Hack from http://tex.stackexchange.com/a/47451/13684:
    \AtBeginDocument{%
        \def\PYZsq{\textquotesingle}% Upright quotes in Pygmentized code
    }
    \usepackage{upquote} % Upright quotes for verbatim code
    \usepackage{eurosym} % defines \euro
    \usepackage[mathletters]{ucs} % Extended unicode (utf-8) support
    \usepackage{fancyvrb} % verbatim replacement that allows latex
    \usepackage{grffile} % extends the file name processing of package graphics 
                         % to support a larger range
    \makeatletter % fix for old versions of grffile with XeLaTeX
    \@ifpackagelater{grffile}{2019/11/01}
    {
      % Do nothing on new versions
    }
    {
      \def\Gread@@xetex#1{%
        \IfFileExists{"\Gin@base".bb}%
        {\Gread@eps{\Gin@base.bb}}%
        {\Gread@@xetex@aux#1}%
      }
    }
    \makeatother
    \usepackage[Export]{adjustbox} % Used to constrain images to a maximum size
    \adjustboxset{max size={0.9\linewidth}{0.9\paperheight}}

    % The hyperref package gives us a pdf with properly built
    % internal navigation ('pdf bookmarks' for the table of contents,
    % internal cross-reference links, web links for URLs, etc.)
    \usepackage{hyperref}
    % The default LaTeX title has an obnoxious amount of whitespace. By default,
    % titling removes some of it. It also provides customization options.
    \usepackage{titling}
    \usepackage{longtable} % longtable support required by pandoc >1.10
    \usepackage{booktabs}  % table support for pandoc > 1.12.2
    \usepackage[inline]{enumitem} % IRkernel/repr support (it uses the enumerate* environment)
    \usepackage[normalem]{ulem} % ulem is needed to support strikethroughs (\sout)
                                % normalem makes italics be italics, not underlines
    \usepackage{mathrsfs}
    

    
    % Colors for the hyperref package
    \definecolor{urlcolor}{rgb}{0,.145,.698}
    \definecolor{linkcolor}{rgb}{.71,0.21,0.01}
    \definecolor{citecolor}{rgb}{.12,.54,.11}

    % ANSI colors
    \definecolor{ansi-black}{HTML}{3E424D}
    \definecolor{ansi-black-intense}{HTML}{282C36}
    \definecolor{ansi-red}{HTML}{E75C58}
    \definecolor{ansi-red-intense}{HTML}{B22B31}
    \definecolor{ansi-green}{HTML}{00A250}
    \definecolor{ansi-green-intense}{HTML}{007427}
    \definecolor{ansi-yellow}{HTML}{DDB62B}
    \definecolor{ansi-yellow-intense}{HTML}{B27D12}
    \definecolor{ansi-blue}{HTML}{208FFB}
    \definecolor{ansi-blue-intense}{HTML}{0065CA}
    \definecolor{ansi-magenta}{HTML}{D160C4}
    \definecolor{ansi-magenta-intense}{HTML}{A03196}
    \definecolor{ansi-cyan}{HTML}{60C6C8}
    \definecolor{ansi-cyan-intense}{HTML}{258F8F}
    \definecolor{ansi-white}{HTML}{C5C1B4}
    \definecolor{ansi-white-intense}{HTML}{A1A6B2}
    \definecolor{ansi-default-inverse-fg}{HTML}{FFFFFF}
    \definecolor{ansi-default-inverse-bg}{HTML}{000000}

    % common color for the border for error outputs.
    \definecolor{outerrorbackground}{HTML}{FFDFDF}

    % commands and environments needed by pandoc snippets
    % extracted from the output of `pandoc -s`
    \providecommand{\tightlist}{%
      \setlength{\itemsep}{0pt}\setlength{\parskip}{0pt}}
    \DefineVerbatimEnvironment{Highlighting}{Verbatim}{commandchars=\\\{\}}
    % Add ',fontsize=\small' for more characters per line
    \newenvironment{Shaded}{}{}
    \newcommand{\KeywordTok}[1]{\textcolor[rgb]{0.00,0.44,0.13}{\textbf{{#1}}}}
    \newcommand{\DataTypeTok}[1]{\textcolor[rgb]{0.56,0.13,0.00}{{#1}}}
    \newcommand{\DecValTok}[1]{\textcolor[rgb]{0.25,0.63,0.44}{{#1}}}
    \newcommand{\BaseNTok}[1]{\textcolor[rgb]{0.25,0.63,0.44}{{#1}}}
    \newcommand{\FloatTok}[1]{\textcolor[rgb]{0.25,0.63,0.44}{{#1}}}
    \newcommand{\CharTok}[1]{\textcolor[rgb]{0.25,0.44,0.63}{{#1}}}
    \newcommand{\StringTok}[1]{\textcolor[rgb]{0.25,0.44,0.63}{{#1}}}
    \newcommand{\CommentTok}[1]{\textcolor[rgb]{0.38,0.63,0.69}{\textit{{#1}}}}
    \newcommand{\OtherTok}[1]{\textcolor[rgb]{0.00,0.44,0.13}{{#1}}}
    \newcommand{\AlertTok}[1]{\textcolor[rgb]{1.00,0.00,0.00}{\textbf{{#1}}}}
    \newcommand{\FunctionTok}[1]{\textcolor[rgb]{0.02,0.16,0.49}{{#1}}}
    \newcommand{\RegionMarkerTok}[1]{{#1}}
    \newcommand{\ErrorTok}[1]{\textcolor[rgb]{1.00,0.00,0.00}{\textbf{{#1}}}}
    \newcommand{\NormalTok}[1]{{#1}}
    
    % Additional commands for more recent versions of Pandoc
    \newcommand{\ConstantTok}[1]{\textcolor[rgb]{0.53,0.00,0.00}{{#1}}}
    \newcommand{\SpecialCharTok}[1]{\textcolor[rgb]{0.25,0.44,0.63}{{#1}}}
    \newcommand{\VerbatimStringTok}[1]{\textcolor[rgb]{0.25,0.44,0.63}{{#1}}}
    \newcommand{\SpecialStringTok}[1]{\textcolor[rgb]{0.73,0.40,0.53}{{#1}}}
    \newcommand{\ImportTok}[1]{{#1}}
    \newcommand{\DocumentationTok}[1]{\textcolor[rgb]{0.73,0.13,0.13}{\textit{{#1}}}}
    \newcommand{\AnnotationTok}[1]{\textcolor[rgb]{0.38,0.63,0.69}{\textbf{\textit{{#1}}}}}
    \newcommand{\CommentVarTok}[1]{\textcolor[rgb]{0.38,0.63,0.69}{\textbf{\textit{{#1}}}}}
    \newcommand{\VariableTok}[1]{\textcolor[rgb]{0.10,0.09,0.49}{{#1}}}
    \newcommand{\ControlFlowTok}[1]{\textcolor[rgb]{0.00,0.44,0.13}{\textbf{{#1}}}}
    \newcommand{\OperatorTok}[1]{\textcolor[rgb]{0.40,0.40,0.40}{{#1}}}
    \newcommand{\BuiltInTok}[1]{{#1}}
    \newcommand{\ExtensionTok}[1]{{#1}}
    \newcommand{\PreprocessorTok}[1]{\textcolor[rgb]{0.74,0.48,0.00}{{#1}}}
    \newcommand{\AttributeTok}[1]{\textcolor[rgb]{0.49,0.56,0.16}{{#1}}}
    \newcommand{\InformationTok}[1]{\textcolor[rgb]{0.38,0.63,0.69}{\textbf{\textit{{#1}}}}}
    \newcommand{\WarningTok}[1]{\textcolor[rgb]{0.38,0.63,0.69}{\textbf{\textit{{#1}}}}}
    
    
    % Define a nice break command that doesn't care if a line doesn't already
    % exist.
    \def\br{\hspace*{\fill} \\* }
    % Math Jax compatibility definitions
    \def\gt{>}
    \def\lt{<}
    \let\Oldtex\TeX
    \let\Oldlatex\LaTeX
    \renewcommand{\TeX}{\textrm{\Oldtex}}
    \renewcommand{\LaTeX}{\textrm{\Oldlatex}}
    % Document parameters
    % Document title
    \title{Tarea\_4\_Optimizacion}
    
    
    
    
    
% Pygments definitions
\makeatletter
\def\PY@reset{\let\PY@it=\relax \let\PY@bf=\relax%
    \let\PY@ul=\relax \let\PY@tc=\relax%
    \let\PY@bc=\relax \let\PY@ff=\relax}
\def\PY@tok#1{\csname PY@tok@#1\endcsname}
\def\PY@toks#1+{\ifx\relax#1\empty\else%
    \PY@tok{#1}\expandafter\PY@toks\fi}
\def\PY@do#1{\PY@bc{\PY@tc{\PY@ul{%
    \PY@it{\PY@bf{\PY@ff{#1}}}}}}}
\def\PY#1#2{\PY@reset\PY@toks#1+\relax+\PY@do{#2}}

\@namedef{PY@tok@w}{\def\PY@tc##1{\textcolor[rgb]{0.73,0.73,0.73}{##1}}}
\@namedef{PY@tok@c}{\let\PY@it=\textit\def\PY@tc##1{\textcolor[rgb]{0.25,0.50,0.50}{##1}}}
\@namedef{PY@tok@cp}{\def\PY@tc##1{\textcolor[rgb]{0.74,0.48,0.00}{##1}}}
\@namedef{PY@tok@k}{\let\PY@bf=\textbf\def\PY@tc##1{\textcolor[rgb]{0.00,0.50,0.00}{##1}}}
\@namedef{PY@tok@kp}{\def\PY@tc##1{\textcolor[rgb]{0.00,0.50,0.00}{##1}}}
\@namedef{PY@tok@kt}{\def\PY@tc##1{\textcolor[rgb]{0.69,0.00,0.25}{##1}}}
\@namedef{PY@tok@o}{\def\PY@tc##1{\textcolor[rgb]{0.40,0.40,0.40}{##1}}}
\@namedef{PY@tok@ow}{\let\PY@bf=\textbf\def\PY@tc##1{\textcolor[rgb]{0.67,0.13,1.00}{##1}}}
\@namedef{PY@tok@nb}{\def\PY@tc##1{\textcolor[rgb]{0.00,0.50,0.00}{##1}}}
\@namedef{PY@tok@nf}{\def\PY@tc##1{\textcolor[rgb]{0.00,0.00,1.00}{##1}}}
\@namedef{PY@tok@nc}{\let\PY@bf=\textbf\def\PY@tc##1{\textcolor[rgb]{0.00,0.00,1.00}{##1}}}
\@namedef{PY@tok@nn}{\let\PY@bf=\textbf\def\PY@tc##1{\textcolor[rgb]{0.00,0.00,1.00}{##1}}}
\@namedef{PY@tok@ne}{\let\PY@bf=\textbf\def\PY@tc##1{\textcolor[rgb]{0.82,0.25,0.23}{##1}}}
\@namedef{PY@tok@nv}{\def\PY@tc##1{\textcolor[rgb]{0.10,0.09,0.49}{##1}}}
\@namedef{PY@tok@no}{\def\PY@tc##1{\textcolor[rgb]{0.53,0.00,0.00}{##1}}}
\@namedef{PY@tok@nl}{\def\PY@tc##1{\textcolor[rgb]{0.63,0.63,0.00}{##1}}}
\@namedef{PY@tok@ni}{\let\PY@bf=\textbf\def\PY@tc##1{\textcolor[rgb]{0.60,0.60,0.60}{##1}}}
\@namedef{PY@tok@na}{\def\PY@tc##1{\textcolor[rgb]{0.49,0.56,0.16}{##1}}}
\@namedef{PY@tok@nt}{\let\PY@bf=\textbf\def\PY@tc##1{\textcolor[rgb]{0.00,0.50,0.00}{##1}}}
\@namedef{PY@tok@nd}{\def\PY@tc##1{\textcolor[rgb]{0.67,0.13,1.00}{##1}}}
\@namedef{PY@tok@s}{\def\PY@tc##1{\textcolor[rgb]{0.73,0.13,0.13}{##1}}}
\@namedef{PY@tok@sd}{\let\PY@it=\textit\def\PY@tc##1{\textcolor[rgb]{0.73,0.13,0.13}{##1}}}
\@namedef{PY@tok@si}{\let\PY@bf=\textbf\def\PY@tc##1{\textcolor[rgb]{0.73,0.40,0.53}{##1}}}
\@namedef{PY@tok@se}{\let\PY@bf=\textbf\def\PY@tc##1{\textcolor[rgb]{0.73,0.40,0.13}{##1}}}
\@namedef{PY@tok@sr}{\def\PY@tc##1{\textcolor[rgb]{0.73,0.40,0.53}{##1}}}
\@namedef{PY@tok@ss}{\def\PY@tc##1{\textcolor[rgb]{0.10,0.09,0.49}{##1}}}
\@namedef{PY@tok@sx}{\def\PY@tc##1{\textcolor[rgb]{0.00,0.50,0.00}{##1}}}
\@namedef{PY@tok@m}{\def\PY@tc##1{\textcolor[rgb]{0.40,0.40,0.40}{##1}}}
\@namedef{PY@tok@gh}{\let\PY@bf=\textbf\def\PY@tc##1{\textcolor[rgb]{0.00,0.00,0.50}{##1}}}
\@namedef{PY@tok@gu}{\let\PY@bf=\textbf\def\PY@tc##1{\textcolor[rgb]{0.50,0.00,0.50}{##1}}}
\@namedef{PY@tok@gd}{\def\PY@tc##1{\textcolor[rgb]{0.63,0.00,0.00}{##1}}}
\@namedef{PY@tok@gi}{\def\PY@tc##1{\textcolor[rgb]{0.00,0.63,0.00}{##1}}}
\@namedef{PY@tok@gr}{\def\PY@tc##1{\textcolor[rgb]{1.00,0.00,0.00}{##1}}}
\@namedef{PY@tok@ge}{\let\PY@it=\textit}
\@namedef{PY@tok@gs}{\let\PY@bf=\textbf}
\@namedef{PY@tok@gp}{\let\PY@bf=\textbf\def\PY@tc##1{\textcolor[rgb]{0.00,0.00,0.50}{##1}}}
\@namedef{PY@tok@go}{\def\PY@tc##1{\textcolor[rgb]{0.53,0.53,0.53}{##1}}}
\@namedef{PY@tok@gt}{\def\PY@tc##1{\textcolor[rgb]{0.00,0.27,0.87}{##1}}}
\@namedef{PY@tok@err}{\def\PY@bc##1{{\setlength{\fboxsep}{\string -\fboxrule}\fcolorbox[rgb]{1.00,0.00,0.00}{1,1,1}{\strut ##1}}}}
\@namedef{PY@tok@kc}{\let\PY@bf=\textbf\def\PY@tc##1{\textcolor[rgb]{0.00,0.50,0.00}{##1}}}
\@namedef{PY@tok@kd}{\let\PY@bf=\textbf\def\PY@tc##1{\textcolor[rgb]{0.00,0.50,0.00}{##1}}}
\@namedef{PY@tok@kn}{\let\PY@bf=\textbf\def\PY@tc##1{\textcolor[rgb]{0.00,0.50,0.00}{##1}}}
\@namedef{PY@tok@kr}{\let\PY@bf=\textbf\def\PY@tc##1{\textcolor[rgb]{0.00,0.50,0.00}{##1}}}
\@namedef{PY@tok@bp}{\def\PY@tc##1{\textcolor[rgb]{0.00,0.50,0.00}{##1}}}
\@namedef{PY@tok@fm}{\def\PY@tc##1{\textcolor[rgb]{0.00,0.00,1.00}{##1}}}
\@namedef{PY@tok@vc}{\def\PY@tc##1{\textcolor[rgb]{0.10,0.09,0.49}{##1}}}
\@namedef{PY@tok@vg}{\def\PY@tc##1{\textcolor[rgb]{0.10,0.09,0.49}{##1}}}
\@namedef{PY@tok@vi}{\def\PY@tc##1{\textcolor[rgb]{0.10,0.09,0.49}{##1}}}
\@namedef{PY@tok@vm}{\def\PY@tc##1{\textcolor[rgb]{0.10,0.09,0.49}{##1}}}
\@namedef{PY@tok@sa}{\def\PY@tc##1{\textcolor[rgb]{0.73,0.13,0.13}{##1}}}
\@namedef{PY@tok@sb}{\def\PY@tc##1{\textcolor[rgb]{0.73,0.13,0.13}{##1}}}
\@namedef{PY@tok@sc}{\def\PY@tc##1{\textcolor[rgb]{0.73,0.13,0.13}{##1}}}
\@namedef{PY@tok@dl}{\def\PY@tc##1{\textcolor[rgb]{0.73,0.13,0.13}{##1}}}
\@namedef{PY@tok@s2}{\def\PY@tc##1{\textcolor[rgb]{0.73,0.13,0.13}{##1}}}
\@namedef{PY@tok@sh}{\def\PY@tc##1{\textcolor[rgb]{0.73,0.13,0.13}{##1}}}
\@namedef{PY@tok@s1}{\def\PY@tc##1{\textcolor[rgb]{0.73,0.13,0.13}{##1}}}
\@namedef{PY@tok@mb}{\def\PY@tc##1{\textcolor[rgb]{0.40,0.40,0.40}{##1}}}
\@namedef{PY@tok@mf}{\def\PY@tc##1{\textcolor[rgb]{0.40,0.40,0.40}{##1}}}
\@namedef{PY@tok@mh}{\def\PY@tc##1{\textcolor[rgb]{0.40,0.40,0.40}{##1}}}
\@namedef{PY@tok@mi}{\def\PY@tc##1{\textcolor[rgb]{0.40,0.40,0.40}{##1}}}
\@namedef{PY@tok@il}{\def\PY@tc##1{\textcolor[rgb]{0.40,0.40,0.40}{##1}}}
\@namedef{PY@tok@mo}{\def\PY@tc##1{\textcolor[rgb]{0.40,0.40,0.40}{##1}}}
\@namedef{PY@tok@ch}{\let\PY@it=\textit\def\PY@tc##1{\textcolor[rgb]{0.25,0.50,0.50}{##1}}}
\@namedef{PY@tok@cm}{\let\PY@it=\textit\def\PY@tc##1{\textcolor[rgb]{0.25,0.50,0.50}{##1}}}
\@namedef{PY@tok@cpf}{\let\PY@it=\textit\def\PY@tc##1{\textcolor[rgb]{0.25,0.50,0.50}{##1}}}
\@namedef{PY@tok@c1}{\let\PY@it=\textit\def\PY@tc##1{\textcolor[rgb]{0.25,0.50,0.50}{##1}}}
\@namedef{PY@tok@cs}{\let\PY@it=\textit\def\PY@tc##1{\textcolor[rgb]{0.25,0.50,0.50}{##1}}}

\def\PYZbs{\char`\\}
\def\PYZus{\char`\_}
\def\PYZob{\char`\{}
\def\PYZcb{\char`\}}
\def\PYZca{\char`\^}
\def\PYZam{\char`\&}
\def\PYZlt{\char`\<}
\def\PYZgt{\char`\>}
\def\PYZsh{\char`\#}
\def\PYZpc{\char`\%}
\def\PYZdl{\char`\$}
\def\PYZhy{\char`\-}
\def\PYZsq{\char`\'}
\def\PYZdq{\char`\"}
\def\PYZti{\char`\~}
% for compatibility with earlier versions
\def\PYZat{@}
\def\PYZlb{[}
\def\PYZrb{]}
\makeatother


    % For linebreaks inside Verbatim environment from package fancyvrb. 
    \makeatletter
        \newbox\Wrappedcontinuationbox 
        \newbox\Wrappedvisiblespacebox 
        \newcommand*\Wrappedvisiblespace {\textcolor{red}{\textvisiblespace}} 
        \newcommand*\Wrappedcontinuationsymbol {\textcolor{red}{\llap{\tiny$\m@th\hookrightarrow$}}} 
        \newcommand*\Wrappedcontinuationindent {3ex } 
        \newcommand*\Wrappedafterbreak {\kern\Wrappedcontinuationindent\copy\Wrappedcontinuationbox} 
        % Take advantage of the already applied Pygments mark-up to insert 
        % potential linebreaks for TeX processing. 
        %        {, <, #, %, $, ' and ": go to next line. 
        %        _, }, ^, &, >, - and ~: stay at end of broken line. 
        % Use of \textquotesingle for straight quote. 
        \newcommand*\Wrappedbreaksatspecials {% 
            \def\PYGZus{\discretionary{\char`\_}{\Wrappedafterbreak}{\char`\_}}% 
            \def\PYGZob{\discretionary{}{\Wrappedafterbreak\char`\{}{\char`\{}}% 
            \def\PYGZcb{\discretionary{\char`\}}{\Wrappedafterbreak}{\char`\}}}% 
            \def\PYGZca{\discretionary{\char`\^}{\Wrappedafterbreak}{\char`\^}}% 
            \def\PYGZam{\discretionary{\char`\&}{\Wrappedafterbreak}{\char`\&}}% 
            \def\PYGZlt{\discretionary{}{\Wrappedafterbreak\char`\<}{\char`\<}}% 
            \def\PYGZgt{\discretionary{\char`\>}{\Wrappedafterbreak}{\char`\>}}% 
            \def\PYGZsh{\discretionary{}{\Wrappedafterbreak\char`\#}{\char`\#}}% 
            \def\PYGZpc{\discretionary{}{\Wrappedafterbreak\char`\%}{\char`\%}}% 
            \def\PYGZdl{\discretionary{}{\Wrappedafterbreak\char`\$}{\char`\$}}% 
            \def\PYGZhy{\discretionary{\char`\-}{\Wrappedafterbreak}{\char`\-}}% 
            \def\PYGZsq{\discretionary{}{\Wrappedafterbreak\textquotesingle}{\textquotesingle}}% 
            \def\PYGZdq{\discretionary{}{\Wrappedafterbreak\char`\"}{\char`\"}}% 
            \def\PYGZti{\discretionary{\char`\~}{\Wrappedafterbreak}{\char`\~}}% 
        } 
        % Some characters . , ; ? ! / are not pygmentized. 
        % This macro makes them "active" and they will insert potential linebreaks 
        \newcommand*\Wrappedbreaksatpunct {% 
            \lccode`\~`\.\lowercase{\def~}{\discretionary{\hbox{\char`\.}}{\Wrappedafterbreak}{\hbox{\char`\.}}}% 
            \lccode`\~`\,\lowercase{\def~}{\discretionary{\hbox{\char`\,}}{\Wrappedafterbreak}{\hbox{\char`\,}}}% 
            \lccode`\~`\;\lowercase{\def~}{\discretionary{\hbox{\char`\;}}{\Wrappedafterbreak}{\hbox{\char`\;}}}% 
            \lccode`\~`\:\lowercase{\def~}{\discretionary{\hbox{\char`\:}}{\Wrappedafterbreak}{\hbox{\char`\:}}}% 
            \lccode`\~`\?\lowercase{\def~}{\discretionary{\hbox{\char`\?}}{\Wrappedafterbreak}{\hbox{\char`\?}}}% 
            \lccode`\~`\!\lowercase{\def~}{\discretionary{\hbox{\char`\!}}{\Wrappedafterbreak}{\hbox{\char`\!}}}% 
            \lccode`\~`\/\lowercase{\def~}{\discretionary{\hbox{\char`\/}}{\Wrappedafterbreak}{\hbox{\char`\/}}}% 
            \catcode`\.\active
            \catcode`\,\active 
            \catcode`\;\active
            \catcode`\:\active
            \catcode`\?\active
            \catcode`\!\active
            \catcode`\/\active 
            \lccode`\~`\~ 	
        }
    \makeatother

    \let\OriginalVerbatim=\Verbatim
    \makeatletter
    \renewcommand{\Verbatim}[1][1]{%
        %\parskip\z@skip
        \sbox\Wrappedcontinuationbox {\Wrappedcontinuationsymbol}%
        \sbox\Wrappedvisiblespacebox {\FV@SetupFont\Wrappedvisiblespace}%
        \def\FancyVerbFormatLine ##1{\hsize\linewidth
            \vtop{\raggedright\hyphenpenalty\z@\exhyphenpenalty\z@
                \doublehyphendemerits\z@\finalhyphendemerits\z@
                \strut ##1\strut}%
        }%
        % If the linebreak is at a space, the latter will be displayed as visible
        % space at end of first line, and a continuation symbol starts next line.
        % Stretch/shrink are however usually zero for typewriter font.
        \def\FV@Space {%
            \nobreak\hskip\z@ plus\fontdimen3\font minus\fontdimen4\font
            \discretionary{\copy\Wrappedvisiblespacebox}{\Wrappedafterbreak}
            {\kern\fontdimen2\font}%
        }%
        
        % Allow breaks at special characters using \PYG... macros.
        \Wrappedbreaksatspecials
        % Breaks at punctuation characters . , ; ? ! and / need catcode=\active 	
        \OriginalVerbatim[#1,codes*=\Wrappedbreaksatpunct]%
    }
    \makeatother

    % Exact colors from NB
    \definecolor{incolor}{HTML}{303F9F}
    \definecolor{outcolor}{HTML}{D84315}
    \definecolor{cellborder}{HTML}{CFCFCF}
    \definecolor{cellbackground}{HTML}{F7F7F7}
    
    % prompt
    \makeatletter
    \newcommand{\boxspacing}{\kern\kvtcb@left@rule\kern\kvtcb@boxsep}
    \makeatother
    \newcommand{\prompt}[4]{
        {\ttfamily\llap{{\color{#2}[#3]:\hspace{3pt}#4}}\vspace{-\baselineskip}}
    }
    

    
    % Prevent overflowing lines due to hard-to-break entities
    \sloppy 
    % Setup hyperref package
    \hypersetup{
      breaklinks=true,  % so long urls are correctly broken across lines
      colorlinks=true,
      urlcolor=urlcolor,
      linkcolor=linkcolor,
      citecolor=citecolor,
      }
    % Slightly bigger margins than the latex defaults
    
    \geometry{verbose,tmargin=1in,bmargin=1in,lmargin=1in,rmargin=1in}
    
    

\begin{document}
    \title{Tarea 4 Optimización}
    \author{Roberto Vásquez Martínez \\ Profesor: Joaquín Peña Acevedo}
    \date{06/Marzo/2022}
    \maketitle   
    
    

    
    \hypertarget{ejercicio-1-5-puntos}{%
\section{Ejercicio 1 (5 puntos)}\label{ejercicio-1-5-puntos}}

Programar el método de descenso máximo con tamaño de paso exacto para
minimizar funciones cuadráticas:

\[ f(x) = \frac{1}{2} x^\top \mathbf{A} x - b^\top x, \]

donde \(\mathbf{A} \in \mathbb{R}^{n \times n}\) y
\(x \in \mathbb{R}^n\).

Dado el vector \(b\), la matriz \(\mathbf{A}\), un punto inicial
\(x_0\), un número máximo de iteraciones \(N\), la tolerancia
\(\tau>0\). Fijar \(k=0\) y repetir los siguientes pasos:

\begin{enumerate}
\def\labelenumi{\arabic{enumi}.}
\item
  Calcular el gradiente en el punto \(x_k\),

  \[g_k = \nabla f(x_k) = \mathbf{A} x_k - b\]
\item
  Si \(\|g_k\| < \tau\), entonces \(x_k\) es (casi) un punto
  estacionario. Hacer \(res=1\) y terminar el ciclo.
\item
  Elegir la dirección de descenso como \(p_k = - g_k\).
\item
  Calcular el tamaño de paso \(\alpha_{k}\) que minimiza el valor de la
  función \[\phi_k(\alpha) =  f(x_k + \alpha p_k)\] es decir, calcular
  \[ \alpha_{k} = -\frac{ g_{k}^{\top} p_{k}}{ p_{k}^{\top}\mathbf{A}p_{k}} \]
\item
  Calcular el siguiente punto de la secuencia como
  \[x_{k+1} = x_k + \alpha_k p_k \]
\item
  Si \(k+1\geq N\), hacer \(res=0\) y terminar.
\item
  Si no, hacer \(k = k+1\) y volver el paso 1.
\item
  Devolver el punto \(x_k\),
  \(f_k= \frac{1}{2} x_k^\top \mathbf{A} x_k - b^\top x_k\), \(g_k\),
  \(k\) y \(res\).
\end{enumerate}

\begin{center}\rule{0.5\linewidth}{0.5pt}\end{center}

\begin{enumerate}
\def\labelenumi{\arabic{enumi}.}
\tightlist
\item
  Escriba una función que implementa el algoritmo anterior usando
  arreglos de Numpy.\\
\item
  Escriba una función para probar el funcionamiento del método de
  descenso máximo. Esta función debe recibir como parámetros el nombre
  de un archivo \texttt{.npy} que contiene las entradas de una matriz
  cuadrada \(\mathbf{A}\), el vector \(b\), un punto inicial \(x_0\), el
  número máximo de iteraciones \(N\) y la tolerancia \(\tau\).
\end{enumerate}

\begin{itemize}
\tightlist
\item
  Esta función debe crear la matriz \(\mathbf{A}\) y el vector \(b\)
  leyendo los archivos de datos.
\item
  Obtener el número de filas \(r\) de la matriz e imprimir este valor.
\item
  Compruebe que la matriz es simétrica y definida positiva calculando e
  imprimiendo el valor \(\|\mathbf{A} - \mathbf{A}^\top\|\) y su
  eigenvalor más pequeño (use la función \texttt{numpy.linalg.eig()}).
\item
  Ejecutar la función del Inciso 1.
\item
  Dependiendo del valor de la variable \(res\), imprima un mensaje que
  diga que el algoritmo convergió (\(res=1\)) o no (\(res=0\)).
\item
  Imprimir \(k\), \(f_k\), la norma de \(g_k\), y los primeros 3 y
  últimos 3 elementos del arreglo \(x_k\).
\item
  Calcule directamente el minimizador resolviendo la ecuación
  \(Ax_* = b\) e imprima el valor del error \(\|x_k - x_* \|\).
\end{itemize}

\begin{enumerate}
\def\labelenumi{\arabic{enumi}.}
\setcounter{enumi}{2}
\tightlist
\item
  Pruebe la función del Inciso 2 usando \(N=1000\), la tolerancia
  \(\tau = \epsilon_m^{1/3}\), donde \(\epsilon_m\) es el épsilon de la
  máquina, y los arreglos que se incluyen en el archivo
  datosTarea04.zip, de la siguiente manera:
\end{enumerate}
\begin{center}
\begin{tabular}{|c|c|c|}
\hline
Matriz & Vector & Punto inicial \\ \hline\hline
A1.npy & b1.npy & $x_0 = (0, -5)$ \\ \hline 
A1.npy & b1.npy & $x_0 = (7045, 7095)$ \\ \hline
A2.npy & b2.npy & $x_0 = (0,0,\dots,0) \in \mathbb{R}^{500}$ \\ \hline
A2.npy & b2.npy &
$x_0 = (10000,10000,\dots,10000) \in \mathbb{R}^{500}$\\ \hline
\end{tabular}
\end{center}
\hypertarget{soluciuxf3n}{%
\subsection{Solución:}\label{soluciuxf3n}}

    En la siguiente celda declaramos el PATH de los archivos \texttt{.npy}
que usaremos para probar el método de descenso máximo con paso exacto
para la funciones cuadráticas de la forma
\[ f(x) = \frac{1}{2} x^\top \mathbf{A} x - b^\top x, \] donde
\(A\in\mathbb{R}^{n\times n}\) es una matriz simétrica positiva
definida.

Importamos el módulo \texttt{lib\_t4} donde se encuentran las funciones
\texttt{grad\_max\_quadratic} y \texttt{proof\_grad\_max\_quadratic},
que implementan los numerales 1 y 2, respectvamente.

    \begin{tcolorbox}[breakable, size=fbox, boxrule=1pt, pad at break*=1mm,colback=cellbackground, colframe=cellborder]
\prompt{In}{incolor}{1}{\boxspacing}
\begin{Verbatim}[commandchars=\\\{\}]
\PY{c+c1}{\PYZsh{} Path de los archivos con los datos}
\PY{n}{data\PYZus{}A1}\PY{o}{=}\PY{l+s+s1}{\PYZsq{}}\PY{l+s+s1}{/datosTarea04/A1.npy}\PY{l+s+s1}{\PYZsq{}}
\PY{n}{data\PYZus{}A2}\PY{o}{=}\PY{l+s+s1}{\PYZsq{}}\PY{l+s+s1}{/datosTarea04/A2.npy}\PY{l+s+s1}{\PYZsq{}}
\PY{n}{data\PYZus{}b1}\PY{o}{=}\PY{l+s+s1}{\PYZsq{}}\PY{l+s+s1}{/datosTarea04/b1.npy}\PY{l+s+s1}{\PYZsq{}}
\PY{n}{data\PYZus{}b2}\PY{o}{=}\PY{l+s+s1}{\PYZsq{}}\PY{l+s+s1}{/datosTarea04/b2.npy}\PY{l+s+s1}{\PYZsq{}}
\end{Verbatim}
\end{tcolorbox}

    \hypertarget{funciuxf3n-1}{%
\subsubsection{Función 1}\label{funciuxf3n-1}}

A continuación realizamos las pruebas correspondiente a dos diferentes
condiciones iniciales \(x_0\) tomando los datos de la matriz en
\texttt{A1.npy} y el vector \texttt{b1.npy} para definir la función
cuadrática. En cada una de las pruebas se utiliza como tolerancia
\(\tau=\epsilon_m^{1/3}\) y un número máximo de iteraciones \(N=1000\),
donde \(\epsilon_m\) es el épsilon de la máquina.

Para \(x_0=(0,-5)\) tenemos el siguiente resultado

    \begin{tcolorbox}[breakable, size=fbox, boxrule=1pt, pad at break*=1mm,colback=cellbackground, colframe=cellborder]
\prompt{In}{incolor}{2}{\boxspacing}
\begin{Verbatim}[commandchars=\\\{\}]
\PY{k+kn}{import} \PY{n+nn}{lib\PYZus{}t4}
\PY{k+kn}{import} \PY{n+nn}{importlib}
\PY{n}{importlib}\PY{o}{.}\PY{n}{reload}\PY{p}{(}\PY{n}{lib\PYZus{}t4}\PY{p}{)}
\PY{k+kn}{from} \PY{n+nn}{lib\PYZus{}t4} \PY{k+kn}{import} \PY{o}{*}
\PY{n}{x0}\PY{o}{=}\PY{n}{np}\PY{o}{.}\PY{n}{array}\PY{p}{(}\PY{p}{[}\PY{l+m+mf}{0.0}\PY{p}{,}\PY{o}{\PYZhy{}}\PY{l+m+mf}{5.0}\PY{p}{]}\PY{p}{)}
\PY{n}{tol}\PY{o}{=}\PY{n}{np}\PY{o}{.}\PY{n}{finfo}\PY{p}{(}\PY{n+nb}{float}\PY{p}{)}\PY{o}{.}\PY{n}{eps}\PY{o}{*}\PY{o}{*}\PY{p}{(}\PY{l+m+mi}{1}\PY{o}{/}\PY{l+m+mi}{3}\PY{p}{)}
\PY{n}{N}\PY{o}{=}\PY{l+m+mi}{1000}
\PY{n}{proof\PYZus{}grad\PYZus{}max\PYZus{}quadratic}\PY{p}{(}\PY{n}{data\PYZus{}A1}\PY{p}{,}\PY{n}{data\PYZus{}b1}\PY{p}{,}\PY{n}{x0}\PY{p}{,}\PY{n}{N}\PY{p}{,}\PY{n}{tol}\PY{p}{)}
\end{Verbatim}
\end{tcolorbox}

    \begin{Verbatim}[commandchars=\\\{\}]
El número de filas de la matriz A es 2
||A-A.T||=  0.0
Es una matriz simétrica
El valor propio más pequeño es:  0.10000000000000003
La matriz A es positiva definida
El método de descenso máximo con paso exacto CONVERGE
k =  69
fk =  -62.749999999933024
||gk|| =  4.767340294300877e-06
xk =  [-24.49997287  25.49997743]
||xk-x*|| =  3.528998540703229e-05
    \end{Verbatim}

    Para la condición inicial \(x_0=(7045,7095)\) obtenemos lo siguiente

    \begin{tcolorbox}[breakable, size=fbox, boxrule=1pt, pad at break*=1mm,colback=cellbackground, colframe=cellborder]
\prompt{In}{incolor}{3}{\boxspacing}
\begin{Verbatim}[commandchars=\\\{\}]
\PY{n}{x0}\PY{o}{=}\PY{n}{np}\PY{o}{.}\PY{n}{array}\PY{p}{(}\PY{p}{[}\PY{l+m+mf}{7045.0}\PY{p}{,}\PY{l+m+mf}{7095.0}\PY{p}{]}\PY{p}{)}
\PY{n}{proof\PYZus{}grad\PYZus{}max\PYZus{}quadratic}\PY{p}{(}\PY{n}{data\PYZus{}A1}\PY{p}{,}\PY{n}{data\PYZus{}b1}\PY{p}{,}\PY{n}{x0}\PY{p}{,}\PY{n}{N}\PY{p}{,}\PY{n}{tol}\PY{p}{)}
\end{Verbatim}
\end{tcolorbox}

    \begin{Verbatim}[commandchars=\\\{\}]
El número de filas de la matriz A es 2
||A-A.T||=  0.0
Es una matriz simétrica
El valor propio más pequeño es:  0.10000000000000003
La matriz A es positiva definida
El método de descenso máximo con paso exacto CONVERGE
k =  1
fk =  -62.75
||gk|| =  6.425429159208664e-13
xk =  [-24.5  25.5]
||xk-x*|| =  9.131096203816022e-13
    \end{Verbatim}

    Aquí llama la atención como en le primer paso se llegó al mínimo de la
función, incluso con mayor exactitud que con la primer condición inicial
como se puede constantar en el valor de \(\lVert x_k-x_\ast\rVert\) para
cada caso.

    \hypertarget{funciuxf3n-2}{%
\subsubsection{Función 2}\label{funciuxf3n-2}}

Ahora usaremos los datos de la matriz en \texttt{A2.npy} y el vector
\texttt{b2.npy} para definir la función cuadrática.

Ejecutando el algoritmo de descenso máximo con paso exacto para dicha
función con la codición inicial \(x_0=(0,0,\dots,0)\in\mathbb{R}^{500}\)
obtenemos lo siguiente

    \begin{tcolorbox}[breakable, size=fbox, boxrule=1pt, pad at break*=1mm,colback=cellbackground, colframe=cellborder]
\prompt{In}{incolor}{4}{\boxspacing}
\begin{Verbatim}[commandchars=\\\{\}]
\PY{n}{x0}\PY{o}{=}\PY{n}{np}\PY{o}{.}\PY{n}{zeros}\PY{p}{(}\PY{l+m+mi}{500}\PY{p}{)}
\PY{n}{proof\PYZus{}grad\PYZus{}max\PYZus{}quadratic}\PY{p}{(}\PY{n}{data\PYZus{}A2}\PY{p}{,}\PY{n}{data\PYZus{}b2}\PY{p}{,}\PY{n}{x0}\PY{p}{,}\PY{n}{N}\PY{p}{,}\PY{n}{tol}\PY{p}{)}
\end{Verbatim}
\end{tcolorbox}

    \begin{Verbatim}[commandchars=\\\{\}]
El número de filas de la matriz A es 500
||A-A.T||=  1.5063918215255853e-14
Es una matriz simétrica
El valor propio más pequeño es:  0.09999999999999767
La matriz A es positiva definida
El método de descenso máximo con paso exacto CONVERGE
k =  332
fk =  -5239.541412076964
||gk|| =  5.996601797613609e-06
Primeras 3 coordenadas de xk son:  [-11.61784712   5.44982336   0.96506345]
Últimas 3 coordenadas de xk son:  [ -1.8852386  -10.34321936  -4.15697837]
||xk-x*|| =  3.9297511569442596e-05
    \end{Verbatim}

    Con la condición inicial
\(x_0=(10000,10000,\dots,10000)\in\mathbb{R}^{500}\) obtenemos

    \begin{tcolorbox}[breakable, size=fbox, boxrule=1pt, pad at break*=1mm,colback=cellbackground, colframe=cellborder]
\prompt{In}{incolor}{5}{\boxspacing}
\begin{Verbatim}[commandchars=\\\{\}]
\PY{n}{x0}\PY{o}{=}\PY{n}{np}\PY{o}{.}\PY{n}{full}\PY{p}{(}\PY{l+m+mi}{500}\PY{p}{,}\PY{l+m+mf}{10000.0}\PY{p}{)}
\PY{n}{proof\PYZus{}grad\PYZus{}max\PYZus{}quadratic}\PY{p}{(}\PY{n}{data\PYZus{}A2}\PY{p}{,}\PY{n}{data\PYZus{}b2}\PY{p}{,}\PY{n}{x0}\PY{p}{,}\PY{n}{N}\PY{p}{,}\PY{n}{tol}\PY{p}{)}
\end{Verbatim}
\end{tcolorbox}

    \begin{Verbatim}[commandchars=\\\{\}]
El número de filas de la matriz A es 500
||A-A.T||=  1.5063918215255853e-14
Es una matriz simétrica
El valor propio más pequeño es:  0.09999999999999767
La matriz A es positiva definida
El método de descenso máximo con paso exacto CONVERGE
k =  453
fk =  -5239.541412076969
||gk|| =  5.700360427845806e-06
Primeras 3 coordenadas de xk son:  [-11.61784726   5.4498234    0.96506346]
Últimas 3 coordenadas de xk son:  [ -1.8852385  -10.34321955  -4.15697837]
||xk-x*|| =  3.859553615080582e-05
    \end{Verbatim}

    A diferencia de la primer función cuadrática no hay una diferencia tan
radical. Con la segunda condición inicial se necesitan más iteraciones
para alcanzar la convergencia que en la primero, pero al final los
resultados son bastante parecidos incluso en la distancia con el punto
estacionario \(x_\ast\),

    \hypertarget{ejercicio-2-5-puntos}{%
\section{Ejercicio 2 (5 puntos)}\label{ejercicio-2-5-puntos}}

Programar el método de descenso máximo con tamaño de paso seleccionado
por la estrategia de backtracking:

\textbf{Algoritmo de descenso máximo con backtracking:}

Dada una función \(f: \mathbb{R}^n \rightarrow \mathbb{R}\), su
gradiente \(g: \mathbb{R}^n \rightarrow \mathbb{R}^n\), un punto inicial
\(x_0\), un número máximo de iteraciones \(N\), una tolerancia
\(\tau>0\). Fijar \(k=0\) y repetir los siguientes pasos:

\begin{enumerate}
\def\labelenumi{\arabic{enumi}.}
\item
  Calcular el gradiente en el punto \(x_k\):

  \[ g_k = \nabla f(x_k) = g(x_k) \]
\item
  Si \(\|g_k\|<\tau\), \(x_k\) es un aproximadamente un punto
  estacionario, por lo que hay que hacer \(res=1\) y terminar el ciclo.
\item
  Eligir la dirección de descenso como \(p_k = - g_k\).
\item
  Calcular el tamaño de paso \(\alpha_k\) mediante la estrategia de
  backtraking, usando el algoritmo que describe más adelante.
\item
  Calcular el siguiente punto de la secuencia como
  \[x_{k+1} = x_k + \alpha_k p_k \]
\item
  Si \({k+1}>N\), hacer \(res=0\) y terminar.
\item
  Si no, hacer \(k = k+1\) y volver el paso 1.
\item
  Devolver el punto \(x_k\), \(f_k= f(x_k)\), \(g_k\), \(k\) y \(res\).
\end{enumerate}

\begin{center}\rule{0.5\linewidth}{0.5pt}\end{center}

** Algoritmo de backtracking **

Backtracking(\(f\), \(f_k\), \(g_k\), \(x_k\), \(p_k\),
\(\alpha_{ini}\), \(\rho\), \(c\))

El algoritmo recibe la función \(f\), el punto \(x_k\),
\(f_k = f(x_k)\), la dirección de descenso \(p_k\), un valor inicial
\(\alpha_{ini}\), \(\rho \in (0,1)\), \(c \in (0,1)\).

Fijar \(\alpha = \alpha_{ini}\) y repetir los siguientes pasos:

\begin{enumerate}
\def\labelenumi{\arabic{enumi}.}
\tightlist
\item
  Si se cumple la condición
  \[ f(x_k+\alpha p_k) \leq f_k + c \alpha g_k^\top p_k, \] terminar el
  ciclo devolviendo
\item
  Hacer \(\alpha = \rho \alpha\) y regresar al paso anterior.
\end{enumerate}

\begin{center}\rule{0.5\linewidth}{0.5pt}\end{center}

\begin{enumerate}
\def\labelenumi{\arabic{enumi}.}
\tightlist
\item
  Escriba una función que implementa el algoritmo de backtracking.
\item
  Escriba la función que implementa el algoritmo de máximo descenso con
  búsqueda inexacta, usando backtraking. Tiene que recibir como
  paramétros todos los elementos que se listaron para ambos algoritmos.
\item
  Escriba una función para probar el funcionamiento del método de
  descenso máximo. Esta función debe recibir la función \(f\), la
  función \(g\) que devuelve su gradiente, el punto inicial \(x_0\), el
  número\\
  máximo de iteraciones \(N\), la tolerancia \(\tau>0\) y el factor
  \(\rho\) del algoritmo de backtracking.
\end{enumerate}

\begin{itemize}
\tightlist
\item
  Fijar los parámetros \(\alpha_{ini}=2\) y \(c=0.0001\) del algoritmo
  de backtracking.
\item
  Ejecutar la función del Inciso 2.
\item
  Dependiendo del valor de la variable \(res\), imprima un mensaje que
  diga que el algoritmo convergió (\(res=1\)) o no (\(res=0\))
\item
  Imprimir \(k\), \(x_k\), \(f_k\) y la norma de \(g_k\).
\end{itemize}

\begin{enumerate}
\def\labelenumi{\arabic{enumi}.}
\setcounter{enumi}{3}
\tightlist
\item
  Pruebe la función del Inciso 3 usando \(N=10000\), \(\rho=0.8\), la
  tolerancia \(\tau = \epsilon_m^{1/3}\), donde \(\epsilon_m\) es el
  épsilon de la máquina. Aplique esta función a:
\end{enumerate}

\begin{itemize}
\tightlist
\item
  La función de Rosenbrock, descrita en la Tarea 3, usando como punto
  inicial \(x_0= (-1.2, 1)\) y \(x_0= (-12, 10)\). Como referencia, el
  minimizador de la función es \(x_* = (1,1)\).
\end{itemize}

\begin{enumerate}
\def\labelenumi{\arabic{enumi}.}
\setcounter{enumi}{4}
\tightlist
\item
  Repita el inciso anterior con \(\rho=0.5\).
\end{enumerate}

\hypertarget{soluciuxf3n}{%
\subsection{Solución:}\label{soluciuxf3n}}

    De manera similar al ejercicio anterior, importamos el módulo
\texttt{lib\_t4} donde se encuentran las funciones
\texttt{backtracking}, \texttt{grad\_max} y \texttt{proof\_grad\_max}
que son las implementaciones de lo que se solicita en los numerales 1,2
y 3, respectivamente.

Realizamos las pruebas del método de descenso máximo con tamaño de paso
obtenido a través del \emph{algoritmo backtracking}.

Fijamos un número máximo de iteraciones \(N=10,000\) y una tolerancia
\(\tau=\epsilon_m^{1/3}\).

\hypertarget{rho0.8}{%
\subsubsection{\texorpdfstring{\(\rho=0.8\)}{\textbackslash rho=0.8}}\label{rho0.8}}

En primer lugar, probamos este método de optimización con la
\emph{Función de Rosenbrock} con la condición inicial \(x_0=(-1.2,1)\) y
tomando \(\rho=0.8\), el resultado es el siguiente

    \begin{tcolorbox}[breakable, size=fbox, boxrule=1pt, pad at break*=1mm,colback=cellbackground, colframe=cellborder]
\prompt{In}{incolor}{6}{\boxspacing}
\begin{Verbatim}[commandchars=\\\{\}]
\PY{n}{importlib}\PY{o}{.}\PY{n}{reload}\PY{p}{(}\PY{n}{lib\PYZus{}t4}\PY{p}{)}
\PY{k+kn}{from} \PY{n+nn}{lib\PYZus{}t4} \PY{k+kn}{import} \PY{o}{*}
\PY{n}{tol}\PY{o}{=}\PY{n}{np}\PY{o}{.}\PY{n}{finfo}\PY{p}{(}\PY{n+nb}{float}\PY{p}{)}\PY{o}{.}\PY{n}{eps}\PY{o}{*}\PY{o}{*}\PY{p}{(}\PY{l+m+mi}{1}\PY{o}{/}\PY{l+m+mi}{3}\PY{p}{)}
\PY{n}{N}\PY{o}{=}\PY{l+m+mi}{10000}
\PY{n}{rho}\PY{o}{=}\PY{l+m+mf}{0.8}
\PY{n}{x0}\PY{o}{=}\PY{n}{np}\PY{o}{.}\PY{n}{array}\PY{p}{(}\PY{p}{[}\PY{o}{\PYZhy{}}\PY{l+m+mf}{1.2}\PY{p}{,}\PY{l+m+mf}{1.0}\PY{p}{]}\PY{p}{)}
\PY{n}{proof\PYZus{}grad\PYZus{}max}\PY{p}{(}\PY{n}{f\PYZus{}Rosenbrock}\PY{p}{,}\PY{n}{grad\PYZus{}Rosenbrock}\PY{p}{,}\PY{n}{x0}\PY{p}{,}\PY{n}{N}\PY{p}{,}\PY{n}{tol}\PY{p}{,}\PY{n}{rho}\PY{p}{)}
\end{Verbatim}
\end{tcolorbox}

    \begin{Verbatim}[commandchars=\\\{\}]
El algoritmo de descenso máximo con backtracking NO CONVERGE
k =  10000
xk =  [1.00079208 1.00158391]
fk =  6.274640640496038e-07
||gk|| =  0.0019653292817284995
    \end{Verbatim}

    Con el mismo factor \(\rho=0.8\) pero con la condición inicial
\(x_0=(-12,10)\) obtenemos

    \begin{tcolorbox}[breakable, size=fbox, boxrule=1pt, pad at break*=1mm,colback=cellbackground, colframe=cellborder]
\prompt{In}{incolor}{7}{\boxspacing}
\begin{Verbatim}[commandchars=\\\{\}]
\PY{n}{x0}\PY{o}{=}\PY{n}{np}\PY{o}{.}\PY{n}{array}\PY{p}{(}\PY{p}{[}\PY{o}{\PYZhy{}}\PY{l+m+mf}{12.0}\PY{p}{,}\PY{l+m+mf}{10.0}\PY{p}{]}\PY{p}{)}
\PY{n}{proof\PYZus{}grad\PYZus{}max}\PY{p}{(}\PY{n}{f\PYZus{}Rosenbrock}\PY{p}{,}\PY{n}{grad\PYZus{}Rosenbrock}\PY{p}{,}\PY{n}{x0}\PY{p}{,}\PY{n}{N}\PY{p}{,}\PY{n}{tol}\PY{p}{,}\PY{n}{rho}\PY{p}{)}
\end{Verbatim}
\end{tcolorbox}

    \begin{Verbatim}[commandchars=\\\{\}]
El algoritmo de descenso máximo con backtracking NO CONVERGE
k =  10000
xk =  [ 3.92075554 15.37404809]
fk =  8.531110164767732
||gk|| =  2.823173435454123
    \end{Verbatim}

    Observamos que en ambos casos se alcanza el máximo número de
iteraciones. Con la condición inicial \(x_0=(-12,10)\), al estar más
lejos de el óptimo \(x_\ast=(1,1)\) el resultado final igual queda más
lejos de \(x_\ast\) que con la primer condición inicial, pues con esta
condición inicial el método de descenso máximo en las 10000 iteraciones
queda mucho más cerca del valor \(x_\ast\).

    \hypertarget{rho0.5}{%
\subsubsection{\texorpdfstring{\(\rho=0.5\)}{\textbackslash rho=0.5}}\label{rho0.5}}

Ahora cambiamos sólo el factor por el cual aumentamos la velocidad de
reducción de el tamaño de paso en el \emph{algoritmo backtracking} a
\(\rho=0.5\).

Para la condición inicial \(x_0=(-1.2,1)\) obtenemos el siguiente
resultado

    \begin{tcolorbox}[breakable, size=fbox, boxrule=1pt, pad at break*=1mm,colback=cellbackground, colframe=cellborder]
\prompt{In}{incolor}{8}{\boxspacing}
\begin{Verbatim}[commandchars=\\\{\}]
\PY{n}{rho}\PY{o}{=}\PY{l+m+mf}{0.5}
\PY{n}{x0}\PY{o}{=}\PY{n}{np}\PY{o}{.}\PY{n}{array}\PY{p}{(}\PY{p}{[}\PY{o}{\PYZhy{}}\PY{l+m+mf}{1.2}\PY{p}{,}\PY{l+m+mf}{1.0}\PY{p}{]}\PY{p}{)}
\PY{n}{proof\PYZus{}grad\PYZus{}max}\PY{p}{(}\PY{n}{f\PYZus{}Rosenbrock}\PY{p}{,}\PY{n}{grad\PYZus{}Rosenbrock}\PY{p}{,}\PY{n}{x0}\PY{p}{,}\PY{n}{N}\PY{p}{,}\PY{n}{tol}\PY{p}{,}\PY{n}{rho}\PY{p}{)}
\end{Verbatim}
\end{tcolorbox}

    \begin{Verbatim}[commandchars=\\\{\}]
El algoritmo de descenso máximo con backtracking NO CONVERGE
k =  10000
xk =  [0.99998357 0.99996704]
fk =  2.7097756887567074e-10
||gk|| =  2.377226279771169e-05
    \end{Verbatim}

    Para la condición inicial \(x_0=(-12,10)\) el resultado es

    \begin{tcolorbox}[breakable, size=fbox, boxrule=1pt, pad at break*=1mm,colback=cellbackground, colframe=cellborder]
\prompt{In}{incolor}{9}{\boxspacing}
\begin{Verbatim}[commandchars=\\\{\}]
\PY{n}{x0}\PY{o}{=}\PY{n}{np}\PY{o}{.}\PY{n}{array}\PY{p}{(}\PY{p}{[}\PY{o}{\PYZhy{}}\PY{l+m+mf}{12.0}\PY{p}{,}\PY{l+m+mf}{10.0}\PY{p}{]}\PY{p}{)}
\PY{n}{proof\PYZus{}grad\PYZus{}max}\PY{p}{(}\PY{n}{f\PYZus{}Rosenbrock}\PY{p}{,}\PY{n}{grad\PYZus{}Rosenbrock}\PY{p}{,}\PY{n}{x0}\PY{p}{,}\PY{n}{N}\PY{p}{,}\PY{n}{tol}\PY{p}{,}\PY{n}{rho}\PY{p}{)}
\end{Verbatim}
\end{tcolorbox}

    \begin{Verbatim}[commandchars=\\\{\}]
El algoritmo de descenso máximo con backtracking NO CONVERGE
k =  10000
xk =  [2.85989511 8.18266786]
fk =  3.46055510931203
||gk|| =  1.121519863018428
    \end{Verbatim}

    Al igual que con el factor \(\rho=0.8\) con ambas condiciones se alcanza
el máximo de iteraciones, sin embargo el valor final de la función de
Rosenbrock en ambos casos queda mucho más cerca del valor óptimo que es
\(f(x_\ast)=0\), lo que sugiere que quizás el factor \(\rho=0.8\) en el
algoritmo backtracking deja aún muy largo el tamaño de paso que puede
ocasionar que rebote la búsqueda en línea.

    Finalmente, vemos que si aumentamos la velocidad a la que el tamaño de
paso \(\alpha\to 0\) tomando \(\rho=0.4\). Para la condición inicial
\(x_0=(-1.2,1)\) el resultado es

    \begin{tcolorbox}[breakable, size=fbox, boxrule=1pt, pad at break*=1mm,colback=cellbackground, colframe=cellborder]
\prompt{In}{incolor}{10}{\boxspacing}
\begin{Verbatim}[commandchars=\\\{\}]
\PY{n}{rho}\PY{o}{=}\PY{l+m+mf}{0.4}
\PY{n}{x0}\PY{o}{=}\PY{n}{np}\PY{o}{.}\PY{n}{array}\PY{p}{(}\PY{p}{[}\PY{o}{\PYZhy{}}\PY{l+m+mf}{1.2}\PY{p}{,}\PY{l+m+mf}{1.0}\PY{p}{]}\PY{p}{)}
\PY{n}{proof\PYZus{}grad\PYZus{}max}\PY{p}{(}\PY{n}{f\PYZus{}Rosenbrock}\PY{p}{,}\PY{n}{grad\PYZus{}Rosenbrock}\PY{p}{,}\PY{n}{x0}\PY{p}{,}\PY{n}{N}\PY{p}{,}\PY{n}{tol}\PY{p}{,}\PY{n}{rho}\PY{p}{)}
\end{Verbatim}
\end{tcolorbox}

    \begin{Verbatim}[commandchars=\\\{\}]
El algoritmo de descenso máximo con backtracking CONVERGE
k =  9989
xk =  [1.00000616 1.00001235]
fk =  3.80249328045783e-11
||gk|| =  5.974110497205854e-06
    \end{Verbatim}

    Para la condición inicial \(x_0=(-12,10)\) los resultados son los
siguientes

    \begin{tcolorbox}[breakable, size=fbox, boxrule=1pt, pad at break*=1mm,colback=cellbackground, colframe=cellborder]
\prompt{In}{incolor}{11}{\boxspacing}
\begin{Verbatim}[commandchars=\\\{\}]
\PY{n}{x0}\PY{o}{=}\PY{n}{np}\PY{o}{.}\PY{n}{array}\PY{p}{(}\PY{p}{[}\PY{o}{\PYZhy{}}\PY{l+m+mf}{12.0}\PY{p}{,}\PY{l+m+mf}{10.0}\PY{p}{]}\PY{p}{)}
\PY{n}{proof\PYZus{}grad\PYZus{}max}\PY{p}{(}\PY{n}{f\PYZus{}Rosenbrock}\PY{p}{,}\PY{n}{grad\PYZus{}Rosenbrock}\PY{p}{,}\PY{n}{x0}\PY{p}{,}\PY{n}{N}\PY{p}{,}\PY{n}{tol}\PY{p}{,}\PY{n}{rho}\PY{p}{)}
\end{Verbatim}
\end{tcolorbox}

    \begin{Verbatim}[commandchars=\\\{\}]
El algoritmo de descenso máximo con backtracking NO CONVERGE
k =  10000
xk =  [1.93435739 3.74235441]
fk =  0.873061658777709
||gk|| =  0.6240981756328119
    \end{Verbatim}

    A diferencia de los otros factores \(\rho\) con \(\rho=0.4\) tenemos
convergencia para la primer condición inicial. Análogamente, para la
segunda condición inicial estamos más cerca del valor óptimo de la
función de Rosenbrock que con los otros factores \(\rho\).

Observando la gráfica de la función de Rosenbrock en la \emph{Tarea 3},
podemos concluir que este comportamiento se puede deber a que en el rayo
\(\lambda (-1.2,1.0)\) con \(\lambda\in \mathbb{R}\) se encuentra un
precipicio algo pronunciado, entonces para pasos grandes nos podemos
salir de ese valle y alejarnos del óptimo de forma más fácil que con
tamaños de paso más pequeños, por lo que eso explica que el
comportamiento del método de descenso máximo mejore cuando se acelera la
velocidad con la que el tamaño de paso \(\alpha\to 0\) en el
\emph{algoritmo backtracking}.


    % Add a bibliography block to the postdoc
    
    
    
\end{document}
