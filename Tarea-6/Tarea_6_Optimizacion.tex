\documentclass[11pt]{article}

    \usepackage[breakable]{tcolorbox}
    \usepackage{parskip} % Stop auto-indenting (to mimic markdown behaviour)
    
    \usepackage{iftex}
    \ifPDFTeX
    	\usepackage[T1]{fontenc}
    	\usepackage{mathpazo}
    \else
    	\usepackage{fontspec}
    \fi

    % Basic figure setup, for now with no caption control since it's done
    % automatically by Pandoc (which extracts ![](path) syntax from Markdown).
    \usepackage{graphicx}
    % Maintain compatibility with old templates. Remove in nbconvert 6.0
    \let\Oldincludegraphics\includegraphics
    % Ensure that by default, figures have no caption (until we provide a
    % proper Figure object with a Caption API and a way to capture that
    % in the conversion process - todo).
    \usepackage{caption}
    \DeclareCaptionFormat{nocaption}{}
    \captionsetup{format=nocaption,aboveskip=0pt,belowskip=0pt}

    \usepackage{float}
    \floatplacement{figure}{H} % forces figures to be placed at the correct location
    \usepackage{xcolor} % Allow colors to be defined
    \usepackage{enumerate} % Needed for markdown enumerations to work
    \usepackage{geometry} % Used to adjust the document margins
    \usepackage{amsmath} % Equations
    \usepackage{amssymb} % Equations
    \usepackage{textcomp} % defines textquotesingle
    % Hack from http://tex.stackexchange.com/a/47451/13684:
    \AtBeginDocument{%
        \def\PYZsq{\textquotesingle}% Upright quotes in Pygmentized code
    }
    \usepackage{upquote} % Upright quotes for verbatim code
    \usepackage{eurosym} % defines \euro
    \usepackage[mathletters]{ucs} % Extended unicode (utf-8) support
    \usepackage{fancyvrb} % verbatim replacement that allows latex
    \usepackage{grffile} % extends the file name processing of package graphics 
                         % to support a larger range
    \makeatletter % fix for old versions of grffile with XeLaTeX
    \@ifpackagelater{grffile}{2019/11/01}
    {
      % Do nothing on new versions
    }
    {
      \def\Gread@@xetex#1{%
        \IfFileExists{"\Gin@base".bb}%
        {\Gread@eps{\Gin@base.bb}}%
        {\Gread@@xetex@aux#1}%
      }
    }
    \makeatother
    \usepackage[Export]{adjustbox} % Used to constrain images to a maximum size
    \adjustboxset{max size={0.9\linewidth}{0.9\paperheight}}

    % The hyperref package gives us a pdf with properly built
    % internal navigation ('pdf bookmarks' for the table of contents,
    % internal cross-reference links, web links for URLs, etc.)
    \usepackage{hyperref}
    % The default LaTeX title has an obnoxious amount of whitespace. By default,
    % titling removes some of it. It also provides customization options.
    \usepackage{titling}
    \usepackage{longtable} % longtable support required by pandoc >1.10
    \usepackage{booktabs}  % table support for pandoc > 1.12.2
    \usepackage[inline]{enumitem} % IRkernel/repr support (it uses the enumerate* environment)
    \usepackage[normalem]{ulem} % ulem is needed to support strikethroughs (\sout)
                                % normalem makes italics be italics, not underlines
    \usepackage{mathrsfs}
    

    
    % Colors for the hyperref package
    \definecolor{urlcolor}{rgb}{0,.145,.698}
    \definecolor{linkcolor}{rgb}{.71,0.21,0.01}
    \definecolor{citecolor}{rgb}{.12,.54,.11}

    % ANSI colors
    \definecolor{ansi-black}{HTML}{3E424D}
    \definecolor{ansi-black-intense}{HTML}{282C36}
    \definecolor{ansi-red}{HTML}{E75C58}
    \definecolor{ansi-red-intense}{HTML}{B22B31}
    \definecolor{ansi-green}{HTML}{00A250}
    \definecolor{ansi-green-intense}{HTML}{007427}
    \definecolor{ansi-yellow}{HTML}{DDB62B}
    \definecolor{ansi-yellow-intense}{HTML}{B27D12}
    \definecolor{ansi-blue}{HTML}{208FFB}
    \definecolor{ansi-blue-intense}{HTML}{0065CA}
    \definecolor{ansi-magenta}{HTML}{D160C4}
    \definecolor{ansi-magenta-intense}{HTML}{A03196}
    \definecolor{ansi-cyan}{HTML}{60C6C8}
    \definecolor{ansi-cyan-intense}{HTML}{258F8F}
    \definecolor{ansi-white}{HTML}{C5C1B4}
    \definecolor{ansi-white-intense}{HTML}{A1A6B2}
    \definecolor{ansi-default-inverse-fg}{HTML}{FFFFFF}
    \definecolor{ansi-default-inverse-bg}{HTML}{000000}

    % common color for the border for error outputs.
    \definecolor{outerrorbackground}{HTML}{FFDFDF}

    % commands and environments needed by pandoc snippets
    % extracted from the output of `pandoc -s`
    \providecommand{\tightlist}{%
      \setlength{\itemsep}{0pt}\setlength{\parskip}{0pt}}
    \DefineVerbatimEnvironment{Highlighting}{Verbatim}{commandchars=\\\{\}}
    % Add ',fontsize=\small' for more characters per line
    \newenvironment{Shaded}{}{}
    \newcommand{\KeywordTok}[1]{\textcolor[rgb]{0.00,0.44,0.13}{\textbf{{#1}}}}
    \newcommand{\DataTypeTok}[1]{\textcolor[rgb]{0.56,0.13,0.00}{{#1}}}
    \newcommand{\DecValTok}[1]{\textcolor[rgb]{0.25,0.63,0.44}{{#1}}}
    \newcommand{\BaseNTok}[1]{\textcolor[rgb]{0.25,0.63,0.44}{{#1}}}
    \newcommand{\FloatTok}[1]{\textcolor[rgb]{0.25,0.63,0.44}{{#1}}}
    \newcommand{\CharTok}[1]{\textcolor[rgb]{0.25,0.44,0.63}{{#1}}}
    \newcommand{\StringTok}[1]{\textcolor[rgb]{0.25,0.44,0.63}{{#1}}}
    \newcommand{\CommentTok}[1]{\textcolor[rgb]{0.38,0.63,0.69}{\textit{{#1}}}}
    \newcommand{\OtherTok}[1]{\textcolor[rgb]{0.00,0.44,0.13}{{#1}}}
    \newcommand{\AlertTok}[1]{\textcolor[rgb]{1.00,0.00,0.00}{\textbf{{#1}}}}
    \newcommand{\FunctionTok}[1]{\textcolor[rgb]{0.02,0.16,0.49}{{#1}}}
    \newcommand{\RegionMarkerTok}[1]{{#1}}
    \newcommand{\ErrorTok}[1]{\textcolor[rgb]{1.00,0.00,0.00}{\textbf{{#1}}}}
    \newcommand{\NormalTok}[1]{{#1}}
    
    % Additional commands for more recent versions of Pandoc
    \newcommand{\ConstantTok}[1]{\textcolor[rgb]{0.53,0.00,0.00}{{#1}}}
    \newcommand{\SpecialCharTok}[1]{\textcolor[rgb]{0.25,0.44,0.63}{{#1}}}
    \newcommand{\VerbatimStringTok}[1]{\textcolor[rgb]{0.25,0.44,0.63}{{#1}}}
    \newcommand{\SpecialStringTok}[1]{\textcolor[rgb]{0.73,0.40,0.53}{{#1}}}
    \newcommand{\ImportTok}[1]{{#1}}
    \newcommand{\DocumentationTok}[1]{\textcolor[rgb]{0.73,0.13,0.13}{\textit{{#1}}}}
    \newcommand{\AnnotationTok}[1]{\textcolor[rgb]{0.38,0.63,0.69}{\textbf{\textit{{#1}}}}}
    \newcommand{\CommentVarTok}[1]{\textcolor[rgb]{0.38,0.63,0.69}{\textbf{\textit{{#1}}}}}
    \newcommand{\VariableTok}[1]{\textcolor[rgb]{0.10,0.09,0.49}{{#1}}}
    \newcommand{\ControlFlowTok}[1]{\textcolor[rgb]{0.00,0.44,0.13}{\textbf{{#1}}}}
    \newcommand{\OperatorTok}[1]{\textcolor[rgb]{0.40,0.40,0.40}{{#1}}}
    \newcommand{\BuiltInTok}[1]{{#1}}
    \newcommand{\ExtensionTok}[1]{{#1}}
    \newcommand{\PreprocessorTok}[1]{\textcolor[rgb]{0.74,0.48,0.00}{{#1}}}
    \newcommand{\AttributeTok}[1]{\textcolor[rgb]{0.49,0.56,0.16}{{#1}}}
    \newcommand{\InformationTok}[1]{\textcolor[rgb]{0.38,0.63,0.69}{\textbf{\textit{{#1}}}}}
    \newcommand{\WarningTok}[1]{\textcolor[rgb]{0.38,0.63,0.69}{\textbf{\textit{{#1}}}}}
    
    
    % Define a nice break command that doesn't care if a line doesn't already
    % exist.
    \def\br{\hspace*{\fill} \\* }
    % Math Jax compatibility definitions
    \def\gt{>}
    \def\lt{<}
    \let\Oldtex\TeX
    \let\Oldlatex\LaTeX
    \renewcommand{\TeX}{\textrm{\Oldtex}}
    \renewcommand{\LaTeX}{\textrm{\Oldlatex}}
    % Document parameters
    % Document title
    \title{Tarea\_6\_Optimizacion}
    
    
    
    
    
% Pygments definitions
\makeatletter
\def\PY@reset{\let\PY@it=\relax \let\PY@bf=\relax%
    \let\PY@ul=\relax \let\PY@tc=\relax%
    \let\PY@bc=\relax \let\PY@ff=\relax}
\def\PY@tok#1{\csname PY@tok@#1\endcsname}
\def\PY@toks#1+{\ifx\relax#1\empty\else%
    \PY@tok{#1}\expandafter\PY@toks\fi}
\def\PY@do#1{\PY@bc{\PY@tc{\PY@ul{%
    \PY@it{\PY@bf{\PY@ff{#1}}}}}}}
\def\PY#1#2{\PY@reset\PY@toks#1+\relax+\PY@do{#2}}

\@namedef{PY@tok@w}{\def\PY@tc##1{\textcolor[rgb]{0.73,0.73,0.73}{##1}}}
\@namedef{PY@tok@c}{\let\PY@it=\textit\def\PY@tc##1{\textcolor[rgb]{0.25,0.50,0.50}{##1}}}
\@namedef{PY@tok@cp}{\def\PY@tc##1{\textcolor[rgb]{0.74,0.48,0.00}{##1}}}
\@namedef{PY@tok@k}{\let\PY@bf=\textbf\def\PY@tc##1{\textcolor[rgb]{0.00,0.50,0.00}{##1}}}
\@namedef{PY@tok@kp}{\def\PY@tc##1{\textcolor[rgb]{0.00,0.50,0.00}{##1}}}
\@namedef{PY@tok@kt}{\def\PY@tc##1{\textcolor[rgb]{0.69,0.00,0.25}{##1}}}
\@namedef{PY@tok@o}{\def\PY@tc##1{\textcolor[rgb]{0.40,0.40,0.40}{##1}}}
\@namedef{PY@tok@ow}{\let\PY@bf=\textbf\def\PY@tc##1{\textcolor[rgb]{0.67,0.13,1.00}{##1}}}
\@namedef{PY@tok@nb}{\def\PY@tc##1{\textcolor[rgb]{0.00,0.50,0.00}{##1}}}
\@namedef{PY@tok@nf}{\def\PY@tc##1{\textcolor[rgb]{0.00,0.00,1.00}{##1}}}
\@namedef{PY@tok@nc}{\let\PY@bf=\textbf\def\PY@tc##1{\textcolor[rgb]{0.00,0.00,1.00}{##1}}}
\@namedef{PY@tok@nn}{\let\PY@bf=\textbf\def\PY@tc##1{\textcolor[rgb]{0.00,0.00,1.00}{##1}}}
\@namedef{PY@tok@ne}{\let\PY@bf=\textbf\def\PY@tc##1{\textcolor[rgb]{0.82,0.25,0.23}{##1}}}
\@namedef{PY@tok@nv}{\def\PY@tc##1{\textcolor[rgb]{0.10,0.09,0.49}{##1}}}
\@namedef{PY@tok@no}{\def\PY@tc##1{\textcolor[rgb]{0.53,0.00,0.00}{##1}}}
\@namedef{PY@tok@nl}{\def\PY@tc##1{\textcolor[rgb]{0.63,0.63,0.00}{##1}}}
\@namedef{PY@tok@ni}{\let\PY@bf=\textbf\def\PY@tc##1{\textcolor[rgb]{0.60,0.60,0.60}{##1}}}
\@namedef{PY@tok@na}{\def\PY@tc##1{\textcolor[rgb]{0.49,0.56,0.16}{##1}}}
\@namedef{PY@tok@nt}{\let\PY@bf=\textbf\def\PY@tc##1{\textcolor[rgb]{0.00,0.50,0.00}{##1}}}
\@namedef{PY@tok@nd}{\def\PY@tc##1{\textcolor[rgb]{0.67,0.13,1.00}{##1}}}
\@namedef{PY@tok@s}{\def\PY@tc##1{\textcolor[rgb]{0.73,0.13,0.13}{##1}}}
\@namedef{PY@tok@sd}{\let\PY@it=\textit\def\PY@tc##1{\textcolor[rgb]{0.73,0.13,0.13}{##1}}}
\@namedef{PY@tok@si}{\let\PY@bf=\textbf\def\PY@tc##1{\textcolor[rgb]{0.73,0.40,0.53}{##1}}}
\@namedef{PY@tok@se}{\let\PY@bf=\textbf\def\PY@tc##1{\textcolor[rgb]{0.73,0.40,0.13}{##1}}}
\@namedef{PY@tok@sr}{\def\PY@tc##1{\textcolor[rgb]{0.73,0.40,0.53}{##1}}}
\@namedef{PY@tok@ss}{\def\PY@tc##1{\textcolor[rgb]{0.10,0.09,0.49}{##1}}}
\@namedef{PY@tok@sx}{\def\PY@tc##1{\textcolor[rgb]{0.00,0.50,0.00}{##1}}}
\@namedef{PY@tok@m}{\def\PY@tc##1{\textcolor[rgb]{0.40,0.40,0.40}{##1}}}
\@namedef{PY@tok@gh}{\let\PY@bf=\textbf\def\PY@tc##1{\textcolor[rgb]{0.00,0.00,0.50}{##1}}}
\@namedef{PY@tok@gu}{\let\PY@bf=\textbf\def\PY@tc##1{\textcolor[rgb]{0.50,0.00,0.50}{##1}}}
\@namedef{PY@tok@gd}{\def\PY@tc##1{\textcolor[rgb]{0.63,0.00,0.00}{##1}}}
\@namedef{PY@tok@gi}{\def\PY@tc##1{\textcolor[rgb]{0.00,0.63,0.00}{##1}}}
\@namedef{PY@tok@gr}{\def\PY@tc##1{\textcolor[rgb]{1.00,0.00,0.00}{##1}}}
\@namedef{PY@tok@ge}{\let\PY@it=\textit}
\@namedef{PY@tok@gs}{\let\PY@bf=\textbf}
\@namedef{PY@tok@gp}{\let\PY@bf=\textbf\def\PY@tc##1{\textcolor[rgb]{0.00,0.00,0.50}{##1}}}
\@namedef{PY@tok@go}{\def\PY@tc##1{\textcolor[rgb]{0.53,0.53,0.53}{##1}}}
\@namedef{PY@tok@gt}{\def\PY@tc##1{\textcolor[rgb]{0.00,0.27,0.87}{##1}}}
\@namedef{PY@tok@err}{\def\PY@bc##1{{\setlength{\fboxsep}{\string -\fboxrule}\fcolorbox[rgb]{1.00,0.00,0.00}{1,1,1}{\strut ##1}}}}
\@namedef{PY@tok@kc}{\let\PY@bf=\textbf\def\PY@tc##1{\textcolor[rgb]{0.00,0.50,0.00}{##1}}}
\@namedef{PY@tok@kd}{\let\PY@bf=\textbf\def\PY@tc##1{\textcolor[rgb]{0.00,0.50,0.00}{##1}}}
\@namedef{PY@tok@kn}{\let\PY@bf=\textbf\def\PY@tc##1{\textcolor[rgb]{0.00,0.50,0.00}{##1}}}
\@namedef{PY@tok@kr}{\let\PY@bf=\textbf\def\PY@tc##1{\textcolor[rgb]{0.00,0.50,0.00}{##1}}}
\@namedef{PY@tok@bp}{\def\PY@tc##1{\textcolor[rgb]{0.00,0.50,0.00}{##1}}}
\@namedef{PY@tok@fm}{\def\PY@tc##1{\textcolor[rgb]{0.00,0.00,1.00}{##1}}}
\@namedef{PY@tok@vc}{\def\PY@tc##1{\textcolor[rgb]{0.10,0.09,0.49}{##1}}}
\@namedef{PY@tok@vg}{\def\PY@tc##1{\textcolor[rgb]{0.10,0.09,0.49}{##1}}}
\@namedef{PY@tok@vi}{\def\PY@tc##1{\textcolor[rgb]{0.10,0.09,0.49}{##1}}}
\@namedef{PY@tok@vm}{\def\PY@tc##1{\textcolor[rgb]{0.10,0.09,0.49}{##1}}}
\@namedef{PY@tok@sa}{\def\PY@tc##1{\textcolor[rgb]{0.73,0.13,0.13}{##1}}}
\@namedef{PY@tok@sb}{\def\PY@tc##1{\textcolor[rgb]{0.73,0.13,0.13}{##1}}}
\@namedef{PY@tok@sc}{\def\PY@tc##1{\textcolor[rgb]{0.73,0.13,0.13}{##1}}}
\@namedef{PY@tok@dl}{\def\PY@tc##1{\textcolor[rgb]{0.73,0.13,0.13}{##1}}}
\@namedef{PY@tok@s2}{\def\PY@tc##1{\textcolor[rgb]{0.73,0.13,0.13}{##1}}}
\@namedef{PY@tok@sh}{\def\PY@tc##1{\textcolor[rgb]{0.73,0.13,0.13}{##1}}}
\@namedef{PY@tok@s1}{\def\PY@tc##1{\textcolor[rgb]{0.73,0.13,0.13}{##1}}}
\@namedef{PY@tok@mb}{\def\PY@tc##1{\textcolor[rgb]{0.40,0.40,0.40}{##1}}}
\@namedef{PY@tok@mf}{\def\PY@tc##1{\textcolor[rgb]{0.40,0.40,0.40}{##1}}}
\@namedef{PY@tok@mh}{\def\PY@tc##1{\textcolor[rgb]{0.40,0.40,0.40}{##1}}}
\@namedef{PY@tok@mi}{\def\PY@tc##1{\textcolor[rgb]{0.40,0.40,0.40}{##1}}}
\@namedef{PY@tok@il}{\def\PY@tc##1{\textcolor[rgb]{0.40,0.40,0.40}{##1}}}
\@namedef{PY@tok@mo}{\def\PY@tc##1{\textcolor[rgb]{0.40,0.40,0.40}{##1}}}
\@namedef{PY@tok@ch}{\let\PY@it=\textit\def\PY@tc##1{\textcolor[rgb]{0.25,0.50,0.50}{##1}}}
\@namedef{PY@tok@cm}{\let\PY@it=\textit\def\PY@tc##1{\textcolor[rgb]{0.25,0.50,0.50}{##1}}}
\@namedef{PY@tok@cpf}{\let\PY@it=\textit\def\PY@tc##1{\textcolor[rgb]{0.25,0.50,0.50}{##1}}}
\@namedef{PY@tok@c1}{\let\PY@it=\textit\def\PY@tc##1{\textcolor[rgb]{0.25,0.50,0.50}{##1}}}
\@namedef{PY@tok@cs}{\let\PY@it=\textit\def\PY@tc##1{\textcolor[rgb]{0.25,0.50,0.50}{##1}}}

\def\PYZbs{\char`\\}
\def\PYZus{\char`\_}
\def\PYZob{\char`\{}
\def\PYZcb{\char`\}}
\def\PYZca{\char`\^}
\def\PYZam{\char`\&}
\def\PYZlt{\char`\<}
\def\PYZgt{\char`\>}
\def\PYZsh{\char`\#}
\def\PYZpc{\char`\%}
\def\PYZdl{\char`\$}
\def\PYZhy{\char`\-}
\def\PYZsq{\char`\'}
\def\PYZdq{\char`\"}
\def\PYZti{\char`\~}
% for compatibility with earlier versions
\def\PYZat{@}
\def\PYZlb{[}
\def\PYZrb{]}
\makeatother


    % For linebreaks inside Verbatim environment from package fancyvrb. 
    \makeatletter
        \newbox\Wrappedcontinuationbox 
        \newbox\Wrappedvisiblespacebox 
        \newcommand*\Wrappedvisiblespace {\textcolor{red}{\textvisiblespace}} 
        \newcommand*\Wrappedcontinuationsymbol {\textcolor{red}{\llap{\tiny$\m@th\hookrightarrow$}}} 
        \newcommand*\Wrappedcontinuationindent {3ex } 
        \newcommand*\Wrappedafterbreak {\kern\Wrappedcontinuationindent\copy\Wrappedcontinuationbox} 
        % Take advantage of the already applied Pygments mark-up to insert 
        % potential linebreaks for TeX processing. 
        %        {, <, #, %, $, ' and ": go to next line. 
        %        _, }, ^, &, >, - and ~: stay at end of broken line. 
        % Use of \textquotesingle for straight quote. 
        \newcommand*\Wrappedbreaksatspecials {% 
            \def\PYGZus{\discretionary{\char`\_}{\Wrappedafterbreak}{\char`\_}}% 
            \def\PYGZob{\discretionary{}{\Wrappedafterbreak\char`\{}{\char`\{}}% 
            \def\PYGZcb{\discretionary{\char`\}}{\Wrappedafterbreak}{\char`\}}}% 
            \def\PYGZca{\discretionary{\char`\^}{\Wrappedafterbreak}{\char`\^}}% 
            \def\PYGZam{\discretionary{\char`\&}{\Wrappedafterbreak}{\char`\&}}% 
            \def\PYGZlt{\discretionary{}{\Wrappedafterbreak\char`\<}{\char`\<}}% 
            \def\PYGZgt{\discretionary{\char`\>}{\Wrappedafterbreak}{\char`\>}}% 
            \def\PYGZsh{\discretionary{}{\Wrappedafterbreak\char`\#}{\char`\#}}% 
            \def\PYGZpc{\discretionary{}{\Wrappedafterbreak\char`\%}{\char`\%}}% 
            \def\PYGZdl{\discretionary{}{\Wrappedafterbreak\char`\$}{\char`\$}}% 
            \def\PYGZhy{\discretionary{\char`\-}{\Wrappedafterbreak}{\char`\-}}% 
            \def\PYGZsq{\discretionary{}{\Wrappedafterbreak\textquotesingle}{\textquotesingle}}% 
            \def\PYGZdq{\discretionary{}{\Wrappedafterbreak\char`\"}{\char`\"}}% 
            \def\PYGZti{\discretionary{\char`\~}{\Wrappedafterbreak}{\char`\~}}% 
        } 
        % Some characters . , ; ? ! / are not pygmentized. 
        % This macro makes them "active" and they will insert potential linebreaks 
        \newcommand*\Wrappedbreaksatpunct {% 
            \lccode`\~`\.\lowercase{\def~}{\discretionary{\hbox{\char`\.}}{\Wrappedafterbreak}{\hbox{\char`\.}}}% 
            \lccode`\~`\,\lowercase{\def~}{\discretionary{\hbox{\char`\,}}{\Wrappedafterbreak}{\hbox{\char`\,}}}% 
            \lccode`\~`\;\lowercase{\def~}{\discretionary{\hbox{\char`\;}}{\Wrappedafterbreak}{\hbox{\char`\;}}}% 
            \lccode`\~`\:\lowercase{\def~}{\discretionary{\hbox{\char`\:}}{\Wrappedafterbreak}{\hbox{\char`\:}}}% 
            \lccode`\~`\?\lowercase{\def~}{\discretionary{\hbox{\char`\?}}{\Wrappedafterbreak}{\hbox{\char`\?}}}% 
            \lccode`\~`\!\lowercase{\def~}{\discretionary{\hbox{\char`\!}}{\Wrappedafterbreak}{\hbox{\char`\!}}}% 
            \lccode`\~`\/\lowercase{\def~}{\discretionary{\hbox{\char`\/}}{\Wrappedafterbreak}{\hbox{\char`\/}}}% 
            \catcode`\.\active
            \catcode`\,\active 
            \catcode`\;\active
            \catcode`\:\active
            \catcode`\?\active
            \catcode`\!\active
            \catcode`\/\active 
            \lccode`\~`\~ 	
        }
    \makeatother

    \let\OriginalVerbatim=\Verbatim
    \makeatletter
    \renewcommand{\Verbatim}[1][1]{%
        %\parskip\z@skip
        \sbox\Wrappedcontinuationbox {\Wrappedcontinuationsymbol}%
        \sbox\Wrappedvisiblespacebox {\FV@SetupFont\Wrappedvisiblespace}%
        \def\FancyVerbFormatLine ##1{\hsize\linewidth
            \vtop{\raggedright\hyphenpenalty\z@\exhyphenpenalty\z@
                \doublehyphendemerits\z@\finalhyphendemerits\z@
                \strut ##1\strut}%
        }%
        % If the linebreak is at a space, the latter will be displayed as visible
        % space at end of first line, and a continuation symbol starts next line.
        % Stretch/shrink are however usually zero for typewriter font.
        \def\FV@Space {%
            \nobreak\hskip\z@ plus\fontdimen3\font minus\fontdimen4\font
            \discretionary{\copy\Wrappedvisiblespacebox}{\Wrappedafterbreak}
            {\kern\fontdimen2\font}%
        }%
        
        % Allow breaks at special characters using \PYG... macros.
        \Wrappedbreaksatspecials
        % Breaks at punctuation characters . , ; ? ! and / need catcode=\active 	
        \OriginalVerbatim[#1,codes*=\Wrappedbreaksatpunct]%
    }
    \makeatother

    % Exact colors from NB
    \definecolor{incolor}{HTML}{303F9F}
    \definecolor{outcolor}{HTML}{D84315}
    \definecolor{cellborder}{HTML}{CFCFCF}
    \definecolor{cellbackground}{HTML}{F7F7F7}
    
    % prompt
    \makeatletter
    \newcommand{\boxspacing}{\kern\kvtcb@left@rule\kern\kvtcb@boxsep}
    \makeatother
    \newcommand{\prompt}[4]{
        {\ttfamily\llap{{\color{#2}[#3]:\hspace{3pt}#4}}\vspace{-\baselineskip}}
    }
    

    
    % Prevent overflowing lines due to hard-to-break entities
    \sloppy 
    % Setup hyperref package
    \hypersetup{
      breaklinks=true,  % so long urls are correctly broken across lines
      colorlinks=true,
      urlcolor=urlcolor,
      linkcolor=linkcolor,
      citecolor=citecolor,
      }
    % Slightly bigger margins than the latex defaults
    
    \geometry{verbose,tmargin=1in,bmargin=1in,lmargin=1in,rmargin=1in}
    
    

\begin{document}
    
    \maketitle
    
    

    
    \hypertarget{ejercicio-1-5-puntos}{%
\section{Ejercicio 1 (5 puntos)}\label{ejercicio-1-5-puntos}}

Programar el método de Newton con tamaño de paso fijo \(\alpha=1\).

La función recibe como parámetros la función que calcula el gradiente
\(g(x)\) de la función objetivo
\(f: \mathbb{R}^n \rightarrow \mathbb{R}\), la función que calcula la
Hessiana \(H(x)\) de \(f\), un punto inicial \(x_0\), un número máximo
de iteraciones \(N\), y la tolerancia \(\tau>0\). Fijar \(k=0\) y
repetir los siguientes pasos:

\begin{enumerate}
\def\labelenumi{\arabic{enumi}.}
\tightlist
\item
  Calcular el gradiente \(g_k\) en el punto \(x_k\), \(g_k = g(x_k)\).
\item
  Si \(\|g_k\| < \tau\), hacer \(res=1\) y terminar.
\item
  Si no se cumple el criterio, calcular la Hessiana \(H_k = H(x_k)\).
\item
  Intentar calcular la factorización de Cholesky de \(H_k\).
\item
  Si la factorización no se puede realizar, imprimir el mensaje de
  error, hacer \(res=0\) y terminar el ciclo.
\item
  Si se obtuvo la factorización, resolver el sistema de ecuaciones
  \(H_k p_k = -g_k\) (esto da la dirección de descenso como \(p_k\)).
\item
  Calcular el siguiente punto de la secuencia como
  \[x_{k+1} = x_k + p_k \]
\item
  Si \(k+1\geq N\), hacer \(res=0\) y terminar.
\item
  Si no, hacer \(k = k+1\) y volver el paso 1.
\item
  Devolver el punto \(x_k\), \(g_k\), \(k\) y \(res\).
\end{enumerate}

\begin{center}\rule{0.5\linewidth}{0.5pt}\end{center}

\textbf{Nota:} Para calcular la factorización de Cholesky y resolver el
sistema de ecuaciones puede usar las funciones
\texttt{scipy.linalg.cho\_factor} y \texttt{scipy.linalg.cho\_solve}. Si
la matriz no es definida positiva, la función \texttt{cho\_factor} lanza
la excepción \texttt{scipy.linalg.LinAlgError}. Puede usar esto para
terminar el ciclo.

\begin{center}\rule{0.5\linewidth}{0.5pt}\end{center}

\begin{enumerate}
\def\labelenumi{\arabic{enumi}.}
\tightlist
\item
  Programe la función que implementa el algoritmo del método de Newton,
  almacenando en una lista los puntos \(x_0, x_1, ..., x_k\) que genera
  el algoritmo. Haga que la función devuelva esta lista.
\item
  Use la función de Rosenbrock, su gradiente y Hessiana para probar el
  algoritmo.
\end{enumerate}

\begin{itemize}
\tightlist
\item
  Use \(N=1000\), la tolerancia \(\tau = \sqrt{\epsilon_m}\), donde
  \(\epsilon_m\) es el épsilon de la máquina, y el punto inicial
  \(x_0= (-1.2, 1)\).
\item
  Si el algoritmo converge, imprima un mensaje que indique esto y genere
  una gráfica que muestre las curvas de nivel de la función \(f\) y la
  trayectoria de los puntos \(x_0, x_1, ..., x_k\). Para generar esta
  gráfica use una discretización de los intervalos \([-1.5, 1.5]\) en la
  dirección \(X\) y \([-1, 2]\) en la dirección \(Y\).
\item
  Imprima el punto final \(x_k\), \(f(x_k)\), la magnitud del gradiente
  \(g_k\) y el número de iteraciones \(k\) realizadas.
\item
  Repita la prueba partiendo del punto inicial \(x_0= (-12, 10)\).
\end{itemize}

\hypertarget{soluciuxf3n}{%
\subsection{Solución:}\label{soluciuxf3n}}

Importaremos el módulo \texttt{lib\_t6}, donde se encuentran
implementadas las función de Rosenbrock, su gradiente y hessiana. Además
el método de Newton con paso fijo (\(\alpha=1\)) está desarrollado en la
función \texttt{newton\_fix\_step}.

A continuación probaremos el método de Newton con paso fijo para la
función de Rosenbrock con un número máximo de iteraciones \(N=1000\) y
una tolerancia para la norma del gradiente en cada iteración de
\(\tau=\sqrt{\epsilon_m}\).

Como lo hemos hecho en tareas anteriores, primero probaremos el método
con la condición inicial \(x_0=(-1.2,1)\) luego con la condición inicial
\((-12,10)\), teniendo en mente que el óptimo de la función de
Rosenbrock se encuentra en \(\mathbf{x}_{\ast}=(1,1)\) con un valor
mínimo para la función de \(f_R(\mathbf{x}_{\ast})=0\).

\hypertarget{condiciuxf3n-inicial-x_0-1.21}{%
\subsubsection{\texorpdfstring{Condición inicial
\(x_0=(-1.2,1)\)}{Condición inicial x\_0=(-1.2,1)}}\label{condiciuxf3n-inicial-x_0-1.21}}

En primer lugar, mostramos el desempeño del método de Newton con paso
fijo con la condición inicial \(x_0=(-1.2,1)\).

    \begin{tcolorbox}[breakable, size=fbox, boxrule=1pt, pad at break*=1mm,colback=cellbackground, colframe=cellborder]
\prompt{In}{incolor}{10}{\boxspacing}
\begin{Verbatim}[commandchars=\\\{\}]
\PY{k+kn}{import} \PY{n+nn}{importlib}
\PY{k+kn}{import} \PY{n+nn}{lib\PYZus{}t6}
\PY{n}{importlib}\PY{o}{.}\PY{n}{reload}\PY{p}{(}\PY{n}{lib\PYZus{}t6}\PY{p}{)}
\PY{k+kn}{from} \PY{n+nn}{lib\PYZus{}t6} \PY{k+kn}{import} \PY{o}{*}

\PY{c+c1}{\PYZsh{} Iteraciones máximas y tolerancia}
\PY{n}{N}\PY{o}{=}\PY{l+m+mi}{1000}
\PY{n}{tol}\PY{o}{=}\PY{n}{np}\PY{o}{.}\PY{n}{finfo}\PY{p}{(}\PY{n+nb}{float}\PY{p}{)}\PY{o}{.}\PY{n}{eps}\PY{o}{*}\PY{o}{*}\PY{p}{(}\PY{l+m+mi}{1}\PY{o}{/}\PY{l+m+mi}{2}\PY{p}{)}

\PY{c+c1}{\PYZsh{} Primer condición inicial}
\PY{n}{x0}\PY{o}{=}\PY{n}{np}\PY{o}{.}\PY{n}{array}\PY{p}{(}\PY{p}{[}\PY{o}{\PYZhy{}}\PY{l+m+mf}{1.2}\PY{p}{,}\PY{l+m+mf}{1.0}\PY{p}{]}\PY{p}{)}\PY{o}{.}\PY{n}{reshape}\PY{p}{(}\PY{o}{\PYZhy{}}\PY{l+m+mi}{1}\PY{p}{,}\PY{l+m+mi}{1}\PY{p}{)}
\PY{n}{proof\PYZus{}newton\PYZus{}fix\PYZus{}step\PYZus{}Rosenbrock}\PY{p}{(}\PY{n}{f\PYZus{}Rosenbrock}\PY{p}{,}
\PY{n}{grad\PYZus{}Rosenbrock}\PY{p}{,}
\PY{n}{hess\PYZus{}Rosenbrock}\PY{p}{,}
\PY{n}{x0}\PY{p}{,}\PY{n}{N}\PY{p}{,}\PY{n}{tol}\PY{p}{,}\PY{n}{f\PYZus{}Rosenbrock\PYZus{}graph}\PY{p}{)}
\end{Verbatim}
\end{tcolorbox}

    \begin{Verbatim}[commandchars=\\\{\}]
El método de Newton con paso fijo CONVERGE
El punto final es:
xk =  [ 1.0000  1.0000]
fk =  3.4326461875363225e-20
||gk|| =  8.285705791275366e-09
k =  6
    \end{Verbatim}

    \begin{center}
    \adjustimage{max size={0.9\linewidth}{0.9\paperheight}}{Tarea_6_Optimizacion_files/Tarea_6_Optimizacion_1_1.png}
    \end{center}
    { \hspace*{\fill} \\}
    
    La convergencia desde el punto inicial \(x_0=(-1.2,1)\) se obtuvo en 6
iteraciones.

Es preciso mencionar que no se dibujó la trayectoria completa, dado que
se da la convergencia solo se dibujo la trayectoria de los puntos
\(\mathbf{x}\) tales ques
\[\lVert \mathbf{x}-\mathbf{x}_\ast \rVert \leq \lVert (-1.5,-1)-\mathbf{x}_\ast \rVert,\]

con el fin de solo capturar los contornos de nivel comprendidos entre
\([-1.5,1.5]\times [-1.0,2.0]\), que son los intervalos discretizados
solicitados.

A continuación imprimimos la trayectoria completa del método junto con
su evaluación en la función de Rosenbrock y la norma del gradiente en
esos puntos.

    \begin{tcolorbox}[breakable, size=fbox, boxrule=1pt, pad at break*=1mm,colback=cellbackground, colframe=cellborder]
\prompt{In}{incolor}{12}{\boxspacing}
\begin{Verbatim}[commandchars=\\\{\}]
\PY{c+c1}{\PYZsh{} Trayectoria explícita}
\PY{n}{dic\PYZus{}results\PYZus{}fix\PYZus{}step\PYZus{}cod\PYZus{}1}\PY{p}{,}\PY{n}{trajectory\PYZus{}fix\PYZus{}step\PYZus{}cod\PYZus{}1}\PY{o}{=}\PY{n}{newton\PYZus{}fix\PYZus{}step}\PY{p}{(}\PY{n}{f\PYZus{}Rosenbrock}\PY{p}{,}
\PY{n}{grad\PYZus{}Rosenbrock}\PY{p}{,}
\PY{n}{hess\PYZus{}Rosenbrock}\PY{p}{,}
\PY{n}{x0}\PY{p}{,}\PY{n}{N}\PY{p}{,}\PY{n}{tol}\PY{p}{)}
\PY{n}{np}\PY{o}{.}\PY{n}{set\PYZus{}printoptions}\PY{p}{(}\PY{n}{precision}\PY{o}{=}\PY{l+m+mi}{4}\PY{p}{,}\PY{n}{formatter}\PY{o}{=}\PY{p}{\PYZob{}}\PY{l+s+s1}{\PYZsq{}}\PY{l+s+s1}{float}\PY{l+s+s1}{\PYZsq{}}\PY{p}{:} \PY{l+s+s1}{\PYZsq{}}\PY{l+s+si}{\PYZob{}: 0.4f\PYZcb{}}\PY{l+s+s1}{\PYZsq{}}\PY{o}{.}\PY{n}{format}\PY{p}{\PYZcb{}}\PY{p}{)}
\PY{n+nb}{print}\PY{p}{(}\PY{l+s+s1}{\PYZsq{}}\PY{l+s+s1}{ k  |      xk        |      fk     |    ||gk||    }\PY{l+s+s1}{\PYZsq{}}\PY{p}{)}
\PY{n}{k}\PY{o}{=}\PY{l+m+mi}{0}
\PY{k}{for} \PY{n}{xk} \PY{o+ow}{in} \PY{n}{trajectory\PYZus{}fix\PYZus{}step\PYZus{}cod\PYZus{}1}\PY{p}{:}
    \PY{n+nb}{print}\PY{p}{(}\PY{l+s+sa}{f}\PY{l+s+s1}{\PYZsq{}}\PY{l+s+s1}{ }\PY{l+s+si}{\PYZob{}}\PY{n}{k}\PY{l+s+si}{\PYZcb{}}\PY{l+s+s1}{  }\PY{l+s+si}{\PYZob{}}\PY{n}{np}\PY{o}{.}\PY{n}{squeeze}\PY{p}{(}\PY{n}{xk}\PY{p}{)}\PY{l+s+si}{\PYZcb{}}\PY{l+s+s1}{    }\PY{l+s+si}{\PYZob{}}\PY{n}{np}\PY{o}{.}\PY{n}{format\PYZus{}float\PYZus{}scientific}\PY{p}{(}\PY{n}{f\PYZus{}Rosenbrock}\PY{p}{(}\PY{n}{xk}\PY{p}{)}\PY{p}{,}\PY{n}{unique}\PY{o}{=}\PY{k+kc}{False}\PY{p}{,}\PY{n}{precision}\PY{o}{=}\PY{l+m+mi}{4}\PY{p}{)}\PY{l+s+si}{\PYZcb{}}\PY{l+s+s1}{    }\PY{l+s+si}{\PYZob{}}\PY{n}{np}\PY{o}{.}\PY{n}{format\PYZus{}float\PYZus{}scientific}\PY{p}{(}\PY{n}{np}\PY{o}{.}\PY{n}{linalg}\PY{o}{.}\PY{n}{norm}\PY{p}{(}\PY{n}{grad\PYZus{}Rosenbrock}\PY{p}{(}\PY{n}{xk}\PY{p}{)}\PY{p}{)}\PY{p}{,}\PY{n}{unique}\PY{o}{=}\PY{k+kc}{False}\PY{p}{,}\PY{n}{precision}\PY{o}{=}\PY{l+m+mi}{4}\PY{p}{)}\PY{l+s+si}{\PYZcb{}}\PY{l+s+s1}{\PYZsq{}}\PY{p}{)}
    \PY{n}{k}\PY{o}{+}\PY{o}{=}\PY{l+m+mi}{1}
\end{Verbatim}
\end{tcolorbox}

    \begin{Verbatim}[commandchars=\\\{\}]
 k  |      xk        |      fk     |    ||gk||
 0  [-1.2000  1.0000]    2.4200e+01    2.3287e+02
 1  [-1.1753  1.3807]    4.7319e+00    4.6394e+00
 2  [ 0.7631 -3.1750]    1.4118e+03    1.3708e+03
 3  [ 0.7634  0.5828]    5.5966e-02    4.7311e-01
 4  [ 1.0000  0.9440]    3.1319e-01    2.5027e+01
 5  [ 1.0000  1.0000]    1.8527e-11    8.6086e-06
 6  [ 1.0000  1.0000]    3.4326e-20    8.2857e-09
    \end{Verbatim}

    Observamos en la trayectoria como en el método de Newton no
necesariamente la función decrece en cada iteración, pues en la tercera
iteración obtuvimos un valor tan grande que ni siquiera queda
comprendido en las curvas de nivel que dibujamos en la gráfica.

\hypertarget{condiciuxf3n-inicial-x_0-1210}{%
\subsubsection{\texorpdfstring{Condición Inicial
\(x_0=(-12,10)\)}{Condición Inicial x\_0=(-12,10)}}\label{condiciuxf3n-inicial-x_0-1210}}

Ahora probaremos con una condición inicial más alejada del óptimo
\(x_\ast=(1,1)\). El resultado es el siguiente

    \begin{tcolorbox}[breakable, size=fbox, boxrule=1pt, pad at break*=1mm,colback=cellbackground, colframe=cellborder]
\prompt{In}{incolor}{13}{\boxspacing}
\begin{Verbatim}[commandchars=\\\{\}]
\PY{c+c1}{\PYZsh{} Segunda condición inicial}
\PY{n}{x0}\PY{o}{=}\PY{n}{np}\PY{o}{.}\PY{n}{array}\PY{p}{(}\PY{p}{[}\PY{o}{\PYZhy{}}\PY{l+m+mf}{12.0}\PY{p}{,}\PY{l+m+mf}{10.0}\PY{p}{]}\PY{p}{)}\PY{o}{.}\PY{n}{reshape}\PY{p}{(}\PY{o}{\PYZhy{}}\PY{l+m+mi}{1}\PY{p}{,}\PY{l+m+mi}{1}\PY{p}{)}
\PY{n}{proof\PYZus{}newton\PYZus{}fix\PYZus{}step\PYZus{}Rosenbrock}\PY{p}{(}\PY{n}{f\PYZus{}Rosenbrock}\PY{p}{,}
\PY{n}{grad\PYZus{}Rosenbrock}\PY{p}{,}
\PY{n}{hess\PYZus{}Rosenbrock}\PY{p}{,}
\PY{n}{x0}\PY{p}{,}\PY{n}{N}\PY{p}{,}\PY{n}{tol}\PY{p}{,}\PY{n}{f\PYZus{}Rosenbrock\PYZus{}graph}\PY{p}{)}
\end{Verbatim}
\end{tcolorbox}

    \begin{Verbatim}[commandchars=\\\{\}]
El método de Newton con paso fijo CONVERGE
El punto final es:
xk =  [ 1.0000  1.0000]
fk =  0.0
||gk|| =  0.0
k =  5
    \end{Verbatim}

    \begin{center}
    \adjustimage{max size={0.9\linewidth}{0.9\paperheight}}{Tarea_6_Optimizacion_files/Tarea_6_Optimizacion_5_1.png}
    \end{center}
    { \hspace*{\fill} \\}
    
    Al igual que hicimos con la condición inicial, imprimimos la trayectoria
completa a pesar de no mostrarla en la gráfica.

    \begin{tcolorbox}[breakable, size=fbox, boxrule=1pt, pad at break*=1mm,colback=cellbackground, colframe=cellborder]
\prompt{In}{incolor}{14}{\boxspacing}
\begin{Verbatim}[commandchars=\\\{\}]
\PY{c+c1}{\PYZsh{} Trayectoria explícita}
\PY{n}{dic\PYZus{}results\PYZus{}fix\PYZus{}step\PYZus{}cod\PYZus{}2}\PY{p}{,}\PY{n}{trajectory\PYZus{}fix\PYZus{}step\PYZus{}cod\PYZus{}2}\PY{o}{=}\PY{n}{newton\PYZus{}fix\PYZus{}step}\PY{p}{(}\PY{n}{f\PYZus{}Rosenbrock}\PY{p}{,}
\PY{n}{grad\PYZus{}Rosenbrock}\PY{p}{,}
\PY{n}{hess\PYZus{}Rosenbrock}\PY{p}{,}
\PY{n}{x0}\PY{p}{,}\PY{n}{N}\PY{p}{,}\PY{n}{tol}\PY{p}{)}
\PY{n}{np}\PY{o}{.}\PY{n}{set\PYZus{}printoptions}\PY{p}{(}\PY{n}{precision}\PY{o}{=}\PY{l+m+mi}{4}\PY{p}{,}\PY{n}{formatter}\PY{o}{=}\PY{p}{\PYZob{}}\PY{l+s+s1}{\PYZsq{}}\PY{l+s+s1}{float}\PY{l+s+s1}{\PYZsq{}}\PY{p}{:} \PY{l+s+s1}{\PYZsq{}}\PY{l+s+si}{\PYZob{}: 0.4f\PYZcb{}}\PY{l+s+s1}{\PYZsq{}}\PY{o}{.}\PY{n}{format}\PY{p}{\PYZcb{}}\PY{p}{)}
\PY{n+nb}{print}\PY{p}{(}\PY{l+s+s1}{\PYZsq{}}\PY{l+s+s1}{ k  |      xk        |      fk     |    ||gk||    }\PY{l+s+s1}{\PYZsq{}}\PY{p}{)}
\PY{n}{k}\PY{o}{=}\PY{l+m+mi}{0}
\PY{k}{for} \PY{n}{xk} \PY{o+ow}{in} \PY{n}{trajectory\PYZus{}fix\PYZus{}step\PYZus{}cod\PYZus{}2}\PY{p}{:}
    \PY{n+nb}{print}\PY{p}{(}\PY{l+s+sa}{f}\PY{l+s+s1}{\PYZsq{}}\PY{l+s+s1}{ }\PY{l+s+si}{\PYZob{}}\PY{n}{k}\PY{l+s+si}{\PYZcb{}}\PY{l+s+s1}{  }\PY{l+s+si}{\PYZob{}}\PY{n}{np}\PY{o}{.}\PY{n}{squeeze}\PY{p}{(}\PY{n}{xk}\PY{p}{)}\PY{l+s+si}{\PYZcb{}}\PY{l+s+s1}{    }\PY{l+s+si}{\PYZob{}}\PY{n}{np}\PY{o}{.}\PY{n}{format\PYZus{}float\PYZus{}scientific}\PY{p}{(}\PY{n}{f\PYZus{}Rosenbrock}\PY{p}{(}\PY{n}{xk}\PY{p}{)}\PY{p}{,}\PY{n}{unique}\PY{o}{=}\PY{k+kc}{False}\PY{p}{,}\PY{n}{precision}\PY{o}{=}\PY{l+m+mi}{4}\PY{p}{)}\PY{l+s+si}{\PYZcb{}}\PY{l+s+s1}{    }\PY{l+s+si}{\PYZob{}}\PY{n}{np}\PY{o}{.}\PY{n}{format\PYZus{}float\PYZus{}scientific}\PY{p}{(}\PY{n}{np}\PY{o}{.}\PY{n}{linalg}\PY{o}{.}\PY{n}{norm}\PY{p}{(}\PY{n}{grad\PYZus{}Rosenbrock}\PY{p}{(}\PY{n}{xk}\PY{p}{)}\PY{p}{)}\PY{p}{,}\PY{n}{unique}\PY{o}{=}\PY{k+kc}{False}\PY{p}{,}\PY{n}{precision}\PY{o}{=}\PY{l+m+mi}{4}\PY{p}{)}\PY{l+s+si}{\PYZcb{}}\PY{l+s+s1}{\PYZsq{}}\PY{p}{)}
    \PY{n}{k}\PY{o}{+}\PY{o}{=}\PY{l+m+mi}{1}
\end{Verbatim}
\end{tcolorbox}

    \begin{Verbatim}[commandchars=\\\{\}]
 k  |      xk        |      fk     |    ||gk||
 0  [-12.0000  10.0000]    1.7958e+06    6.4378e+05
 1  [-11.9995  143.9884]    1.6899e+02    2.6000e+01
 2  [ 0.9994 -167.9727]    2.8551e+06    7.5529e+04
 3  [ 0.9994  0.9988]    3.7413e-07    1.2233e-03
 4  [ 1.0000  1.0000]    1.3997e-11    1.6731e-04
 5  [ 1.0000  1.0000]    0.0000e+00    0.0000e+00
    \end{Verbatim}

    Observando la trayectoria anterior y la gráfica podemos ver que solo se
muestran en las gráfica los último 3 puntos (redondendeamos a 4
decimales al momento de imprimir \(x_k\)) además de ver otra vez que la
trayectoria en el método de Newton no necesariamente es monótona
decreciente como en el método de descenso máximo, incluso los primeros 3
valores de la función son valores considerablemente alejados del valor
óptimo de la función de Rosenbrock que es \(f_R(x_\ast)=0\)

    \hypertarget{ejercicio-2-5-puntos}{%
\section{Ejercicio 2 (5 puntos)}\label{ejercicio-2-5-puntos}}

Programar el método de Newton con tamaño de paso ajustado por el
algoritmo de backtracking.

\begin{enumerate}
\def\labelenumi{\arabic{enumi}.}
\tightlist
\item
  Modifique la función del Ejercicio 1 que implementa el algoritmo del
  método de Newton para incluir como parámetros a la función objetivo
  \(f(x)\) y los parámetros \(\rho\) y \(c_1\) del algoritmo de
  backtraking.
\end{enumerate}

\begin{itemize}
\tightlist
\item
  Después de obtener la dirección de descenso \(p_k\), calcular el
  tamaño de paso \(\alpha_k\) usando como valor inicial
  \(\bar{\alpha}_0 = 1\) en el algoritmo de backtracking.
\item
  Hacer \[x_{k+1} = x_k + \alpha_k p_k. \]
\end{itemize}

\begin{enumerate}
\def\labelenumi{\arabic{enumi}.}
\setcounter{enumi}{1}
\tightlist
\item
  Repita la prueba del algoritmo, como se indicó en el Ejercicio 1, a la
  función de Rosenbrock, para ver como cambia la trayectoria de los
  puntos \(x_0, x_1, ..., x_k\) en comparación con el resultado
  anterior, partiendo de \(x_0= (-1.2, 1)\) y de \(x_0= (-12, 10)\).
\end{enumerate}

\hypertarget{soluciuxf3n}{%
\subsection{Solución}\label{soluciuxf3n}}

Implementamos el método de Newton modificando el tamaño de paso con el
algoritmo de backtracking con \(\alpha_{0}=1\), \(c=0.0001\) y
\(\rho=0.4\) en cada iteración. Al igual que en el Ejercicio 1
consideramos un número de máximo de iteraciones \(N=1000\) y una
tolerancia \(\tau=\sqrt{\epsilon_m}\).

\hypertarget{condiciuxf3n-inicial-x_0-1.21}{%
\subsubsection{\texorpdfstring{Condición inicial
\(x_0=(-1.2,1)\)}{Condición inicial x\_0=(-1.2,1)}}\label{condiciuxf3n-inicial-x_0-1.21}}

En la siguiente celda ejecutamos el método de Newton con backtracking
parar optimizar la función de Rosenbrock con la condición inicial
\(x_0=(-1.2,1)\)

    \begin{tcolorbox}[breakable, size=fbox, boxrule=1pt, pad at break*=1mm,colback=cellbackground, colframe=cellborder]
\prompt{In}{incolor}{15}{\boxspacing}
\begin{Verbatim}[commandchars=\\\{\}]
\PY{n}{a0}\PY{p}{,}\PY{n}{c}\PY{p}{,}\PY{n}{rho}\PY{o}{=}\PY{l+m+mf}{1.0}\PY{p}{,}\PY{l+m+mf}{1e\PYZhy{}4}\PY{p}{,}\PY{l+m+mf}{0.4}
\PY{n}{importlib}\PY{o}{.}\PY{n}{reload}\PY{p}{(}\PY{n}{lib\PYZus{}t6}\PY{p}{)}
\PY{k+kn}{from} \PY{n+nn}{lib\PYZus{}t6} \PY{k+kn}{import} \PY{o}{*}

\PY{c+c1}{\PYZsh{} Primer condición inicial}
\PY{n}{x0}\PY{o}{=}\PY{n}{np}\PY{o}{.}\PY{n}{array}\PY{p}{(}\PY{p}{[}\PY{o}{\PYZhy{}}\PY{l+m+mf}{1.2}\PY{p}{,}\PY{l+m+mf}{1.0}\PY{p}{]}\PY{p}{)}\PY{o}{.}\PY{n}{reshape}\PY{p}{(}\PY{o}{\PYZhy{}}\PY{l+m+mi}{1}\PY{p}{,}\PY{l+m+mi}{1}\PY{p}{)}
\PY{n}{proof\PYZus{}newton\PYZus{}backtracking\PYZus{}Rosenbrock}\PY{p}{(}\PY{n}{f\PYZus{}Rosenbrock}\PY{p}{,}
\PY{n}{grad\PYZus{}Rosenbrock}\PY{p}{,}
\PY{n}{hess\PYZus{}Rosenbrock}\PY{p}{,}
\PY{n}{x0}\PY{p}{,}\PY{n}{N}\PY{p}{,}\PY{n}{tol}\PY{p}{,}\PY{n}{a0}\PY{p}{,}\PY{n}{rho}\PY{p}{,}\PY{n}{c}\PY{p}{,}
\PY{n}{f\PYZus{}Rosenbrock\PYZus{}graph}\PY{p}{)}
\end{Verbatim}
\end{tcolorbox}

    \begin{Verbatim}[commandchars=\\\{\}]
El método de Newton con paso fijo CONVERGE
El punto final es:
xk =  [ 1.0000  1.0000]
fk =  2.4528510178244796e-21
||gk|| =  2.1606802683671457e-09
k =  21
    \end{Verbatim}

    \begin{center}
    \adjustimage{max size={0.9\linewidth}{0.9\paperheight}}{Tarea_6_Optimizacion_files/Tarea_6_Optimizacion_10_1.png}
    \end{center}
    { \hspace*{\fill} \\}
    
    La convergencia se alcanzó en 21 iteraciones y es más lenta que con paso
fijo pues acortamos el tamaño de paso. Al utilizar backtracking como
vimos en clase, la sucesión \(\{f(x_{i})\}_{i=1}^{k}\) es decreciente,
por lo que en la gráfica anterior vemos la trayectoria completa pues el
valor de la función objetivo va decreciendo y
\(x_0\in [-1.5,1.5]\times [-1,2]\), a diferencia de lo obsevado en el
Ejercicio 1.

\hypertarget{condiciuxf3n-inicial-x_0-1210}{%
\subsubsection{\texorpdfstring{Condición inicial
\(x_0=(-12,10)\)}{Condición inicial x\_0=(-12,10)}}\label{condiciuxf3n-inicial-x_0-1210}}

Ahora realizamos la prueba con la condición inicial más alejada del
óptimo \(\mathbf{x}_\ast=(1,1)\). El resultado del método fue el
siguiente.

    \begin{tcolorbox}[breakable, size=fbox, boxrule=1pt, pad at break*=1mm,colback=cellbackground, colframe=cellborder]
\prompt{In}{incolor}{16}{\boxspacing}
\begin{Verbatim}[commandchars=\\\{\}]
\PY{c+c1}{\PYZsh{} Segunda condición inicial}
\PY{n}{x0}\PY{o}{=}\PY{n}{np}\PY{o}{.}\PY{n}{array}\PY{p}{(}\PY{p}{[}\PY{o}{\PYZhy{}}\PY{l+m+mf}{12.0}\PY{p}{,}\PY{l+m+mf}{10.0}\PY{p}{]}\PY{p}{)}\PY{o}{.}\PY{n}{reshape}\PY{p}{(}\PY{o}{\PYZhy{}}\PY{l+m+mi}{1}\PY{p}{,}\PY{l+m+mi}{1}\PY{p}{)}
\PY{n}{proof\PYZus{}newton\PYZus{}backtracking\PYZus{}Rosenbrock}\PY{p}{(}\PY{n}{f\PYZus{}Rosenbrock}\PY{p}{,}
\PY{n}{grad\PYZus{}Rosenbrock}\PY{p}{,}
\PY{n}{hess\PYZus{}Rosenbrock}\PY{p}{,}
\PY{n}{x0}\PY{p}{,}\PY{n}{N}\PY{p}{,}\PY{n}{tol}\PY{p}{,}\PY{n}{a0}\PY{p}{,}\PY{n}{rho}\PY{p}{,}\PY{n}{c}\PY{p}{,}
\PY{n}{f\PYZus{}Rosenbrock\PYZus{}graph}\PY{p}{)}
\end{Verbatim}
\end{tcolorbox}

    \begin{Verbatim}[commandchars=\\\{\}]
El método de Newton con paso fijo CONVERGE
El punto final es:
xk =  [ 1.0000  1.0000]
fk =  1.449591655998633e-23
||gk|| =  9.049471772251831e-11
k =  60
    \end{Verbatim}

    \begin{center}
    \adjustimage{max size={0.9\linewidth}{0.9\paperheight}}{Tarea_6_Optimizacion_files/Tarea_6_Optimizacion_12_1.png}
    \end{center}
    { \hspace*{\fill} \\}
    
    Como solo mostramos las contornos de nivel de la función de Rosenbrock
en la región \([-1.5,1.5]\times [-1,2]\) y la condición inicial en este
caso es \(x_0=(-12,10)\) no podemos mostrar toda la trayectoria, solo la
comprendida en dicha región.

La convergencia se alcanzó en 60 iteraciones, era de esperar que se iban
a necesitar más iteraciones en este caso pues aquí la condición inicial
\((-12,10)\) está más alejada del óptimo que el punto \((-1.2,1)\) y
acortamos el tamaño de paso en cada iteración.

Concluimos que la segunda condición inicial es un poco mejor que la
primera considerando el paso fijo \(\alpha=1\), pero al acortar el
tamaño de paso, para obtener una sucesión monótona y evitar una mala
aproximación cuadrática, claramente la primer condición inicial tiene
mejor desempeño que la segunda por estar más cerca del óptimo
\(\mathbf{x}_\ast=(1,1)\).


    % Add a bibliography block to the postdoc
    
    
    
\end{document}
