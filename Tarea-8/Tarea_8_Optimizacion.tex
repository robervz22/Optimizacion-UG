\documentclass[11pt]{article}

    \usepackage[breakable]{tcolorbox}
    \usepackage{parskip} % Stop auto-indenting (to mimic markdown behaviour)
    
    \usepackage{iftex}
    \ifPDFTeX
    	\usepackage[T1]{fontenc}
    	\usepackage{mathpazo}
    \else
    	\usepackage{fontspec}
    \fi

    % Basic figure setup, for now with no caption control since it's done
    % automatically by Pandoc (which extracts ![](path) syntax from Markdown).
    \usepackage{graphicx}
    % Maintain compatibility with old templates. Remove in nbconvert 6.0
    \let\Oldincludegraphics\includegraphics
    % Ensure that by default, figures have no caption (until we provide a
    % proper Figure object with a Caption API and a way to capture that
    % in the conversion process - todo).
    \usepackage{caption}
    \DeclareCaptionFormat{nocaption}{}
    \captionsetup{format=nocaption,aboveskip=0pt,belowskip=0pt}

    \usepackage[Export]{adjustbox} % Used to constrain images to a maximum size
    \adjustboxset{max size={0.9\linewidth}{0.9\paperheight}}
    \usepackage{float}
    \floatplacement{figure}{H} % forces figures to be placed at the correct location
    \usepackage{xcolor} % Allow colors to be defined
    \usepackage{enumerate} % Needed for markdown enumerations to work
    \usepackage{geometry} % Used to adjust the document margins
    \usepackage{amsmath} % Equations
    \usepackage{amssymb} % Equations
    \usepackage{textcomp} % defines textquotesingle
    % Hack from http://tex.stackexchange.com/a/47451/13684:
    \AtBeginDocument{%
        \def\PYZsq{\textquotesingle}% Upright quotes in Pygmentized code
    }
    \usepackage{upquote} % Upright quotes for verbatim code
    \usepackage{eurosym} % defines \euro
    \usepackage[mathletters]{ucs} % Extended unicode (utf-8) support
    \usepackage{fancyvrb} % verbatim replacement that allows latex
    \usepackage{grffile} % extends the file name processing of package graphics 
                         % to support a larger range
    \makeatletter % fix for grffile with XeLaTeX
    \def\Gread@@xetex#1{%
      \IfFileExists{"\Gin@base".bb}%
      {\Gread@eps{\Gin@base.bb}}%
      {\Gread@@xetex@aux#1}%
    }
    \makeatother

    % The hyperref package gives us a pdf with properly built
    % internal navigation ('pdf bookmarks' for the table of contents,
    % internal cross-reference links, web links for URLs, etc.)
    \usepackage{hyperref}
    % The default LaTeX title has an obnoxious amount of whitespace. By default,
    % titling removes some of it. It also provides customization options.
    \usepackage{titling}
    \usepackage{longtable} % longtable support required by pandoc >1.10
    \usepackage{booktabs}  % table support for pandoc > 1.12.2
    \usepackage[inline]{enumitem} % IRkernel/repr support (it uses the enumerate* environment)
    \usepackage[normalem]{ulem} % ulem is needed to support strikethroughs (\sout)
                                % normalem makes italics be italics, not underlines
    \usepackage{mathrsfs}
    

    
    % Colors for the hyperref package
    \definecolor{urlcolor}{rgb}{0,.145,.698}
    \definecolor{linkcolor}{rgb}{.71,0.21,0.01}
    \definecolor{citecolor}{rgb}{.12,.54,.11}

    % ANSI colors
    \definecolor{ansi-black}{HTML}{3E424D}
    \definecolor{ansi-black-intense}{HTML}{282C36}
    \definecolor{ansi-red}{HTML}{E75C58}
    \definecolor{ansi-red-intense}{HTML}{B22B31}
    \definecolor{ansi-green}{HTML}{00A250}
    \definecolor{ansi-green-intense}{HTML}{007427}
    \definecolor{ansi-yellow}{HTML}{DDB62B}
    \definecolor{ansi-yellow-intense}{HTML}{B27D12}
    \definecolor{ansi-blue}{HTML}{208FFB}
    \definecolor{ansi-blue-intense}{HTML}{0065CA}
    \definecolor{ansi-magenta}{HTML}{D160C4}
    \definecolor{ansi-magenta-intense}{HTML}{A03196}
    \definecolor{ansi-cyan}{HTML}{60C6C8}
    \definecolor{ansi-cyan-intense}{HTML}{258F8F}
    \definecolor{ansi-white}{HTML}{C5C1B4}
    \definecolor{ansi-white-intense}{HTML}{A1A6B2}
    \definecolor{ansi-default-inverse-fg}{HTML}{FFFFFF}
    \definecolor{ansi-default-inverse-bg}{HTML}{000000}

    % commands and environments needed by pandoc snippets
    % extracted from the output of `pandoc -s`
    \providecommand{\tightlist}{%
      \setlength{\itemsep}{0pt}\setlength{\parskip}{0pt}}
    \DefineVerbatimEnvironment{Highlighting}{Verbatim}{commandchars=\\\{\}}
    % Add ',fontsize=\small' for more characters per line
    \newenvironment{Shaded}{}{}
    \newcommand{\KeywordTok}[1]{\textcolor[rgb]{0.00,0.44,0.13}{\textbf{{#1}}}}
    \newcommand{\DataTypeTok}[1]{\textcolor[rgb]{0.56,0.13,0.00}{{#1}}}
    \newcommand{\DecValTok}[1]{\textcolor[rgb]{0.25,0.63,0.44}{{#1}}}
    \newcommand{\BaseNTok}[1]{\textcolor[rgb]{0.25,0.63,0.44}{{#1}}}
    \newcommand{\FloatTok}[1]{\textcolor[rgb]{0.25,0.63,0.44}{{#1}}}
    \newcommand{\CharTok}[1]{\textcolor[rgb]{0.25,0.44,0.63}{{#1}}}
    \newcommand{\StringTok}[1]{\textcolor[rgb]{0.25,0.44,0.63}{{#1}}}
    \newcommand{\CommentTok}[1]{\textcolor[rgb]{0.38,0.63,0.69}{\textit{{#1}}}}
    \newcommand{\OtherTok}[1]{\textcolor[rgb]{0.00,0.44,0.13}{{#1}}}
    \newcommand{\AlertTok}[1]{\textcolor[rgb]{1.00,0.00,0.00}{\textbf{{#1}}}}
    \newcommand{\FunctionTok}[1]{\textcolor[rgb]{0.02,0.16,0.49}{{#1}}}
    \newcommand{\RegionMarkerTok}[1]{{#1}}
    \newcommand{\ErrorTok}[1]{\textcolor[rgb]{1.00,0.00,0.00}{\textbf{{#1}}}}
    \newcommand{\NormalTok}[1]{{#1}}
    
    % Additional commands for more recent versions of Pandoc
    \newcommand{\ConstantTok}[1]{\textcolor[rgb]{0.53,0.00,0.00}{{#1}}}
    \newcommand{\SpecialCharTok}[1]{\textcolor[rgb]{0.25,0.44,0.63}{{#1}}}
    \newcommand{\VerbatimStringTok}[1]{\textcolor[rgb]{0.25,0.44,0.63}{{#1}}}
    \newcommand{\SpecialStringTok}[1]{\textcolor[rgb]{0.73,0.40,0.53}{{#1}}}
    \newcommand{\ImportTok}[1]{{#1}}
    \newcommand{\DocumentationTok}[1]{\textcolor[rgb]{0.73,0.13,0.13}{\textit{{#1}}}}
    \newcommand{\AnnotationTok}[1]{\textcolor[rgb]{0.38,0.63,0.69}{\textbf{\textit{{#1}}}}}
    \newcommand{\CommentVarTok}[1]{\textcolor[rgb]{0.38,0.63,0.69}{\textbf{\textit{{#1}}}}}
    \newcommand{\VariableTok}[1]{\textcolor[rgb]{0.10,0.09,0.49}{{#1}}}
    \newcommand{\ControlFlowTok}[1]{\textcolor[rgb]{0.00,0.44,0.13}{\textbf{{#1}}}}
    \newcommand{\OperatorTok}[1]{\textcolor[rgb]{0.40,0.40,0.40}{{#1}}}
    \newcommand{\BuiltInTok}[1]{{#1}}
    \newcommand{\ExtensionTok}[1]{{#1}}
    \newcommand{\PreprocessorTok}[1]{\textcolor[rgb]{0.74,0.48,0.00}{{#1}}}
    \newcommand{\AttributeTok}[1]{\textcolor[rgb]{0.49,0.56,0.16}{{#1}}}
    \newcommand{\InformationTok}[1]{\textcolor[rgb]{0.38,0.63,0.69}{\textbf{\textit{{#1}}}}}
    \newcommand{\WarningTok}[1]{\textcolor[rgb]{0.38,0.63,0.69}{\textbf{\textit{{#1}}}}}
    
    
    % Define a nice break command that doesn't care if a line doesn't already
    % exist.
    \def\br{\hspace*{\fill} \\* }
    % Math Jax compatibility definitions
    \def\gt{>}
    \def\lt{<}
    \let\Oldtex\TeX
    \let\Oldlatex\LaTeX
    \renewcommand{\TeX}{\textrm{\Oldtex}}
    \renewcommand{\LaTeX}{\textrm{\Oldlatex}}
    % Document parameters
    % Document title
    \title{Tarea\_8\_Optimizacion}
    
    
    
    
    
% Pygments definitions
\makeatletter
\def\PY@reset{\let\PY@it=\relax \let\PY@bf=\relax%
    \let\PY@ul=\relax \let\PY@tc=\relax%
    \let\PY@bc=\relax \let\PY@ff=\relax}
\def\PY@tok#1{\csname PY@tok@#1\endcsname}
\def\PY@toks#1+{\ifx\relax#1\empty\else%
    \PY@tok{#1}\expandafter\PY@toks\fi}
\def\PY@do#1{\PY@bc{\PY@tc{\PY@ul{%
    \PY@it{\PY@bf{\PY@ff{#1}}}}}}}
\def\PY#1#2{\PY@reset\PY@toks#1+\relax+\PY@do{#2}}

\expandafter\def\csname PY@tok@w\endcsname{\def\PY@tc##1{\textcolor[rgb]{0.73,0.73,0.73}{##1}}}
\expandafter\def\csname PY@tok@c\endcsname{\let\PY@it=\textit\def\PY@tc##1{\textcolor[rgb]{0.25,0.50,0.50}{##1}}}
\expandafter\def\csname PY@tok@cp\endcsname{\def\PY@tc##1{\textcolor[rgb]{0.74,0.48,0.00}{##1}}}
\expandafter\def\csname PY@tok@k\endcsname{\let\PY@bf=\textbf\def\PY@tc##1{\textcolor[rgb]{0.00,0.50,0.00}{##1}}}
\expandafter\def\csname PY@tok@kp\endcsname{\def\PY@tc##1{\textcolor[rgb]{0.00,0.50,0.00}{##1}}}
\expandafter\def\csname PY@tok@kt\endcsname{\def\PY@tc##1{\textcolor[rgb]{0.69,0.00,0.25}{##1}}}
\expandafter\def\csname PY@tok@o\endcsname{\def\PY@tc##1{\textcolor[rgb]{0.40,0.40,0.40}{##1}}}
\expandafter\def\csname PY@tok@ow\endcsname{\let\PY@bf=\textbf\def\PY@tc##1{\textcolor[rgb]{0.67,0.13,1.00}{##1}}}
\expandafter\def\csname PY@tok@nb\endcsname{\def\PY@tc##1{\textcolor[rgb]{0.00,0.50,0.00}{##1}}}
\expandafter\def\csname PY@tok@nf\endcsname{\def\PY@tc##1{\textcolor[rgb]{0.00,0.00,1.00}{##1}}}
\expandafter\def\csname PY@tok@nc\endcsname{\let\PY@bf=\textbf\def\PY@tc##1{\textcolor[rgb]{0.00,0.00,1.00}{##1}}}
\expandafter\def\csname PY@tok@nn\endcsname{\let\PY@bf=\textbf\def\PY@tc##1{\textcolor[rgb]{0.00,0.00,1.00}{##1}}}
\expandafter\def\csname PY@tok@ne\endcsname{\let\PY@bf=\textbf\def\PY@tc##1{\textcolor[rgb]{0.82,0.25,0.23}{##1}}}
\expandafter\def\csname PY@tok@nv\endcsname{\def\PY@tc##1{\textcolor[rgb]{0.10,0.09,0.49}{##1}}}
\expandafter\def\csname PY@tok@no\endcsname{\def\PY@tc##1{\textcolor[rgb]{0.53,0.00,0.00}{##1}}}
\expandafter\def\csname PY@tok@nl\endcsname{\def\PY@tc##1{\textcolor[rgb]{0.63,0.63,0.00}{##1}}}
\expandafter\def\csname PY@tok@ni\endcsname{\let\PY@bf=\textbf\def\PY@tc##1{\textcolor[rgb]{0.60,0.60,0.60}{##1}}}
\expandafter\def\csname PY@tok@na\endcsname{\def\PY@tc##1{\textcolor[rgb]{0.49,0.56,0.16}{##1}}}
\expandafter\def\csname PY@tok@nt\endcsname{\let\PY@bf=\textbf\def\PY@tc##1{\textcolor[rgb]{0.00,0.50,0.00}{##1}}}
\expandafter\def\csname PY@tok@nd\endcsname{\def\PY@tc##1{\textcolor[rgb]{0.67,0.13,1.00}{##1}}}
\expandafter\def\csname PY@tok@s\endcsname{\def\PY@tc##1{\textcolor[rgb]{0.73,0.13,0.13}{##1}}}
\expandafter\def\csname PY@tok@sd\endcsname{\let\PY@it=\textit\def\PY@tc##1{\textcolor[rgb]{0.73,0.13,0.13}{##1}}}
\expandafter\def\csname PY@tok@si\endcsname{\let\PY@bf=\textbf\def\PY@tc##1{\textcolor[rgb]{0.73,0.40,0.53}{##1}}}
\expandafter\def\csname PY@tok@se\endcsname{\let\PY@bf=\textbf\def\PY@tc##1{\textcolor[rgb]{0.73,0.40,0.13}{##1}}}
\expandafter\def\csname PY@tok@sr\endcsname{\def\PY@tc##1{\textcolor[rgb]{0.73,0.40,0.53}{##1}}}
\expandafter\def\csname PY@tok@ss\endcsname{\def\PY@tc##1{\textcolor[rgb]{0.10,0.09,0.49}{##1}}}
\expandafter\def\csname PY@tok@sx\endcsname{\def\PY@tc##1{\textcolor[rgb]{0.00,0.50,0.00}{##1}}}
\expandafter\def\csname PY@tok@m\endcsname{\def\PY@tc##1{\textcolor[rgb]{0.40,0.40,0.40}{##1}}}
\expandafter\def\csname PY@tok@gh\endcsname{\let\PY@bf=\textbf\def\PY@tc##1{\textcolor[rgb]{0.00,0.00,0.50}{##1}}}
\expandafter\def\csname PY@tok@gu\endcsname{\let\PY@bf=\textbf\def\PY@tc##1{\textcolor[rgb]{0.50,0.00,0.50}{##1}}}
\expandafter\def\csname PY@tok@gd\endcsname{\def\PY@tc##1{\textcolor[rgb]{0.63,0.00,0.00}{##1}}}
\expandafter\def\csname PY@tok@gi\endcsname{\def\PY@tc##1{\textcolor[rgb]{0.00,0.63,0.00}{##1}}}
\expandafter\def\csname PY@tok@gr\endcsname{\def\PY@tc##1{\textcolor[rgb]{1.00,0.00,0.00}{##1}}}
\expandafter\def\csname PY@tok@ge\endcsname{\let\PY@it=\textit}
\expandafter\def\csname PY@tok@gs\endcsname{\let\PY@bf=\textbf}
\expandafter\def\csname PY@tok@gp\endcsname{\let\PY@bf=\textbf\def\PY@tc##1{\textcolor[rgb]{0.00,0.00,0.50}{##1}}}
\expandafter\def\csname PY@tok@go\endcsname{\def\PY@tc##1{\textcolor[rgb]{0.53,0.53,0.53}{##1}}}
\expandafter\def\csname PY@tok@gt\endcsname{\def\PY@tc##1{\textcolor[rgb]{0.00,0.27,0.87}{##1}}}
\expandafter\def\csname PY@tok@err\endcsname{\def\PY@bc##1{\setlength{\fboxsep}{0pt}\fcolorbox[rgb]{1.00,0.00,0.00}{1,1,1}{\strut ##1}}}
\expandafter\def\csname PY@tok@kc\endcsname{\let\PY@bf=\textbf\def\PY@tc##1{\textcolor[rgb]{0.00,0.50,0.00}{##1}}}
\expandafter\def\csname PY@tok@kd\endcsname{\let\PY@bf=\textbf\def\PY@tc##1{\textcolor[rgb]{0.00,0.50,0.00}{##1}}}
\expandafter\def\csname PY@tok@kn\endcsname{\let\PY@bf=\textbf\def\PY@tc##1{\textcolor[rgb]{0.00,0.50,0.00}{##1}}}
\expandafter\def\csname PY@tok@kr\endcsname{\let\PY@bf=\textbf\def\PY@tc##1{\textcolor[rgb]{0.00,0.50,0.00}{##1}}}
\expandafter\def\csname PY@tok@bp\endcsname{\def\PY@tc##1{\textcolor[rgb]{0.00,0.50,0.00}{##1}}}
\expandafter\def\csname PY@tok@fm\endcsname{\def\PY@tc##1{\textcolor[rgb]{0.00,0.00,1.00}{##1}}}
\expandafter\def\csname PY@tok@vc\endcsname{\def\PY@tc##1{\textcolor[rgb]{0.10,0.09,0.49}{##1}}}
\expandafter\def\csname PY@tok@vg\endcsname{\def\PY@tc##1{\textcolor[rgb]{0.10,0.09,0.49}{##1}}}
\expandafter\def\csname PY@tok@vi\endcsname{\def\PY@tc##1{\textcolor[rgb]{0.10,0.09,0.49}{##1}}}
\expandafter\def\csname PY@tok@vm\endcsname{\def\PY@tc##1{\textcolor[rgb]{0.10,0.09,0.49}{##1}}}
\expandafter\def\csname PY@tok@sa\endcsname{\def\PY@tc##1{\textcolor[rgb]{0.73,0.13,0.13}{##1}}}
\expandafter\def\csname PY@tok@sb\endcsname{\def\PY@tc##1{\textcolor[rgb]{0.73,0.13,0.13}{##1}}}
\expandafter\def\csname PY@tok@sc\endcsname{\def\PY@tc##1{\textcolor[rgb]{0.73,0.13,0.13}{##1}}}
\expandafter\def\csname PY@tok@dl\endcsname{\def\PY@tc##1{\textcolor[rgb]{0.73,0.13,0.13}{##1}}}
\expandafter\def\csname PY@tok@s2\endcsname{\def\PY@tc##1{\textcolor[rgb]{0.73,0.13,0.13}{##1}}}
\expandafter\def\csname PY@tok@sh\endcsname{\def\PY@tc##1{\textcolor[rgb]{0.73,0.13,0.13}{##1}}}
\expandafter\def\csname PY@tok@s1\endcsname{\def\PY@tc##1{\textcolor[rgb]{0.73,0.13,0.13}{##1}}}
\expandafter\def\csname PY@tok@mb\endcsname{\def\PY@tc##1{\textcolor[rgb]{0.40,0.40,0.40}{##1}}}
\expandafter\def\csname PY@tok@mf\endcsname{\def\PY@tc##1{\textcolor[rgb]{0.40,0.40,0.40}{##1}}}
\expandafter\def\csname PY@tok@mh\endcsname{\def\PY@tc##1{\textcolor[rgb]{0.40,0.40,0.40}{##1}}}
\expandafter\def\csname PY@tok@mi\endcsname{\def\PY@tc##1{\textcolor[rgb]{0.40,0.40,0.40}{##1}}}
\expandafter\def\csname PY@tok@il\endcsname{\def\PY@tc##1{\textcolor[rgb]{0.40,0.40,0.40}{##1}}}
\expandafter\def\csname PY@tok@mo\endcsname{\def\PY@tc##1{\textcolor[rgb]{0.40,0.40,0.40}{##1}}}
\expandafter\def\csname PY@tok@ch\endcsname{\let\PY@it=\textit\def\PY@tc##1{\textcolor[rgb]{0.25,0.50,0.50}{##1}}}
\expandafter\def\csname PY@tok@cm\endcsname{\let\PY@it=\textit\def\PY@tc##1{\textcolor[rgb]{0.25,0.50,0.50}{##1}}}
\expandafter\def\csname PY@tok@cpf\endcsname{\let\PY@it=\textit\def\PY@tc##1{\textcolor[rgb]{0.25,0.50,0.50}{##1}}}
\expandafter\def\csname PY@tok@c1\endcsname{\let\PY@it=\textit\def\PY@tc##1{\textcolor[rgb]{0.25,0.50,0.50}{##1}}}
\expandafter\def\csname PY@tok@cs\endcsname{\let\PY@it=\textit\def\PY@tc##1{\textcolor[rgb]{0.25,0.50,0.50}{##1}}}

\def\PYZbs{\char`\\}
\def\PYZus{\char`\_}
\def\PYZob{\char`\{}
\def\PYZcb{\char`\}}
\def\PYZca{\char`\^}
\def\PYZam{\char`\&}
\def\PYZlt{\char`\<}
\def\PYZgt{\char`\>}
\def\PYZsh{\char`\#}
\def\PYZpc{\char`\%}
\def\PYZdl{\char`\$}
\def\PYZhy{\char`\-}
\def\PYZsq{\char`\'}
\def\PYZdq{\char`\"}
\def\PYZti{\char`\~}
% for compatibility with earlier versions
\def\PYZat{@}
\def\PYZlb{[}
\def\PYZrb{]}
\makeatother


    % For linebreaks inside Verbatim environment from package fancyvrb. 
    \makeatletter
        \newbox\Wrappedcontinuationbox 
        \newbox\Wrappedvisiblespacebox 
        \newcommand*\Wrappedvisiblespace {\textcolor{red}{\textvisiblespace}} 
        \newcommand*\Wrappedcontinuationsymbol {\textcolor{red}{\llap{\tiny$\m@th\hookrightarrow$}}} 
        \newcommand*\Wrappedcontinuationindent {3ex } 
        \newcommand*\Wrappedafterbreak {\kern\Wrappedcontinuationindent\copy\Wrappedcontinuationbox} 
        % Take advantage of the already applied Pygments mark-up to insert 
        % potential linebreaks for TeX processing. 
        %        {, <, #, %, $, ' and ": go to next line. 
        %        _, }, ^, &, >, - and ~: stay at end of broken line. 
        % Use of \textquotesingle for straight quote. 
        \newcommand*\Wrappedbreaksatspecials {% 
            \def\PYGZus{\discretionary{\char`\_}{\Wrappedafterbreak}{\char`\_}}% 
            \def\PYGZob{\discretionary{}{\Wrappedafterbreak\char`\{}{\char`\{}}% 
            \def\PYGZcb{\discretionary{\char`\}}{\Wrappedafterbreak}{\char`\}}}% 
            \def\PYGZca{\discretionary{\char`\^}{\Wrappedafterbreak}{\char`\^}}% 
            \def\PYGZam{\discretionary{\char`\&}{\Wrappedafterbreak}{\char`\&}}% 
            \def\PYGZlt{\discretionary{}{\Wrappedafterbreak\char`\<}{\char`\<}}% 
            \def\PYGZgt{\discretionary{\char`\>}{\Wrappedafterbreak}{\char`\>}}% 
            \def\PYGZsh{\discretionary{}{\Wrappedafterbreak\char`\#}{\char`\#}}% 
            \def\PYGZpc{\discretionary{}{\Wrappedafterbreak\char`\%}{\char`\%}}% 
            \def\PYGZdl{\discretionary{}{\Wrappedafterbreak\char`\$}{\char`\$}}% 
            \def\PYGZhy{\discretionary{\char`\-}{\Wrappedafterbreak}{\char`\-}}% 
            \def\PYGZsq{\discretionary{}{\Wrappedafterbreak\textquotesingle}{\textquotesingle}}% 
            \def\PYGZdq{\discretionary{}{\Wrappedafterbreak\char`\"}{\char`\"}}% 
            \def\PYGZti{\discretionary{\char`\~}{\Wrappedafterbreak}{\char`\~}}% 
        } 
        % Some characters . , ; ? ! / are not pygmentized. 
        % This macro makes them "active" and they will insert potential linebreaks 
        \newcommand*\Wrappedbreaksatpunct {% 
            \lccode`\~`\.\lowercase{\def~}{\discretionary{\hbox{\char`\.}}{\Wrappedafterbreak}{\hbox{\char`\.}}}% 
            \lccode`\~`\,\lowercase{\def~}{\discretionary{\hbox{\char`\,}}{\Wrappedafterbreak}{\hbox{\char`\,}}}% 
            \lccode`\~`\;\lowercase{\def~}{\discretionary{\hbox{\char`\;}}{\Wrappedafterbreak}{\hbox{\char`\;}}}% 
            \lccode`\~`\:\lowercase{\def~}{\discretionary{\hbox{\char`\:}}{\Wrappedafterbreak}{\hbox{\char`\:}}}% 
            \lccode`\~`\?\lowercase{\def~}{\discretionary{\hbox{\char`\?}}{\Wrappedafterbreak}{\hbox{\char`\?}}}% 
            \lccode`\~`\!\lowercase{\def~}{\discretionary{\hbox{\char`\!}}{\Wrappedafterbreak}{\hbox{\char`\!}}}% 
            \lccode`\~`\/\lowercase{\def~}{\discretionary{\hbox{\char`\/}}{\Wrappedafterbreak}{\hbox{\char`\/}}}% 
            \catcode`\.\active
            \catcode`\,\active 
            \catcode`\;\active
            \catcode`\:\active
            \catcode`\?\active
            \catcode`\!\active
            \catcode`\/\active 
            \lccode`\~`\~ 	
        }
    \makeatother

    \let\OriginalVerbatim=\Verbatim
    \makeatletter
    \renewcommand{\Verbatim}[1][1]{%
        %\parskip\z@skip
        \sbox\Wrappedcontinuationbox {\Wrappedcontinuationsymbol}%
        \sbox\Wrappedvisiblespacebox {\FV@SetupFont\Wrappedvisiblespace}%
        \def\FancyVerbFormatLine ##1{\hsize\linewidth
            \vtop{\raggedright\hyphenpenalty\z@\exhyphenpenalty\z@
                \doublehyphendemerits\z@\finalhyphendemerits\z@
                \strut ##1\strut}%
        }%
        % If the linebreak is at a space, the latter will be displayed as visible
        % space at end of first line, and a continuation symbol starts next line.
        % Stretch/shrink are however usually zero for typewriter font.
        \def\FV@Space {%
            \nobreak\hskip\z@ plus\fontdimen3\font minus\fontdimen4\font
            \discretionary{\copy\Wrappedvisiblespacebox}{\Wrappedafterbreak}
            {\kern\fontdimen2\font}%
        }%
        
        % Allow breaks at special characters using \PYG... macros.
        \Wrappedbreaksatspecials
        % Breaks at punctuation characters . , ; ? ! and / need catcode=\active 	
        \OriginalVerbatim[#1,codes*=\Wrappedbreaksatpunct]%
    }
    \makeatother

    % Exact colors from NB
    \definecolor{incolor}{HTML}{303F9F}
    \definecolor{outcolor}{HTML}{D84315}
    \definecolor{cellborder}{HTML}{CFCFCF}
    \definecolor{cellbackground}{HTML}{F7F7F7}
    
    % prompt
    \makeatletter
    \newcommand{\boxspacing}{\kern\kvtcb@left@rule\kern\kvtcb@boxsep}
    \makeatother
    \newcommand{\prompt}[4]{
        \ttfamily\llap{{\color{#2}[#3]:\hspace{3pt}#4}}\vspace{-\baselineskip}
    }
    

    
    % Prevent overflowing lines due to hard-to-break entities
    \sloppy 
    % Setup hyperref package
    \hypersetup{
      breaklinks=true,  % so long urls are correctly broken across lines
      colorlinks=true,
      urlcolor=urlcolor,
      linkcolor=linkcolor,
      citecolor=citecolor,
      }
    % Slightly bigger margins than the latex defaults
    
    \geometry{verbose,tmargin=1in,bmargin=1in,lmargin=1in,rmargin=1in}
    
    

\begin{document}
    
\title{Tarea 8 Optimización}
\author{Roberto Vásquez Martínez \\ Profesor: Joaquín Peña Acevedo}
\date{01/Mayo/2022}
\maketitle   
    
    

    
    \hypertarget{ejercicio-1-3-puntos}{%
\section{Ejercicio 1 (3 puntos)}\label{ejercicio-1-3-puntos}}

Sea \(x=(x_1, x_2, ..., x_n)\) la variable independiente.

Programar las siguientes funciones y sus gradientes:

\begin{itemize}
\tightlist
\item
  Función cuadrática
\end{itemize}

\[ f(\mathbf{x}) = 0.5\mathbf{x}^\top \mathbf{A}\mathbf{x} - \mathbf{b}^\top\mathbf{x}. \]

Si \(\mathbf{I}\) es la matriz identidad y \(\mathbf{1}\) es la matriz
llena de 1's, ambas de tamaño \(n\), entonces

\[ \mathbf{A} = n\mathbf{I} + \mathbf{1} = 
\left[\begin{array}{llll} n      & 0      & \cdots & 0 \\
                       0      & n      & \cdots & 0 \\ 
                       \vdots & \vdots & \ddots & \vdots \\
                       0      & 0      & \cdots & n \end{array}\right]
+ \left[\begin{array}{llll} 1    & 1      & \cdots & 1 \\
                       1      & 1      & \cdots & 1 \\ 
                       \vdots & \vdots & \ddots & \vdots \\
                       1      & 1      & \cdots & 1 \end{array}\right],  \qquad
\mathbf{b} = \left[\begin{array}{l} 1 \\ 1 \\ \vdots \\ 1 \end{array}\right] \]

\begin{itemize}
\tightlist
\item
  Función generalizada de Rosenbrock
\end{itemize}

\[  f(x) = \sum_{i=1}^{n-1} 100(x_{i+1} - x_i^2)^2 + (1 - x_{i} )^2  \]

\[ x_0 = (-1.2, 1, -1.2, 1, ..., -1.2, 1) \]

En la implementación de cada función y de su gradiente, se recibe como
argumento la variable \(x\) y definimos \(n\) como la longitud del
arreglo \(x\), y con esos datos aplicamos la definición correspondiente.

Estas funciones van a ser usadas para probar los algoritmos de
optimización. El punto \(x_0\) que aparece en la definición de cada
función es el punto inicial que se sugiere para el algoritmo de
optimización.

\hypertarget{soluciuxf3n}{%
\subsection{Solución:}\label{soluciuxf3n}}

    A continuación, importamos el módulo \texttt{lib\_t8.py} que contiene la
definición de las funciones anteriores y sus gradientes. Sea \(f_Q\) y
\(f_R\) las funciones cuadráticas y generalizada de Rosenbrock
respectivamente. Vamos a probar estas funciones en el punto
\(\mathbf{x}=(1,1)\).

Sabemos que \(\mathbf{x}\) es punto crítico de \(f_R\) con óptimo
\(f_R(\mathbf{x})=0\), mientras que, haciendo unos cuantos cálculos
podemos ver que \(f_Q(\mathbf{x})=2\) y
\(\nabla f_Q(\mathbf{x})=(3,3)^T\). Compararemos estos valores con los
resultados numéricos.

En primer lugar, evaluamos en \(\mathbf{x}\) la función cuadrática
\(f_Q\) programada

    \begin{tcolorbox}[breakable, size=fbox, boxrule=1pt, pad at break*=1mm,colback=cellbackground, colframe=cellborder]
\prompt{In}{incolor}{2}{\boxspacing}
\begin{Verbatim}[commandchars=\\\{\}]
\PY{k+kn}{import} \PY{n+nn}{lib\PYZus{}t8}
\PY{k+kn}{import} \PY{n+nn}{importlib}
\PY{n}{importlib}\PY{o}{.}\PY{n}{reload}\PY{p}{(}\PY{n}{lib\PYZus{}t8}\PY{p}{)}
\PY{k+kn}{from} \PY{n+nn}{lib\PYZus{}t8} \PY{k+kn}{import} \PY{o}{*}

\PY{c+c1}{\PYZsh{} Punto de prueba}
\PY{n}{x\PYZus{}proof}\PY{o}{=}\PY{n}{np}\PY{o}{.}\PY{n}{array}\PY{p}{(}\PY{p}{[}\PY{l+m+mf}{1.0}\PY{p}{,}\PY{l+m+mf}{1.0}\PY{p}{]}\PY{p}{)}

\PY{c+c1}{\PYZsh{} Implementación de la función cuadrática y su gradiente}
\PY{n+nb}{print}\PY{p}{(}\PY{l+s+s1}{\PYZsq{}}\PY{l+s+s1}{Valor de la funcion cuadrática en (1,1): }\PY{l+s+s1}{\PYZsq{}}\PY{p}{,}\PY{n}{quad\PYZus{}fun}\PY{p}{(}\PY{n}{x\PYZus{}proof}\PY{p}{)}\PY{p}{)}
\PY{n+nb}{print}\PY{p}{(}\PY{l+s+s1}{\PYZsq{}}\PY{l+s+s1}{Valor del gradiente de la funcion cuadrática en (1,1): }\PY{l+s+s1}{\PYZsq{}}\PY{p}{,}\PY{n}{grad\PYZus{}quad\PYZus{}fun}\PY{p}{(}\PY{n}{x\PYZus{}proof}\PY{p}{)}\PY{p}{)}
\end{Verbatim}
\end{tcolorbox}

    \begin{Verbatim}[commandchars=\\\{\}]
Valor de la funcion cuadrática en (1,1):  2.0
Valor del gradiente de la funcion cuadrática en (1,1):  [[3.]
 [3.]]
    \end{Verbatim}

    Realizamos la misma prueba para la función generalizada de Rosenbrock,
sabiendo que \(f_R(\mathbf{x})=0\) y \(\nabla f_R(\mathbf{x})=0\)

    \begin{tcolorbox}[breakable, size=fbox, boxrule=1pt, pad at break*=1mm,colback=cellbackground, colframe=cellborder]
\prompt{In}{incolor}{3}{\boxspacing}
\begin{Verbatim}[commandchars=\\\{\}]
\PY{c+c1}{\PYZsh{}  Implementación de la función generalizada de Rosenbrock y su gradiente}
\PY{n+nb}{print}\PY{p}{(}\PY{l+s+s1}{\PYZsq{}}\PY{l+s+s1}{Valor de la funcion generalizada de Rosenbrock en (1,1): }\PY{l+s+s1}{\PYZsq{}}\PY{p}{,}\PY{n}{gen\PYZus{}rosenbrock}\PY{p}{(}\PY{n}{x\PYZus{}proof}\PY{p}{)}\PY{p}{)}
\PY{n+nb}{print}\PY{p}{(}\PY{l+s+s1}{\PYZsq{}}\PY{l+s+s1}{Valor del gradiente de la funcion generalizada de Rosenbrock en (1,1): }\PY{l+s+s1}{\PYZsq{}}\PY{p}{,}\PY{n}{grad\PYZus{}gen\PYZus{}rosenbrock}\PY{p}{(}\PY{n}{x\PYZus{}proof}\PY{p}{)}\PY{p}{)}
\end{Verbatim}
\end{tcolorbox}

    \begin{Verbatim}[commandchars=\\\{\}]
Valor de la funcion generalizada de Rosenbrock en (1,1):  0.0
Valor del gradiente de la funcion generalizada de Rosenbrock en (1,1):  [-0.
0.]
    \end{Verbatim}

    \hypertarget{ejercicio-2-3.5-puntos}{%
\section{Ejercicio 2 (3.5 puntos)}\label{ejercicio-2-3.5-puntos}}

Programar el método de gradiente conjugado no lineal de Fletcher-Reeves:

\begin{center}\rule{0.5\linewidth}{0.5pt}\end{center}

La implementación recibe como argumentos a la función objetivo \(f\), su
gradiente \(\nabla f\), un punto inicial \(x_0\), el máximo número de
iteraciones \(N\) y una tolerancia \(\tau>0\).

\begin{enumerate}
\def\labelenumi{\arabic{enumi}.}
\tightlist
\item
  Calcular \(\nabla f_0 = \nabla f(x_0)\), \(p_0 = -\nabla f_0\) y hacer
  \(res=0\).
\item
  Para \(k=0,1,..., N\):
\end{enumerate}

\begin{itemize}
\tightlist
\item
  Si \(\|\nabla f_k\|< \tau\), hacer \(res=1\) y terminar el ciclo
\item
  Usando backtracking calcular el tamaño de paso \(\alpha_k\)
\item
  Calcular \(x_{k+1} = x_k + \alpha_k p_k\)
\item
  Calcular \(\nabla f_{k+1} = \nabla f(x_{k+1})\)
\item
  Calcular
\end{itemize}

\[ \beta_{k+1} = \frac{\nabla f_{k+1}^\top \nabla f_{k+1}}{\nabla f_{k}^\top\nabla f_{k}}  \]

\begin{itemize}
\tightlist
\item
  Calcular
\end{itemize}

\[ p_{k+1} = -\nabla f_{k+1} + \beta_{k+1} p_k \]

\begin{enumerate}
\def\labelenumi{\arabic{enumi}.}
\setcounter{enumi}{2}
\tightlist
\item
  Devolver \(x_k, \nabla f_k, k, res\)
\end{enumerate}

\begin{verbatim}
\end{verbatim}

\begin{center}\rule{0.5\linewidth}{0.5pt}\end{center}

\begin{enumerate}
\def\labelenumi{\arabic{enumi}.}
\tightlist
\item
  Escriba la función que implemente el algoritmo anterior.
\item
  Pruebe el algoritmo usando para cada una de las funciones del
  Ejercicio 1, tomando el punto \(x_0\) que se indica.
\item
  Fije \(N=50000\), \(\tau = \epsilon_m^{1/3}\).
\item
  Para cada función del Ejercicio 1 cree el punto \(x_0\)
  correspondiente usado \(n=2, 10, 20\) y ejecute el algoritmo. Imprima
\end{enumerate}

\begin{itemize}
\tightlist
\item
  n,
\item
  f(x0),
\item
  las primeras y últimas 4 entradas del punto \(x_k\) que devuelve el
  algoritmo,
\item
  f(xk),
\item
  la norma del vector \(\nabla f_k\),
\item
  el número \(k\) de iteraciones realizadas,
\item
  la variable \(res\) para saber si el algoritmo puedo converger.
\end{itemize}

\hypertarget{soluciuxf3n}{%
\subsection{Solución:}\label{soluciuxf3n}}

    Actualizamos el módulo con el que estamos trabajando, definimos la
tolerancia, el número máximo de iteraciones y el parámetro \(\rho\) del
algoritmo de backtracking

    \begin{tcolorbox}[breakable, size=fbox, boxrule=1pt, pad at break*=1mm,colback=cellbackground, colframe=cellborder]
\prompt{In}{incolor}{71}{\boxspacing}
\begin{Verbatim}[commandchars=\\\{\}]
\PY{n}{importlib}\PY{o}{.}\PY{n}{reload}\PY{p}{(}\PY{n}{lib\PYZus{}t8}\PY{p}{)}
\PY{k+kn}{import} \PY{n+nn}{lib\PYZus{}t8}
\PY{k+kn}{from} \PY{n+nn}{lib\PYZus{}t8} \PY{k+kn}{import} \PY{o}{*}

\PY{c+c1}{\PYZsh{} Tolerancia y numero maximo de iteraciones}
\PY{n}{tol}\PY{o}{=}\PY{n}{np}\PY{o}{.}\PY{n}{finfo}\PY{p}{(}\PY{n+nb}{float}\PY{p}{)}\PY{o}{.}\PY{n}{eps}\PY{o}{*}\PY{o}{*}\PY{p}{(}\PY{l+m+mi}{1}\PY{o}{/}\PY{l+m+mi}{3}\PY{p}{)}
\PY{n}{N}\PY{o}{=}\PY{l+m+mi}{50000}
\PY{n}{rho}\PY{o}{=}\PY{l+m+mf}{0.8}
\end{Verbatim}
\end{tcolorbox}

    \hypertarget{funciuxf3n-cuadruxe1tica}{%
\paragraph{Función cuadrática}\label{funciuxf3n-cuadruxe1tica}}

En primer lugar, probamos el algoritmo de Fletcher-Reeves como en clase,
en este caso como es una función cuadrática es equivalente al método de
gradiente conjugado. A continuación, probamos el desempeño en esta
función para \(n=2,10,20\)

Por la proposición 1 de la Clase 17, el método de gradiente conjugado en
este caso es globalmente convergente y el punto óptimo es
\(\mathbf{x}_\ast=\frac{1}{2n}(1,1,\dots,1)^T\). Daremos cualquier punto
inicial y compararemos con el óptimo correspondiente, en este caso
usaremos la misma condición que se sugiere en la función generalizada de
Rosenbrock

    \begin{tcolorbox}[breakable, size=fbox, boxrule=1pt, pad at break*=1mm,colback=cellbackground, colframe=cellborder]
\prompt{In}{incolor}{72}{\boxspacing}
\begin{Verbatim}[commandchars=\\\{\}]
\PY{c+c1}{\PYZsh{} Valores de n}
\PY{n}{n}\PY{o}{=}\PY{p}{[}\PY{l+m+mi}{2}\PY{p}{,}\PY{l+m+mi}{10}\PY{p}{,}\PY{l+m+mi}{20}\PY{p}{]}
\PY{c+c1}{\PYZsh{} Paras para cada valor de n}
\PY{k}{for} \PY{n}{nn} \PY{o+ow}{in} \PY{n}{n}\PY{p}{:}
    \PY{n}{x0}\PY{o}{=}\PY{n}{np}\PY{o}{.}\PY{n}{tile}\PY{p}{(}\PY{p}{[}\PY{o}{\PYZhy{}}\PY{l+m+mf}{1.2}\PY{p}{,}\PY{l+m+mf}{1.0}\PY{p}{]}\PY{p}{,}\PY{n+nb}{int}\PY{p}{(}\PY{n}{nn}\PY{o}{/}\PY{l+m+mi}{2}\PY{p}{)}\PY{p}{)}
    \PY{n}{proof\PYZus{}fletcher\PYZus{}reeves}\PY{p}{(}\PY{n}{quad\PYZus{}fun}\PY{p}{,}\PY{n}{grad\PYZus{}quad\PYZus{}fun}\PY{p}{,}\PY{n}{x0}\PY{p}{,}\PY{n}{N}\PY{p}{,}\PY{n}{tol}\PY{p}{,}\PY{n}{rho}\PY{p}{)}
    \PY{n+nb}{print}\PY{p}{(}\PY{l+s+s1}{\PYZsq{}}\PY{l+s+se}{\PYZbs{}n}\PY{l+s+se}{\PYZbs{}n}\PY{l+s+s1}{\PYZsq{}}\PY{p}{)}
\end{Verbatim}
\end{tcolorbox}

    \begin{Verbatim}[commandchars=\\\{\}]
El algoritmo de Fletcher-Reeves CONVERGE
n =  2
f(x0) =  2.66
xk =  [0.25000229 0.2499987 ]
k =  91
fk =  -0.24999999999256428
||gk|| =  5.803891132856387e-06



El algoritmo de Fletcher-Reeves CONVERGE
n =  10
f(x0) =  62.49999999999999
Primer y últimas 4 entradas de xk =  [0.04999983 0.05000018 0.04999983
0.05000018] {\ldots} [0.04999983 0.05000018 0.04999983 0.05000018]
k =  134
fk =  -0.24999999999843664
||gk|| =  5.603306622825684e-06



El algoritmo de Fletcher-Reeves CONVERGE
n =  20
f(x0) =  248.0
Primer y últimas 4 entradas de xk =  [0.02499995 0.02500007 0.02499995
0.02500007] {\ldots} [0.02499995 0.02500007 0.02499995 0.02500007]
k =  138
fk =  -0.2499999999992326
||gk|| =  5.707639527877831e-06



    \end{Verbatim}

    Como usamos backtracking para estimar el óptimo del tamaño de paso no se
necesariamente se obtiene la solución en \(n\) pasos y en efecto hemos
obtenido los vectores solución de la forma \(\frac{1}{2n}(1,1,\dots,1)\)
para \(n=2,10,20\), que corresponden a los valores donde la función
alcanza su óptimo en cada caso.

    \hypertarget{funciuxf3n-generalizada-de-rosenbrock}{%
\paragraph{Función Generalizada de
Rosenbrock}\label{funciuxf3n-generalizada-de-rosenbrock}}

Repetimos la prueba ahora con la función generalizada de Rosenbrock,
sabiendo que el óptimo es \(\mathbf{x}_\ast=(1,1,\dots,1)\)

    \begin{tcolorbox}[breakable, size=fbox, boxrule=1pt, pad at break*=1mm,colback=cellbackground, colframe=cellborder]
\prompt{In}{incolor}{73}{\boxspacing}
\begin{Verbatim}[commandchars=\\\{\}]
\PY{c+c1}{\PYZsh{} Paras para cada valor de n}
\PY{k}{for} \PY{n}{nn} \PY{o+ow}{in} \PY{n}{n}\PY{p}{:}
    \PY{n}{x0}\PY{o}{=}\PY{n}{np}\PY{o}{.}\PY{n}{tile}\PY{p}{(}\PY{p}{[}\PY{o}{\PYZhy{}}\PY{l+m+mf}{1.2}\PY{p}{,}\PY{l+m+mf}{1.0}\PY{p}{]}\PY{p}{,}\PY{n+nb}{int}\PY{p}{(}\PY{n}{nn}\PY{o}{/}\PY{l+m+mi}{2}\PY{p}{)}\PY{p}{)}
    \PY{n}{proof\PYZus{}fletcher\PYZus{}reeves}\PY{p}{(}\PY{n}{gen\PYZus{}rosenbrock}\PY{p}{,}\PY{n}{grad\PYZus{}gen\PYZus{}rosenbrock}\PY{p}{,}\PY{n}{x0}\PY{p}{,}\PY{n}{N}\PY{p}{,}\PY{n}{tol}\PY{p}{,}\PY{n}{rho}\PY{p}{)}
    \PY{n+nb}{print}\PY{p}{(}\PY{l+s+s1}{\PYZsq{}}\PY{l+s+se}{\PYZbs{}n}\PY{l+s+se}{\PYZbs{}n}\PY{l+s+s1}{\PYZsq{}}\PY{p}{)}
\end{Verbatim}
\end{tcolorbox}

    \begin{Verbatim}[commandchars=\\\{\}]
El algoritmo de Fletcher-Reeves CONVERGE
n =  2
f(x0) =  24.199999999999996
xk =  [0.99999539 0.99999077]
k =  251
fk =  2.1277093635685042e-11
||gk|| =  4.780235391540514e-06



El algoritmo de Fletcher-Reeves CONVERGE
n =  10
f(x0) =  2057.0
Primer y últimas 4 entradas de xk =  [0.99999999 1.         1.         1.
] {\ldots} [0.99999999 0.99999998 0.99999995 0.9999999 ]
k =  2019
fk =  2.193108694121752e-14
||gk|| =  5.749420469861276e-06



El algoritmo de Fletcher-Reeves CONVERGE
n =  20
f(x0) =  4598.0
Primer y últimas 4 entradas de xk =  [1. 1. 1. 1.] {\ldots} [0.99999999 0.99999998
0.99999997 0.99999993]
k =  38376
fk =  1.5946132549678812e-14
||gk|| =  6.039770533007446e-06



    \end{Verbatim}

    \hypertarget{ejercicio-3-3.5-puntos}{%
\section{Ejercicio 3 (3.5 puntos)}\label{ejercicio-3-3.5-puntos}}

Programar el método de gradiente conjugado no lineal de usando la
fórmula de Hestenes-Stiefel:

En este caso el algoritmo es igual al del Ejercicio 2, con excepción del
cálculo de \(\beta_{k+1}\). Primero se calcula el vector
\(\mathbf{y}_k\) y luego \(\beta_{k+1}\):

\[ \mathbf{y}_k =  \nabla f_{k+1}-\nabla f_{k} \]
\[ \beta_{k+1} =   \frac{\nabla f_{k+1}^\top\mathbf{y}_k }{p_{k}^\top\mathbf{y}_k}  \]

\begin{enumerate}
\def\labelenumi{\arabic{enumi}.}
\tightlist
\item
  Repita el Ejercicio 2 usando la fórmula de Hestenes-Stiefel.
\item
  ¿Cuál de los métodos es mejor para encontrar los óptimos de las
  funciones de prueba?
\end{enumerate}

\hypertarget{soluciuxf3n}{%
\subsection{Solución:}\label{soluciuxf3n}}

    Haremos las mismas pruebas tanto para la función cuadrática como la
función generalizada de Rosenbrock ahora utilizando las actualización de
Hestenes-Stiefel.

\hypertarget{funciuxf3n-cuadruxe1tica}{%
\paragraph{Función cuadrática}\label{funciuxf3n-cuadruxe1tica}}

En la siguiente celda presentamos el resultado de Hestenes-Stiefel para
la función \(f_Q\)

    \begin{tcolorbox}[breakable, size=fbox, boxrule=1pt, pad at break*=1mm,colback=cellbackground, colframe=cellborder]
\prompt{In}{incolor}{74}{\boxspacing}
\begin{Verbatim}[commandchars=\\\{\}]
\PY{n}{importlib}\PY{o}{.}\PY{n}{reload}\PY{p}{(}\PY{n}{lib\PYZus{}t8}\PY{p}{)}
\PY{k+kn}{import} \PY{n+nn}{lib\PYZus{}t8}
\PY{k+kn}{from} \PY{n+nn}{lib\PYZus{}t8} \PY{k+kn}{import} \PY{o}{*}

\PY{c+c1}{\PYZsh{} Paras para cada valor de n}
\PY{k}{for} \PY{n}{nn} \PY{o+ow}{in} \PY{n}{n}\PY{p}{:}
    \PY{n}{x0}\PY{o}{=}\PY{n}{np}\PY{o}{.}\PY{n}{tile}\PY{p}{(}\PY{p}{[}\PY{o}{\PYZhy{}}\PY{l+m+mf}{1.2}\PY{p}{,}\PY{l+m+mf}{1.0}\PY{p}{]}\PY{p}{,}\PY{n+nb}{int}\PY{p}{(}\PY{n}{nn}\PY{o}{/}\PY{l+m+mi}{2}\PY{p}{)}\PY{p}{)}
    \PY{n}{proof\PYZus{}hestenes\PYZus{}stiefel}\PY{p}{(}\PY{n}{quad\PYZus{}fun}\PY{p}{,}\PY{n}{grad\PYZus{}quad\PYZus{}fun}\PY{p}{,}\PY{n}{x0}\PY{p}{,}\PY{n}{N}\PY{p}{,}\PY{n}{tol}\PY{p}{,}\PY{n}{rho}\PY{p}{)}
    \PY{n+nb}{print}\PY{p}{(}\PY{l+s+s1}{\PYZsq{}}\PY{l+s+se}{\PYZbs{}n}\PY{l+s+se}{\PYZbs{}n}\PY{l+s+s1}{\PYZsq{}}\PY{p}{)}
\end{Verbatim}
\end{tcolorbox}

    \begin{Verbatim}[commandchars=\\\{\}]
El algoritmo de Hestenes-Stiefel CONVERGE
n =  2
f(x0) =  2.66
xk =  [0.25000217 0.24999881]
k =  66
fk =  -0.24999999999340244
||gk|| =  5.497058549749721e-06



El algoritmo de Hestenes-Stiefel CONVERGE
n =  10
f(x0) =  62.49999999999999
Primer y últimas 4 entradas de xk =  [0.05000019 0.04999985 0.05000019
0.04999985] {\ldots} [0.05000019 0.04999985 0.05000019 0.04999985]
k =  162
fk =  -0.24999999999844713
||gk|| =  5.650924470696626e-06



El algoritmo de Hestenes-Stiefel CONVERGE
n =  20
f(x0) =  248.0
Primer y últimas 4 entradas de xk =  [0.02499993 0.02500004 0.02499993
0.02500004] {\ldots} [0.02499993 0.02500004 0.02499993 0.02500004]
k =  227
fk =  -0.24999999999929282
||gk|| =  5.793279751916127e-06



    \end{Verbatim}

    \hypertarget{funciuxf3n-generalizada-de-rosenbrock}{%
\paragraph{Función Generalizada de
Rosenbrock}\label{funciuxf3n-generalizada-de-rosenbrock}}

Se puede ver que en el caso cuadrático el desempeño es el mismo porque
ambas actualizaciones son equivalente al método de gradiente conjugado
para el caso cuadrática.

El caso de mayor interés, y en el que esperamos descrepancias es
utilizando como función de prueba a la función \(f_R\).

A continuación, presentamos el resumen de las pruebas

    \begin{tcolorbox}[breakable, size=fbox, boxrule=1pt, pad at break*=1mm,colback=cellbackground, colframe=cellborder]
\prompt{In}{incolor}{75}{\boxspacing}
\begin{Verbatim}[commandchars=\\\{\}]
\PY{c+c1}{\PYZsh{} Paras para cada valor de n}
\PY{k}{for} \PY{n}{nn} \PY{o+ow}{in} \PY{n}{n}\PY{p}{:}
    \PY{n}{x0}\PY{o}{=}\PY{n}{np}\PY{o}{.}\PY{n}{tile}\PY{p}{(}\PY{p}{[}\PY{o}{\PYZhy{}}\PY{l+m+mf}{1.2}\PY{p}{,}\PY{l+m+mf}{1.0}\PY{p}{]}\PY{p}{,}\PY{n+nb}{int}\PY{p}{(}\PY{n}{nn}\PY{o}{/}\PY{l+m+mi}{2}\PY{p}{)}\PY{p}{)}
    \PY{n}{proof\PYZus{}hestenes\PYZus{}stiefel}\PY{p}{(}\PY{n}{gen\PYZus{}rosenbrock}\PY{p}{,}\PY{n}{grad\PYZus{}gen\PYZus{}rosenbrock}\PY{p}{,}\PY{n}{x0}\PY{p}{,}\PY{n}{N}\PY{p}{,}\PY{n}{tol}\PY{p}{,}\PY{n}{rho}\PY{p}{)}
    \PY{n+nb}{print}\PY{p}{(}\PY{l+s+s1}{\PYZsq{}}\PY{l+s+se}{\PYZbs{}n}\PY{l+s+se}{\PYZbs{}n}\PY{l+s+s1}{\PYZsq{}}\PY{p}{)}
\end{Verbatim}
\end{tcolorbox}

    \begin{Verbatim}[commandchars=\\\{\}]
El algoritmo de Hestenes-Stiefel CONVERGE
n =  2
f(x0) =  24.199999999999996
xk =  [1.0000003 1.0000006]
k =  110
fk =  1.0800792500369892e-13
||gk|| =  5.942872244700956e-06



El algoritmo de Hestenes-Stiefel CONVERGE
n =  10
f(x0) =  2057.0
Primer y últimas 4 entradas de xk =  [1. 1. 1. 1.] {\ldots} [1.00000001 1.00000002
1.00000004 1.00000009]
k =  783
fk =  1.4272406448603126e-14
||gk|| =  5.906937683335181e-06



El algoritmo de Hestenes-Stiefel CONVERGE
n =  20
f(x0) =  4598.0
Primer y últimas 4 entradas de xk =  [1. 1. 1. 1.] {\ldots} [0.99999994 0.99999988
0.99999976 0.99999953]
k =  503
fk =  8.817321104442124e-14
||gk|| =  5.955100340841743e-06



    \end{Verbatim}

    Comparando Frechet-Reeves con Hestenes-Stiefel, Frechet-Reeves tiene un
mejor comportamiento en el caso de la función \(f_Q\) en general, sin
embargo, en el caso de la función de Rosenbrock, el comportamiento de
Hestenes-Stiefel es mucho mejor que Frechet-Reeves respecto al número de
iteraciones.


    % Add a bibliography block to the postdoc
    
    
    
\end{document}
