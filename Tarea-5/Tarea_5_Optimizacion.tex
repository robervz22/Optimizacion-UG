\documentclass[11pt]{article}

    \usepackage[breakable]{tcolorbox}
    \usepackage{parskip} % Stop auto-indenting (to mimic markdown behaviour)
    
    \usepackage{iftex}
    \ifPDFTeX
    	\usepackage[T1]{fontenc}
    	\usepackage{mathpazo}
    \else
    	\usepackage{fontspec}
    \fi

    % Basic figure setup, for now with no caption control since it's done
    % automatically by Pandoc (which extracts ![](path) syntax from Markdown).
    \usepackage{graphicx}
    % Maintain compatibility with old templates. Remove in nbconvert 6.0
    \let\Oldincludegraphics\includegraphics
    % Ensure that by default, figures have no caption (until we provide a
    % proper Figure object with a Caption API and a way to capture that
    % in the conversion process - todo).
    \usepackage{caption}
    \DeclareCaptionFormat{nocaption}{}
    \captionsetup{format=nocaption,aboveskip=0pt,belowskip=0pt}

    \usepackage{float}
    \floatplacement{figure}{H} % forces figures to be placed at the correct location
    \usepackage{xcolor} % Allow colors to be defined
    \usepackage{enumerate} % Needed for markdown enumerations to work
    \usepackage{geometry} % Used to adjust the document margins
    \usepackage{amsmath} % Equations
    \usepackage{amssymb} % Equations
    \usepackage{textcomp} % defines textquotesingle
    % Hack from http://tex.stackexchange.com/a/47451/13684:
    \AtBeginDocument{%
        \def\PYZsq{\textquotesingle}% Upright quotes in Pygmentized code
    }
    \usepackage{upquote} % Upright quotes for verbatim code
    \usepackage{eurosym} % defines \euro
    \usepackage[mathletters]{ucs} % Extended unicode (utf-8) support
    \usepackage{fancyvrb} % verbatim replacement that allows latex
    \usepackage{grffile} % extends the file name processing of package graphics 
                         % to support a larger range
    \makeatletter % fix for old versions of grffile with XeLaTeX
    \@ifpackagelater{grffile}{2019/11/01}
    {
      % Do nothing on new versions
    }
    {
      \def\Gread@@xetex#1{%
        \IfFileExists{"\Gin@base".bb}%
        {\Gread@eps{\Gin@base.bb}}%
        {\Gread@@xetex@aux#1}%
      }
    }
    \makeatother
    \usepackage[Export]{adjustbox} % Used to constrain images to a maximum size
    \adjustboxset{max size={0.9\linewidth}{0.9\paperheight}}

    % The hyperref package gives us a pdf with properly built
    % internal navigation ('pdf bookmarks' for the table of contents,
    % internal cross-reference links, web links for URLs, etc.)
    \usepackage{hyperref}
    % The default LaTeX title has an obnoxious amount of whitespace. By default,
    % titling removes some of it. It also provides customization options.
    \usepackage{titling}
    \usepackage{longtable} % longtable support required by pandoc >1.10
    \usepackage{booktabs}  % table support for pandoc > 1.12.2
    \usepackage[inline]{enumitem} % IRkernel/repr support (it uses the enumerate* environment)
    \usepackage[normalem]{ulem} % ulem is needed to support strikethroughs (\sout)
                                % normalem makes italics be italics, not underlines
    \usepackage{mathrsfs}
    

    
    % Colors for the hyperref package
    \definecolor{urlcolor}{rgb}{0,.145,.698}
    \definecolor{linkcolor}{rgb}{.71,0.21,0.01}
    \definecolor{citecolor}{rgb}{.12,.54,.11}

    % ANSI colors
    \definecolor{ansi-black}{HTML}{3E424D}
    \definecolor{ansi-black-intense}{HTML}{282C36}
    \definecolor{ansi-red}{HTML}{E75C58}
    \definecolor{ansi-red-intense}{HTML}{B22B31}
    \definecolor{ansi-green}{HTML}{00A250}
    \definecolor{ansi-green-intense}{HTML}{007427}
    \definecolor{ansi-yellow}{HTML}{DDB62B}
    \definecolor{ansi-yellow-intense}{HTML}{B27D12}
    \definecolor{ansi-blue}{HTML}{208FFB}
    \definecolor{ansi-blue-intense}{HTML}{0065CA}
    \definecolor{ansi-magenta}{HTML}{D160C4}
    \definecolor{ansi-magenta-intense}{HTML}{A03196}
    \definecolor{ansi-cyan}{HTML}{60C6C8}
    \definecolor{ansi-cyan-intense}{HTML}{258F8F}
    \definecolor{ansi-white}{HTML}{C5C1B4}
    \definecolor{ansi-white-intense}{HTML}{A1A6B2}
    \definecolor{ansi-default-inverse-fg}{HTML}{FFFFFF}
    \definecolor{ansi-default-inverse-bg}{HTML}{000000}

    % common color for the border for error outputs.
    \definecolor{outerrorbackground}{HTML}{FFDFDF}

    % commands and environments needed by pandoc snippets
    % extracted from the output of `pandoc -s`
    \providecommand{\tightlist}{%
      \setlength{\itemsep}{0pt}\setlength{\parskip}{0pt}}
    \DefineVerbatimEnvironment{Highlighting}{Verbatim}{commandchars=\\\{\}}
    % Add ',fontsize=\small' for more characters per line
    \newenvironment{Shaded}{}{}
    \newcommand{\KeywordTok}[1]{\textcolor[rgb]{0.00,0.44,0.13}{\textbf{{#1}}}}
    \newcommand{\DataTypeTok}[1]{\textcolor[rgb]{0.56,0.13,0.00}{{#1}}}
    \newcommand{\DecValTok}[1]{\textcolor[rgb]{0.25,0.63,0.44}{{#1}}}
    \newcommand{\BaseNTok}[1]{\textcolor[rgb]{0.25,0.63,0.44}{{#1}}}
    \newcommand{\FloatTok}[1]{\textcolor[rgb]{0.25,0.63,0.44}{{#1}}}
    \newcommand{\CharTok}[1]{\textcolor[rgb]{0.25,0.44,0.63}{{#1}}}
    \newcommand{\StringTok}[1]{\textcolor[rgb]{0.25,0.44,0.63}{{#1}}}
    \newcommand{\CommentTok}[1]{\textcolor[rgb]{0.38,0.63,0.69}{\textit{{#1}}}}
    \newcommand{\OtherTok}[1]{\textcolor[rgb]{0.00,0.44,0.13}{{#1}}}
    \newcommand{\AlertTok}[1]{\textcolor[rgb]{1.00,0.00,0.00}{\textbf{{#1}}}}
    \newcommand{\FunctionTok}[1]{\textcolor[rgb]{0.02,0.16,0.49}{{#1}}}
    \newcommand{\RegionMarkerTok}[1]{{#1}}
    \newcommand{\ErrorTok}[1]{\textcolor[rgb]{1.00,0.00,0.00}{\textbf{{#1}}}}
    \newcommand{\NormalTok}[1]{{#1}}
    
    % Additional commands for more recent versions of Pandoc
    \newcommand{\ConstantTok}[1]{\textcolor[rgb]{0.53,0.00,0.00}{{#1}}}
    \newcommand{\SpecialCharTok}[1]{\textcolor[rgb]{0.25,0.44,0.63}{{#1}}}
    \newcommand{\VerbatimStringTok}[1]{\textcolor[rgb]{0.25,0.44,0.63}{{#1}}}
    \newcommand{\SpecialStringTok}[1]{\textcolor[rgb]{0.73,0.40,0.53}{{#1}}}
    \newcommand{\ImportTok}[1]{{#1}}
    \newcommand{\DocumentationTok}[1]{\textcolor[rgb]{0.73,0.13,0.13}{\textit{{#1}}}}
    \newcommand{\AnnotationTok}[1]{\textcolor[rgb]{0.38,0.63,0.69}{\textbf{\textit{{#1}}}}}
    \newcommand{\CommentVarTok}[1]{\textcolor[rgb]{0.38,0.63,0.69}{\textbf{\textit{{#1}}}}}
    \newcommand{\VariableTok}[1]{\textcolor[rgb]{0.10,0.09,0.49}{{#1}}}
    \newcommand{\ControlFlowTok}[1]{\textcolor[rgb]{0.00,0.44,0.13}{\textbf{{#1}}}}
    \newcommand{\OperatorTok}[1]{\textcolor[rgb]{0.40,0.40,0.40}{{#1}}}
    \newcommand{\BuiltInTok}[1]{{#1}}
    \newcommand{\ExtensionTok}[1]{{#1}}
    \newcommand{\PreprocessorTok}[1]{\textcolor[rgb]{0.74,0.48,0.00}{{#1}}}
    \newcommand{\AttributeTok}[1]{\textcolor[rgb]{0.49,0.56,0.16}{{#1}}}
    \newcommand{\InformationTok}[1]{\textcolor[rgb]{0.38,0.63,0.69}{\textbf{\textit{{#1}}}}}
    \newcommand{\WarningTok}[1]{\textcolor[rgb]{0.38,0.63,0.69}{\textbf{\textit{{#1}}}}}
    
    
    % Define a nice break command that doesn't care if a line doesn't already
    % exist.
    \def\br{\hspace*{\fill} \\* }
    % Math Jax compatibility definitions
    \def\gt{>}
    \def\lt{<}
    \let\Oldtex\TeX
    \let\Oldlatex\LaTeX
    \renewcommand{\TeX}{\textrm{\Oldtex}}
    \renewcommand{\LaTeX}{\textrm{\Oldlatex}}
    % Document parameters
    % Document title
    \title{Tarea\_5\_Optimizacion}
    
    
    
    
    
% Pygments definitions
\makeatletter
\def\PY@reset{\let\PY@it=\relax \let\PY@bf=\relax%
    \let\PY@ul=\relax \let\PY@tc=\relax%
    \let\PY@bc=\relax \let\PY@ff=\relax}
\def\PY@tok#1{\csname PY@tok@#1\endcsname}
\def\PY@toks#1+{\ifx\relax#1\empty\else%
    \PY@tok{#1}\expandafter\PY@toks\fi}
\def\PY@do#1{\PY@bc{\PY@tc{\PY@ul{%
    \PY@it{\PY@bf{\PY@ff{#1}}}}}}}
\def\PY#1#2{\PY@reset\PY@toks#1+\relax+\PY@do{#2}}

\@namedef{PY@tok@w}{\def\PY@tc##1{\textcolor[rgb]{0.73,0.73,0.73}{##1}}}
\@namedef{PY@tok@c}{\let\PY@it=\textit\def\PY@tc##1{\textcolor[rgb]{0.25,0.50,0.50}{##1}}}
\@namedef{PY@tok@cp}{\def\PY@tc##1{\textcolor[rgb]{0.74,0.48,0.00}{##1}}}
\@namedef{PY@tok@k}{\let\PY@bf=\textbf\def\PY@tc##1{\textcolor[rgb]{0.00,0.50,0.00}{##1}}}
\@namedef{PY@tok@kp}{\def\PY@tc##1{\textcolor[rgb]{0.00,0.50,0.00}{##1}}}
\@namedef{PY@tok@kt}{\def\PY@tc##1{\textcolor[rgb]{0.69,0.00,0.25}{##1}}}
\@namedef{PY@tok@o}{\def\PY@tc##1{\textcolor[rgb]{0.40,0.40,0.40}{##1}}}
\@namedef{PY@tok@ow}{\let\PY@bf=\textbf\def\PY@tc##1{\textcolor[rgb]{0.67,0.13,1.00}{##1}}}
\@namedef{PY@tok@nb}{\def\PY@tc##1{\textcolor[rgb]{0.00,0.50,0.00}{##1}}}
\@namedef{PY@tok@nf}{\def\PY@tc##1{\textcolor[rgb]{0.00,0.00,1.00}{##1}}}
\@namedef{PY@tok@nc}{\let\PY@bf=\textbf\def\PY@tc##1{\textcolor[rgb]{0.00,0.00,1.00}{##1}}}
\@namedef{PY@tok@nn}{\let\PY@bf=\textbf\def\PY@tc##1{\textcolor[rgb]{0.00,0.00,1.00}{##1}}}
\@namedef{PY@tok@ne}{\let\PY@bf=\textbf\def\PY@tc##1{\textcolor[rgb]{0.82,0.25,0.23}{##1}}}
\@namedef{PY@tok@nv}{\def\PY@tc##1{\textcolor[rgb]{0.10,0.09,0.49}{##1}}}
\@namedef{PY@tok@no}{\def\PY@tc##1{\textcolor[rgb]{0.53,0.00,0.00}{##1}}}
\@namedef{PY@tok@nl}{\def\PY@tc##1{\textcolor[rgb]{0.63,0.63,0.00}{##1}}}
\@namedef{PY@tok@ni}{\let\PY@bf=\textbf\def\PY@tc##1{\textcolor[rgb]{0.60,0.60,0.60}{##1}}}
\@namedef{PY@tok@na}{\def\PY@tc##1{\textcolor[rgb]{0.49,0.56,0.16}{##1}}}
\@namedef{PY@tok@nt}{\let\PY@bf=\textbf\def\PY@tc##1{\textcolor[rgb]{0.00,0.50,0.00}{##1}}}
\@namedef{PY@tok@nd}{\def\PY@tc##1{\textcolor[rgb]{0.67,0.13,1.00}{##1}}}
\@namedef{PY@tok@s}{\def\PY@tc##1{\textcolor[rgb]{0.73,0.13,0.13}{##1}}}
\@namedef{PY@tok@sd}{\let\PY@it=\textit\def\PY@tc##1{\textcolor[rgb]{0.73,0.13,0.13}{##1}}}
\@namedef{PY@tok@si}{\let\PY@bf=\textbf\def\PY@tc##1{\textcolor[rgb]{0.73,0.40,0.53}{##1}}}
\@namedef{PY@tok@se}{\let\PY@bf=\textbf\def\PY@tc##1{\textcolor[rgb]{0.73,0.40,0.13}{##1}}}
\@namedef{PY@tok@sr}{\def\PY@tc##1{\textcolor[rgb]{0.73,0.40,0.53}{##1}}}
\@namedef{PY@tok@ss}{\def\PY@tc##1{\textcolor[rgb]{0.10,0.09,0.49}{##1}}}
\@namedef{PY@tok@sx}{\def\PY@tc##1{\textcolor[rgb]{0.00,0.50,0.00}{##1}}}
\@namedef{PY@tok@m}{\def\PY@tc##1{\textcolor[rgb]{0.40,0.40,0.40}{##1}}}
\@namedef{PY@tok@gh}{\let\PY@bf=\textbf\def\PY@tc##1{\textcolor[rgb]{0.00,0.00,0.50}{##1}}}
\@namedef{PY@tok@gu}{\let\PY@bf=\textbf\def\PY@tc##1{\textcolor[rgb]{0.50,0.00,0.50}{##1}}}
\@namedef{PY@tok@gd}{\def\PY@tc##1{\textcolor[rgb]{0.63,0.00,0.00}{##1}}}
\@namedef{PY@tok@gi}{\def\PY@tc##1{\textcolor[rgb]{0.00,0.63,0.00}{##1}}}
\@namedef{PY@tok@gr}{\def\PY@tc##1{\textcolor[rgb]{1.00,0.00,0.00}{##1}}}
\@namedef{PY@tok@ge}{\let\PY@it=\textit}
\@namedef{PY@tok@gs}{\let\PY@bf=\textbf}
\@namedef{PY@tok@gp}{\let\PY@bf=\textbf\def\PY@tc##1{\textcolor[rgb]{0.00,0.00,0.50}{##1}}}
\@namedef{PY@tok@go}{\def\PY@tc##1{\textcolor[rgb]{0.53,0.53,0.53}{##1}}}
\@namedef{PY@tok@gt}{\def\PY@tc##1{\textcolor[rgb]{0.00,0.27,0.87}{##1}}}
\@namedef{PY@tok@err}{\def\PY@bc##1{{\setlength{\fboxsep}{\string -\fboxrule}\fcolorbox[rgb]{1.00,0.00,0.00}{1,1,1}{\strut ##1}}}}
\@namedef{PY@tok@kc}{\let\PY@bf=\textbf\def\PY@tc##1{\textcolor[rgb]{0.00,0.50,0.00}{##1}}}
\@namedef{PY@tok@kd}{\let\PY@bf=\textbf\def\PY@tc##1{\textcolor[rgb]{0.00,0.50,0.00}{##1}}}
\@namedef{PY@tok@kn}{\let\PY@bf=\textbf\def\PY@tc##1{\textcolor[rgb]{0.00,0.50,0.00}{##1}}}
\@namedef{PY@tok@kr}{\let\PY@bf=\textbf\def\PY@tc##1{\textcolor[rgb]{0.00,0.50,0.00}{##1}}}
\@namedef{PY@tok@bp}{\def\PY@tc##1{\textcolor[rgb]{0.00,0.50,0.00}{##1}}}
\@namedef{PY@tok@fm}{\def\PY@tc##1{\textcolor[rgb]{0.00,0.00,1.00}{##1}}}
\@namedef{PY@tok@vc}{\def\PY@tc##1{\textcolor[rgb]{0.10,0.09,0.49}{##1}}}
\@namedef{PY@tok@vg}{\def\PY@tc##1{\textcolor[rgb]{0.10,0.09,0.49}{##1}}}
\@namedef{PY@tok@vi}{\def\PY@tc##1{\textcolor[rgb]{0.10,0.09,0.49}{##1}}}
\@namedef{PY@tok@vm}{\def\PY@tc##1{\textcolor[rgb]{0.10,0.09,0.49}{##1}}}
\@namedef{PY@tok@sa}{\def\PY@tc##1{\textcolor[rgb]{0.73,0.13,0.13}{##1}}}
\@namedef{PY@tok@sb}{\def\PY@tc##1{\textcolor[rgb]{0.73,0.13,0.13}{##1}}}
\@namedef{PY@tok@sc}{\def\PY@tc##1{\textcolor[rgb]{0.73,0.13,0.13}{##1}}}
\@namedef{PY@tok@dl}{\def\PY@tc##1{\textcolor[rgb]{0.73,0.13,0.13}{##1}}}
\@namedef{PY@tok@s2}{\def\PY@tc##1{\textcolor[rgb]{0.73,0.13,0.13}{##1}}}
\@namedef{PY@tok@sh}{\def\PY@tc##1{\textcolor[rgb]{0.73,0.13,0.13}{##1}}}
\@namedef{PY@tok@s1}{\def\PY@tc##1{\textcolor[rgb]{0.73,0.13,0.13}{##1}}}
\@namedef{PY@tok@mb}{\def\PY@tc##1{\textcolor[rgb]{0.40,0.40,0.40}{##1}}}
\@namedef{PY@tok@mf}{\def\PY@tc##1{\textcolor[rgb]{0.40,0.40,0.40}{##1}}}
\@namedef{PY@tok@mh}{\def\PY@tc##1{\textcolor[rgb]{0.40,0.40,0.40}{##1}}}
\@namedef{PY@tok@mi}{\def\PY@tc##1{\textcolor[rgb]{0.40,0.40,0.40}{##1}}}
\@namedef{PY@tok@il}{\def\PY@tc##1{\textcolor[rgb]{0.40,0.40,0.40}{##1}}}
\@namedef{PY@tok@mo}{\def\PY@tc##1{\textcolor[rgb]{0.40,0.40,0.40}{##1}}}
\@namedef{PY@tok@ch}{\let\PY@it=\textit\def\PY@tc##1{\textcolor[rgb]{0.25,0.50,0.50}{##1}}}
\@namedef{PY@tok@cm}{\let\PY@it=\textit\def\PY@tc##1{\textcolor[rgb]{0.25,0.50,0.50}{##1}}}
\@namedef{PY@tok@cpf}{\let\PY@it=\textit\def\PY@tc##1{\textcolor[rgb]{0.25,0.50,0.50}{##1}}}
\@namedef{PY@tok@c1}{\let\PY@it=\textit\def\PY@tc##1{\textcolor[rgb]{0.25,0.50,0.50}{##1}}}
\@namedef{PY@tok@cs}{\let\PY@it=\textit\def\PY@tc##1{\textcolor[rgb]{0.25,0.50,0.50}{##1}}}

\def\PYZbs{\char`\\}
\def\PYZus{\char`\_}
\def\PYZob{\char`\{}
\def\PYZcb{\char`\}}
\def\PYZca{\char`\^}
\def\PYZam{\char`\&}
\def\PYZlt{\char`\<}
\def\PYZgt{\char`\>}
\def\PYZsh{\char`\#}
\def\PYZpc{\char`\%}
\def\PYZdl{\char`\$}
\def\PYZhy{\char`\-}
\def\PYZsq{\char`\'}
\def\PYZdq{\char`\"}
\def\PYZti{\char`\~}
% for compatibility with earlier versions
\def\PYZat{@}
\def\PYZlb{[}
\def\PYZrb{]}
\makeatother


    % For linebreaks inside Verbatim environment from package fancyvrb. 
    \makeatletter
        \newbox\Wrappedcontinuationbox 
        \newbox\Wrappedvisiblespacebox 
        \newcommand*\Wrappedvisiblespace {\textcolor{red}{\textvisiblespace}} 
        \newcommand*\Wrappedcontinuationsymbol {\textcolor{red}{\llap{\tiny$\m@th\hookrightarrow$}}} 
        \newcommand*\Wrappedcontinuationindent {3ex } 
        \newcommand*\Wrappedafterbreak {\kern\Wrappedcontinuationindent\copy\Wrappedcontinuationbox} 
        % Take advantage of the already applied Pygments mark-up to insert 
        % potential linebreaks for TeX processing. 
        %        {, <, #, %, $, ' and ": go to next line. 
        %        _, }, ^, &, >, - and ~: stay at end of broken line. 
        % Use of \textquotesingle for straight quote. 
        \newcommand*\Wrappedbreaksatspecials {% 
            \def\PYGZus{\discretionary{\char`\_}{\Wrappedafterbreak}{\char`\_}}% 
            \def\PYGZob{\discretionary{}{\Wrappedafterbreak\char`\{}{\char`\{}}% 
            \def\PYGZcb{\discretionary{\char`\}}{\Wrappedafterbreak}{\char`\}}}% 
            \def\PYGZca{\discretionary{\char`\^}{\Wrappedafterbreak}{\char`\^}}% 
            \def\PYGZam{\discretionary{\char`\&}{\Wrappedafterbreak}{\char`\&}}% 
            \def\PYGZlt{\discretionary{}{\Wrappedafterbreak\char`\<}{\char`\<}}% 
            \def\PYGZgt{\discretionary{\char`\>}{\Wrappedafterbreak}{\char`\>}}% 
            \def\PYGZsh{\discretionary{}{\Wrappedafterbreak\char`\#}{\char`\#}}% 
            \def\PYGZpc{\discretionary{}{\Wrappedafterbreak\char`\%}{\char`\%}}% 
            \def\PYGZdl{\discretionary{}{\Wrappedafterbreak\char`\$}{\char`\$}}% 
            \def\PYGZhy{\discretionary{\char`\-}{\Wrappedafterbreak}{\char`\-}}% 
            \def\PYGZsq{\discretionary{}{\Wrappedafterbreak\textquotesingle}{\textquotesingle}}% 
            \def\PYGZdq{\discretionary{}{\Wrappedafterbreak\char`\"}{\char`\"}}% 
            \def\PYGZti{\discretionary{\char`\~}{\Wrappedafterbreak}{\char`\~}}% 
        } 
        % Some characters . , ; ? ! / are not pygmentized. 
        % This macro makes them "active" and they will insert potential linebreaks 
        \newcommand*\Wrappedbreaksatpunct {% 
            \lccode`\~`\.\lowercase{\def~}{\discretionary{\hbox{\char`\.}}{\Wrappedafterbreak}{\hbox{\char`\.}}}% 
            \lccode`\~`\,\lowercase{\def~}{\discretionary{\hbox{\char`\,}}{\Wrappedafterbreak}{\hbox{\char`\,}}}% 
            \lccode`\~`\;\lowercase{\def~}{\discretionary{\hbox{\char`\;}}{\Wrappedafterbreak}{\hbox{\char`\;}}}% 
            \lccode`\~`\:\lowercase{\def~}{\discretionary{\hbox{\char`\:}}{\Wrappedafterbreak}{\hbox{\char`\:}}}% 
            \lccode`\~`\?\lowercase{\def~}{\discretionary{\hbox{\char`\?}}{\Wrappedafterbreak}{\hbox{\char`\?}}}% 
            \lccode`\~`\!\lowercase{\def~}{\discretionary{\hbox{\char`\!}}{\Wrappedafterbreak}{\hbox{\char`\!}}}% 
            \lccode`\~`\/\lowercase{\def~}{\discretionary{\hbox{\char`\/}}{\Wrappedafterbreak}{\hbox{\char`\/}}}% 
            \catcode`\.\active
            \catcode`\,\active 
            \catcode`\;\active
            \catcode`\:\active
            \catcode`\?\active
            \catcode`\!\active
            \catcode`\/\active 
            \lccode`\~`\~ 	
        }
    \makeatother

    \let\OriginalVerbatim=\Verbatim
    \makeatletter
    \renewcommand{\Verbatim}[1][1]{%
        %\parskip\z@skip
        \sbox\Wrappedcontinuationbox {\Wrappedcontinuationsymbol}%
        \sbox\Wrappedvisiblespacebox {\FV@SetupFont\Wrappedvisiblespace}%
        \def\FancyVerbFormatLine ##1{\hsize\linewidth
            \vtop{\raggedright\hyphenpenalty\z@\exhyphenpenalty\z@
                \doublehyphendemerits\z@\finalhyphendemerits\z@
                \strut ##1\strut}%
        }%
        % If the linebreak is at a space, the latter will be displayed as visible
        % space at end of first line, and a continuation symbol starts next line.
        % Stretch/shrink are however usually zero for typewriter font.
        \def\FV@Space {%
            \nobreak\hskip\z@ plus\fontdimen3\font minus\fontdimen4\font
            \discretionary{\copy\Wrappedvisiblespacebox}{\Wrappedafterbreak}
            {\kern\fontdimen2\font}%
        }%
        
        % Allow breaks at special characters using \PYG... macros.
        \Wrappedbreaksatspecials
        % Breaks at punctuation characters . , ; ? ! and / need catcode=\active 	
        \OriginalVerbatim[#1,codes*=\Wrappedbreaksatpunct]%
    }
    \makeatother

    % Exact colors from NB
    \definecolor{incolor}{HTML}{303F9F}
    \definecolor{outcolor}{HTML}{D84315}
    \definecolor{cellborder}{HTML}{CFCFCF}
    \definecolor{cellbackground}{HTML}{F7F7F7}
    
    % prompt
    \makeatletter
    \newcommand{\boxspacing}{\kern\kvtcb@left@rule\kern\kvtcb@boxsep}
    \makeatother
    \newcommand{\prompt}[4]{
        {\ttfamily\llap{{\color{#2}[#3]:\hspace{3pt}#4}}\vspace{-\baselineskip}}
    }
    

    
    % Prevent overflowing lines due to hard-to-break entities
    \sloppy 
    % Setup hyperref package
    \hypersetup{
      breaklinks=true,  % so long urls are correctly broken across lines
      colorlinks=true,
      urlcolor=urlcolor,
      linkcolor=linkcolor,
      citecolor=citecolor,
      }
    % Slightly bigger margins than the latex defaults
    
    \geometry{verbose,tmargin=1in,bmargin=1in,lmargin=1in,rmargin=1in}
    
    

\begin{document}
    
    \title{Tarea 5 Optimización}
    \author{Roberto Vásquez Martínez \\ Profesor: Joaquín Peña Acevedo}
    \date{13/Marzo/2022}
    \maketitle   
    
    

    
    \hypertarget{ejercicio-1-6-puntos}{%
\section{Ejercicio 1 (6 puntos)}\label{ejercicio-1-6-puntos}}

Programar el método de descenso máximo con tamaño de paso fijo y
probarlo.

El algoritmo recibe como parámetros la función gradiente \(g(x)\) de la
función objetivo, un punto inicial \(x_0\), el valor del tamaño de paso
\(\alpha\), un número máximo de iteraciones \(N\), la tolerancia
\(\tau>0\). Fijar \(k=0\) y repetir los siguientes pasos:

\begin{enumerate}
\def\labelenumi{\arabic{enumi}.}
\tightlist
\item
  Calcular el gradiente \(g_k\) en el punto \(x_k\), \(g_k = g(x_k)\).
\item
  Si \(\|g_k\| < \tau\), hacer \(res=1\) y terminar.
\item
  Elegir la dirección de descenso como \(p_k = - g_k\).
\item
  Calcular el siguiente punto de la secuencia como
  \[x_{k+1} = x_k + \alpha p_k \]
\item
  Si \(k+1\geq N\), hacer \(res=0\) y terminar.
\item
  Si no, hacer \(k = k+1\) y volver el paso 1.
\item
  Devolver el punto \(x_k\), \(g_k\), \(k\) y \(res\).
\end{enumerate}

\begin{center}\rule{0.5\linewidth}{0.5pt}\end{center}

De acuerdo con la proposición vista en la clase 12, para que el método
de máximo descenso con paso fijo para funciones cuadráticas converja se
requiere que el tamaño de paso \(\alpha\) cumpla con
\[ 0 < \alpha < \frac{2}{\lambda_{\max}(A)} = \alpha_{\max}, \] donde
\(\lambda_{\max}(A)\) es el eigenvalor más grande de \(A\).

\begin{enumerate}
\def\labelenumi{\arabic{enumi}.}
\item
  Escriba una función que implementa el algoritmo de descenso máximo con
  paso fijo.
\item
  Programe las funciones cuadráticas y sus gradientes
  \[ f_i(x) = \frac{1}{2} x^\top \mathbf{A}_i x - \mathbf{b}^\top_i x, \quad i=1,2 \]
  donde \[ \mathbf{A}_1 = \left[ \begin{array}{cc}
  1.18 & 0.69 \\
  0.69 & 3.01 
  \end{array} \right],\qquad 
  \mathbf{b}_1 = \left( \begin{array}{r} -0.24 \\ 0.99 \end{array} \right).\]

  y

  \[\mathbf{A}_2=\begin{pmatrix}6.36 & -3.07 & -2.8  & -3.42 & -0.68\\-3.07 & 10.19 &  0.74 &  0.5  & 0.72\\-2.8  & 0.74  &  4.97 & -1.48 & 1.93\\-3.42 & 0.5   & -1.48 &  4.9  & -0.97\\-0.68 & 0.72  &  1.93 & -0.97 & 3.21\end{pmatrix},\qquad\mathbf{b}_2 = \left( \begin{array}{r} 0.66 \\ 0.37  \\ -2.06  \\ 0.14 \\ 1.36 \end{array} \right).\]
\item
  Fije el número máximo de iteraciones \(N=2000\) y la tolerancia
  \(\tau =\sqrt{\epsilon_m}\), donde \(\epsilon_m\) es el épsilon de la
  máquina. Para cada función cuadrática, calcule \(\alpha_{\max}\) de la
  matriz \(\mathbf{A}_i\). Pruebe con los tamaños de paso \(\alpha\)
  igual a \(1.1\alpha_{\max}\) y \(0.9\alpha_{\max}\). Use el punto
  inicial

  \[
  \mathbf{x}_0 = 
  \left( \begin{array}{r} -38.12 \\ -55.87  \end{array} \right) \quad \text{para} \quad f_1
  \]

  \[
  \mathbf{x}_0 = 
  \left( \begin{array}{r} 4.60 \\  6.85 \\  4.31 \\  4.79 \\  8.38  
  \end{array} \right) \quad \text{para} \quad f_2
  \]
\item
  En cada caso imprima \(x_k\), \(\|g_k\|\), el número de iteraciones
  \(k\) y el valor de \(res\).
\end{enumerate}

\hypertarget{soluciuxf3n}{%
\subsection{Solución:}\label{soluciuxf3n}}

    El número de iteraciones que consideraremos en el método de descenso
máximo con paso fijo será \(N=2000\) y la tolerancia
\(\tau=\sqrt{\epsilon_m}\), donde \(\epsilon_m\) es el épsilon de la
máquina.

Calcularemos \(\alpha_{\max}\) para cada función cuadrática \(f_i\).
Probaremos el método de descenso máximo con paso fijo con los tamaños de
paso \(0.9\alpha_{\max}\) y \(1.1\alpha_{\max}\).

\hypertarget{funciuxf3n-cuadruxe1tica-1}{%
\subsubsection{Función cuadrática 1}\label{funciuxf3n-cuadruxe1tica-1}}

Haremos las pruebas correspondientes a la función \(f_1\). En primer
lugar, determinamos \(\alpha_{\max}\), además de validar si
\(\mathbf{A}_1\) es positiva definida.

    \begin{tcolorbox}[breakable, size=fbox, boxrule=1pt, pad at break*=1mm,colback=cellbackground, colframe=cellborder]
\prompt{In}{incolor}{75}{\boxspacing}
\begin{Verbatim}[commandchars=\\\{\}]
\PY{k+kn}{import} \PY{n+nn}{lib\PYZus{}t5}
\PY{k+kn}{import} \PY{n+nn}{importlib}
\PY{n}{importlib}\PY{o}{.}\PY{n}{reload}\PY{p}{(}\PY{n}{lib\PYZus{}t5}\PY{p}{)}
\PY{k+kn}{from} \PY{n+nn}{lib\PYZus{}t5} \PY{k+kn}{import} \PY{o}{*}

\PY{c+c1}{\PYZsh{} Iteraciones maximas, tolerancia}
\PY{n}{N}\PY{o}{=}\PY{l+m+mi}{2000}
\PY{n}{tol}\PY{o}{=}\PY{n}{np}\PY{o}{.}\PY{n}{finfo}\PY{p}{(}\PY{n+nb}{float}\PY{p}{)}\PY{o}{.}\PY{n}{eps}\PY{o}{*}\PY{o}{*}\PY{p}{(}\PY{l+m+mi}{1}\PY{o}{/}\PY{l+m+mi}{2}\PY{p}{)}

\PY{c+c1}{\PYZsh{} Matriz y vector modelo cuadrático}
\PY{n}{A1}\PY{p}{,}\PY{n}{b1}\PY{o}{=}\PY{n}{np}\PY{o}{.}\PY{n}{array}\PY{p}{(}\PY{p}{[}\PY{p}{[}\PY{l+m+mf}{1.18}\PY{p}{,}\PY{l+m+mf}{0.69}\PY{p}{]}\PY{p}{,}\PY{p}{[}\PY{l+m+mf}{0.69}\PY{p}{,}\PY{l+m+mf}{3.01}\PY{p}{]}\PY{p}{]}\PY{p}{)}\PY{p}{,}\PY{n}{np}\PY{o}{.}\PY{n}{array}\PY{p}{(}\PY{p}{[}\PY{o}{\PYZhy{}}\PY{l+m+mf}{0.24}\PY{p}{,}\PY{l+m+mf}{0.99}\PY{p}{]}\PY{p}{)}\PY{o}{.}\PY{n}{reshape}\PY{p}{(}\PY{o}{\PYZhy{}}\PY{l+m+mi}{1}\PY{p}{,}\PY{l+m+mi}{1}\PY{p}{)}
\PY{c+c1}{\PYZsh{} Condición inicial}
\PY{n}{x0}\PY{o}{=}\PY{n}{np}\PY{o}{.}\PY{n}{array}\PY{p}{(}\PY{p}{[}\PY{o}{\PYZhy{}}\PY{l+m+mf}{38.12}\PY{p}{,}\PY{o}{\PYZhy{}}\PY{l+m+mf}{55.87}\PY{p}{]}\PY{p}{)}\PY{o}{.}\PY{n}{reshape}\PY{p}{(}\PY{o}{\PYZhy{}}\PY{l+m+mi}{1}\PY{p}{,}\PY{l+m+mi}{1}\PY{p}{)}

\PY{c+c1}{\PYZsh{} Verificar si es positiva definida}
\PY{n}{eig\PYZus{}val\PYZus{}A1}\PY{o}{=}\PY{n}{np}\PY{o}{.}\PY{n}{real\PYZus{}if\PYZus{}close}\PY{p}{(}\PY{n}{np}\PY{o}{.}\PY{n}{linalg}\PY{o}{.}\PY{n}{eigvals}\PY{p}{(}\PY{n}{A1}\PY{p}{)}\PY{p}{)}
\PY{n+nb}{print}\PY{p}{(}\PY{l+s+s1}{\PYZsq{}}\PY{l+s+s1}{El valor eigenvalor mínimo es: }\PY{l+s+s1}{\PYZsq{}}\PY{p}{,}\PY{n}{np}\PY{o}{.}\PY{n}{min}\PY{p}{(}\PY{n}{eig\PYZus{}val\PYZus{}A1}\PY{p}{)}\PY{p}{)}
\PY{c+c1}{\PYZsh{} Calcula a\PYZus{}max}
\PY{n}{a\PYZus{}max\PYZus{}1}\PY{o}{=}\PY{l+m+mf}{2.0}\PY{o}{/}\PY{n}{np}\PY{o}{.}\PY{n}{max}\PY{p}{(}\PY{n}{eig\PYZus{}val\PYZus{}A1}\PY{p}{)}
\PY{n+nb}{print}\PY{p}{(}\PY{l+s+s1}{\PYZsq{}}\PY{l+s+s1}{a\PYZus{}max = }\PY{l+s+s1}{\PYZsq{}}\PY{p}{,}\PY{n}{a\PYZus{}max\PYZus{}1}\PY{p}{)}
\end{Verbatim}
\end{tcolorbox}

    \begin{Verbatim}[commandchars=\\\{\}]
El valor eigenvalor mínimo es:  0.9489960733051563
a\_max =  0.6170927420133021
    \end{Verbatim}

    De lo anterior tenemos que \(\mathbf{A}_1\) es positiva definida, luego
el punto crítico es mínimo global. Ahora haremos las pruebas con los
distintos tamaños de pasos.

\hypertarget{alpha_max}{%
\paragraph{\texorpdfstring{\(0.9\alpha_{\max}\)}{0.9\textbackslash alpha\_\{\textbackslash max\}}}\label{alpha_max}}

Para este tamaño de paso, imprimimos el número de iteraciones, el valor
en el mínimo y la distancia respecto al punto crítico para validar que
en efecto convergemos al mínimo global

    \begin{tcolorbox}[breakable, size=fbox, boxrule=1pt, pad at break*=1mm,colback=cellbackground, colframe=cellborder]
\prompt{In}{incolor}{76}{\boxspacing}
\begin{Verbatim}[commandchars=\\\{\}]
\PY{n}{proof\PYZus{}grad\PYZus{}max\PYZus{}fix\PYZus{}quad}\PY{p}{(}\PY{n}{quad\PYZus{}fun}\PY{p}{,}\PY{n}{grad\PYZus{}quad\PYZus{}fun}\PY{p}{,}\PY{n}{x0}\PY{p}{,}\PY{n}{N}\PY{p}{,}\PY{n}{tol}\PY{p}{,}\PY{l+m+mf}{0.9}\PY{o}{*}\PY{n}{a\PYZus{}max\PYZus{}1}\PY{p}{,}\PY{n}{args}\PY{o}{=}\PY{p}{[}\PY{n}{A1}\PY{p}{,}\PY{n}{b1}\PY{p}{]}\PY{p}{)}
\end{Verbatim}
\end{tcolorbox}

    \begin{Verbatim}[commandchars=\\\{\}]
res =  1
El método de descenso máximo con paso fijo CONVERGE
k =  105
fk =  -0.26949669993822545
||gk|| =  1.4137071703555776e-08
xk =  [-0.45696914  0.43365738]
||xk-x*|| =  4.361942126279292e-09
    \end{Verbatim}

    La convergencia esta asegurada por la Proposición de la Clase 12 para
cualquier condición inicial pues \(0.9\alpha_{\max}<\alpha_{\max}\), y
en este caso la convergencia fue relativamente rápida.

    \hypertarget{alpha_max}{%
\paragraph{\texorpdfstring{\(1.1\alpha_{\max}\)}{1.1\textbackslash alpha\_\{\textbackslash max\}}}\label{alpha_max}}

En este caso no tenemos segura la convergencia, realizamos la misma
prueba para este tamaño de paso

    \begin{tcolorbox}[breakable, size=fbox, boxrule=1pt, pad at break*=1mm,colback=cellbackground, colframe=cellborder]
\prompt{In}{incolor}{78}{\boxspacing}
\begin{Verbatim}[commandchars=\\\{\}]
\PY{n}{proof\PYZus{}grad\PYZus{}max\PYZus{}fix\PYZus{}quad}\PY{p}{(}\PY{n}{quad\PYZus{}fun}\PY{p}{,}\PY{n}{grad\PYZus{}quad\PYZus{}fun}\PY{p}{,}\PY{n}{x0}\PY{p}{,}\PY{n}{N}\PY{p}{,}\PY{n}{tol}\PY{p}{,}\PY{l+m+mf}{1.1}\PY{o}{*}\PY{n}{a\PYZus{}max\PYZus{}1}\PY{p}{,}\PY{n}{args}\PY{o}{=}\PY{p}{[}\PY{n}{A1}\PY{p}{,}\PY{n}{b1}\PY{p}{]}\PY{p}{)}
\end{Verbatim}
\end{tcolorbox}

    \begin{Verbatim}[commandchars=\\\{\}]
res =  0
El método de descenso máximo con paso exacto NO CONVERGE
k =  2000
fk =  inf
||gk|| =  inf
xk =  [-4.77996787e+159 -1.42775834e+160]
||xk-x*|| =  inf
    \end{Verbatim}

    En este caso por completo divergemos, el valor de la función aumenta
cada vez más, incluso diverge la norma del gradiente. Nos alejamos tanto
del punto crítico que la distancia final respecto al punto crítico es
\(\infty\).

El cambio fue sutil en el tamaño de paso fijo, pero nos llevó a un
resultado degenerado.

\hypertarget{funciuxf3n-cuadruxe1tica-2}{%
\subsubsection{Función cuadrática 2}\label{funciuxf3n-cuadruxe1tica-2}}

Al igual que la función \(f_1\), primero validamos si \(\mathbf{A}_2\) es
positiva definida además de calcular \(\alpha_{\max}\) para esta función
cuadrática.

    \begin{tcolorbox}[breakable, size=fbox, boxrule=1pt, pad at break*=1mm,colback=cellbackground, colframe=cellborder]
\prompt{In}{incolor}{79}{\boxspacing}
\begin{Verbatim}[commandchars=\\\{\}]
\PY{c+c1}{\PYZsh{} Matriz y vector modelo cuadrático}
\PY{n}{A2}\PY{p}{,}\PY{n}{b2}\PY{o}{=}\PY{n}{np}\PY{o}{.}\PY{n}{array}\PY{p}{(}\PY{p}{[}\PY{p}{[}\PY{l+m+mf}{6.36}\PY{p}{,}\PY{o}{\PYZhy{}}\PY{l+m+mf}{3.07}\PY{p}{,}\PY{o}{\PYZhy{}}\PY{l+m+mf}{2.8}\PY{p}{,}\PY{o}{\PYZhy{}}\PY{l+m+mf}{3.42}\PY{p}{,}\PY{o}{\PYZhy{}}\PY{l+m+mf}{0.68}\PY{p}{]}\PY{p}{,}\PY{p}{[}\PY{o}{\PYZhy{}}\PY{l+m+mf}{3.07}\PY{p}{,}\PY{l+m+mf}{10.19}\PY{p}{,}\PY{l+m+mf}{0.74}\PY{p}{,}\PY{l+m+mf}{0.5}\PY{p}{,}\PY{l+m+mf}{0.72}\PY{p}{]}\PY{p}{,}\PY{p}{[}\PY{o}{\PYZhy{}}\PY{l+m+mf}{2.8}\PY{p}{,}\PY{l+m+mf}{0.74}\PY{p}{,}\PY{l+m+mf}{4.97}\PY{p}{,}\PY{o}{\PYZhy{}}\PY{l+m+mf}{1.48}\PY{p}{,}\PY{l+m+mf}{1.93}\PY{p}{]}\PY{p}{,}\PY{p}{[}\PY{o}{\PYZhy{}}\PY{l+m+mf}{3.42}\PY{p}{,}\PY{l+m+mf}{0.5}\PY{p}{,}\PY{o}{\PYZhy{}}\PY{l+m+mf}{1.48}\PY{p}{,}\PY{l+m+mf}{4.9}\PY{p}{,}\PY{o}{\PYZhy{}}\PY{l+m+mf}{0.97}\PY{p}{]}\PY{p}{,}\PY{p}{[}\PY{o}{\PYZhy{}}\PY{l+m+mf}{0.68}\PY{p}{,}\PY{l+m+mf}{0.72}\PY{p}{,}\PY{l+m+mf}{1.93}\PY{p}{,}\PY{o}{\PYZhy{}}\PY{l+m+mf}{0.97}\PY{p}{,}\PY{l+m+mf}{3.21}\PY{p}{]}\PY{p}{]}\PY{p}{)}\PY{p}{,}\PY{n}{np}\PY{o}{.}\PY{n}{array}\PY{p}{(}\PY{p}{[}\PY{l+m+mf}{0.66}\PY{p}{,}\PY{l+m+mf}{0.37}\PY{p}{,}\PY{o}{\PYZhy{}}\PY{l+m+mf}{2.06}\PY{p}{,}\PY{l+m+mf}{0.14}\PY{p}{,}\PY{l+m+mf}{1.36}\PY{p}{]}\PY{p}{)}\PY{o}{.}\PY{n}{reshape}\PY{p}{(}\PY{o}{\PYZhy{}}\PY{l+m+mi}{1}\PY{p}{,}\PY{l+m+mi}{1}\PY{p}{)}

\PY{c+c1}{\PYZsh{} Condición inicial}
\PY{n}{x0}\PY{o}{=}\PY{n}{np}\PY{o}{.}\PY{n}{array}\PY{p}{(}\PY{p}{[}\PY{l+m+mf}{4.60}\PY{p}{,}\PY{l+m+mf}{6.85}\PY{p}{,}\PY{l+m+mf}{4.31}\PY{p}{,}\PY{l+m+mf}{4.79}\PY{p}{,}\PY{l+m+mf}{8.38}\PY{p}{]}\PY{p}{)}\PY{o}{.}\PY{n}{reshape}\PY{p}{(}\PY{o}{\PYZhy{}}\PY{l+m+mi}{1}\PY{p}{,}\PY{l+m+mi}{1}\PY{p}{)}

\PY{c+c1}{\PYZsh{} Verificar si es positiva definida}
\PY{n}{eig\PYZus{}val\PYZus{}A2}\PY{o}{=}\PY{n}{np}\PY{o}{.}\PY{n}{real\PYZus{}if\PYZus{}close}\PY{p}{(}\PY{n}{np}\PY{o}{.}\PY{n}{linalg}\PY{o}{.}\PY{n}{eigvals}\PY{p}{(}\PY{n}{A2}\PY{p}{)}\PY{p}{)}
\PY{n+nb}{print}\PY{p}{(}\PY{l+s+s1}{\PYZsq{}}\PY{l+s+s1}{El valor eigenvalor mínimo es: }\PY{l+s+s1}{\PYZsq{}}\PY{p}{,}\PY{n}{np}\PY{o}{.}\PY{n}{min}\PY{p}{(}\PY{n}{eig\PYZus{}val\PYZus{}A2}\PY{p}{)}\PY{p}{)}

\PY{c+c1}{\PYZsh{} Calcula a\PYZus{}max}
\PY{n}{a\PYZus{}max\PYZus{}2}\PY{o}{=}\PY{l+m+mf}{2.0}\PY{o}{/}\PY{n}{np}\PY{o}{.}\PY{n}{max}\PY{p}{(}\PY{n}{eig\PYZus{}val\PYZus{}A2}\PY{p}{)}
\PY{n+nb}{print}\PY{p}{(}\PY{l+s+s1}{\PYZsq{}}\PY{l+s+s1}{a\PYZus{}max = }\PY{l+s+s1}{\PYZsq{}}\PY{p}{,}\PY{n}{a\PYZus{}max\PYZus{}2}\PY{p}{)}
\end{Verbatim}
\end{tcolorbox}

    \begin{Verbatim}[commandchars=\\\{\}]
El valor eigenvalor mínimo es:  0.12547412774810465
a\_max =  0.1529725843120685
    \end{Verbatim}

    Del resultado anterior concluimos que \(\mathbf{A}_2\) es una matriz simétrica
positiva definida, luego el punto crítico es óptimo global. Haremos las
pruebas con los distintos tamaños de pasos.

\hypertarget{alpha_max}{%
\paragraph{\texorpdfstring{\(0.9\alpha_{\max}\)}{0.9\textbackslash alpha\_\{\textbackslash max\}}}\label{alpha_max}}

Imprimimos el número de iteraciones, el valor en el mínimo y la
distancia respecto al punto crítico para validar que en efecto
convergemos al mínimo global.

    \begin{tcolorbox}[breakable, size=fbox, boxrule=1pt, pad at break*=1mm,colback=cellbackground, colframe=cellborder]
\prompt{In}{incolor}{80}{\boxspacing}
\begin{Verbatim}[commandchars=\\\{\}]
\PY{n}{proof\PYZus{}grad\PYZus{}max\PYZus{}fix\PYZus{}quad}\PY{p}{(}\PY{n}{quad\PYZus{}fun}\PY{p}{,}\PY{n}{grad\PYZus{}quad\PYZus{}fun}\PY{p}{,}\PY{n}{x0}\PY{p}{,}\PY{n}{N}\PY{p}{,}\PY{n}{tol}\PY{p}{,}\PY{l+m+mf}{0.9}\PY{o}{*}\PY{n}{a\PYZus{}max\PYZus{}2}\PY{p}{,}\PY{n}{args}\PY{o}{=}\PY{p}{[}\PY{n}{A2}\PY{p}{,}\PY{n}{b2}\PY{p}{]}\PY{p}{)}
\end{Verbatim}
\end{tcolorbox}

    \begin{Verbatim}[commandchars=\\\{\}]
res =  1
El método de descenso máximo con paso fijo CONVERGE
k =  1064
fk =  -2.6497175235052906
||gk|| =  1.4711921029622859e-08
xk =  [-2.77194407 -0.52190805 -3.05959477 -2.57614049  1.01464594]
||xk-x*|| =  1.1725061853105777e-07
    \end{Verbatim}

    Ya sabíamos que la convergencia se obtenía por la proposición de la
Clase 12, en este caso tomó más iteraciones que con la función \(f_1\)
pero esto se puede deber a que \(\alpha_{\max}\) para \(f_2\) es más
pequeño que el correspondiente a \(f_1\).

    \hypertarget{alpha_max}{%
\paragraph{\texorpdfstring{\(1.1\alpha_{\max}\)}{1.1\textbackslash alpha\_\{\textbackslash max\}}}\label{alpha_max}}

    En este caso no tenemos segura la convergencia, realizamos la misma
prueba para este tamaño de paso

    \begin{tcolorbox}[breakable, size=fbox, boxrule=1pt, pad at break*=1mm,colback=cellbackground, colframe=cellborder]
\prompt{In}{incolor}{81}{\boxspacing}
\begin{Verbatim}[commandchars=\\\{\}]
\PY{n}{proof\PYZus{}grad\PYZus{}max\PYZus{}fix\PYZus{}quad}\PY{p}{(}\PY{n}{quad\PYZus{}fun}\PY{p}{,}\PY{n}{grad\PYZus{}quad\PYZus{}fun}\PY{p}{,}\PY{n}{x0}\PY{p}{,}\PY{n}{N}\PY{p}{,}\PY{n}{tol}\PY{p}{,}\PY{l+m+mf}{1.1}\PY{o}{*}\PY{n}{a\PYZus{}max\PYZus{}2}\PY{p}{,}\PY{n}{args}\PY{o}{=}\PY{p}{[}\PY{n}{A2}\PY{p}{,}\PY{n}{b2}\PY{p}{]}\PY{p}{)}
\end{Verbatim}
\end{tcolorbox}

    \begin{Verbatim}[commandchars=\\\{\}]
res =  0
El método de descenso máximo con paso exacto NO CONVERGE
k =  2000
fk =  inf
||gk|| =  inf
xk =  [-7.41437598e+158  9.63088421e+158  3.28353818e+158  2.91033121e+158
  1.57034121e+158]
||xk-x*|| =  inf
    \end{Verbatim}

    Al igual que con la función \(f_1\), con el tamaño de paso
\(1.1\alpha_{\max}\) el método de descenso máximo con paso fijo diverge,
tanto que la norma del gradiente y el valor de la función en la última
iteración es infinito, lo que puede significar que con este tamaño de
paso, en lugar de reducir el valor de la función este se incrementa cada
vez más, a pesar de estar considerando direcciones de descenso pues
seguro no se satisface la condición de descenso suficiente.

    \hypertarget{ejercicio-2-4-puntos}{%
\subsection{Ejercicio 2 (4 puntos)}\label{ejercicio-2-4-puntos}}

Pruebe el método de descenso máximo con paso fijo aplicado a la función
de Rosenbrock.

Encuentre un valor adecuado para \(\alpha\) para que el algoritmo
converja. Use como punto inicial el punto \((-12, 10)\).

Imprima \(x_k\), \(\|g_k\|\), el número de iteraciones \(k\) y el valor
de \(res\).

\hypertarget{soluciuxf3n}{%
\subsubsection{Solución:}\label{soluciuxf3n}}

    Haremos la prueba del método de descenso máximo con paso fijo con la
función de Rosenbrock, del que sabemos tiene su mínimo en
\(x_\ast=(1,1)\)

    \begin{tcolorbox}[breakable, size=fbox, boxrule=1pt, pad at break*=1mm,colback=cellbackground, colframe=cellborder]
\prompt{In}{incolor}{82}{\boxspacing}
\begin{Verbatim}[commandchars=\\\{\}]
\PY{n}{importlib}\PY{o}{.}\PY{n}{reload}\PY{p}{(}\PY{n}{lib\PYZus{}t5}\PY{p}{)}
\PY{k+kn}{from} \PY{n+nn}{lib\PYZus{}t5} \PY{k+kn}{import} \PY{o}{*}
\PY{n}{x0}\PY{o}{=}\PY{n}{np}\PY{o}{.}\PY{n}{array}\PY{p}{(}\PY{p}{[}\PY{o}{\PYZhy{}}\PY{l+m+mf}{12.0}\PY{p}{,}\PY{l+m+mf}{10.0}\PY{p}{]}\PY{p}{)}\PY{o}{.}\PY{n}{reshape}\PY{p}{(}\PY{o}{\PYZhy{}}\PY{l+m+mi}{1}\PY{p}{,}\PY{l+m+mi}{1}\PY{p}{)}
\PY{n}{N}\PY{o}{=}\PY{l+m+mf}{1e7}
\PY{n}{proof\PYZus{}grad\PYZus{}max\PYZus{}fix}\PY{p}{(}\PY{n}{f\PYZus{}Rosenbrock}\PY{p}{,}\PY{n}{grad\PYZus{}Rosenbrock}\PY{p}{,}\PY{n}{x0}\PY{p}{,}\PY{n}{N}\PY{p}{,}\PY{n}{tol}\PY{p}{,}\PY{n}{a}\PY{o}{=}\PY{l+m+mf}{0.000036}\PY{p}{,}\PY{n}{args}\PY{o}{=}\PY{k+kc}{None}\PY{p}{)}
\end{Verbatim}
\end{tcolorbox}

    \begin{Verbatim}[commandchars=\\\{\}]
res =  1
El método de descenso máximo con paso fijo CONVERGE
k =  1407272
fk =  2.77999204102933e-16
||gk|| =  1.4901139451473246e-08
xk =  [0.99999998 0.99999997]
    \end{Verbatim}

    En efecto, hemos alcanzado convergencia en \(k=1,407,272\) iteraciones
con tamaño de paso fijo \(\alpha=3.5\times 10^{-5}\). Como observamos en
la tarea anterior, el punto inicial \(x_0=(-12,10)\) se encuentra en un
valle y relativamente alejado del óptimo, incluso con backtracking no
pudimos hallar un tamaño de paso eficiente, así que recurrimos a un
tamaño de paso pequeño pero aumentando el número de iteraciones máximas
para lograr la convergencia. El tiempo de cómputo para la convergencia
fue de 25.4s.

Cabe resaltar que con tamaños de paso un poco más grandes se obtienen
\texttt{NaN}'s al momento de correr el algoritmo, por eso un tamaño de
paso chico fue una mejor opción.


    % Add a bibliography block to the postdoc
    
    
    
\end{document}
