\documentclass[11pt]{article}

    \usepackage[breakable]{tcolorbox}
    \usepackage{parskip} % Stop auto-indenting (to mimic markdown behaviour)
    
    \usepackage{iftex}
    \ifPDFTeX
    	\usepackage[T1]{fontenc}
    	\usepackage{mathpazo}
    \else
    	\usepackage{fontspec}
    \fi

    % Basic figure setup, for now with no caption control since it's done
    % automatically by Pandoc (which extracts ![](path) syntax from Markdown).
    \usepackage{graphicx}
    % Maintain compatibility with old templates. Remove in nbconvert 6.0
    \let\Oldincludegraphics\includegraphics
    % Ensure that by default, figures have no caption (until we provide a
    % proper Figure object with a Caption API and a way to capture that
    % in the conversion process - todo).
    \usepackage{caption}
    \DeclareCaptionFormat{nocaption}{}
    \captionsetup{format=nocaption,aboveskip=0pt,belowskip=0pt}

    \usepackage{float}
    \floatplacement{figure}{H} % forces figures to be placed at the correct location
    \usepackage{xcolor} % Allow colors to be defined
    \usepackage{enumerate} % Needed for markdown enumerations to work
    \usepackage{geometry} % Used to adjust the document margins
    \usepackage{amsmath} % Equations
    \usepackage{amssymb} % Equations
    \usepackage{textcomp} % defines textquotesingle
    % Hack from http://tex.stackexchange.com/a/47451/13684:
    \AtBeginDocument{%
        \def\PYZsq{\textquotesingle}% Upright quotes in Pygmentized code
    }
    \usepackage{upquote} % Upright quotes for verbatim code
    \usepackage{eurosym} % defines \euro
    \usepackage[mathletters]{ucs} % Extended unicode (utf-8) support
    \usepackage{fancyvrb} % verbatim replacement that allows latex
    \usepackage{grffile} % extends the file name processing of package graphics 
                         % to support a larger range
    \makeatletter % fix for old versions of grffile with XeLaTeX
    \@ifpackagelater{grffile}{2019/11/01}
    {
      % Do nothing on new versions
    }
    {
      \def\Gread@@xetex#1{%
        \IfFileExists{"\Gin@base".bb}%
        {\Gread@eps{\Gin@base.bb}}%
        {\Gread@@xetex@aux#1}%
      }
    }
    \makeatother
    \usepackage[Export]{adjustbox} % Used to constrain images to a maximum size
    \adjustboxset{max size={0.9\linewidth}{0.9\paperheight}}

    % The hyperref package gives us a pdf with properly built
    % internal navigation ('pdf bookmarks' for the table of contents,
    % internal cross-reference links, web links for URLs, etc.)
    \usepackage{hyperref}
    % The default LaTeX title has an obnoxious amount of whitespace. By default,
    % titling removes some of it. It also provides customization options.
    \usepackage{titling}
    \usepackage{longtable} % longtable support required by pandoc >1.10
    \usepackage{booktabs}  % table support for pandoc > 1.12.2
    \usepackage[inline]{enumitem} % IRkernel/repr support (it uses the enumerate* environment)
    \usepackage[normalem]{ulem} % ulem is needed to support strikethroughs (\sout)
                                % normalem makes italics be italics, not underlines
    \usepackage{mathrsfs}
    

    
    % Colors for the hyperref package
    \definecolor{urlcolor}{rgb}{0,.145,.698}
    \definecolor{linkcolor}{rgb}{.71,0.21,0.01}
    \definecolor{citecolor}{rgb}{.12,.54,.11}

    % ANSI colors
    \definecolor{ansi-black}{HTML}{3E424D}
    \definecolor{ansi-black-intense}{HTML}{282C36}
    \definecolor{ansi-red}{HTML}{E75C58}
    \definecolor{ansi-red-intense}{HTML}{B22B31}
    \definecolor{ansi-green}{HTML}{00A250}
    \definecolor{ansi-green-intense}{HTML}{007427}
    \definecolor{ansi-yellow}{HTML}{DDB62B}
    \definecolor{ansi-yellow-intense}{HTML}{B27D12}
    \definecolor{ansi-blue}{HTML}{208FFB}
    \definecolor{ansi-blue-intense}{HTML}{0065CA}
    \definecolor{ansi-magenta}{HTML}{D160C4}
    \definecolor{ansi-magenta-intense}{HTML}{A03196}
    \definecolor{ansi-cyan}{HTML}{60C6C8}
    \definecolor{ansi-cyan-intense}{HTML}{258F8F}
    \definecolor{ansi-white}{HTML}{C5C1B4}
    \definecolor{ansi-white-intense}{HTML}{A1A6B2}
    \definecolor{ansi-default-inverse-fg}{HTML}{FFFFFF}
    \definecolor{ansi-default-inverse-bg}{HTML}{000000}

    % common color for the border for error outputs.
    \definecolor{outerrorbackground}{HTML}{FFDFDF}

    % commands and environments needed by pandoc snippets
    % extracted from the output of `pandoc -s`
    \providecommand{\tightlist}{%
      \setlength{\itemsep}{0pt}\setlength{\parskip}{0pt}}
    \DefineVerbatimEnvironment{Highlighting}{Verbatim}{commandchars=\\\{\}}
    % Add ',fontsize=\small' for more characters per line
    \newenvironment{Shaded}{}{}
    \newcommand{\KeywordTok}[1]{\textcolor[rgb]{0.00,0.44,0.13}{\textbf{{#1}}}}
    \newcommand{\DataTypeTok}[1]{\textcolor[rgb]{0.56,0.13,0.00}{{#1}}}
    \newcommand{\DecValTok}[1]{\textcolor[rgb]{0.25,0.63,0.44}{{#1}}}
    \newcommand{\BaseNTok}[1]{\textcolor[rgb]{0.25,0.63,0.44}{{#1}}}
    \newcommand{\FloatTok}[1]{\textcolor[rgb]{0.25,0.63,0.44}{{#1}}}
    \newcommand{\CharTok}[1]{\textcolor[rgb]{0.25,0.44,0.63}{{#1}}}
    \newcommand{\StringTok}[1]{\textcolor[rgb]{0.25,0.44,0.63}{{#1}}}
    \newcommand{\CommentTok}[1]{\textcolor[rgb]{0.38,0.63,0.69}{\textit{{#1}}}}
    \newcommand{\OtherTok}[1]{\textcolor[rgb]{0.00,0.44,0.13}{{#1}}}
    \newcommand{\AlertTok}[1]{\textcolor[rgb]{1.00,0.00,0.00}{\textbf{{#1}}}}
    \newcommand{\FunctionTok}[1]{\textcolor[rgb]{0.02,0.16,0.49}{{#1}}}
    \newcommand{\RegionMarkerTok}[1]{{#1}}
    \newcommand{\ErrorTok}[1]{\textcolor[rgb]{1.00,0.00,0.00}{\textbf{{#1}}}}
    \newcommand{\NormalTok}[1]{{#1}}
    
    % Additional commands for more recent versions of Pandoc
    \newcommand{\ConstantTok}[1]{\textcolor[rgb]{0.53,0.00,0.00}{{#1}}}
    \newcommand{\SpecialCharTok}[1]{\textcolor[rgb]{0.25,0.44,0.63}{{#1}}}
    \newcommand{\VerbatimStringTok}[1]{\textcolor[rgb]{0.25,0.44,0.63}{{#1}}}
    \newcommand{\SpecialStringTok}[1]{\textcolor[rgb]{0.73,0.40,0.53}{{#1}}}
    \newcommand{\ImportTok}[1]{{#1}}
    \newcommand{\DocumentationTok}[1]{\textcolor[rgb]{0.73,0.13,0.13}{\textit{{#1}}}}
    \newcommand{\AnnotationTok}[1]{\textcolor[rgb]{0.38,0.63,0.69}{\textbf{\textit{{#1}}}}}
    \newcommand{\CommentVarTok}[1]{\textcolor[rgb]{0.38,0.63,0.69}{\textbf{\textit{{#1}}}}}
    \newcommand{\VariableTok}[1]{\textcolor[rgb]{0.10,0.09,0.49}{{#1}}}
    \newcommand{\ControlFlowTok}[1]{\textcolor[rgb]{0.00,0.44,0.13}{\textbf{{#1}}}}
    \newcommand{\OperatorTok}[1]{\textcolor[rgb]{0.40,0.40,0.40}{{#1}}}
    \newcommand{\BuiltInTok}[1]{{#1}}
    \newcommand{\ExtensionTok}[1]{{#1}}
    \newcommand{\PreprocessorTok}[1]{\textcolor[rgb]{0.74,0.48,0.00}{{#1}}}
    \newcommand{\AttributeTok}[1]{\textcolor[rgb]{0.49,0.56,0.16}{{#1}}}
    \newcommand{\InformationTok}[1]{\textcolor[rgb]{0.38,0.63,0.69}{\textbf{\textit{{#1}}}}}
    \newcommand{\WarningTok}[1]{\textcolor[rgb]{0.38,0.63,0.69}{\textbf{\textit{{#1}}}}}
    
    
    % Define a nice break command that doesn't care if a line doesn't already
    % exist.
    \def\br{\hspace*{\fill} \\* }
    % Math Jax compatibility definitions
    \def\gt{>}
    \def\lt{<}
    \let\Oldtex\TeX
    \let\Oldlatex\LaTeX
    \renewcommand{\TeX}{\textrm{\Oldtex}}
    \renewcommand{\LaTeX}{\textrm{\Oldlatex}}
    % Document parameters
    % Document title
    \title{parcial2\_optimizacion}
    
    
    
    
    
% Pygments definitions
\makeatletter
\def\PY@reset{\let\PY@it=\relax \let\PY@bf=\relax%
    \let\PY@ul=\relax \let\PY@tc=\relax%
    \let\PY@bc=\relax \let\PY@ff=\relax}
\def\PY@tok#1{\csname PY@tok@#1\endcsname}
\def\PY@toks#1+{\ifx\relax#1\empty\else%
    \PY@tok{#1}\expandafter\PY@toks\fi}
\def\PY@do#1{\PY@bc{\PY@tc{\PY@ul{%
    \PY@it{\PY@bf{\PY@ff{#1}}}}}}}
\def\PY#1#2{\PY@reset\PY@toks#1+\relax+\PY@do{#2}}

\@namedef{PY@tok@w}{\def\PY@tc##1{\textcolor[rgb]{0.73,0.73,0.73}{##1}}}
\@namedef{PY@tok@c}{\let\PY@it=\textit\def\PY@tc##1{\textcolor[rgb]{0.25,0.50,0.50}{##1}}}
\@namedef{PY@tok@cp}{\def\PY@tc##1{\textcolor[rgb]{0.74,0.48,0.00}{##1}}}
\@namedef{PY@tok@k}{\let\PY@bf=\textbf\def\PY@tc##1{\textcolor[rgb]{0.00,0.50,0.00}{##1}}}
\@namedef{PY@tok@kp}{\def\PY@tc##1{\textcolor[rgb]{0.00,0.50,0.00}{##1}}}
\@namedef{PY@tok@kt}{\def\PY@tc##1{\textcolor[rgb]{0.69,0.00,0.25}{##1}}}
\@namedef{PY@tok@o}{\def\PY@tc##1{\textcolor[rgb]{0.40,0.40,0.40}{##1}}}
\@namedef{PY@tok@ow}{\let\PY@bf=\textbf\def\PY@tc##1{\textcolor[rgb]{0.67,0.13,1.00}{##1}}}
\@namedef{PY@tok@nb}{\def\PY@tc##1{\textcolor[rgb]{0.00,0.50,0.00}{##1}}}
\@namedef{PY@tok@nf}{\def\PY@tc##1{\textcolor[rgb]{0.00,0.00,1.00}{##1}}}
\@namedef{PY@tok@nc}{\let\PY@bf=\textbf\def\PY@tc##1{\textcolor[rgb]{0.00,0.00,1.00}{##1}}}
\@namedef{PY@tok@nn}{\let\PY@bf=\textbf\def\PY@tc##1{\textcolor[rgb]{0.00,0.00,1.00}{##1}}}
\@namedef{PY@tok@ne}{\let\PY@bf=\textbf\def\PY@tc##1{\textcolor[rgb]{0.82,0.25,0.23}{##1}}}
\@namedef{PY@tok@nv}{\def\PY@tc##1{\textcolor[rgb]{0.10,0.09,0.49}{##1}}}
\@namedef{PY@tok@no}{\def\PY@tc##1{\textcolor[rgb]{0.53,0.00,0.00}{##1}}}
\@namedef{PY@tok@nl}{\def\PY@tc##1{\textcolor[rgb]{0.63,0.63,0.00}{##1}}}
\@namedef{PY@tok@ni}{\let\PY@bf=\textbf\def\PY@tc##1{\textcolor[rgb]{0.60,0.60,0.60}{##1}}}
\@namedef{PY@tok@na}{\def\PY@tc##1{\textcolor[rgb]{0.49,0.56,0.16}{##1}}}
\@namedef{PY@tok@nt}{\let\PY@bf=\textbf\def\PY@tc##1{\textcolor[rgb]{0.00,0.50,0.00}{##1}}}
\@namedef{PY@tok@nd}{\def\PY@tc##1{\textcolor[rgb]{0.67,0.13,1.00}{##1}}}
\@namedef{PY@tok@s}{\def\PY@tc##1{\textcolor[rgb]{0.73,0.13,0.13}{##1}}}
\@namedef{PY@tok@sd}{\let\PY@it=\textit\def\PY@tc##1{\textcolor[rgb]{0.73,0.13,0.13}{##1}}}
\@namedef{PY@tok@si}{\let\PY@bf=\textbf\def\PY@tc##1{\textcolor[rgb]{0.73,0.40,0.53}{##1}}}
\@namedef{PY@tok@se}{\let\PY@bf=\textbf\def\PY@tc##1{\textcolor[rgb]{0.73,0.40,0.13}{##1}}}
\@namedef{PY@tok@sr}{\def\PY@tc##1{\textcolor[rgb]{0.73,0.40,0.53}{##1}}}
\@namedef{PY@tok@ss}{\def\PY@tc##1{\textcolor[rgb]{0.10,0.09,0.49}{##1}}}
\@namedef{PY@tok@sx}{\def\PY@tc##1{\textcolor[rgb]{0.00,0.50,0.00}{##1}}}
\@namedef{PY@tok@m}{\def\PY@tc##1{\textcolor[rgb]{0.40,0.40,0.40}{##1}}}
\@namedef{PY@tok@gh}{\let\PY@bf=\textbf\def\PY@tc##1{\textcolor[rgb]{0.00,0.00,0.50}{##1}}}
\@namedef{PY@tok@gu}{\let\PY@bf=\textbf\def\PY@tc##1{\textcolor[rgb]{0.50,0.00,0.50}{##1}}}
\@namedef{PY@tok@gd}{\def\PY@tc##1{\textcolor[rgb]{0.63,0.00,0.00}{##1}}}
\@namedef{PY@tok@gi}{\def\PY@tc##1{\textcolor[rgb]{0.00,0.63,0.00}{##1}}}
\@namedef{PY@tok@gr}{\def\PY@tc##1{\textcolor[rgb]{1.00,0.00,0.00}{##1}}}
\@namedef{PY@tok@ge}{\let\PY@it=\textit}
\@namedef{PY@tok@gs}{\let\PY@bf=\textbf}
\@namedef{PY@tok@gp}{\let\PY@bf=\textbf\def\PY@tc##1{\textcolor[rgb]{0.00,0.00,0.50}{##1}}}
\@namedef{PY@tok@go}{\def\PY@tc##1{\textcolor[rgb]{0.53,0.53,0.53}{##1}}}
\@namedef{PY@tok@gt}{\def\PY@tc##1{\textcolor[rgb]{0.00,0.27,0.87}{##1}}}
\@namedef{PY@tok@err}{\def\PY@bc##1{{\setlength{\fboxsep}{\string -\fboxrule}\fcolorbox[rgb]{1.00,0.00,0.00}{1,1,1}{\strut ##1}}}}
\@namedef{PY@tok@kc}{\let\PY@bf=\textbf\def\PY@tc##1{\textcolor[rgb]{0.00,0.50,0.00}{##1}}}
\@namedef{PY@tok@kd}{\let\PY@bf=\textbf\def\PY@tc##1{\textcolor[rgb]{0.00,0.50,0.00}{##1}}}
\@namedef{PY@tok@kn}{\let\PY@bf=\textbf\def\PY@tc##1{\textcolor[rgb]{0.00,0.50,0.00}{##1}}}
\@namedef{PY@tok@kr}{\let\PY@bf=\textbf\def\PY@tc##1{\textcolor[rgb]{0.00,0.50,0.00}{##1}}}
\@namedef{PY@tok@bp}{\def\PY@tc##1{\textcolor[rgb]{0.00,0.50,0.00}{##1}}}
\@namedef{PY@tok@fm}{\def\PY@tc##1{\textcolor[rgb]{0.00,0.00,1.00}{##1}}}
\@namedef{PY@tok@vc}{\def\PY@tc##1{\textcolor[rgb]{0.10,0.09,0.49}{##1}}}
\@namedef{PY@tok@vg}{\def\PY@tc##1{\textcolor[rgb]{0.10,0.09,0.49}{##1}}}
\@namedef{PY@tok@vi}{\def\PY@tc##1{\textcolor[rgb]{0.10,0.09,0.49}{##1}}}
\@namedef{PY@tok@vm}{\def\PY@tc##1{\textcolor[rgb]{0.10,0.09,0.49}{##1}}}
\@namedef{PY@tok@sa}{\def\PY@tc##1{\textcolor[rgb]{0.73,0.13,0.13}{##1}}}
\@namedef{PY@tok@sb}{\def\PY@tc##1{\textcolor[rgb]{0.73,0.13,0.13}{##1}}}
\@namedef{PY@tok@sc}{\def\PY@tc##1{\textcolor[rgb]{0.73,0.13,0.13}{##1}}}
\@namedef{PY@tok@dl}{\def\PY@tc##1{\textcolor[rgb]{0.73,0.13,0.13}{##1}}}
\@namedef{PY@tok@s2}{\def\PY@tc##1{\textcolor[rgb]{0.73,0.13,0.13}{##1}}}
\@namedef{PY@tok@sh}{\def\PY@tc##1{\textcolor[rgb]{0.73,0.13,0.13}{##1}}}
\@namedef{PY@tok@s1}{\def\PY@tc##1{\textcolor[rgb]{0.73,0.13,0.13}{##1}}}
\@namedef{PY@tok@mb}{\def\PY@tc##1{\textcolor[rgb]{0.40,0.40,0.40}{##1}}}
\@namedef{PY@tok@mf}{\def\PY@tc##1{\textcolor[rgb]{0.40,0.40,0.40}{##1}}}
\@namedef{PY@tok@mh}{\def\PY@tc##1{\textcolor[rgb]{0.40,0.40,0.40}{##1}}}
\@namedef{PY@tok@mi}{\def\PY@tc##1{\textcolor[rgb]{0.40,0.40,0.40}{##1}}}
\@namedef{PY@tok@il}{\def\PY@tc##1{\textcolor[rgb]{0.40,0.40,0.40}{##1}}}
\@namedef{PY@tok@mo}{\def\PY@tc##1{\textcolor[rgb]{0.40,0.40,0.40}{##1}}}
\@namedef{PY@tok@ch}{\let\PY@it=\textit\def\PY@tc##1{\textcolor[rgb]{0.25,0.50,0.50}{##1}}}
\@namedef{PY@tok@cm}{\let\PY@it=\textit\def\PY@tc##1{\textcolor[rgb]{0.25,0.50,0.50}{##1}}}
\@namedef{PY@tok@cpf}{\let\PY@it=\textit\def\PY@tc##1{\textcolor[rgb]{0.25,0.50,0.50}{##1}}}
\@namedef{PY@tok@c1}{\let\PY@it=\textit\def\PY@tc##1{\textcolor[rgb]{0.25,0.50,0.50}{##1}}}
\@namedef{PY@tok@cs}{\let\PY@it=\textit\def\PY@tc##1{\textcolor[rgb]{0.25,0.50,0.50}{##1}}}

\def\PYZbs{\char`\\}
\def\PYZus{\char`\_}
\def\PYZob{\char`\{}
\def\PYZcb{\char`\}}
\def\PYZca{\char`\^}
\def\PYZam{\char`\&}
\def\PYZlt{\char`\<}
\def\PYZgt{\char`\>}
\def\PYZsh{\char`\#}
\def\PYZpc{\char`\%}
\def\PYZdl{\char`\$}
\def\PYZhy{\char`\-}
\def\PYZsq{\char`\'}
\def\PYZdq{\char`\"}
\def\PYZti{\char`\~}
% for compatibility with earlier versions
\def\PYZat{@}
\def\PYZlb{[}
\def\PYZrb{]}
\makeatother


    % For linebreaks inside Verbatim environment from package fancyvrb. 
    \makeatletter
        \newbox\Wrappedcontinuationbox 
        \newbox\Wrappedvisiblespacebox 
        \newcommand*\Wrappedvisiblespace {\textcolor{red}{\textvisiblespace}} 
        \newcommand*\Wrappedcontinuationsymbol {\textcolor{red}{\llap{\tiny$\m@th\hookrightarrow$}}} 
        \newcommand*\Wrappedcontinuationindent {3ex } 
        \newcommand*\Wrappedafterbreak {\kern\Wrappedcontinuationindent\copy\Wrappedcontinuationbox} 
        % Take advantage of the already applied Pygments mark-up to insert 
        % potential linebreaks for TeX processing. 
        %        {, <, #, %, $, ' and ": go to next line. 
        %        _, }, ^, &, >, - and ~: stay at end of broken line. 
        % Use of \textquotesingle for straight quote. 
        \newcommand*\Wrappedbreaksatspecials {% 
            \def\PYGZus{\discretionary{\char`\_}{\Wrappedafterbreak}{\char`\_}}% 
            \def\PYGZob{\discretionary{}{\Wrappedafterbreak\char`\{}{\char`\{}}% 
            \def\PYGZcb{\discretionary{\char`\}}{\Wrappedafterbreak}{\char`\}}}% 
            \def\PYGZca{\discretionary{\char`\^}{\Wrappedafterbreak}{\char`\^}}% 
            \def\PYGZam{\discretionary{\char`\&}{\Wrappedafterbreak}{\char`\&}}% 
            \def\PYGZlt{\discretionary{}{\Wrappedafterbreak\char`\<}{\char`\<}}% 
            \def\PYGZgt{\discretionary{\char`\>}{\Wrappedafterbreak}{\char`\>}}% 
            \def\PYGZsh{\discretionary{}{\Wrappedafterbreak\char`\#}{\char`\#}}% 
            \def\PYGZpc{\discretionary{}{\Wrappedafterbreak\char`\%}{\char`\%}}% 
            \def\PYGZdl{\discretionary{}{\Wrappedafterbreak\char`\$}{\char`\$}}% 
            \def\PYGZhy{\discretionary{\char`\-}{\Wrappedafterbreak}{\char`\-}}% 
            \def\PYGZsq{\discretionary{}{\Wrappedafterbreak\textquotesingle}{\textquotesingle}}% 
            \def\PYGZdq{\discretionary{}{\Wrappedafterbreak\char`\"}{\char`\"}}% 
            \def\PYGZti{\discretionary{\char`\~}{\Wrappedafterbreak}{\char`\~}}% 
        } 
        % Some characters . , ; ? ! / are not pygmentized. 
        % This macro makes them "active" and they will insert potential linebreaks 
        \newcommand*\Wrappedbreaksatpunct {% 
            \lccode`\~`\.\lowercase{\def~}{\discretionary{\hbox{\char`\.}}{\Wrappedafterbreak}{\hbox{\char`\.}}}% 
            \lccode`\~`\,\lowercase{\def~}{\discretionary{\hbox{\char`\,}}{\Wrappedafterbreak}{\hbox{\char`\,}}}% 
            \lccode`\~`\;\lowercase{\def~}{\discretionary{\hbox{\char`\;}}{\Wrappedafterbreak}{\hbox{\char`\;}}}% 
            \lccode`\~`\:\lowercase{\def~}{\discretionary{\hbox{\char`\:}}{\Wrappedafterbreak}{\hbox{\char`\:}}}% 
            \lccode`\~`\?\lowercase{\def~}{\discretionary{\hbox{\char`\?}}{\Wrappedafterbreak}{\hbox{\char`\?}}}% 
            \lccode`\~`\!\lowercase{\def~}{\discretionary{\hbox{\char`\!}}{\Wrappedafterbreak}{\hbox{\char`\!}}}% 
            \lccode`\~`\/\lowercase{\def~}{\discretionary{\hbox{\char`\/}}{\Wrappedafterbreak}{\hbox{\char`\/}}}% 
            \catcode`\.\active
            \catcode`\,\active 
            \catcode`\;\active
            \catcode`\:\active
            \catcode`\?\active
            \catcode`\!\active
            \catcode`\/\active 
            \lccode`\~`\~ 	
        }
    \makeatother

    \let\OriginalVerbatim=\Verbatim
    \makeatletter
    \renewcommand{\Verbatim}[1][1]{%
        %\parskip\z@skip
        \sbox\Wrappedcontinuationbox {\Wrappedcontinuationsymbol}%
        \sbox\Wrappedvisiblespacebox {\FV@SetupFont\Wrappedvisiblespace}%
        \def\FancyVerbFormatLine ##1{\hsize\linewidth
            \vtop{\raggedright\hyphenpenalty\z@\exhyphenpenalty\z@
                \doublehyphendemerits\z@\finalhyphendemerits\z@
                \strut ##1\strut}%
        }%
        % If the linebreak is at a space, the latter will be displayed as visible
        % space at end of first line, and a continuation symbol starts next line.
        % Stretch/shrink are however usually zero for typewriter font.
        \def\FV@Space {%
            \nobreak\hskip\z@ plus\fontdimen3\font minus\fontdimen4\font
            \discretionary{\copy\Wrappedvisiblespacebox}{\Wrappedafterbreak}
            {\kern\fontdimen2\font}%
        }%
        
        % Allow breaks at special characters using \PYG... macros.
        \Wrappedbreaksatspecials
        % Breaks at punctuation characters . , ; ? ! and / need catcode=\active 	
        \OriginalVerbatim[#1,codes*=\Wrappedbreaksatpunct]%
    }
    \makeatother

    % Exact colors from NB
    \definecolor{incolor}{HTML}{303F9F}
    \definecolor{outcolor}{HTML}{D84315}
    \definecolor{cellborder}{HTML}{CFCFCF}
    \definecolor{cellbackground}{HTML}{F7F7F7}
    
    % prompt
    \makeatletter
    \newcommand{\boxspacing}{\kern\kvtcb@left@rule\kern\kvtcb@boxsep}
    \makeatother
    \newcommand{\prompt}[4]{
        {\ttfamily\llap{{\color{#2}[#3]:\hspace{3pt}#4}}\vspace{-\baselineskip}}
    }
    

    
    % Prevent overflowing lines due to hard-to-break entities
    \sloppy 
    % Setup hyperref package
    \hypersetup{
      breaklinks=true,  % so long urls are correctly broken across lines
      colorlinks=true,
      urlcolor=urlcolor,
      linkcolor=linkcolor,
      citecolor=citecolor,
      }
    % Slightly bigger margins than the latex defaults
    
    \geometry{verbose,tmargin=1in,bmargin=1in,lmargin=1in,rmargin=1in}
    
    

\begin{document}
    
\title{Examen Parcial 2 Optimización}
\author{Roberto Vásquez Martínez \\ Profesor: Joaquín Peña Acevedo}
\date{25/Mayo/2022}
\maketitle 
    
    

    
    \hypertarget{curso-de-optimizaciuxf3n-demat}{%
\section{Curso de Optimización
(DEMAT)}\label{curso-de-optimizaciuxf3n-demat}}

\hypertarget{parcial-2}{%
\subsection{Parcial 2}\label{parcial-2}}

\begin{longtable}[]{@{}ll@{}}
\toprule
Descripción: & Fechas \\
\midrule
\endhead
Fecha de publicación del documento: & \textbf{Mayo 25, 2022} \\
Hora de inicio: & \textbf{15:00} \\
Hora límite de entrega: & \textbf{18:00} \\
\bottomrule
\end{longtable}

\hypertarget{indicaciones}{%
\subsubsection{Indicaciones}\label{indicaciones}}

Lea con cuidado los ejercicios.

Puede usar las notas de clase y las tareas hechas para resolver el
examen.

Al final, entregue el notebook con sus respuestas, junto con los códigos
que hagan falta para reproducir los resultados. Si es más de un archivo,
genere un archivo ZIP que contenga el notebook y los scripts
adicionales.

A partir del notebook genere un archivo PDF con las respuestas y envíelo
por separado antes de la hora límite.

    \hypertarget{ejercicio-1.-5-puntos}{%
\subsection{Ejercicio 1. (5 puntos)}\label{ejercicio-1.-5-puntos}}

Sea

\[ f(\mathbf{x}) = f(x_1, x_2) = 2x_1^{2} + x_2^{2} - x_1 x_2 - 6.5x_1  + 2.5x_2. \]

Considere el problema

\%
\[ \min\; f(\mathbf{x}) \quad \text{sujeto a} \quad -1\leq x_1 \leq 1, \; -1\leq x_2 \leq 1.\]

Encontrar la solución usando un método de barrera logarítmica (Clase
25).

Para esto construimos la función barrera logarítmica

\[
b_{log}(\mathbf{x}; \mu) = f(x_1,x_2) + \frac{1}{\mu}(-log(1-x_1)-log(1-x_2)-log(1+x_1)-log(1+x_2) )
\]

\begin{enumerate}
\def\labelenumi{\arabic{enumi}.}
\tightlist
\item
  Programar las funciones y sus gradientes
\end{enumerate}

\[f(\mathbf{x}) \quad \text{y} \quad b_{log}(\mathbf{x}; \mu) \]

\begin{enumerate}
\def\labelenumi{\arabic{enumi}.}
\setcounter{enumi}{1}
\tightlist
\item
  Programar el método de barrera logarítmica:
\end{enumerate}

\begin{enumerate}
\def\labelenumi{\alph{enumi})}
\item
  Dar un punto inicial \(\mathbf{x}_0\), la función
  \(b_{log}(\mathbf{x}; \mu)\), su gradiente, \(\mu_0\), una tolerancia
  \(\tau>0\) y los parámetros que se necesiten para usar el método BFGS.
\item
  Hacer \(k=0\) y repetir los siguientes pasos:
\end{enumerate}

b1) Usando el método BFGS, calcular el mínimo \(\mathbf{x}_{k+1}\) de
\(b_{log}(\mathbf{x}; \mu_k)\). Para el método BFGS use la tolerancia
\(\epsilon_m^{1/4}\) porque no necesitamos tener demasiada precisión en
el cálculo. Además, para el algoritmo de backtracking use
\(\alpha_{ini}=1\), para reducir la posibilidad de evaluar la función de
barrera en puntos que están fuera de la región factible.

b2) Imprimir si el algoritmo BFGS encuentra la solución, y en ese caso
imprimir

\begin{itemize}
\tightlist
\item
  el entero \(k\)
\item
  el valor \(\mu_k\),
\item
  el número de iteraciones que realizó el algoritmo BFGS,
\item
  el punto \(\mathbf{x}_{k+1}\) y
\item
  el valor \(f(\mathbf{x}_{k+1})\).
\end{itemize}

b3) Si \(\|\mathbf{x}_{k+1} - \mathbf{x}_k\|<\tau\), terminar
devolviendo \(\mathbf{x}_{k+1}\)

b4) En caso contrario, hacer \(\mu_{k+1} = 10\mu_k\), \(k=k+1\) y volver
al paso (b1)

\begin{enumerate}
\def\labelenumi{\arabic{enumi}.}
\setcounter{enumi}{2}
\item
  Probar el algoritmo usando \(\mathbf{x}_0 = (-0.75, 0.5)\),
  \(\mu_0=1\) y \(\tau=0.001\).
\item
  Si \((x_1^{k}, x_2^k)\) es el punto que devuelve el algoritmo de
  barrera logarítmica y si está en frontera, genere la gráfica de la
  función evaluándola en puntos sobre la arista en donde está punto para
  visualmente verificar que tiene sentido el resultado.
\end{enumerate}

\hypertarget{soluciuxf3n}{%
\subsubsection{Solución:}\label{soluciuxf3n}}

    A continuación probamos el algoritmo con un número de iteraciones máximo
de \(1000\) para el método de barrera y \(50000\) para el BFGS,
utilizando las tolerancias correspondientes.

    \begin{tcolorbox}[breakable, size=fbox, boxrule=1pt, pad at break*=1mm,colback=cellbackground, colframe=cellborder]
\prompt{In}{incolor}{1}{\boxspacing}
\begin{Verbatim}[commandchars=\\\{\}]
\PY{k+kn}{import} \PY{n+nn}{numpy} \PY{k}{as} \PY{n+nn}{np}
\PY{k+kn}{import} \PY{n+nn}{importlib}

\PY{c+c1}{\PYZsh{} Tolerancia y numero maximo de iteraciones BFGS}
\PY{n}{tol\PYZus{}BFGS}\PY{o}{=}\PY{n}{np}\PY{o}{.}\PY{n}{finfo}\PY{p}{(}\PY{n+nb}{float}\PY{p}{)}\PY{o}{.}\PY{n}{eps}\PY{o}{*}\PY{o}{*}\PY{p}{(}\PY{l+m+mi}{1}\PY{o}{/}\PY{l+m+mi}{4}\PY{p}{)}
\PY{n}{N\PYZus{}BFGS}\PY{o}{=}\PY{l+m+mi}{5000}
\PY{n}{rho}\PY{o}{=}\PY{l+m+mf}{0.1}

\PY{c+c1}{\PYZsh{} Tolerancia y numero de iteraciones BARRERA}
\PY{n}{tol\PYZus{}BARRERA}\PY{o}{=}\PY{l+m+mf}{0.001}
\PY{n}{N\PYZus{}BARRERA}\PY{o}{=}\PY{l+m+mi}{1000}
\PY{n}{x0}\PY{o}{=}\PY{n}{np}\PY{o}{.}\PY{n}{array}\PY{p}{(}\PY{p}{[}\PY{o}{\PYZhy{}}\PY{l+m+mf}{0.75}\PY{p}{,}\PY{l+m+mf}{0.5}\PY{p}{]}\PY{p}{)}
\PY{n}{mu0}\PY{o}{=}\PY{l+m+mf}{0.1}

\PY{k+kn}{import} \PY{n+nn}{lib\PYZus{}examen\PYZus{}2}
\PY{n}{importlib}\PY{o}{.}\PY{n}{reload}\PY{p}{(}\PY{n}{lib\PYZus{}examen\PYZus{}2}\PY{p}{)}
\PY{k+kn}{from} \PY{n+nn}{lib\PYZus{}examen\PYZus{}2} \PY{k+kn}{import} \PY{o}{*}

\PY{c+c1}{\PYZsh{} Prueba}
\PY{n}{xk}\PY{o}{=}\PY{n}{proof\PYZus{}barrera}\PY{p}{(}\PY{n}{barrera\PYZus{}f}\PY{p}{,}\PY{n}{grad\PYZus{}barrera\PYZus{}f}\PY{p}{,}\PY{n}{x0}\PY{p}{,}\PY{n}{mu0}\PY{p}{,}\PY{n}{N\PYZus{}BARRERA}\PY{p}{,}\PY{n}{N\PYZus{}BFGS}\PY{p}{,}\PY{n}{tol\PYZus{}BARRERA}\PY{p}{,}\PY{n}{tol\PYZus{}BFGS}\PY{p}{,}\PY{n}{rho}\PY{p}{,}\PY{n}{eje1\PYZus{}f}\PY{p}{)}
\end{Verbatim}
\end{tcolorbox}

    \begin{Verbatim}[commandchars=\\\{\}]
k =  6
mu\_k =  100000.0
iteraciones BFGS =  30
xk =  [[ 0.99999429]
 [-0.7499858 ]]
fk =  [-5.06249]
    \end{Verbatim}

    \begin{Verbatim}[commandchars=\\\{\}]
/home/roberto/Documentos/DEMAT/10mo Semestre/Materias/Optimización/Examen
Parcial 2/lib\_examen\_2.py:19: RuntimeWarning: invalid value encountered in log
  aux=-(np.log(1.0-x\_squeezed[0])+np.log(1.0+x\_squeezed[0])+np.log(1.0-x\_squeeze
d[1])+np.log(1.0+x\_squeezed[1]))
    \end{Verbatim}

    Finalmente, evaluamos en la arista a la función \(f\)

    \begin{tcolorbox}[breakable, size=fbox, boxrule=1pt, pad at break*=1mm,colback=cellbackground, colframe=cellborder]
\prompt{In}{incolor}{2}{\boxspacing}
\begin{Verbatim}[commandchars=\\\{\}]
\PY{k+kn}{import} \PY{n+nn}{warnings}
\PY{n}{warnings}\PY{o}{.}\PY{n}{filterwarnings}\PY{p}{(}\PY{l+s+s2}{\PYZdq{}}\PY{l+s+s2}{ignore}\PY{l+s+s2}{\PYZdq{}}\PY{p}{)}


\PY{k}{def} \PY{n+nf}{profile\PYZus{}f}\PY{p}{(}\PY{n}{x2}\PY{p}{)}\PY{p}{:}
    \PY{n}{vec}\PY{o}{=}\PY{n}{np}\PY{o}{.}\PY{n}{array}\PY{p}{(}\PY{p}{[}\PY{l+m+mf}{1.0}\PY{p}{,}\PY{n}{x2}\PY{p}{]}\PY{p}{)}
    \PY{k}{return} \PY{n}{eje1\PYZus{}f}\PY{p}{(}\PY{n}{vec}\PY{p}{)}
\PY{k+kn}{import} \PY{n+nn}{matplotlib}\PY{n+nn}{.}\PY{n+nn}{pyplot} \PY{k}{as} \PY{n+nn}{plt}
\PY{n}{x2\PYZus{}linspace}\PY{o}{=}\PY{n}{np}\PY{o}{.}\PY{n}{linspace}\PY{p}{(}\PY{o}{\PYZhy{}}\PY{l+m+mf}{1.0}\PY{p}{,}\PY{l+m+mf}{1.0}\PY{p}{,}\PY{l+m+mi}{100}\PY{p}{)}
\PY{n}{profile\PYZus{}f\PYZus{}x2\PYZus{}linspace}\PY{o}{=}\PY{n}{np}\PY{o}{.}\PY{n}{array}\PY{p}{(}\PY{p}{[}\PY{n}{profile\PYZus{}f}\PY{p}{(}\PY{n}{x2}\PY{p}{)} \PY{k}{for} \PY{n}{x2} \PY{o+ow}{in} \PY{n}{x2\PYZus{}linspace}\PY{p}{]}\PY{p}{)}

\PY{n}{plt}\PY{o}{.}\PY{n}{plot}\PY{p}{(}\PY{n}{x2\PYZus{}linspace}\PY{p}{,}\PY{n}{profile\PYZus{}f\PYZus{}x2\PYZus{}linspace}\PY{p}{,}\PY{n}{color}\PY{o}{=}\PY{l+s+s1}{\PYZsq{}}\PY{l+s+s1}{blue}\PY{l+s+s1}{\PYZsq{}}\PY{p}{)}
\PY{n}{fk}\PY{o}{=}\PY{n}{profile\PYZus{}f}\PY{p}{(}\PY{n}{xk}\PY{p}{[}\PY{l+m+mi}{1}\PY{p}{]}\PY{p}{)}
\PY{n}{plt}\PY{o}{.}\PY{n}{scatter}\PY{p}{(}\PY{n}{xk}\PY{p}{[}\PY{l+m+mi}{1}\PY{p}{]}\PY{p}{,}\PY{n}{fk}\PY{p}{,} \PY{n}{marker}\PY{o}{=}\PY{l+s+s2}{\PYZdq{}}\PY{l+s+s2}{o}\PY{l+s+s2}{\PYZdq{}}\PY{p}{,}\PY{n}{color}\PY{o}{=}\PY{l+s+s1}{\PYZsq{}}\PY{l+s+s1}{red}\PY{l+s+s1}{\PYZsq{}}\PY{p}{)}
\end{Verbatim}
\end{tcolorbox}

            \begin{tcolorbox}[breakable, size=fbox, boxrule=.5pt, pad at break*=1mm, opacityfill=0]
\prompt{Out}{outcolor}{2}{\boxspacing}
\begin{Verbatim}[commandchars=\\\{\}]
<matplotlib.collections.PathCollection at 0x7f6eb3ab1730>
\end{Verbatim}
\end{tcolorbox}
        
    \begin{center}
    \adjustimage{max size={0.9\linewidth}{0.9\paperheight}}{parcial2_optimizacion_files/parcial2_optimizacion_5_1.png}
    \end{center}
    { \hspace*{\fill} \\}
    
    De lo que podemos observar que al menos el punto hallado
\(\mathbf{x}^\ast\), cerca de la frontera, es un mínimo local de la
función \(f\) considerando al menos la sección de la función
correspondiente a la arista de la región factible.

    \hypertarget{ejercicio-2.-5-puntos}{%
\subsection{Ejercicio 2. (5 puntos)}\label{ejercicio-2.-5-puntos}}

Consideremos la función \(f(z; x_1, ..., x_{rs})\) definida por una
combinación lineal de funciones trigonométricas:

\[
\begin{array}{rcl}
f(z; x_1, ...,  x_{rs}) &=&
x_1 \sin(\omega_1 z + \phi_1) + ... + x_{r} \sin(\omega_1 z + \phi_r)\\
& & + x_{r+1} \sin(\omega_2 z + \phi_1) + ... + x_{2r} \sin(\omega_2 z + \phi_r) \\
& & + ... \\
& & + x_{r(s-1)+1} \sin(\omega_{s}  z + \phi_1) + ... + x_{rs} \sin(\omega_{s} z + \phi_r)
\end{array}
\]

Para \$ x\_1, \ldots, x\_\{rs\}\$ fijos y unos puntos
\(z_1, z_2, ..., z_m\) datos, calculamos
\(b_i = f(z_i; x_1, ..., x_{rs})\) para \(i=1,2,...,m\). Es decir, si
construimos la matriz

\[
\left(
\begin{array}{l}
b_1 \\ b_2 \\ \vdots \\ b_m
\end{array}
\right)
=
\left[
\begin{array}{rrrrrr}
 \sin(\omega_1 z_1 + \phi_1) & \cdots & \sin(\omega_1 z_1 + \phi_r) & \sin(\omega_2 z_1 + \phi_1) &\cdots & \sin(\omega_{s} z_1 + \phi_r) \\
 \sin(\omega_1 z_2 + \phi_1) & \cdots & \sin(\omega_1 z_2 + \phi_r) & \sin(\omega_2 z_2 + \phi_1) &\cdots & \sin(\omega_{s} z_2 + \phi_r) \\
 \vdots                      & \cdots & \vdots                      & \vdots                      &\cdots & \vdots \\
 \sin(\omega_1 z_m + \phi_1) & \cdots & \sin(\omega_1 z_m + \phi_r) & \sin(\omega_2 z_m + \phi_1) &\cdots & \sin(\omega_{s} z_m + \phi_r) 
\end{array}
\right]
\left(
\begin{array}{l}
x_1 \\ x_2 \\ \vdots \\ x_{rs}
\end{array}
\right)
\]

Es decir,

\[ \mathbf{b} = \mathbf{A} \mathbf{x} \]

El siguiente código contruye la matriz \(\mathbf{A}\), propone un vector
\(\mathbf{x}_{real}\) y genera el vector\\
\(\mathbf{b} = \mathbf{A} \mathbf{x}_{real}\).

    \begin{tcolorbox}[breakable, size=fbox, boxrule=1pt, pad at break*=1mm,colback=cellbackground, colframe=cellborder]
\prompt{In}{incolor}{3}{\boxspacing}
\begin{Verbatim}[commandchars=\\\{\}]
\PY{k+kn}{import} \PY{n+nn}{numpy} \PY{k}{as} \PY{n+nn}{np}
\PY{k+kn}{import} \PY{n+nn}{matplotlib}\PY{n+nn}{.}\PY{n+nn}{pyplot} \PY{k}{as} \PY{n+nn}{plt}

\PY{c+c1}{\PYZsh{} Evalua cada función trigonométrica. }
\PY{k}{def} \PY{n+nf}{fnctrig}\PY{p}{(}\PY{n}{x}\PY{p}{,} \PY{n}{fparam}\PY{p}{)}\PY{p}{:}
    \PY{n}{omega} \PY{o}{=} \PY{n}{fparam}\PY{p}{[}\PY{l+s+s1}{\PYZsq{}}\PY{l+s+s1}{omega}\PY{l+s+s1}{\PYZsq{}}\PY{p}{]}
    \PY{n}{phi}   \PY{o}{=} \PY{n}{fparam}\PY{p}{[}\PY{l+s+s1}{\PYZsq{}}\PY{l+s+s1}{phi}\PY{l+s+s1}{\PYZsq{}}\PY{p}{]}
    \PY{k}{return} \PY{n}{np}\PY{o}{.}\PY{n}{sin}\PY{p}{(}\PY{n}{omega}\PY{o}{*}\PY{n}{x} \PY{o}{+} \PY{n}{phi}\PY{p}{)}

\PY{n}{m}    \PY{o}{=} \PY{l+m+mi}{15}   \PY{c+c1}{\PYZsh{} Número de puntos z\PYZus{}i}
\PY{n}{r}    \PY{o}{=} \PY{l+m+mi}{10}   \PY{c+c1}{\PYZsh{} Número de angulos phi}
\PY{n}{s}    \PY{o}{=} \PY{l+m+mi}{20}   \PY{c+c1}{\PYZsh{} Número de frecuencias omega}

\PY{n}{z}     \PY{o}{=} \PY{n}{np}\PY{o}{.}\PY{n}{linspace}\PY{p}{(}\PY{l+m+mi}{0}\PY{p}{,} \PY{l+m+mi}{9}\PY{p}{,} \PY{n}{m}\PY{p}{)}
\PY{n}{phi}   \PY{o}{=} \PY{n}{np}\PY{o}{.}\PY{n}{linspace}\PY{p}{(}\PY{l+m+mi}{0}\PY{p}{,} \PY{n}{np}\PY{o}{.}\PY{n}{pi}\PY{p}{,} \PY{n}{r}\PY{p}{)}
\PY{n}{omega} \PY{o}{=} \PY{n}{np}\PY{o}{.}\PY{n}{linspace}\PY{p}{(}\PY{l+m+mf}{0.5}\PY{p}{,} \PY{l+m+mf}{2.0}\PY{p}{,} \PY{n}{s}\PY{p}{)}

\PY{n+nb}{print}\PY{p}{(}\PY{l+s+s1}{\PYZsq{}}\PY{l+s+s1}{Vector de frecuencias omega:}\PY{l+s+se}{\PYZbs{}n}\PY{l+s+s1}{\PYZsq{}}\PY{p}{,} \PY{n}{omega}\PY{p}{)}
\PY{n+nb}{print}\PY{p}{(}\PY{l+s+s1}{\PYZsq{}}\PY{l+s+se}{\PYZbs{}n}\PY{l+s+s1}{Vector de ángulos de fase phi:}\PY{l+s+se}{\PYZbs{}n}\PY{l+s+s1}{\PYZsq{}}\PY{p}{,} \PY{n}{phi}\PY{p}{)}

\PY{n}{n}  \PY{o}{=} \PY{n}{r}\PY{o}{*}\PY{n}{s}

\PY{c+c1}{\PYZsh{} Se crea la matriz A }
\PY{n}{A}  \PY{o}{=} \PY{n}{np}\PY{o}{.}\PY{n}{zeros}\PY{p}{(}\PY{p}{(}\PY{n}{m}\PY{p}{,} \PY{n}{n}\PY{p}{)}\PY{p}{)}
\PY{k}{for} \PY{n}{i}\PY{p}{,}\PY{n}{z\PYZus{}i} \PY{o+ow}{in} \PY{n+nb}{enumerate}\PY{p}{(}\PY{n}{z}\PY{p}{)}\PY{p}{:}
    \PY{n}{mc} \PY{o}{=} \PY{l+m+mi}{0}
    \PY{k}{for} \PY{n}{j}\PY{p}{,}\PY{n}{phi\PYZus{}j} \PY{o+ow}{in} \PY{n+nb}{enumerate}\PY{p}{(}\PY{n}{phi}\PY{p}{)}\PY{p}{:}
        \PY{k}{for} \PY{n}{l}\PY{p}{,}\PY{n}{omega\PYZus{}l} \PY{o+ow}{in} \PY{n+nb}{enumerate}\PY{p}{(}\PY{n}{omega}\PY{p}{)}\PY{p}{:}
            \PY{n}{fparam}   \PY{o}{=} \PY{p}{\PYZob{}}\PY{l+s+s1}{\PYZsq{}}\PY{l+s+s1}{omega}\PY{l+s+s1}{\PYZsq{}}\PY{p}{:}\PY{n}{omega\PYZus{}l}\PY{p}{,} \PY{l+s+s1}{\PYZsq{}}\PY{l+s+s1}{phi}\PY{l+s+s1}{\PYZsq{}}\PY{p}{:}\PY{n}{phi\PYZus{}j}\PY{p}{\PYZcb{}}
            \PY{n}{A}\PY{p}{[}\PY{n}{i}\PY{p}{,}\PY{n}{mc}\PY{p}{]} \PY{o}{=} \PY{n}{fnctrig}\PY{p}{(}\PY{n}{z\PYZus{}i}\PY{p}{,} \PY{n}{fparam}\PY{p}{)}
            \PY{n}{mc} \PY{o}{+}\PY{o}{=} \PY{l+m+mi}{1}

\PY{n}{x\PYZus{}real} \PY{o}{=} \PY{n}{np}\PY{o}{.}\PY{n}{zeros}\PY{p}{(}\PY{n}{n}\PY{p}{)}
\PY{n}{nc} \PY{o}{=} \PY{l+m+mi}{1} 
\PY{k}{for} \PY{n}{i} \PY{o+ow}{in} \PY{n+nb}{range}\PY{p}{(}\PY{n}{n}\PY{p}{)}\PY{p}{:}
    \PY{k}{if} \PY{n}{i}\PY{o}{\PYZpc{}}\PY{k}{r}==3 and (i//s)\PYZpc{}2==0:
        \PY{n}{x\PYZus{}real}\PY{p}{[}\PY{n}{i}\PY{p}{]} \PY{o}{=} \PY{p}{(}\PY{o}{\PYZhy{}}\PY{l+m+mi}{1}\PY{p}{)}\PY{o}{*}\PY{o}{*}\PY{n}{nc}
        \PY{n}{nc}  \PY{o}{+}\PY{o}{=} \PY{l+m+mi}{1}

\PY{n}{b}  \PY{o}{=} \PY{n}{A} \PY{o}{@} \PY{n}{x\PYZus{}real}

\PY{n+nb}{print}\PY{p}{(}\PY{l+s+s1}{\PYZsq{}}\PY{l+s+se}{\PYZbs{}n}\PY{l+s+s1}{Dimensiones de la matriz A:}\PY{l+s+s1}{\PYZsq{}}\PY{p}{,} \PY{n}{A}\PY{o}{.}\PY{n}{shape}\PY{p}{)}

\PY{n}{plt}\PY{o}{.}\PY{n}{figure}\PY{p}{(}\PY{p}{)}
\PY{n}{plt}\PY{o}{.}\PY{n}{bar}\PY{p}{(}\PY{n}{np}\PY{o}{.}\PY{n}{arange}\PY{p}{(}\PY{n}{n}\PY{p}{)}\PY{p}{,} \PY{n}{np}\PY{o}{.}\PY{n}{squeeze}\PY{p}{(}\PY{n}{x\PYZus{}real}\PY{p}{)}\PY{p}{)}
\PY{n}{plt}\PY{o}{.}\PY{n}{xlabel}\PY{p}{(}\PY{l+s+s1}{\PYZsq{}}\PY{l+s+s1}{Índice i}\PY{l+s+s1}{\PYZsq{}}\PY{p}{)}
\PY{n}{plt}\PY{o}{.}\PY{n}{ylabel}\PY{p}{(}\PY{l+s+sa}{r}\PY{l+s+s1}{\PYZsq{}}\PY{l+s+s1}{\PYZdl{}x\PYZus{}i\PYZdl{}}\PY{l+s+s1}{\PYZsq{}}\PY{p}{)}
\PY{n}{plt}\PY{o}{.}\PY{n}{title}\PY{p}{(}\PY{l+s+sa}{r}\PY{l+s+s1}{\PYZsq{}}\PY{l+s+s1}{Componentes del vector \PYZdl{}x\PYZus{}}\PY{l+s+si}{\PYZob{}real\PYZcb{}}\PY{l+s+s1}{\PYZdl{} que genera los datos}\PY{l+s+s1}{\PYZsq{}}\PY{p}{,} \PY{n}{fontsize}\PY{o}{=}\PY{l+m+mi}{14}\PY{p}{)}

\PY{n}{plt}\PY{o}{.}\PY{n}{figure}\PY{p}{(}\PY{p}{)}
\PY{n}{plt}\PY{o}{.}\PY{n}{plot}\PY{p}{(}\PY{n}{np}\PY{o}{.}\PY{n}{arange}\PY{p}{(}\PY{n}{m}\PY{p}{)}\PY{p}{,} \PY{n}{b}\PY{p}{,} \PY{l+s+s1}{\PYZsq{}}\PY{l+s+s1}{r}\PY{l+s+s1}{\PYZsq{}}\PY{p}{,} \PY{n}{linewidth}\PY{o}{=}\PY{l+m+mi}{1}\PY{p}{)}
\PY{n}{plt}\PY{o}{.}\PY{n}{xlabel}\PY{p}{(}\PY{l+s+s1}{\PYZsq{}}\PY{l+s+s1}{Índice i}\PY{l+s+s1}{\PYZsq{}}\PY{p}{)}
\PY{n}{plt}\PY{o}{.}\PY{n}{ylabel}\PY{p}{(}\PY{l+s+sa}{r}\PY{l+s+s1}{\PYZsq{}}\PY{l+s+s1}{\PYZdl{}b\PYZus{}i\PYZdl{}}\PY{l+s+s1}{\PYZsq{}}\PY{p}{)}
\PY{n}{\PYZus{}}\PY{o}{=}\PY{n}{plt}\PY{o}{.}\PY{n}{title}\PY{p}{(}\PY{l+s+sa}{r}\PY{l+s+s1}{\PYZsq{}}\PY{l+s+s1}{Datos generados \PYZdl{}b\PYZus{}i\PYZdl{}}\PY{l+s+s1}{\PYZsq{}}\PY{p}{,} \PY{n}{fontsize}\PY{o}{=}\PY{l+m+mi}{14}\PY{p}{)}
\end{Verbatim}
\end{tcolorbox}

    \begin{Verbatim}[commandchars=\\\{\}]
Vector de frecuencias omega:
 [0.5        0.57894737 0.65789474 0.73684211 0.81578947 0.89473684
 0.97368421 1.05263158 1.13157895 1.21052632 1.28947368 1.36842105
 1.44736842 1.52631579 1.60526316 1.68421053 1.76315789 1.84210526
 1.92105263 2.        ]

Vector de ángulos de fase phi:
 [0.         0.34906585 0.6981317  1.04719755 1.3962634  1.74532925
 2.0943951  2.44346095 2.7925268  3.14159265]

Dimensiones de la matriz A: (15, 200)
    \end{Verbatim}

    \begin{center}
    \adjustimage{max size={0.9\linewidth}{0.9\paperheight}}{parcial2_optimizacion_files/parcial2_optimizacion_8_1.png}
    \end{center}
    { \hspace*{\fill} \\}
    
    \begin{center}
    \adjustimage{max size={0.9\linewidth}{0.9\paperheight}}{parcial2_optimizacion_files/parcial2_optimizacion_8_2.png}
    \end{center}
    { \hspace*{\fill} \\}
    
    Dados la matriz \(\mathbf{A}\) y el vector \(\mathbf{b}\), queremos
calcular un vector \(\mathbf{x}\) tal que

\[ \mathbf{b} = \mathbf{A} \mathbf{x}. \]

Como \(\mathbf{A}\) es de tamaño \(m \times n\), con \(m<n\), no se
puede resolver el sistema de ecuaciones o aplicar el método de mínimos
cuadrados.

Como hay es un sistema subdeterminado, puede haber una infinidad de
soluciones Una manera obtener un vector \(\mathbf{x}\) se seleccionar
aquel que satisface el sistema y tiene norma mínima (para que no ocurra
como en las soluciones de mínimos cuadrados en los que las componentes
toman valores en el orden de decenas de miles mientras que el
componentes de \(\mathbf{b}\) están en orden de unidades). Así se puede
plantear el problema

\[  \min \|\mathbf{x}\|_2 \quad \text{sujeto a} \quad \mathbf{A} \mathbf{x} = \mathbf{b}. \]

Se puede ver que la solución de este problema está dada por (medio punto
adicional si hacen la deducción)

\[ \mathbf{x}_{mn2} = \mathbf{A}^\top(\mathbf{A}\mathbf{A}^\top)^{-1} \mathbf{b}.\]

El problema es que esta solución no es rala, es decir, que la mayoría de
sus componentes son cero, tal como el vector \(\mathbf{x}_{real}\) que
generó los datos \(\mathbf{b}\).

El método ``basis pursuit'' propone calcular calcular \(\mathbf{x}\)
resolviendo el problema

\[  \min \|\mathbf{x}\|_1 \quad \text{sujeto a} \quad \mathbf{A} \mathbf{x} = \mathbf{b}. \]

Al minimizar la suma de valores absolutos de las variables \(x_i\), se
espera que la mayoría de éstas se hagan cero para reducir la suma, en
especial si usamos un método como el método simplex que calcula
soluciones básicas factibles (todas la variables no básicas son cero).

Este problema lo podemos plantear como un problema de programación
lineal, expresando el vector \(\mathbf{x}\) como la resta de dos nuevos
vectores de variables \(\mathbf{u}\) y \(\mathbf{v}\) no negativos:

\[ \mathbf{x} = \mathbf{u} - \mathbf{v}.\]

Así, se plantea el problema de programación lineal

\[  \min \; 
\mathbf{c}^\top 
\left(
\begin{array}{l}
\mathbf{u}  \\ \mathbf{v}
\end{array}
\right)
\quad \text{sujeto a} \quad 
[\mathbf{A}\;\; -\mathbf{A}] 
\left(
\begin{array}{l}
\mathbf{u}  \\ \mathbf{v}
\end{array}
\right)
= \mathbf{b}, \quad (\mathbf{u}, \mathbf{v}) \geq  \mathbf{0}, \]

donde \(\mathbf{c}\) es el vector con todas sus componentes iguales a 1.

\begin{enumerate}
\def\labelenumi{\arabic{enumi}.}
\tightlist
\item
  Calcule la solución de mínima norma 2,
\end{enumerate}

\[ \mathbf{x}_{mn2}. \]

\begin{itemize}
\tightlist
\item
  Imprima el valor del error \(\|\mathbf{A} \mathbf{x}-\mathbf{b}\|\)
\item
  Grafique las entradas del vector mínima norma 2 para que pueda
  constatar que no es vector ralo y que no tiene parecido con
  \(\mathbf{x}_{real}\).
\end{itemize}

\begin{enumerate}
\def\labelenumi{\arabic{enumi}.}
\setcounter{enumi}{1}
\tightlist
\item
  Resuelva el problema de programación lineal con alguna librería de
  Python.
\item
  Construya y resuelva el problema dual para obtener el vector
  \(\lambda\) y defina \(\mathbf{s}\) como las variables de holgura del
  problema dual.
\item
  Verique si se cumplen las condiciones KKT para la variable primal
  \((\mathbf{u}, \mathbf{v})\) y las variables duales
  \((\lambda, \mathbf{s})\) (pueden que no se cumplan si la librería
  envió warnings sobre el mal condicionamiento de una matriz).
\item
  Imprima y compare los valores de la función objetivo del problema
  primal y del problema dual.
\item
  Calcule \(\mathbf{x} = \mathbf{u} - \mathbf{v}\). Imprima el error
  \(\|\mathbf{A} \mathbf{x}-\mathbf{b}\|\) y grafique las entradas del
  vector \(\mathbf{x}\). Comparado con la gráfica de
  \(\mathbf{x}_{real}\), describa las similutudes o diferencias, y si
  menos se cumple que es un vector ralo, que era el propósito original
  de cambiar la norma 2 por la norma 1 en la función objetivo del
  problema.
\item
  Repita los pasos 2 al 6, cambiando \(m=50\). Vuelva a ejecutar el
  código que genera a la matriz \(\mathbf{A}\) y al vector
  \(\mathbf{b}\). Describa si hay algún cambio en los resultados por el
  hecho de tener más ecuaciones.
\end{enumerate}

\hypertarget{soluciuxf3n}{%
\subsubsection{Solución:}\label{soluciuxf3n}}

    En primer lugar, calculamos la solución al problema de minimización con
la norma 2 utilizando la fórmula proporcionada, este vector lo denotamos
por \(\mathbf{x}^\ast_2\)

    \begin{tcolorbox}[breakable, size=fbox, boxrule=1pt, pad at break*=1mm,colback=cellbackground, colframe=cellborder]
\prompt{In}{incolor}{4}{\boxspacing}
\begin{Verbatim}[commandchars=\\\{\}]
\PY{c+c1}{\PYZsh{} Norma 2}
\PY{n}{x\PYZus{}2norm}\PY{o}{=}\PY{n}{A}\PY{o}{.}\PY{n}{T}\PY{n+nd}{@np}\PY{o}{.}\PY{n}{linalg}\PY{o}{.}\PY{n}{inv}\PY{p}{(}\PY{n}{A}\PY{n+nd}{@A}\PY{o}{.}\PY{n}{T}\PY{p}{)}\PY{n+nd}{@b}
\PY{n+nb}{print}\PY{p}{(}\PY{l+s+s1}{\PYZsq{}}\PY{l+s+s1}{|Ax\PYZhy{}b| = }\PY{l+s+s1}{\PYZsq{}}\PY{p}{,}\PY{n}{np}\PY{o}{.}\PY{n}{linalg}\PY{o}{.}\PY{n}{norm}\PY{p}{(}\PY{n}{A}\PY{n+nd}{@x\PYZus{}2norm}\PY{o}{\PYZhy{}}\PY{n}{b}\PY{p}{)}\PY{p}{)}
\end{Verbatim}
\end{tcolorbox}

    \begin{Verbatim}[commandchars=\\\{\}]
|Ax-b| =  0.0012767576284011112
    \end{Verbatim}

    Hacemos la gráfica de las componentes de \(\mathbf{x}^\ast_2\)

    \begin{tcolorbox}[breakable, size=fbox, boxrule=1pt, pad at break*=1mm,colback=cellbackground, colframe=cellborder]
\prompt{In}{incolor}{5}{\boxspacing}
\begin{Verbatim}[commandchars=\\\{\}]
\PY{n}{plt}\PY{o}{.}\PY{n}{bar}\PY{p}{(}\PY{n}{np}\PY{o}{.}\PY{n}{arange}\PY{p}{(}\PY{n}{n}\PY{p}{)}\PY{p}{,} \PY{n}{np}\PY{o}{.}\PY{n}{squeeze}\PY{p}{(}\PY{n}{x\PYZus{}2norm}\PY{p}{)}\PY{p}{)}
\PY{n}{plt}\PY{o}{.}\PY{n}{xlabel}\PY{p}{(}\PY{l+s+s1}{\PYZsq{}}\PY{l+s+s1}{Índice i}\PY{l+s+s1}{\PYZsq{}}\PY{p}{)}
\PY{n}{plt}\PY{o}{.}\PY{n}{ylabel}\PY{p}{(}\PY{l+s+sa}{r}\PY{l+s+s1}{\PYZsq{}}\PY{l+s+s1}{\PYZdl{}x\PYZus{}i\PYZdl{}}\PY{l+s+s1}{\PYZsq{}}\PY{p}{)}
\PY{n}{plt}\PY{o}{.}\PY{n}{title}\PY{p}{(}\PY{l+s+sa}{r}\PY{l+s+s1}{\PYZsq{}}\PY{l+s+s1}{Componentes del vector \PYZdl{}}\PY{l+s+s1}{\PYZbs{}}\PY{l+s+s1}{mathbf}\PY{l+s+si}{\PYZob{}x\PYZcb{}}\PY{l+s+s1}{\PYZca{}}\PY{l+s+s1}{\PYZbs{}}\PY{l+s+s1}{ast\PYZus{}2\PYZdl{} que genera los datos}\PY{l+s+s1}{\PYZsq{}}\PY{p}{,} \PY{n}{fontsize}\PY{o}{=}\PY{l+m+mi}{14}\PY{p}{)}
\end{Verbatim}
\end{tcolorbox}

            \begin{tcolorbox}[breakable, size=fbox, boxrule=.5pt, pad at break*=1mm, opacityfill=0]
\prompt{Out}{outcolor}{5}{\boxspacing}
\begin{Verbatim}[commandchars=\\\{\}]
Text(0.5, 1.0, 'Componentes del vector \$\textbackslash{}\textbackslash{}mathbf\{x\}\^{}\textbackslash{}\textbackslash{}ast\_2\$ que genera los
datos')
\end{Verbatim}
\end{tcolorbox}
        
    \begin{center}
    \adjustimage{max size={0.9\linewidth}{0.9\paperheight}}{parcial2_optimizacion_files/parcial2_optimizacion_13_1.png}
    \end{center}
    { \hspace*{\fill} \\}
    
    Y claramente no es igual al vector \(\mathbf{x}_{\mathrm{real}}\)
propuesto.

    Ahora, resolvemos el problema de programación lineal primal

    \begin{tcolorbox}[breakable, size=fbox, boxrule=1pt, pad at break*=1mm,colback=cellbackground, colframe=cellborder]
\prompt{In}{incolor}{6}{\boxspacing}
\begin{Verbatim}[commandchars=\\\{\}]
\PY{k+kn}{from} \PY{n+nn}{scipy}\PY{n+nn}{.}\PY{n+nn}{optimize} \PY{k+kn}{import} \PY{n}{linprog}
\PY{k+kn}{import} \PY{n+nn}{scipy}

\PY{n}{c} \PY{o}{=} \PY{n}{np}\PY{o}{.}\PY{n}{ones}\PY{p}{(}\PY{l+m+mi}{2}\PY{o}{*}\PY{n}{n}\PY{p}{)}
\PY{n}{A\PYZus{}primal}\PY{o}{=}\PY{n}{np}\PY{o}{.}\PY{n}{concatenate}\PY{p}{(}\PY{p}{(}\PY{n}{A}\PY{p}{,} \PY{o}{\PYZhy{}}\PY{n}{A}\PY{p}{)}\PY{p}{,} \PY{n}{axis}\PY{o}{=}\PY{l+m+mi}{1}\PY{p}{)}

\PY{c+c1}{\PYZsh{} Cotas de las variables}
\PY{n}{bnd} \PY{o}{=} \PY{p}{[}\PY{p}{(}\PY{l+m+mi}{0}\PY{p}{,} \PY{n}{scipy}\PY{o}{.}\PY{n}{inf}\PY{p}{)} \PY{k}{for} \PY{n}{i} \PY{o+ow}{in} \PY{n+nb}{range}\PY{p}{(}\PY{l+m+mi}{2}\PY{o}{*}\PY{n}{n}\PY{p}{)}\PY{p}{]}

\PY{n}{opt\PYZus{}primal} \PY{o}{=} \PY{n}{linprog}\PY{p}{(}\PY{n}{c}\PY{o}{=}\PY{n}{c}\PY{p}{,} \PY{n}{A\PYZus{}eq}\PY{o}{=}\PY{n}{A\PYZus{}primal}\PY{p}{,} \PY{n}{b\PYZus{}eq}\PY{o}{=}\PY{n}{b}\PY{p}{,} \PY{n}{bounds}\PY{o}{=}\PY{n}{bnd}\PY{p}{,}
              \PY{n}{method}\PY{o}{=}\PY{l+s+s2}{\PYZdq{}}\PY{l+s+s2}{simplex}\PY{l+s+s2}{\PYZdq{}}\PY{p}{)}

\PY{n+nb}{print}\PY{p}{(}\PY{l+s+s1}{\PYZsq{}}\PY{l+s+se}{\PYZbs{}n}\PY{l+s+s1}{Resultado del proceso:}\PY{l+s+s1}{\PYZsq{}}\PY{p}{,} \PY{n}{opt\PYZus{}primal}\PY{o}{.}\PY{n}{message}\PY{p}{)}
\PY{k}{if} \PY{n}{opt\PYZus{}primal}\PY{o}{.}\PY{n}{success}\PY{p}{:}
    \PY{n+nb}{print}\PY{p}{(}\PY{l+s+s1}{\PYZsq{}}\PY{l+s+s1}{Valor de la función objetivo:}\PY{l+s+s1}{\PYZsq{}}\PY{p}{,} \PY{n}{opt\PYZus{}primal}\PY{o}{.}\PY{n}{fun}\PY{p}{)}
    \PY{n}{x\PYZus{}1norm}\PY{o}{=}\PY{n}{opt\PYZus{}primal}\PY{o}{.}\PY{n}{x}
    
\end{Verbatim}
\end{tcolorbox}

    \begin{Verbatim}[commandchars=\\\{\}]

Resultado del proceso: Optimization terminated successfully.
Valor de la función objetivo: 5.758770483774055
    \end{Verbatim}

    Posteriormente, resolvemos el problema dual, en este caso consideramos
el problema de \(\min -\mathbf{b}^T\lambda\)

    \begin{tcolorbox}[breakable, size=fbox, boxrule=1pt, pad at break*=1mm,colback=cellbackground, colframe=cellborder]
\prompt{In}{incolor}{7}{\boxspacing}
\begin{Verbatim}[commandchars=\\\{\}]
\PY{c+c1}{\PYZsh{} Coeficientes de la funcion objetivo}
\PY{n}{c\PYZus{}dual} \PY{o}{=} \PY{o}{\PYZhy{}}\PY{n}{b}

\PY{c+c1}{\PYZsh{} Coeficientes del lado izquierdo de las desigualdades del tipo \PYZdq{}menor o igual a\PYZdq{}}
\PY{n}{A\PYZus{}dual} \PY{o}{=} \PY{n}{A\PYZus{}primal}\PY{o}{.}\PY{n}{T}

\PY{c+c1}{\PYZsh{} Coeficientes del vector del lado derecho de las desigualdades del tipo \PYZdq{}menor o igual a\PYZdq{}}
\PY{n}{b\PYZus{}dual} \PY{o}{=} \PY{n}{c}

\PY{c+c1}{\PYZsh{} Cotas para las variables}
\PY{n}{bnd} \PY{o}{=} \PY{p}{[}\PY{p}{(}\PY{o}{\PYZhy{}}\PY{n}{scipy}\PY{o}{.}\PY{n}{inf}\PY{p}{,} \PY{n}{scipy}\PY{o}{.}\PY{n}{inf}\PY{p}{)} \PY{k}{for} \PY{n}{i} \PY{o+ow}{in} \PY{n+nb}{range}\PY{p}{(}\PY{n}{m}\PY{p}{)}\PY{p}{]}


\PY{n}{opt\PYZus{}dual} \PY{o}{=} \PY{n}{linprog}\PY{p}{(}\PY{n}{c}\PY{o}{=}\PY{n}{c\PYZus{}dual}\PY{p}{,} \PY{n}{A\PYZus{}ub}\PY{o}{=}\PY{n}{A\PYZus{}dual}\PY{p}{,} \PY{n}{b\PYZus{}ub}\PY{o}{=}\PY{n}{b\PYZus{}dual}\PY{p}{,} \PY{n}{bounds}\PY{o}{=}\PY{n}{bnd}\PY{p}{,}
              \PY{n}{method}\PY{o}{=}\PY{l+s+s1}{\PYZsq{}}\PY{l+s+s1}{interior\PYZhy{}point}\PY{l+s+s1}{\PYZsq{}}\PY{p}{)}

\PY{n+nb}{print}\PY{p}{(}\PY{l+s+s1}{\PYZsq{}}\PY{l+s+se}{\PYZbs{}n}\PY{l+s+s1}{Resultado del proceso:}\PY{l+s+s1}{\PYZsq{}}\PY{p}{,} \PY{n}{opt\PYZus{}dual}\PY{o}{.}\PY{n}{message}\PY{p}{)}
\PY{k}{if} \PY{n}{opt\PYZus{}dual}\PY{o}{.}\PY{n}{success}\PY{p}{:}
    \PY{n+nb}{print}\PY{p}{(}\PY{l+s+s1}{\PYZsq{}}\PY{l+s+s1}{Valor de la función objetivo:}\PY{l+s+s1}{\PYZsq{}}\PY{p}{,} \PY{n}{opt\PYZus{}dual}\PY{o}{.}\PY{n}{fun}\PY{p}{)}
    \PY{n}{lamb}\PY{o}{=}\PY{n}{opt\PYZus{}dual}\PY{o}{.}\PY{n}{x}
    \PY{n}{slack}\PY{o}{=}\PY{n}{opt\PYZus{}dual}\PY{o}{.}\PY{n}{slack}
\end{Verbatim}
\end{tcolorbox}

    \begin{Verbatim}[commandchars=\\\{\}]

Resultado del proceso: Optimization terminated successfully.
Valor de la función objetivo: -5.75877048293428
    \end{Verbatim}

    Por lo que \(\max \mathbf{b}^T\lambda=\min\mathbf{c}^T\mathbf{x}\), que
es lo que pasa en dualidad fuerte.

    A continuación, verificamos las condiciones de KKT con un tolerancia
\(\tau=\sqrt{\epsilon_m}\)

    \begin{tcolorbox}[breakable, size=fbox, boxrule=1pt, pad at break*=1mm,colback=cellbackground, colframe=cellborder]
\prompt{In}{incolor}{8}{\boxspacing}
\begin{Verbatim}[commandchars=\\\{\}]
\PY{n}{importlib}\PY{o}{.}\PY{n}{reload}\PY{p}{(}\PY{n}{lib\PYZus{}examen\PYZus{}2}\PY{p}{)}
\PY{k+kn}{from} \PY{n+nn}{lib\PYZus{}examen\PYZus{}2} \PY{k+kn}{import} \PY{o}{*}

\PY{n}{tol\PYZus{}KKT}\PY{o}{=}\PY{n}{np}\PY{o}{.}\PY{n}{finfo}\PY{p}{(}\PY{n+nb}{float}\PY{p}{)}\PY{o}{.}\PY{n}{eps}\PY{o}{*}\PY{o}{*}\PY{p}{(}\PY{l+m+mi}{1}\PY{o}{/}\PY{l+m+mi}{2}\PY{p}{)}
\PY{n}{KKT\PYZus{}cond}\PY{p}{(}\PY{n}{tol\PYZus{}KKT}\PY{p}{,}\PY{n}{b}\PY{p}{,}\PY{n}{c}\PY{p}{,}\PY{n}{x\PYZus{}1norm}\PY{p}{,}\PY{n}{lamb}\PY{p}{,}\PY{n}{slack}\PY{p}{,}\PY{n}{A\PYZus{}primal}\PY{p}{)}
\end{Verbatim}
\end{tcolorbox}

    \begin{Verbatim}[commandchars=\\\{\}]
Condicion 1: |AT*lamb+s-c| = 3.8459253727671276e-16
Condicion 2: |Ax-b| =  1.1050046463299629e-14
SI se cumple la condicion de no negatividad de x
SI se cumple la condicion de no negatividad de s
SI se cumple la condicion de complentariedad
    \end{Verbatim}

    Finalmente hayamos el \(\mathbf{x}^\ast_1\), que es el vector solución
en el problema con la norma \(1\), con el problema de programación
lineal hemos obtenido tanto su parte posivia como negativa.

    \begin{tcolorbox}[breakable, size=fbox, boxrule=1pt, pad at break*=1mm,colback=cellbackground, colframe=cellborder]
\prompt{In}{incolor}{9}{\boxspacing}
\begin{Verbatim}[commandchars=\\\{\}]
\PY{n}{x\PYZus{}1norm\PYZus{}pos\PYZus{}part}\PY{o}{=}\PY{n}{x\PYZus{}1norm}\PY{p}{[}\PY{p}{:}\PY{n}{n}\PY{p}{]}
\PY{n}{x\PYZus{}1norm\PYZus{}neg\PYZus{}part}\PY{o}{=}\PY{n}{x\PYZus{}1norm}\PY{p}{[}\PY{n}{n}\PY{p}{:}\PY{p}{]}
\PY{n}{x\PYZus{}1norm\PYZus{}gen}\PY{o}{=}\PY{n}{x\PYZus{}1norm\PYZus{}pos\PYZus{}part}\PY{o}{\PYZhy{}}\PY{n}{x\PYZus{}1norm\PYZus{}neg\PYZus{}part}

\PY{n+nb}{print}\PY{p}{(}\PY{l+s+s1}{\PYZsq{}}\PY{l+s+s1}{|Ax\PYZhy{}b| = }\PY{l+s+s1}{\PYZsq{}}\PY{p}{,}\PY{n}{np}\PY{o}{.}\PY{n}{linalg}\PY{o}{.}\PY{n}{norm}\PY{p}{(}\PY{n}{A}\PY{n+nd}{@x\PYZus{}1norm\PYZus{}gen}\PY{o}{\PYZhy{}}\PY{n}{b}\PY{p}{)}\PY{p}{)}
\end{Verbatim}
\end{tcolorbox}

    \begin{Verbatim}[commandchars=\\\{\}]
|Ax-b| =  1.1219401770546083e-14
    \end{Verbatim}

    La gráfica de los componentes de \(\mathbf{x}^\ast_1\) es

    \begin{tcolorbox}[breakable, size=fbox, boxrule=1pt, pad at break*=1mm,colback=cellbackground, colframe=cellborder]
\prompt{In}{incolor}{10}{\boxspacing}
\begin{Verbatim}[commandchars=\\\{\}]
\PY{n}{plt}\PY{o}{.}\PY{n}{bar}\PY{p}{(}\PY{n}{np}\PY{o}{.}\PY{n}{arange}\PY{p}{(}\PY{n}{n}\PY{p}{)}\PY{p}{,} \PY{n}{np}\PY{o}{.}\PY{n}{squeeze}\PY{p}{(}\PY{n}{x\PYZus{}1norm\PYZus{}gen}\PY{p}{)}\PY{p}{)}
\PY{n}{plt}\PY{o}{.}\PY{n}{xlabel}\PY{p}{(}\PY{l+s+s1}{\PYZsq{}}\PY{l+s+s1}{Índice i}\PY{l+s+s1}{\PYZsq{}}\PY{p}{)}
\PY{n}{plt}\PY{o}{.}\PY{n}{ylabel}\PY{p}{(}\PY{l+s+sa}{r}\PY{l+s+s1}{\PYZsq{}}\PY{l+s+s1}{\PYZdl{}x\PYZus{}i\PYZdl{}}\PY{l+s+s1}{\PYZsq{}}\PY{p}{)}
\PY{n}{plt}\PY{o}{.}\PY{n}{title}\PY{p}{(}\PY{l+s+sa}{r}\PY{l+s+s1}{\PYZsq{}}\PY{l+s+s1}{Componentes del vector \PYZdl{}}\PY{l+s+s1}{\PYZbs{}}\PY{l+s+s1}{mathbf}\PY{l+s+si}{\PYZob{}x\PYZcb{}}\PY{l+s+s1}{\PYZca{}}\PY{l+s+s1}{\PYZbs{}}\PY{l+s+s1}{ast\PYZus{}1\PYZdl{} que genera los datos}\PY{l+s+s1}{\PYZsq{}}\PY{p}{,} \PY{n}{fontsize}\PY{o}{=}\PY{l+m+mi}{14}\PY{p}{)}
\end{Verbatim}
\end{tcolorbox}

            \begin{tcolorbox}[breakable, size=fbox, boxrule=.5pt, pad at break*=1mm, opacityfill=0]
\prompt{Out}{outcolor}{10}{\boxspacing}
\begin{Verbatim}[commandchars=\\\{\}]
Text(0.5, 1.0, 'Componentes del vector \$\textbackslash{}\textbackslash{}mathbf\{x\}\^{}\textbackslash{}\textbackslash{}ast\_1\$ que genera los
datos')
\end{Verbatim}
\end{tcolorbox}
        
    \begin{center}
    \adjustimage{max size={0.9\linewidth}{0.9\paperheight}}{parcial2_optimizacion_files/parcial2_optimizacion_25_1.png}
    \end{center}
    { \hspace*{\fill} \\}
    
    Evidentemente este vector es mucho más ralo que el real, lo que sugiere
un modelo más simple aunque los datos fueron generados por un modelo más
complejo.

    \hypertarget{caso-m50}{%
\paragraph{\texorpdfstring{Caso \(m=50\)}{Caso m=50}}\label{caso-m50}}

    Ahora replicamos los pasos, aumentando el número de ecuaciones

    \begin{tcolorbox}[breakable, size=fbox, boxrule=1pt, pad at break*=1mm,colback=cellbackground, colframe=cellborder]
\prompt{In}{incolor}{11}{\boxspacing}
\begin{Verbatim}[commandchars=\\\{\}]
\PY{k+kn}{import} \PY{n+nn}{numpy} \PY{k}{as} \PY{n+nn}{np}
\PY{k+kn}{import} \PY{n+nn}{matplotlib}\PY{n+nn}{.}\PY{n+nn}{pyplot} \PY{k}{as} \PY{n+nn}{plt}

\PY{c+c1}{\PYZsh{} Evalua cada función trigonométrica. }
\PY{k}{def} \PY{n+nf}{fnctrig}\PY{p}{(}\PY{n}{x}\PY{p}{,} \PY{n}{fparam}\PY{p}{)}\PY{p}{:}
    \PY{n}{omega} \PY{o}{=} \PY{n}{fparam}\PY{p}{[}\PY{l+s+s1}{\PYZsq{}}\PY{l+s+s1}{omega}\PY{l+s+s1}{\PYZsq{}}\PY{p}{]}
    \PY{n}{phi}   \PY{o}{=} \PY{n}{fparam}\PY{p}{[}\PY{l+s+s1}{\PYZsq{}}\PY{l+s+s1}{phi}\PY{l+s+s1}{\PYZsq{}}\PY{p}{]}
    \PY{k}{return} \PY{n}{np}\PY{o}{.}\PY{n}{sin}\PY{p}{(}\PY{n}{omega}\PY{o}{*}\PY{n}{x} \PY{o}{+} \PY{n}{phi}\PY{p}{)}

\PY{n}{m}    \PY{o}{=} \PY{l+m+mi}{50}   \PY{c+c1}{\PYZsh{} Número de puntos z\PYZus{}i}
\PY{n}{r}    \PY{o}{=} \PY{l+m+mi}{10}   \PY{c+c1}{\PYZsh{} Número de angulos phi}
\PY{n}{s}    \PY{o}{=} \PY{l+m+mi}{20}   \PY{c+c1}{\PYZsh{} Número de frecuencias omega}

\PY{n}{z}     \PY{o}{=} \PY{n}{np}\PY{o}{.}\PY{n}{linspace}\PY{p}{(}\PY{l+m+mi}{0}\PY{p}{,} \PY{l+m+mi}{9}\PY{p}{,} \PY{n}{m}\PY{p}{)}
\PY{n}{phi}   \PY{o}{=} \PY{n}{np}\PY{o}{.}\PY{n}{linspace}\PY{p}{(}\PY{l+m+mi}{0}\PY{p}{,} \PY{n}{np}\PY{o}{.}\PY{n}{pi}\PY{p}{,} \PY{n}{r}\PY{p}{)}
\PY{n}{omega} \PY{o}{=} \PY{n}{np}\PY{o}{.}\PY{n}{linspace}\PY{p}{(}\PY{l+m+mf}{0.5}\PY{p}{,} \PY{l+m+mf}{2.0}\PY{p}{,} \PY{n}{s}\PY{p}{)}

\PY{n+nb}{print}\PY{p}{(}\PY{l+s+s1}{\PYZsq{}}\PY{l+s+s1}{Vector de frecuencias omega:}\PY{l+s+se}{\PYZbs{}n}\PY{l+s+s1}{\PYZsq{}}\PY{p}{,} \PY{n}{omega}\PY{p}{)}
\PY{n+nb}{print}\PY{p}{(}\PY{l+s+s1}{\PYZsq{}}\PY{l+s+se}{\PYZbs{}n}\PY{l+s+s1}{Vector de ángulos de fase phi:}\PY{l+s+se}{\PYZbs{}n}\PY{l+s+s1}{\PYZsq{}}\PY{p}{,} \PY{n}{phi}\PY{p}{)}

\PY{n}{n}  \PY{o}{=} \PY{n}{r}\PY{o}{*}\PY{n}{s}

\PY{c+c1}{\PYZsh{} Se crea la matriz A }
\PY{n}{A}  \PY{o}{=} \PY{n}{np}\PY{o}{.}\PY{n}{zeros}\PY{p}{(}\PY{p}{(}\PY{n}{m}\PY{p}{,} \PY{n}{n}\PY{p}{)}\PY{p}{)}
\PY{k}{for} \PY{n}{i}\PY{p}{,}\PY{n}{z\PYZus{}i} \PY{o+ow}{in} \PY{n+nb}{enumerate}\PY{p}{(}\PY{n}{z}\PY{p}{)}\PY{p}{:}
    \PY{n}{mc} \PY{o}{=} \PY{l+m+mi}{0}
    \PY{k}{for} \PY{n}{j}\PY{p}{,}\PY{n}{phi\PYZus{}j} \PY{o+ow}{in} \PY{n+nb}{enumerate}\PY{p}{(}\PY{n}{phi}\PY{p}{)}\PY{p}{:}
        \PY{k}{for} \PY{n}{l}\PY{p}{,}\PY{n}{omega\PYZus{}l} \PY{o+ow}{in} \PY{n+nb}{enumerate}\PY{p}{(}\PY{n}{omega}\PY{p}{)}\PY{p}{:}
            \PY{n}{fparam}   \PY{o}{=} \PY{p}{\PYZob{}}\PY{l+s+s1}{\PYZsq{}}\PY{l+s+s1}{omega}\PY{l+s+s1}{\PYZsq{}}\PY{p}{:}\PY{n}{omega\PYZus{}l}\PY{p}{,} \PY{l+s+s1}{\PYZsq{}}\PY{l+s+s1}{phi}\PY{l+s+s1}{\PYZsq{}}\PY{p}{:}\PY{n}{phi\PYZus{}j}\PY{p}{\PYZcb{}}
            \PY{n}{A}\PY{p}{[}\PY{n}{i}\PY{p}{,}\PY{n}{mc}\PY{p}{]} \PY{o}{=} \PY{n}{fnctrig}\PY{p}{(}\PY{n}{z\PYZus{}i}\PY{p}{,} \PY{n}{fparam}\PY{p}{)}
            \PY{n}{mc} \PY{o}{+}\PY{o}{=} \PY{l+m+mi}{1}

\PY{n}{x\PYZus{}real} \PY{o}{=} \PY{n}{np}\PY{o}{.}\PY{n}{zeros}\PY{p}{(}\PY{n}{n}\PY{p}{)}
\PY{n}{nc} \PY{o}{=} \PY{l+m+mi}{1} 
\PY{k}{for} \PY{n}{i} \PY{o+ow}{in} \PY{n+nb}{range}\PY{p}{(}\PY{n}{n}\PY{p}{)}\PY{p}{:}
    \PY{k}{if} \PY{n}{i}\PY{o}{\PYZpc{}}\PY{k}{r}==3 and (i//s)\PYZpc{}2==0:
        \PY{n}{x\PYZus{}real}\PY{p}{[}\PY{n}{i}\PY{p}{]} \PY{o}{=} \PY{p}{(}\PY{o}{\PYZhy{}}\PY{l+m+mi}{1}\PY{p}{)}\PY{o}{*}\PY{o}{*}\PY{n}{nc}
        \PY{n}{nc}  \PY{o}{+}\PY{o}{=} \PY{l+m+mi}{1}

\PY{n}{b}  \PY{o}{=} \PY{n}{A} \PY{o}{@} \PY{n}{x\PYZus{}real}

\PY{n+nb}{print}\PY{p}{(}\PY{l+s+s1}{\PYZsq{}}\PY{l+s+se}{\PYZbs{}n}\PY{l+s+s1}{Dimensiones de la matriz A:}\PY{l+s+s1}{\PYZsq{}}\PY{p}{,} \PY{n}{A}\PY{o}{.}\PY{n}{shape}\PY{p}{)}

\PY{n}{plt}\PY{o}{.}\PY{n}{figure}\PY{p}{(}\PY{p}{)}
\PY{n}{plt}\PY{o}{.}\PY{n}{bar}\PY{p}{(}\PY{n}{np}\PY{o}{.}\PY{n}{arange}\PY{p}{(}\PY{n}{n}\PY{p}{)}\PY{p}{,} \PY{n}{np}\PY{o}{.}\PY{n}{squeeze}\PY{p}{(}\PY{n}{x\PYZus{}real}\PY{p}{)}\PY{p}{)}
\PY{n}{plt}\PY{o}{.}\PY{n}{xlabel}\PY{p}{(}\PY{l+s+s1}{\PYZsq{}}\PY{l+s+s1}{Índice i}\PY{l+s+s1}{\PYZsq{}}\PY{p}{)}
\PY{n}{plt}\PY{o}{.}\PY{n}{ylabel}\PY{p}{(}\PY{l+s+sa}{r}\PY{l+s+s1}{\PYZsq{}}\PY{l+s+s1}{\PYZdl{}x\PYZus{}i\PYZdl{}}\PY{l+s+s1}{\PYZsq{}}\PY{p}{)}
\PY{n}{plt}\PY{o}{.}\PY{n}{title}\PY{p}{(}\PY{l+s+sa}{r}\PY{l+s+s1}{\PYZsq{}}\PY{l+s+s1}{Componentes del vector \PYZdl{}x\PYZus{}}\PY{l+s+si}{\PYZob{}real\PYZcb{}}\PY{l+s+s1}{\PYZdl{} que genera los datos}\PY{l+s+s1}{\PYZsq{}}\PY{p}{,} \PY{n}{fontsize}\PY{o}{=}\PY{l+m+mi}{14}\PY{p}{)}

\PY{n}{plt}\PY{o}{.}\PY{n}{figure}\PY{p}{(}\PY{p}{)}
\PY{n}{plt}\PY{o}{.}\PY{n}{plot}\PY{p}{(}\PY{n}{np}\PY{o}{.}\PY{n}{arange}\PY{p}{(}\PY{n}{m}\PY{p}{)}\PY{p}{,} \PY{n}{b}\PY{p}{,} \PY{l+s+s1}{\PYZsq{}}\PY{l+s+s1}{r}\PY{l+s+s1}{\PYZsq{}}\PY{p}{,} \PY{n}{linewidth}\PY{o}{=}\PY{l+m+mi}{1}\PY{p}{)}
\PY{n}{plt}\PY{o}{.}\PY{n}{xlabel}\PY{p}{(}\PY{l+s+s1}{\PYZsq{}}\PY{l+s+s1}{Índice i}\PY{l+s+s1}{\PYZsq{}}\PY{p}{)}
\PY{n}{plt}\PY{o}{.}\PY{n}{ylabel}\PY{p}{(}\PY{l+s+sa}{r}\PY{l+s+s1}{\PYZsq{}}\PY{l+s+s1}{\PYZdl{}b\PYZus{}i\PYZdl{}}\PY{l+s+s1}{\PYZsq{}}\PY{p}{)}
\PY{n}{\PYZus{}}\PY{o}{=}\PY{n}{plt}\PY{o}{.}\PY{n}{title}\PY{p}{(}\PY{l+s+sa}{r}\PY{l+s+s1}{\PYZsq{}}\PY{l+s+s1}{Datos generados \PYZdl{}b\PYZus{}i\PYZdl{}}\PY{l+s+s1}{\PYZsq{}}\PY{p}{,} \PY{n}{fontsize}\PY{o}{=}\PY{l+m+mi}{14}\PY{p}{)}
\end{Verbatim}
\end{tcolorbox}

    \begin{Verbatim}[commandchars=\\\{\}]
Vector de frecuencias omega:
 [0.5        0.57894737 0.65789474 0.73684211 0.81578947 0.89473684
 0.97368421 1.05263158 1.13157895 1.21052632 1.28947368 1.36842105
 1.44736842 1.52631579 1.60526316 1.68421053 1.76315789 1.84210526
 1.92105263 2.        ]

Vector de ángulos de fase phi:
 [0.         0.34906585 0.6981317  1.04719755 1.3962634  1.74532925
 2.0943951  2.44346095 2.7925268  3.14159265]

Dimensiones de la matriz A: (50, 200)
    \end{Verbatim}

    \begin{center}
    \adjustimage{max size={0.9\linewidth}{0.9\paperheight}}{parcial2_optimizacion_files/parcial2_optimizacion_29_1.png}
    \end{center}
    { \hspace*{\fill} \\}
    
    \begin{center}
    \adjustimage{max size={0.9\linewidth}{0.9\paperheight}}{parcial2_optimizacion_files/parcial2_optimizacion_29_2.png}
    \end{center}
    { \hspace*{\fill} \\}
    
    La solución con norma 2 la obtenemos a continuación. El error es

    \begin{tcolorbox}[breakable, size=fbox, boxrule=1pt, pad at break*=1mm,colback=cellbackground, colframe=cellborder]
\prompt{In}{incolor}{12}{\boxspacing}
\begin{Verbatim}[commandchars=\\\{\}]
\PY{c+c1}{\PYZsh{} Norma 2}
\PY{n}{x\PYZus{}2norm}\PY{o}{=}\PY{n}{A}\PY{o}{.}\PY{n}{T}\PY{n+nd}{@np}\PY{o}{.}\PY{n}{linalg}\PY{o}{.}\PY{n}{inv}\PY{p}{(}\PY{n}{A}\PY{n+nd}{@A}\PY{o}{.}\PY{n}{T}\PY{p}{)}\PY{n+nd}{@b}
\PY{n+nb}{print}\PY{p}{(}\PY{l+s+s1}{\PYZsq{}}\PY{l+s+s1}{|Ax\PYZhy{}b| = }\PY{l+s+s1}{\PYZsq{}}\PY{p}{,}\PY{n}{np}\PY{o}{.}\PY{n}{linalg}\PY{o}{.}\PY{n}{norm}\PY{p}{(}\PY{n}{A}\PY{n+nd}{@x\PYZus{}2norm}\PY{o}{\PYZhy{}}\PY{n}{b}\PY{p}{)}\PY{p}{)}
\end{Verbatim}
\end{tcolorbox}

    \begin{Verbatim}[commandchars=\\\{\}]
|Ax-b| =  114.54984322728232
    \end{Verbatim}

    Sus componentes son

    \begin{tcolorbox}[breakable, size=fbox, boxrule=1pt, pad at break*=1mm,colback=cellbackground, colframe=cellborder]
\prompt{In}{incolor}{13}{\boxspacing}
\begin{Verbatim}[commandchars=\\\{\}]
\PY{n}{plt}\PY{o}{.}\PY{n}{bar}\PY{p}{(}\PY{n}{np}\PY{o}{.}\PY{n}{arange}\PY{p}{(}\PY{n}{n}\PY{p}{)}\PY{p}{,} \PY{n}{np}\PY{o}{.}\PY{n}{squeeze}\PY{p}{(}\PY{n}{x\PYZus{}2norm}\PY{p}{)}\PY{p}{)}
\PY{n}{plt}\PY{o}{.}\PY{n}{xlabel}\PY{p}{(}\PY{l+s+s1}{\PYZsq{}}\PY{l+s+s1}{Índice i}\PY{l+s+s1}{\PYZsq{}}\PY{p}{)}
\PY{n}{plt}\PY{o}{.}\PY{n}{ylabel}\PY{p}{(}\PY{l+s+sa}{r}\PY{l+s+s1}{\PYZsq{}}\PY{l+s+s1}{\PYZdl{}x\PYZus{}i\PYZdl{}}\PY{l+s+s1}{\PYZsq{}}\PY{p}{)}
\PY{n}{plt}\PY{o}{.}\PY{n}{title}\PY{p}{(}\PY{l+s+sa}{r}\PY{l+s+s1}{\PYZsq{}}\PY{l+s+s1}{Componentes del vector \PYZdl{}}\PY{l+s+s1}{\PYZbs{}}\PY{l+s+s1}{mathbf}\PY{l+s+si}{\PYZob{}x\PYZcb{}}\PY{l+s+s1}{\PYZca{}}\PY{l+s+s1}{\PYZbs{}}\PY{l+s+s1}{ast\PYZus{}2\PYZdl{} que genera los datos}\PY{l+s+s1}{\PYZsq{}}\PY{p}{,} \PY{n}{fontsize}\PY{o}{=}\PY{l+m+mi}{14}\PY{p}{)}
\end{Verbatim}
\end{tcolorbox}

            \begin{tcolorbox}[breakable, size=fbox, boxrule=.5pt, pad at break*=1mm, opacityfill=0]
\prompt{Out}{outcolor}{13}{\boxspacing}
\begin{Verbatim}[commandchars=\\\{\}]
Text(0.5, 1.0, 'Componentes del vector \$\textbackslash{}\textbackslash{}mathbf\{x\}\^{}\textbackslash{}\textbackslash{}ast\_2\$ que genera los
datos')
\end{Verbatim}
\end{tcolorbox}
        
    \begin{center}
    \adjustimage{max size={0.9\linewidth}{0.9\paperheight}}{parcial2_optimizacion_files/parcial2_optimizacion_33_1.png}
    \end{center}
    { \hspace*{\fill} \\}
    
    El problema primal en este caso es

    \begin{tcolorbox}[breakable, size=fbox, boxrule=1pt, pad at break*=1mm,colback=cellbackground, colframe=cellborder]
\prompt{In}{incolor}{14}{\boxspacing}
\begin{Verbatim}[commandchars=\\\{\}]
\PY{k+kn}{from} \PY{n+nn}{scipy}\PY{n+nn}{.}\PY{n+nn}{optimize} \PY{k+kn}{import} \PY{n}{linprog}
\PY{k+kn}{import} \PY{n+nn}{scipy}

\PY{n}{c} \PY{o}{=} \PY{n}{np}\PY{o}{.}\PY{n}{ones}\PY{p}{(}\PY{l+m+mi}{2}\PY{o}{*}\PY{n}{n}\PY{p}{)}
\PY{n}{A\PYZus{}primal}\PY{o}{=}\PY{n}{np}\PY{o}{.}\PY{n}{concatenate}\PY{p}{(}\PY{p}{(}\PY{n}{A}\PY{p}{,} \PY{o}{\PYZhy{}}\PY{n}{A}\PY{p}{)}\PY{p}{,} \PY{n}{axis}\PY{o}{=}\PY{l+m+mi}{1}\PY{p}{)}

\PY{c+c1}{\PYZsh{} Cotas de las variables}
\PY{n}{bnd} \PY{o}{=} \PY{p}{[}\PY{p}{(}\PY{l+m+mi}{0}\PY{p}{,} \PY{n}{scipy}\PY{o}{.}\PY{n}{inf}\PY{p}{)} \PY{k}{for} \PY{n}{i} \PY{o+ow}{in} \PY{n+nb}{range}\PY{p}{(}\PY{l+m+mi}{2}\PY{o}{*}\PY{n}{n}\PY{p}{)}\PY{p}{]}

\PY{n}{opt\PYZus{}primal} \PY{o}{=} \PY{n}{linprog}\PY{p}{(}\PY{n}{c}\PY{o}{=}\PY{n}{c}\PY{p}{,} \PY{n}{A\PYZus{}eq}\PY{o}{=}\PY{n}{A\PYZus{}primal}\PY{p}{,} \PY{n}{b\PYZus{}eq}\PY{o}{=}\PY{n}{b}\PY{p}{,} \PY{n}{bounds}\PY{o}{=}\PY{n}{bnd}\PY{p}{,}
              \PY{n}{method}\PY{o}{=}\PY{l+s+s1}{\PYZsq{}}\PY{l+s+s1}{interior\PYZhy{}point}\PY{l+s+s1}{\PYZsq{}}\PY{p}{)}

\PY{n+nb}{print}\PY{p}{(}\PY{l+s+s1}{\PYZsq{}}\PY{l+s+se}{\PYZbs{}n}\PY{l+s+s1}{Resultado del proceso:}\PY{l+s+s1}{\PYZsq{}}\PY{p}{,} \PY{n}{opt\PYZus{}primal}\PY{o}{.}\PY{n}{message}\PY{p}{)}
\PY{k}{if} \PY{n}{opt\PYZus{}primal}\PY{o}{.}\PY{n}{success}\PY{p}{:}
    \PY{n+nb}{print}\PY{p}{(}\PY{l+s+s1}{\PYZsq{}}\PY{l+s+s1}{Valor de la función objetivo:}\PY{l+s+s1}{\PYZsq{}}\PY{p}{,} \PY{n}{opt\PYZus{}primal}\PY{o}{.}\PY{n}{fun}\PY{p}{)}
    \PY{n}{x\PYZus{}1norm}\PY{o}{=}\PY{n}{opt\PYZus{}primal}\PY{o}{.}\PY{n}{x}
\end{Verbatim}
\end{tcolorbox}

    \begin{Verbatim}[commandchars=\\\{\}]

Resultado del proceso: Optimization terminated successfully.
Valor de la función objetivo: 5.758770483204005
    \end{Verbatim}

    El dual correspondiente

    \begin{tcolorbox}[breakable, size=fbox, boxrule=1pt, pad at break*=1mm,colback=cellbackground, colframe=cellborder]
\prompt{In}{incolor}{15}{\boxspacing}
\begin{Verbatim}[commandchars=\\\{\}]
\PY{c+c1}{\PYZsh{} Coeficientes de la funcion objetivo}
\PY{n}{c\PYZus{}dual} \PY{o}{=} \PY{o}{\PYZhy{}}\PY{n}{b}

\PY{c+c1}{\PYZsh{} Coeficientes del lado izquierdo de las desigualdades del tipo \PYZdq{}menor o igual a\PYZdq{}}
\PY{n}{A\PYZus{}dual} \PY{o}{=} \PY{n}{A\PYZus{}primal}\PY{o}{.}\PY{n}{T}

\PY{c+c1}{\PYZsh{} Coeficientes del vector del lado derecho de las desigualdades del tipo \PYZdq{}menor o igual a\PYZdq{}}
\PY{n}{b\PYZus{}dual} \PY{o}{=} \PY{n}{c}

\PY{c+c1}{\PYZsh{} Cotas para las variables}
\PY{n}{bnd} \PY{o}{=} \PY{p}{[}\PY{p}{(}\PY{o}{\PYZhy{}}\PY{n}{scipy}\PY{o}{.}\PY{n}{inf}\PY{p}{,} \PY{n}{scipy}\PY{o}{.}\PY{n}{inf}\PY{p}{)} \PY{k}{for} \PY{n}{i} \PY{o+ow}{in} \PY{n+nb}{range}\PY{p}{(}\PY{n}{m}\PY{p}{)}\PY{p}{]}


\PY{n}{opt\PYZus{}dual} \PY{o}{=} \PY{n}{linprog}\PY{p}{(}\PY{n}{c}\PY{o}{=}\PY{n}{c\PYZus{}dual}\PY{p}{,} \PY{n}{A\PYZus{}ub}\PY{o}{=}\PY{n}{A\PYZus{}dual}\PY{p}{,} \PY{n}{b\PYZus{}ub}\PY{o}{=}\PY{n}{b\PYZus{}dual}\PY{p}{,} \PY{n}{bounds}\PY{o}{=}\PY{n}{bnd}\PY{p}{,}
              \PY{n}{method}\PY{o}{=}\PY{l+s+s1}{\PYZsq{}}\PY{l+s+s1}{interior\PYZhy{}point}\PY{l+s+s1}{\PYZsq{}}\PY{p}{)}

\PY{n+nb}{print}\PY{p}{(}\PY{l+s+s1}{\PYZsq{}}\PY{l+s+se}{\PYZbs{}n}\PY{l+s+s1}{Resultado del proceso:}\PY{l+s+s1}{\PYZsq{}}\PY{p}{,} \PY{n}{opt\PYZus{}dual}\PY{o}{.}\PY{n}{message}\PY{p}{)}
\PY{k}{if} \PY{n}{opt\PYZus{}dual}\PY{o}{.}\PY{n}{success}\PY{p}{:}
    \PY{n+nb}{print}\PY{p}{(}\PY{l+s+s1}{\PYZsq{}}\PY{l+s+s1}{Valor de la función objetivo:}\PY{l+s+s1}{\PYZsq{}}\PY{p}{,} \PY{n}{opt\PYZus{}dual}\PY{o}{.}\PY{n}{fun}\PY{p}{)}
    \PY{n}{lamb}\PY{o}{=}\PY{n}{opt\PYZus{}dual}\PY{o}{.}\PY{n}{x}
    \PY{n}{slack}\PY{o}{=}\PY{n}{opt\PYZus{}dual}\PY{o}{.}\PY{n}{slack}
\end{Verbatim}
\end{tcolorbox}

    \begin{Verbatim}[commandchars=\\\{\}]

Resultado del proceso: Optimization terminated successfully.
Valor de la función objetivo: -5.758770483051036
    \end{Verbatim}

    Verificamos las condiciones KKT

    \begin{tcolorbox}[breakable, size=fbox, boxrule=1pt, pad at break*=1mm,colback=cellbackground, colframe=cellborder]
\prompt{In}{incolor}{16}{\boxspacing}
\begin{Verbatim}[commandchars=\\\{\}]
\PY{n}{importlib}\PY{o}{.}\PY{n}{reload}\PY{p}{(}\PY{n}{lib\PYZus{}examen\PYZus{}2}\PY{p}{)}
\PY{k+kn}{from} \PY{n+nn}{lib\PYZus{}examen\PYZus{}2} \PY{k+kn}{import} \PY{o}{*}

\PY{n}{tol\PYZus{}KKT}\PY{o}{=}\PY{n}{np}\PY{o}{.}\PY{n}{finfo}\PY{p}{(}\PY{n+nb}{float}\PY{p}{)}\PY{o}{.}\PY{n}{eps}\PY{o}{*}\PY{o}{*}\PY{p}{(}\PY{l+m+mi}{1}\PY{o}{/}\PY{l+m+mi}{2}\PY{p}{)}
\PY{n}{KKT\PYZus{}cond}\PY{p}{(}\PY{n}{tol\PYZus{}KKT}\PY{p}{,}\PY{n}{b}\PY{p}{,}\PY{n}{c}\PY{p}{,}\PY{n}{x\PYZus{}1norm}\PY{p}{,}\PY{n}{lamb}\PY{p}{,}\PY{n}{slack}\PY{p}{,}\PY{n}{A\PYZus{}primal}\PY{p}{)}
\end{Verbatim}
\end{tcolorbox}

    \begin{Verbatim}[commandchars=\\\{\}]
Condicion 1: |AT*lamb+s-c| = 2.9373740229761033e-16
Condicion 2: |Ax-b| =  3.67522363563323e-10
SI se cumple la condicion de no negatividad de x
SI se cumple la condicion de no negatividad de s
SI se cumple la condicion de complentariedad
    \end{Verbatim}

    El error en este caso es

    \begin{tcolorbox}[breakable, size=fbox, boxrule=1pt, pad at break*=1mm,colback=cellbackground, colframe=cellborder]
\prompt{In}{incolor}{17}{\boxspacing}
\begin{Verbatim}[commandchars=\\\{\}]
\PY{n}{x\PYZus{}1norm\PYZus{}pos\PYZus{}part}\PY{o}{=}\PY{n}{x\PYZus{}1norm}\PY{p}{[}\PY{p}{:}\PY{n}{n}\PY{p}{]}
\PY{n}{x\PYZus{}1norm\PYZus{}neg\PYZus{}part}\PY{o}{=}\PY{n}{x\PYZus{}1norm}\PY{p}{[}\PY{n}{n}\PY{p}{:}\PY{p}{]}
\PY{n}{x\PYZus{}1norm\PYZus{}gen}\PY{o}{=}\PY{n}{x\PYZus{}1norm\PYZus{}pos\PYZus{}part}\PY{o}{\PYZhy{}}\PY{n}{x\PYZus{}1norm\PYZus{}neg\PYZus{}part}

\PY{n+nb}{print}\PY{p}{(}\PY{l+s+s1}{\PYZsq{}}\PY{l+s+s1}{|Ax\PYZhy{}b| = }\PY{l+s+s1}{\PYZsq{}}\PY{p}{,}\PY{n}{np}\PY{o}{.}\PY{n}{linalg}\PY{o}{.}\PY{n}{norm}\PY{p}{(}\PY{n}{A}\PY{n+nd}{@x\PYZus{}1norm\PYZus{}gen}\PY{o}{\PYZhy{}}\PY{n}{b}\PY{p}{)}\PY{p}{)}
\end{Verbatim}
\end{tcolorbox}

    \begin{Verbatim}[commandchars=\\\{\}]
|Ax-b| =  3.675217025201218e-10
    \end{Verbatim}

    Y las componentes de la solución con la norma 1 las visualizamos a
continuación

    \begin{tcolorbox}[breakable, size=fbox, boxrule=1pt, pad at break*=1mm,colback=cellbackground, colframe=cellborder]
\prompt{In}{incolor}{18}{\boxspacing}
\begin{Verbatim}[commandchars=\\\{\}]
\PY{n}{plt}\PY{o}{.}\PY{n}{bar}\PY{p}{(}\PY{n}{np}\PY{o}{.}\PY{n}{arange}\PY{p}{(}\PY{n}{n}\PY{p}{)}\PY{p}{,} \PY{n}{np}\PY{o}{.}\PY{n}{squeeze}\PY{p}{(}\PY{n}{x\PYZus{}1norm\PYZus{}gen}\PY{p}{)}\PY{p}{)}
\PY{n}{plt}\PY{o}{.}\PY{n}{xlabel}\PY{p}{(}\PY{l+s+s1}{\PYZsq{}}\PY{l+s+s1}{Índice i}\PY{l+s+s1}{\PYZsq{}}\PY{p}{)}
\PY{n}{plt}\PY{o}{.}\PY{n}{ylabel}\PY{p}{(}\PY{l+s+sa}{r}\PY{l+s+s1}{\PYZsq{}}\PY{l+s+s1}{\PYZdl{}x\PYZus{}i\PYZdl{}}\PY{l+s+s1}{\PYZsq{}}\PY{p}{)}
\PY{n}{plt}\PY{o}{.}\PY{n}{title}\PY{p}{(}\PY{l+s+sa}{r}\PY{l+s+s1}{\PYZsq{}}\PY{l+s+s1}{Componentes del vector \PYZdl{}}\PY{l+s+s1}{\PYZbs{}}\PY{l+s+s1}{mathbf}\PY{l+s+si}{\PYZob{}x\PYZcb{}}\PY{l+s+s1}{\PYZca{}}\PY{l+s+s1}{\PYZbs{}}\PY{l+s+s1}{ast\PYZus{}1\PYZdl{} que genera los datos}\PY{l+s+s1}{\PYZsq{}}\PY{p}{,} \PY{n}{fontsize}\PY{o}{=}\PY{l+m+mi}{14}\PY{p}{)}
\end{Verbatim}
\end{tcolorbox}

            \begin{tcolorbox}[breakable, size=fbox, boxrule=.5pt, pad at break*=1mm, opacityfill=0]
\prompt{Out}{outcolor}{18}{\boxspacing}
\begin{Verbatim}[commandchars=\\\{\}]
Text(0.5, 1.0, 'Componentes del vector \$\textbackslash{}\textbackslash{}mathbf\{x\}\^{}\textbackslash{}\textbackslash{}ast\_1\$ que genera los
datos')
\end{Verbatim}
\end{tcolorbox}
        
    \begin{center}
    \adjustimage{max size={0.9\linewidth}{0.9\paperheight}}{parcial2_optimizacion_files/parcial2_optimizacion_43_1.png}
    \end{center}
    { \hspace*{\fill} \\}
    
    De igual forma tenemos una solución igual de sparse a pesar de aumentar
el numero de ecuaciones

    Para justificar la fórmula cerrada que tenemos para el caso del problema
de optimización con la norma 2 utilizaremos el método de multiplicadores
de Lagrange.

La función langrangiana es la siguiente \[ 
\mathcal{L}(\mathbf{x},\lambda)=\lVert \mathbf{x} \rVert_2-\lambda^T(\mathbf{A}\mathbf{x}-\mathbf{b}),
\]

luego el gradiente es \[
\nabla \mathcal{L}(\mathbf{x},\lambda)=\begin{bmatrix}2\mathbf{x}-\mathbf{A}^T\lambda \\
\mathbf{A}\mathbf{x}-\mathbf{b}\end{bmatrix}
\]

    Por lo tanto, un punto critico es a que que anula el gradiente anterior,
derivamos que \(\lambda^\ast=(\mathbf{A}\mathbf{A}^T)^{-1}b\) por lo que
\[ 
    \mathbf{x}^\ast=\mathbf{A}^T(\mathbf{A}\mathbf{A}^T)^{-1}b,
\] y se puede ver que \((\mathbf{x}^\ast,\lambda^\ast)\) cumplen las
condiciones de segundo orden por lo que es mínimo local, y en virtud de
estar en un problema convexo se tiene que \(\mathbf{x}^\ast\) es mínimo
global del problema de optimización con restricciones.


    % Add a bibliography block to the postdoc
    
    
    
\end{document}
